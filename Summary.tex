\documentclass[cyan]{elegantnote}
\author{Yuyang Songsheng}
\email{songshengyuyang@gmail.com}
\zhtitle{物理}
\entitle{Physics}
\version{1.00}
\myquote{Summary is the best way to say "Good Bye"}
\logo{logo.jpg}
\cover{cover.pdf}
%green color
   \definecolor{main1}{RGB}{210,168,75}
   \definecolor{seco1}{RGB}{9,80,3}
   \definecolor{thid1}{RGB}{0,175,152}
%cyan color
   \definecolor{main2}{RGB}{239,126,30}
   \definecolor{seco2}{RGB}{0,175,152}
   \definecolor{thid2}{RGB}{236,74,53}
%cyan color
   \definecolor{main3}{RGB}{127,191,51}
   \definecolor{seco3}{RGB}{0,145,215}
   \definecolor{thid3}{RGB}{180,27,131}


\usepackage{makecell}
\usepackage{lipsum}
\usepackage{amssymb}
\usepackage{float}
\usepackage{wrapfig}
\usepackage{latexsym}
\usepackage{hyperref}
\usepackage{feynmf}
\usepackage{exscale}
\usepackage{relsize}
\usepackage{slashed}
\usepackage{bm}%bold math, for vector


\begin{document}
\maketitle
\tableofcontents
\part{Classical Mechanics}
\documentclass{article}
\usepackage[left=1.5cm, right=1.5cm, top=3cm, bottom = 3cm]{geometry}
\usepackage{amsmath}
\usepackage{mathrsfs}
\usepackage{amsfonts}
\usepackage{amssymb}
\usepackage{graphicx}
\usepackage{float}
\usepackage{wrapfig}
\usepackage{latexsym}
\usepackage{hyperref}
\usepackage{feynmf}
\usepackage{exscale}
\usepackage{relsize}
\usepackage{bm}%bold math, for vector
\linespread{1.1}


\author{Yuyang Songsheng}
\title{Summary on Classical Mechanics}

\begin{document}
\maketitle
\section{Lagrangian Mechanics}
Lagrangian and Action:
\begin{equation}
S=\int_{t_1}^{t_2}L(q,\dot{q},t)dt
\end{equation}
Hamilton Principle:
\begin{equation}
\delta S=0
\end{equation}
Euler-Lagrangian equation:
\begin{equation}
\frac{d}{dt}(\frac{\partial L}{\partial \dot{q_i}}) - \frac{\partial L}{\partial q_i}=0
\end{equation}
The form of Lagrangian for a system of particles in inertial frame:
\begin{equation}
L=\sum_a \frac{1}{2}m_a v_a^2 -U(\vec{r_1},\vec{r_2},\cdots,)
\end{equation}
Notes: To get the form of Lagrangian for a system of interacting particles, we must assume:\\
(1) Space and time are homogeneous and isotropic in inertial frame;\\
(2) Galileo's relativity principle and Galilean transformation;\\
(3) Spontaneous interaction between particles;\\

\section{Symmetry and Conservation Laws}
\paragraph{Nother's theorem}
For $q_i \to q_i+\delta q_i$ and $L \to L+\delta L$, if $\delta L= \frac{d f(q,\dot{q},t)}{dt}$,then we get
\begin{equation}
\frac{d}{dt}(p_i \delta q_i-f)=0 \ \ \ (p_i=\frac{\partial L}{\partial \dot{q_i}})
\end{equation}
This can imply the conservation laws of momentum and angular momentum.
\paragraph{Homogeneity of time}
If $\frac{\partial L}{\partial t}=0$,then we get
\begin{equation}
\frac{dE}{dt}=0 \ \ \ (E=\sum_i \dot{q_i}p_i-L)
\end{equation}

\section{Hamilton Mechanics}
\subsection{Hamilton equation}
\begin{equation}
H(q,p,t)=\sum_i p_i \dot{q_i}-L
\end{equation}
\begin{equation}
\dot{p_i}=-\frac{\partial H}{\partial q_i} \ \ \ \ \ \ \dot{q_i}=\frac{\partial H}{\partial p_i}
\end{equation}
 
\subsection{Poisson Brackets}
Operation properties:
\[ \left\{f,g\right\}=-\left\{g,f\right\} \]
\[\left\{\alpha_1 f_1+\alpha_2 f_2,\beta_1 g_1+\beta_2 g_2\right\}=\alpha_1 \beta_1\left\{f_1,g_1\right\}
+\alpha_1 \beta_2\left\{f_1,g_2\right\}+\alpha_2 \beta_1\left\{f_2,g_1\right\}+\alpha_2 \beta_2\left\{f_2,g_2\right\}\]
\[\left\{f_1 f_2,g_1 g_2\right\}=f_1\left\{f_2,g_1\right\}g_2+f_1 g_1\left\{f_2,g_2\right\}+g_1\left\{f_1,g_2\right\}f_2 +\left\{f_1,f_2\right\}g_2 f_2 \]
\[\left\{f,\left\{g,h\right\}\right\}+\left\{g,\left\{h,f\right\}\right\}+\left\{h,\left\{f,g\right\}\right\}=0\]
Here, $f,g,h$ are functions of $p_i,q_i,t$.
If we assume that
\[ \left \{q_i,q_k\right \}=0,\left \{p_i,p_k\right \}=0,\left \{q_i,p_k\right \}=\delta_{ik}\]
we can deduce that 
\[ \left\{f,g\right\}=\sum_k(\frac{\partial f}{\partial q_k} \frac{\partial g}{\partial p_k} - \frac{\partial f}{\partial p_k} \frac{\partial g}{\partial q_k}  )\]
The Hamilton equation can be written as
\begin{equation}
\dot{p_i}=\left\{ p_i,H \right\} \ \ \ \ \dot{q_i}=\left\{ q_i,H \right\}
\end{equation}
\end{document}
\part{Classical Field Theory}
\chapter{Mechanics within special relativity}
\section{Basic Assumption}
First, we assume there is an upper limit of velocity of propagation of interaction $c$. Second, we assume that inertial reference frame are all the same in describing the law of physics. Then, we can find the invariant intervals when transforming from one inertial reference frame to another, $ds^2 = -c^2 dt^2 + dx^2 + dy^2 + dz^2$. 
(In the following, we assume that $c=1$.)This transformation is called Lorentz transformation, which can be written as
\[\bar{x}^{\mu} = \Lambda^{\mu}_{\phantom{\mu}\nu} x^{\nu}\]
and it is easy to verify that
\[\eta_{\mu \nu} \Lambda^{\mu}_{\phantom{\mu}\rho} \Lambda^{\nu}_{\phantom{\nu}\sigma} = \eta_{\rho \sigma},\]
where,
\[\eta_{\mu \nu} = \left[ 
\begin{matrix} 
-1& & & \\ 
& +1 & & \\
& & +1 & \\
& & & +1
\end{matrix} 
\right]\]
and
\[(\Lambda^{-1})^{\rho}_{\phantom{\rho}\nu} = \Lambda_{\nu}^{\phantom{\mu}\rho}\]

In a special case when the new reference frame move along $\hat{1}$ direction with velocity $\beta$, we have
\[\bar{x}^{0} = \gamma x^0 - \gamma \beta x^1\]
\[\bar{x}^{1} = -\gamma \beta x^0 + \gamma x^1\]

Some physical quantity will behave like a tensor (vector,scalar) when transforming form one inertial frame to another. For example,
\paragraph{scalar} proper time: $d \tau$, mass: $m$, electrical charge $e$
\paragraph{vector} four velocity: $v^{\mu} = \frac{dx^{\mu}}{d \tau}$, four momentum: $p^{\mu} = m v^{\mu}$, four acceleration: $a^{\mu} = \frac{du^{\mu}}{d \tau}$, four force: $f^{\mu} = m a^{\mu}$

\section{"Three vector"}
\noindent
three velocity: $\hat{u}^{i} = \frac{dx^i}{dt}$
\[u^0 = \gamma_v, u^i = \gamma \hat{u}^i\]
transformation of three velocity when we boost along $\hat{1}$ direction:
\[\bar{\hat{v}}^1 = \frac{\hat{v}^1 - \beta}{1-\hat{v}^1 \beta}\]
\[\bar{\hat{v}}^2 = \frac{\hat{v}^2}{\gamma(1-\hat{v}^2 \beta})\]
\[\bar{\hat{v}}^3 = \frac{\hat{v}^3}{\gamma(1-\hat{v}^3 \beta})\]
three momentum: $\hat{p}^{i} = p^i$
\[\hat{p}^{i} \gamma_v \hat{v}^i\]
three acceleration: $\hat{a}^{i} = \frac{dv^i}{dt}$\\
three force: $\hat{f}^i = \frac{d\hat{p}^i}{dt}$
\[f^i = \gamma_v \hat{f}^i\]
Energy: $E = p^0 = m u^0 = \gamma_v m$

\section{Mechanics}
\noindent
Revised newton's second law:
\[f^{\mu} = \frac{dp^{\mu}}{d\tau}\]
It can be written in three vector language as
\[\hat{f}^i = \gamma_v m \hat{a}^i + \gamma_v^3 (\hat{a}^j \hat{v}_j) m \hat{v}^i\]

\section{Lagrangian formulation}
\[S=-m\int_{a}^{b} d\tau, \ \ \ \ \delta x^{\mu}(a) = \delta x^{\mu}(b) = 0\]
\[\delta S = 0 \Rightarrow m\frac{du^{\mu}}{d\tau} = 0\]

\section{Hamiltonian formulation}
\[S = -m \int_{t_1}^{t_2} \sqrt{1-\dot{x}_i\dot{x}^i} dt\]
\[L = - m \sqrt{1-\dot{x}_i\dot{x}^i}\]
\[\pi^i = \frac{\partial L}{\partial \dot{x}_i} = \gamma m \eta^{ij}\dot{x}_j\]
\[H = \pi^i \dot{x}_i - L = \gamma m = \sqrt{m^2 + \pi^i \pi_i}\]

\subsubsection{Hamilton equation}
\[\dot{\pi}^i = 0, \ \ \ \ \dot{x_i} = \eta_{ij}\frac{\pi^j}{\sqrt{m^2 + \pi^k \pi_k}}\]

\subsubsection{Hamiltonian-Jacobi equation}
\[H = -\frac{\partial S}{\partial t}, \ \ \ \ \pi^i = \frac{\partial S}{\partial x_i}\]
If we define $p^0 = H$, $p^i = \pi^i$, then we can verify that $p^{\mu} = \frac{\partial S}{\partial x_{\mu}}$. So, $p^{\mu}$ is a vector under Lorentz transformation. The Hamiltonian-Jacobi equation can be written as
\[(\frac{\partial S}{\partial t})^2 = m^2 + (\frac{\partial S}{\partial x})^2 + (\frac{\partial S}{\partial y})^2 + (\frac{\partial S}{\partial z})^2\]

\section{Symmetry and conservation law}
\subsubsection{Translational symmetry and conservation of momentum}
\[\bar{x}^{\mu} = x^{\mu} + \delta x^{\mu}\]
\[\delta S = \sum mu_{\mu} \delta x^{\mu}|_a^b = 0 \]
$\sum p^{\mu}$ is conserved.
\subsubsection{Rotational symmetry and conservation of angular momentum}
\[\bar{x}^{\mu} = x^{\mu} + x_{\nu}\delta \Omega^{\mu \nu}\]
\[\delta S = \sum mu^{\mu} x^{\nu} \delta \Omega_{\mu \nu}|_a^b = 0 \]
$\sum M^{\mu \nu} $ is conserved, where $M^{\mu \nu} = x^{\mu}p^{\nu} - x^{\nu}p^{\mu}$.

\chapter{Classical field theory}
\section{Lagrangian formulation}
\[S = \int \mathcal{L}(\phi_a,\dot{\phi}_a,\nabla \phi_a) d^4 x, \ \ \ \ \delta \phi_a |_{\Sigma} = 0\]
\[\delta S = 0 \Rightarrow \partial_{\mu} \left (\frac{\partial \mathcal{L}}{\partial (\partial_{\mu} \phi_a)} \right ) - \frac{\partial \mathcal{L}}{\partial \phi_a} = 0\]
\subsubsection{Locality of the theory}
There are no terms in the Lagrangian coupling $\phi(\vec{x},t)$ directly to  $\phi(\vec{y},t)$ with $\vec{x} \neq \vec{y}$. The closet we get for the $\vec{x}$ label is coupling between $\phi(\vec{x},t)$ and $\phi(\vec{x}+\delta\vec{x},t)$ through the gradient term $\nabla \phi$.
\subsubsection{Lorentz invariance}
\noindent
Scalar fields:
\[\bar{\phi}(x) = \phi(\Lambda^{-1} x)\]
Vector fields:
\[\bar{A}^{\mu}(x) = \Lambda^{\mu}_{\phantom{\mu}\nu} A^{\nu}(\Lambda^{-1}x)\]
\[\bar{A}_{\mu}(x) = (\Lambda^{-1})^{\nu}_{\phantom{\mu}\mu} A_{\nu}(\Lambda^{-1}x) = \Lambda_{\mu}^{\phantom{\mu}\nu}A_{\nu}(\Lambda^{-1}x)\]
\[\overline{\partial_{\mu}\phi}(x) = (\Lambda^{-1})^{\nu}_{\phantom{\mu}\mu} \partial_{\nu} \phi (\Lambda^{-1}x) = \Lambda_{\mu}^{\phantom{\mu}\nu} \partial_{\nu} \phi (\Lambda^{-1}x)\]
Lagrangian is a scalar, or more loosely, action is invariant under Lorentz transformation.

\section{Symmetry and conservation law}
\begin{newthem}[Noether's theorem]
Every continuous symmetry of the Lagrangian gives rise to a conserved current $j^{\mu}(x)$ such that the equation of motion imply $\partial_{\mu} j^{\mu} = 0$.
Suppose that the infinitesimal transformation is
\[\phi_a \rightarrow \phi_a + \delta \phi_a\]
\[\mathcal{L} \rightarrow + \mathcal{L} + \delta \mathcal{L} \]
and if $\delta \mathcal{L} = \partial_{\mu} K^{\mu} = 0$, we can get
\[j^{\mu} = \frac{\partial \mathcal{L}}{\partial (\partial_{\mu} \phi_a)} \delta \phi_a - K^{\mu}\]
\end{newthem}

\subsubsection{space-time translation}
\noindent
$\bar{x} = x - a$ 
\[j^{\mu} = -a_{\nu} T^{\mu \nu}\]
\[T^{\mu \nu} \equiv -\frac{\partial \mathcal{L}}{\partial(\partial_{\mu}\phi_a)} \partial^{\nu} \phi_a + \eta^{\mu \nu} \mathcal{L}\]
If we define $P^{\mu} = \int T^{0 \mu} d^3 x$,then we have
\[\frac{d P^{\mu}}{dt} = 0\]

\subsubsection{Lorentz Transformation} 
\noindent
$\bar{x}^{\mu} = x^{\mu} + \delta \omega^{\mu}_{\phantom{\mu}\nu} x^{\nu}$\\
The infinitesimal Lorentz transformation can be written as $I+\delta \omega^{\mu}_{\phantom{\mu}\nu}$
\[\delta \omega^{\mu}_{\phantom{\mu}\nu} = \left[ 
\begin{matrix} 
0       & \beta_1   & \beta_2   & \beta_3   \\ 
\beta_1 & 0         & -\theta_3 & \theta_2  \\
\beta_2 & \theta_3  & 0         & -\theta_1 \\
\beta_3 & -\theta_2 & \theta_1  & 0
\end{matrix} 
\right]\]
This time, we assume that
\[\bar{\phi}_a(x) = S_{a}^{\phantom{a}b}\phi_b(\Lambda^{-1}x)\]
In the limit of infinitesimal Lorentz transformation, we have
\[S_{a}^{\phantom{a}b} = \delta_{a}^{\phantom{a}b}+\frac{1}{2} \delta \omega_{\alpha \beta} (\Sigma^{\alpha \beta})_{a}^{\phantom{a}b} \]
\[j^{\mu} = -\frac{1}{2} M^{\mu \nu \rho}  \delta \omega_{\nu \rho}\]
\[M^{\mu \nu \rho} \equiv x^{\nu}T^{\mu \rho} - x^{\rho} T^{\mu \nu} - \frac{\partial \mathcal{L}}{\partial (\partial_{\mu}\phi_a)}(\Sigma^{\nu \rho})_{a}^{\phantom{a}b}\phi_b\]
If we define $M^{\nu \rho} = \int M^{0 \nu \rho} d^3 x$, then we have
\[\frac{dM^{\nu \rho}}{dt} = 0\]

\section{Functional derivatives}
\begin{newdef}[Functional derivatives]
Given a manifold $M$ representing (continuous/smooth) functions $\rho$ (with certain boundary conditions etc.), and a functional $F$ defined as
\[F\colon M\rightarrow \mathbb {R} \quad {\mbox{or}}\quad F\colon M\rightarrow \mathbb {C} \,,\]
the functional derivative of $F[\rho]$, denoted $\frac{\delta F}{\delta \rho}$,is defined by
\[{\begin{aligned}\int {\frac {\delta F}{\delta \rho }}(x)\phi (x)\;dx&=\lim _{\varepsilon \to 0}{\frac {F[\rho +\varepsilon \phi ]-F[\rho ]}{\varepsilon }}\\&=\left[{\frac {d}{d\epsilon }}F[\rho +\epsilon \phi ]\right]_{\epsilon =0},\end{aligned}}\]
where $\phi$ is an arbitrary function. The quantity $\epsilon \phi$ is called the variation of $\rho$. 
\end{newdef}

Like the derivative of a function, the functional derivative satisfies the following properties, where $F[\rho]$ and $G[\rho]$ are functionals:\\
Linearity:
\[{\frac {\delta (\lambda F+\mu G)[\rho ]}{\delta \rho (x)}}=\lambda {\frac {\delta F[\rho ]}{\delta \rho (x)}}+\mu {\frac {\delta G[\rho ]}{\delta \rho (x)}},\]
where $\lambda$, $\mu$ are constants.\\
Product rule:
\[{\frac {\delta (FG)[\rho ]}{\delta \rho (x)}}={\frac {\delta F[\rho ]}{\delta \rho (x)}}G[\rho ]+F[\rho ]{\frac {\delta G[\rho ]}{\delta \rho (x)}}\,,\]
Chain rules:\\
If $F$ is a functional and $G$ an operator, then
\[{\displaystyle \displaystyle {\frac {\delta F[G[\rho ]]}{\delta \rho (y)}}=\int dx{\frac {\delta F[G]}{\delta G(x)}}_{G=G[\rho ]}\cdot {\frac {\delta G[\rho ](x)}{\delta \rho (y)}}\ .}\]
If $G$ is an ordinary differentiable function $g$, then this reduces to
\[{\displaystyle \displaystyle {\frac {\delta F[g(\rho )]}{\delta \rho (y)}}={\frac {\delta F[g(\rho )]}{\delta g[\rho (y)]}}\ {\frac {dg(\rho )}{d\rho (y)}}\ .} \]
\begin{newprop}[Properties of functional derivatives]
\[\frac{\delta F}{\delta \rho} (y) = \lim_{\epsilon \to \infty} \frac{1}{\epsilon} \{ F[\rho(x) + \epsilon \delta(x-y)] - F[\rho(x)] \}\]
\[\frac{\delta f(x)}{\delta f(y)} = \delta(x-y)\]
\[\frac{\delta}{\delta f(y)} \int g\left( f(x)\right) dx =  g'(f(y))\]
\[\frac{\delta f'(x)}{\delta f(y)} = \frac{d}{dx}\delta(x-y)\]
\[\frac{\delta}{\delta f(y)} \int g\left( f'(x)\right) dx = -\frac{d}{dy} g'(f'(y))\]
\end{newprop}

\section{Hamiltonian formulation}
\[\pi^a(x) = \frac{\partial \mathcal{L}}{\partial \dot{\phi_a}}\]
\[\mathcal{H}(\phi_a,\nabla \phi_a,\pi^a) = \pi^a \dot{\phi_a} - \mathcal{L}\]
\[H = \int \mathcal{H} d^3 x\]
Now, we can get the Hamilton equation form $\delta S =0$,
\[\dot{\phi_a}(\vec{x},t) = \frac{\delta}{\delta \pi^a(\vec{x},t)} H = \frac{\partial \mathcal{H}}{\partial \pi^a}\]
\[\dot{\pi^a}(\vec{x},t) = -\frac{\delta}{\delta \phi_a(\vec{x},t)} H = - \frac{\partial \mathcal{H}}{\partial \phi_a} + \left(\frac{\partial \mathcal{H}}{\partial \phi_{a,b}}\right)_{,b}\]

\subsection{Poission bracket}
\noindent
First, we demand that
\[[\phi_a(\vec{x}),\phi_b(\vec{y})] = [\pi^a(\vec{x}),\phi_b(\vec{y})]= 0\]
\[[\phi_a(\vec{x}),\pi^b(\vec{y})] = \delta^{b}_{a} \delta(\vec{x}-\vec{y})\]
then, we assume the bracket operation has the same properties as the Poission bracket in classical mechanics. And we also assume that
\[[\partial_x A(\vec{x}),B(\vec{y})] = \partial_x [A(\vec{x}),B(\vec{y})]\]
and
\[\left[\int d^3 x A(\vec{x}),B(\vec{y})\right] = \int d^3 x [A(\vec{x}),B(\vec{y})]\]
We can verify that
\[[W[\phi(\vec{x}),\pi(\vec{x})],Z[\phi(\vec{x}),\pi(\vec{x})]] = \int d^3x \left\{ \frac{\delta W}{\delta \phi(\vec{x})} \frac{\delta Z}{\delta \pi(\vec{x})} - \frac{\delta W}{\delta \pi(\vec{x})} \frac{\delta Z}{\delta \phi(\vec{x})} \right\}\]
Specially,
\[[\phi_a(\vec{x}),H] = \frac{\delta }{\delta \pi^a(\vec{x})} H, \ \ \ \ [\pi^a(\vec{x}),H] = -\frac{\delta }{\delta \phi_a(\vec{x})} H\]
So, the Hamilton equation can be written as
\[\dot{\phi_a} = [\phi_a,H], \ \ \ \ \dot{\pi^a} = [\pi^a,H]\]
Further more, we can prove
\[\frac{dO(\phi,\pi,t)}{dt} = [O,H] + \frac{\partial O}{\partial t}\]
and
\[\frac{d[A,B]}{dt} = [A,\frac{dB}{dt}] + [\frac{dA}{dt},B]  \]

\subsection{Momentum}
\noindent
It is easy to verify that
\[P^{0} = H, \ \ \ \ P^{i} = \int -\pi^a \partial^i \phi_a d^3 x\]
And we can get the commutation relationship that
\begin{eqnarray}
\left[\phi_a,P^{\mu}\right] &=& -\partial^{\mu} \phi_a \nonumber \\
\left[\pi^a,P^{\mu}\right] &=& -\partial^{\mu} \pi^a \nonumber \\
\left[P^{\mu},P^{\nu}\right] &=& 0 \nonumber 
\end{eqnarray}


\subsection{Angular momentum}
\noindent
It is easy to verify that
\[M^{\mu \nu} = \int (x^{\mu}T^{0\nu}-x^{\nu}T^{0\mu}-\pi^a(\Sigma^{\mu \nu})_{a}^{\phantom{a}b}\phi_b) d^3 x\]
We denote that
\[M_{L}^{\mu \nu} = \int (x^{\mu}T^{0\nu}-x^{\nu}T^{0\mu}) d^3 x \quad M_S^{\mu \nu} = \int (-\pi^a(\Sigma^{\mu \nu})_{a}^{\phantom{a}b}\phi_b) d^3 x\]
\[(L^{\mu \nu})_a^{\phantom{a}b} = -(x^{\mu}\partial^{\nu}-x^{\nu}\partial^{\mu})\delta_a^{\phantom{a}b} \quad (S^{\mu \nu})_a^{\phantom{a}b} = -(\Sigma^{\mu \nu})_a^{\phantom{a}b}\]
So, we have the commutation relationship that
\[M^{\mu \nu} = M_L^{\mu \nu} + M_S^{\mu \nu}\]
\[[\phi_a,M_L^{\mu \nu}] = (L^{\mu \nu})_a^{\phantom{a}b} \phi_b \quad [\phi_a,M_S^{\mu \nu}] = (S^{\mu \nu})_a^{\phantom{a}b} \phi_b\]
\[[\pi^a,M_L^{\mu \nu}] = (L^{\mu \nu})_b^{\phantom{b}a}\pi^{b}  \quad [\pi^a,M_S^{\mu \nu}] = - (S^{\mu \nu})_b^{\phantom{b}a} \pi^b \]
Because $\frac{d M^{\mu \nu}}{dt} = 0$, we can prove that
\[[[\phi(x),M^{\mu \nu}],M^{\rho \sigma}] = (L^{\mu \nu}+S^{\mu \nu})(L^{\rho \sigma}+S^{\rho \sigma})\phi(x)\]
and then we can get the communication relationship from the Jacobi identity,
\[[\phi(x),[M^{\mu \nu},M^{\rho \sigma}]] = (L^{\mu \nu}L^{\rho \sigma}-L^{\rho \sigma}L^{\mu \nu} + S^{\mu \nu}S^{\rho \sigma}-S^{\rho \sigma}S^{\mu \nu})\phi(x)\]
We can prove that
\[L^{\mu \nu}L^{\rho \sigma}-L^{\rho \sigma}L^{\mu \nu} = -g^{\nu \rho}L^{\mu \sigma} + g^{\sigma \mu}L^{\rho \nu} + g^{\mu \rho}L^{\nu \sigma} - g^{\sigma \nu}L^{\rho \mu}\]
If we demand that
\[S^{\mu \nu}S^{\rho \sigma}-S^{\rho \sigma}S^{\mu \nu} = -g^{\nu \rho}S^{\mu \sigma} + g^{\sigma \mu}S^{\rho \nu} + g^{\mu \rho}S^{\nu \sigma} - g^{\sigma \nu}S^{\rho \mu}\]
We can get get the communication relationship of the $M^{\mu \nu}$,
\[[M^{\mu \nu},M^{\rho \sigma}] = -g^{\nu \rho}M^{\mu \sigma} + g^{\sigma \mu}M^{\rho \nu} + g^{\mu \rho}M^{\nu \sigma} - g^{\sigma \nu}M^{\rho \mu}\]
up to the possibility of a term on the right-hand side that commutes with $\phi(x)$ and its derivatives.\\
We now define $J_i \equiv \frac{1}{2} \epsilon_{ijk} M^{jk}$ and $K_i \equiv M^{i0}$,so
\[M^{\mu \nu} = \left[ 
\begin{matrix} 
0   & -K_1 & -K_2 & -K_3 \\ 
K_1 & 0    & J_3  & -J_2 \\
K_2 & -J_3 & 0    &  J_1 \\
K_3 & J_2  & -J_1 &  0
\end{matrix} 
\right]\] 
the communication relationship can be written as
\begin{eqnarray}
\left[J_i,J_j\right] &=& \epsilon_{ijk}J_k \nonumber \\
\left[J_i,K_j\right] &=& \epsilon_{ijk}K_k \nonumber \\
\left[K_i,K_j\right] &=& -\epsilon_{ijk}J_k \nonumber
\end{eqnarray}
We can use the similar method to derive that
\[[P^{\mu},M^{\rho \sigma}] = g^{\mu \sigma}P^{\mu} - g^{\mu \rho}P^{\sigma}\]
It can also be written as
\begin{eqnarray}
\left[J_i,H\right] &=& 0 \nonumber \\
\left[J_i,P_j\right] &=& \epsilon_{ijk}P_k \nonumber \\
\left[K_i,H\right] &=& P_i \nonumber \\
\left[K_i,P_j\right] &=& \delta_{ij}H \nonumber
\end{eqnarray}
At last, we define $L_i \equiv \frac{1}{2} \epsilon_{ijk} M_L^{jk}$ and $S_i \equiv \frac{1}{2} \epsilon_{ijk} M_S^{jk}$
we can demonstrate that
\begin{eqnarray}
\left[L_i,S_j\right] &=& 0 \nonumber \\
\left[S_i,P_j\right] &=& 0 \nonumber \\
\left[L_i,P_j\right] &=& \epsilon_{ijk}P_k \nonumber
\end{eqnarray}

\part{General relativity}
\documentclass{article}
\usepackage[left=1.5cm, right=1.5cm, top=3cm, bottom = 3cm]{geometry}
\usepackage{amsmath}
\usepackage{mathrsfs}
\usepackage{amsfonts}
\usepackage{amssymb}
\usepackage{graphicx}
\usepackage{float}
\usepackage{wrapfig}
\usepackage{latexsym}
\usepackage{hyperref}
\usepackage{feynmf}
\usepackage{exscale}
\usepackage{relsize}
\usepackage{bm}%bold math, for vector
\linespread{1.1}

\usepackage{hyperref}
\hypersetup{
  pdfauthor={Yuyang Songsheng},
  pdftitle={Summary on General Relativity},
  pdfsubject={Summary on General Relativity},
  urlcolor=blue,
}




\author{Yuyang Songsheng}
\title{Summary on General Relativity}

\begin{document}
\maketitle
\section{Differential Geometry}
\subsection{Fundamental conception on differential manifolds}
\href{https://en.wikipedia.org/wiki/Manifold}{\textbf{Manifold}} 
Formally, a topological manifold is a second countable Hausdorff space that is locally homeomorphic to Euclidean space.\\
\href{https://en.wikipedia.org/wiki/Differentiable_manifold}{\textbf{Differentiable manifold}} 
In formal terms, a differentiable manifold is a topological manifold with a globally defined differential structure. \\
\href{https://en.wikipedia.org/wiki/Tangent_space}{\textbf{Tangent space}} 
In mathematics, the tangent space of a manifold facilitates the generalization of vectors from affine spaces to general manifolds, since in the latter case one cannot simply subtract two points to obtain a vector pointing from one to the other.\\
\href{https://en.wikipedia.org/wiki/Cotangent_space}{\textbf{Cotangent space}} 
Typically, the cotangent space is defined as the dual space of the tangent space at $x$.\\
\href{https://en.wikipedia.org/wiki/Submanifold}{\textbf{Submanifold}}\\
\textbf{Immersed submanifolds} An immersed submanifold of a manifold $M$ is the image $S$ of an immersion map $f:N \to M$; in general this image will not be a submanifold as a subset, and an immersion map need not even be injective (one-to-one) – it can have self-intersections.\\
\textbf{Injective immersion submanifolds} More narrowly, one can require that the map $f:N \to M$ be an inclusion (one-to-one), in which we call it an injective immersion, and define an immersed submanifold to be the image subset S together with a topology and differential structure such that S is a manifold and the inclusion f is a diffeomorphism: this is just the topology on N, which in general will not agree with the subset topology: in general the subset S is not a submanifold of M, in the subset topology.\\
\textbf{Open submanifolds}\\
\textbf{Closed submanifolds}\\
\textbf{Closed submanifolds of an open submanifold}\\
If an n dimensional injective immersed submanifold $N$ of a $m$ dimensional manifold $M$ is a closed submanifold of an open submanifold of $M$, then for every point $p \in f(N)$ there exists a chart ($U \subset M$,$\phi:U \to R_n $) containing $p$ such that $\phi(f(N) \cap U)$ is the intersection of a $n$-dimensional plane with $\phi(U)$.\\
\textbf{Embedded submanifolds} An embedded submanifold (also called a regular submanifold), is an immersed submanifold for which the inclusion map is a topological embedding. That is, the submanifold topology on $S$ is the same as the subspace topology. Given any embedding $f:N \to M$ of a manifold $N$ in $M$ the image $f(N)$ naturally has the structure of an embedded submanifold. That is, embedded submanifolds are precisely the images of embeddings. Closed submanifolds of an open submanifold are equal to embedded submanifolds.\\

\subsection{Multi linear algebra}
\textbf{Vector space}\ \ \ \ \textbf{Dual space}\\
\textbf{Tensor product}
\[V \otimes W = \mbox{ Span}\{ v \otimes w \} = \mathcal{L}(V^*,W^*;F)\]
\[V^* \otimes W^* = \mbox{ Span}\{ v^* \otimes w^* \} = \mathcal{L}(V,W;F)\]
\[\mathcal{L}(V,W;Z)=\mathcal{L}(V \otimes W;Z)\]
\[(\phi \otimes \psi)\otimes \xi = \phi \otimes (\psi \otimes \xi)\]
\textbf{Tensor}
\[V_s^r = V \otimes \cdots \otimes V \otimes V^* \otimes \cdots \otimes V^*\]
\[x=x^{i_1 \cdots i_r}_{\phantom{i_1 \cdots i_r} k_1 \cdots k_s} e_{i_1} \otimes \cdots \otimes e_{i_r} \otimes e^{*k_1} \otimes \cdots \otimes e^{*k_s}\]
\[(x \otimes y)^{i_1 \cdots i_{r_1+r_2}}_{\phantom{i_1 \cdots i_{r_1+r_2}} k_1 \cdots k_{s_1+s_2}} = 
x^{i_1 \cdots i_{r_1}}_{\phantom{i_1 \cdots i_{r_1}}k_1 \cdots k_{s_1}} \cdot y^{i_{r_1+1} \cdots i_{r_1+r_2}}_{\phantom{i_{r_1+1} \cdots i_{r_1+r_2}}k_{s_1+1} \cdots k_{s_1+s_2}}\]
\textbf{Permutation}($\sigma \in \mathcal{P}(r)$)
\[\sigma x(v^{*1},\cdots,v^{*r})=x(v^{*\sigma(1)},\cdots,v^{*\sigma(r)})\]
\textbf{Symmetric contra-variant tensor}
\[\sigma x =x\]
\textbf{Antisymmetric contra-variant tensor}
\[\sigma x =\mbox{ sgn }\sigma \cdot x\]
\textbf{Symmetrization operator}
\[S_r(x) = \frac{1}{r!} \sum_{\sigma \in \mathcal{P}(x)} \sigma x\]
\textbf{Antisymmetrization operator}
\[A_r(x) = \frac{1}{r!} \sum_{\sigma \in \mathcal{P}(x)} \mbox{ sgn }\cdot \sigma x\]
\textbf{Exterior vector space}
\[\Lambda^r(V) = A_r(T^r(V))\]
\[\Lambda^0(V)=F \ \ \ \Lambda^1(V)=V\]
\textbf{Wedge product}
\[\xi \wedge \eta \equiv \frac{(k+l)!}{k!l!}A_{k+l}(\xi \otimes \eta)\]
\[(\xi_1+\xi_2) \wedge \eta = \xi_1 \wedge \eta + \xi_2 \wedge \eta\]
\[\xi \wedge (\eta_1+\eta_2) = \xi \wedge \eta_1 + \xi \wedge \eta_2\]
\[\xi \wedge \eta = (-1)^{kl} \eta \wedge \xi\]
\[(\xi \wedge \eta) \wedge \zeta = \xi \wedge (\eta \wedge \zeta) = \frac{(k+l+h)!}{k!l!h!}A_{k+l+h}(\xi \otimes \eta \otimes \zeta)\]
\[e_{i_1} \wedge \cdots \wedge e_{i_r}(v^{*1},\cdots,v^{*r}) = \mbox{ det } \langle e_{i_{\alpha}},v^{*\beta} \rangle\]
\[e_{i_1} \wedge \cdots \wedge e_{i_r}(e^{*j_1},\cdots,e^{*j_r}) = \mbox{ det } \langle e_{i_{\alpha}},e^{*j_{\beta}} \rangle = \delta^{j_1 \cdots j_r}_{i_1 \cdots i_r}\]
\[\Lambda^r(V) = \mbox{ Span }\{e_{i_1} \wedge \cdots \wedge e_{i_r},1\leq i_1 < \cdots < i_r \leq n  \}\]
\[(\Lambda^r(V))^* = \Lambda^r(V^*)\]
\textbf{Pull-back mapping}
$f:V \to W$ is a linear mapping, we define $f^*:\Lambda^r(W^*) \to \Lambda^r(V^*)$ as
\[f^* \phi(v_1,\cdots,v_r) = \phi(f(v_1),\cdots,f(v_r)).\]
Theorem:
\[f^*(\phi \wedge \psi) = f^*\phi \wedge f^*\psi\]

\subsection{Vector Bundle}
\href{https://en.wikipedia.org/wiki/Fiber_bundle}{\textbf{Fiber bundle}} 
In mathematics, and particularly topology, a fiber bundle is a space that is locally a product space, but globally may have a different topological structure. Specifically, the similarity between a space E and a product space B × F is defined using a continuous surjective map $\pi :E \to B$ that in small regions of $E$ behaves just like a projection from corresponding regions of $B \times F$ to $B$. The map $\pi$, called the projection or submersion of the bundle, is regarded as part of the structure of the bundle. The space $E$ is known as the total space of the fiber bundle, $B$ as the base space, and $F$ the fiber.\\ \\
\href{https://en.wikipedia.org/wiki/Vector_bundle}{\textbf{Vector Bundle}} 
In mathematics, a vector bundle is a topological construction that makes precise the idea of a family of vector spaces parameterized by another space $X$ (for example $X$ could be a topological space, a manifold, or an algebraic variety): to every point $x$ of the space $X$ we associate (or "attach") a vector space $V(x)$ in such a way that these vector spaces fit together to form another space of the same kind as $X$ (e.g. a topological space, manifold, or algebraic variety), which is then called a vector bundle over $X$.\\ \\
\href{https://en.wikipedia.org/wiki/Tangent_bundle}{\textbf{Tangent bundle}} 
In differential geometry, the tangent bundle of a differentiable manifold $M$ is a manifold $TM$, which assembles all the tangent vectors in $M$. As a set, it is given by the disjoint union of the tangent spaces of $M$. That is,
\[
TM = \bigsqcup_{x \in M}T_xM = \bigcup_{x \in M} \{x\} \times T_xM = \bigcup_{x \in M}\{(x,y)| y \in T_xM \}
\]
where $T_xM$ denotes the tangent space to $M$ at the point $x$. So, an element of $TM$ can be thought of as a pair $(x,v)$ , where $x$ is a point in $M$ and $v$ is a tangent vector to $M$ at $x$. There is a natural projection $\pi : TM \to M$ defined by $\pi(x,v) = x$. This projection maps each tangent space $T_xM$ to the single point $x$.
The tangent bundle comes equipped with a natural topology. With this topology, the tangent bundle to a manifold is the prototypical example of a vector bundle (a fiber bundle whose fibers are vector spaces). A section of $TM$ is a vector field on $M$, and the dual bundle to $TM$ is the cotangent bundle, which is the disjoint union of the cotangent spaces of $M$. By definition, a manifold $M$ is parallelizable if and only if the tangent bundle is trivial. By definition, a manifold $M$ is framed if and only if the tangent bundle $TM$ is stably trivial, meaning that for some trivial bundle $E$ the Whitney sum $TM \oplus E$ is trivial. For example, the $n$-dimensional sphere $S_n$ is framed for all $n$, but parallelizable only for $n=1,3,7$ (by results of Bott-Milnor and Kervaire).\\
A smooth assignment of a tangent vector to each point of a manifold is called a vector field. Specifically, a vector field on a manifold $M$ is a smooth map $V: M \to TM$ such that the image of $x$, denoted $Vx$, lies in $T_xM$, the tangent space at $x$. In the language of fiber bundles, such a map is called a section. A vector field on $M$ is therefore a section of the tangent bundle of $M$.\\
Generalization:\\
\textbf{Cotangent bundle} $T^*M = \bigcup_{x \in M} T^*_{\phantom{*}x}M$ \\
\textbf{Tensor bundle} $T^r_sM = \bigcup_{x \in M} T^{r\phantom{x}}_{sx}M$ \\
\textbf{Exterior vector bundle} $\Lambda^rM = \bigcup_{x \in M} \Lambda^r(T_x)$\\
\textbf{Exterior form bundle} $\Lambda^rM^* = \bigcup_{x \in M} \Lambda^r(T^{*}_{\phantom{*}x})$

\subsection{Tangent vector field}
\textbf{Theorem 1} Let $M$ be a smooth manifold, and let $Y:M \to TM$ be a vector field. If $(U, (X_i))$ is any smooth coordinate chart on $M$, then $Y$ is smooth on $U$ if and only if its component functions with respect to this chart are smooth.\\
\textbf{Theorem 2} Let $M$ be a $m$ dimensional smooth manifold and $v$ a smooth tangent vector field on $M$. $v:C^{\infty}(M) \to C^{\infty}$ satisfy that\\
(1) $\forall f,g \in C^{\infty}(M),v(f+g)=v(f)+v(g)$;\\
(2) $\forall f \in C^{\infty}(M),\alpha \in \mathbf{R},v(\alpha f)=\alpha \cdot v(f)$;\\
(3) $\forall f,g \in C^{\infty}(M),v(f,g) = f \cdot v(g) + g \cdot v(f)$.\\
If $\alpha:C^{\infty}(M) \to C^{\infty}(M)$ satisfy the three conditions above, there exists a unique smooth vector field $v$ on $M$ that $\forall f \in C^{\infty}(M),v(f)=\alpha(f)$.\\ \\
\textbf{Theorem 3} $\forall X,Y \in \mathcal{H}(M),[X,Y]=X \circ Y -Y \circ X \in \mathcal{H}(M)$.\\ \\
\textbf{Useful formulas 1}\\
(1) $[aX+bY,Z]=a[X,Z]+b[Y,Z]$;$[Z,aX+bY]=a[Z,X]+b[Z,Y]$;\\
(2) $[X,Y]=-[Y,X]$;\\
(3) $[X,[Y,Z]] + [Y,[Z,X]] +[Z,[X,Y]]=0$;\\
(4) $[X,Y]|_{U} = [X|_{U},Y|_{U}] = (X_i \frac{\partial Y^j}{\partial u^i} - Y^i \frac{\partial X^j}{\partial u^i}) \frac{\partial}{\partial u^j}$;\\
(5) $f_{*}[X,Y] = [f_{*}X,f_{*}Y]$;\\ \\
\textbf{definition 1} Let $M$ be a smooth manifold and $\phi:\mathbf{R} \times M \to M$ a smooth mapping, and $\forall (t,p) \in \mathbf{R} \times M$,denote $\phi_{t}(p) = \phi(t,p)$. If $\phi$ satisfy that\\
(1) $\phi_0 = \mathrm{id}:M \to M$;\\
(2) $\forall s,t \in \mathbf{R}, \phi_s \circ \phi_t = \phi_{s+t}$;\\
then $\phi$ is called a one parameter differentiable transformation group acting on $M$.\\
Trajectory of $\phi$ through $p$ on $M$:$\gamma_p(t) = \phi(t,p)$.\\
Vector field induced by $\phi$: $X_p(f) = \langle \gamma_p , f \rangle$.\\ \\
\textbf{Useful Formulas 2}\\
(1) $\gamma_q(t) = \phi(t,\phi(s,p)) = \phi(t+s,p) = \gamma_p(t+s)$;\\
(2) $(\phi_s)_{*}X_p = X_{\phi_s(p)}$;\\
(3) $\psi_{*} X_p = \tilde{X}_{\psi(p)}$ if $X$ is induced by $\phi$ and $\tilde{X}$ is induced by $\psi \circ \phi \circ \psi^{-1} $. $\psi$ is a smooth  homeomorphism.\\
(4) $[X,Y] = \lim_{t \to 0} \frac{Y_p-(\phi_t)_* Y_{\phi_{-t}(p)}}{t} = \lim_{t \to 0} \frac{(\phi_{-t})_*Y_{\phi_t(p)}- Y}{t}$ if $X$ is induced by $\phi$.\\ \\
\textbf{Lie derivative(1)}\\
\[\mathcal{L}_{X}Y= \lim_{t \to 0} \frac{(\phi_{-t})_*Y_{\phi_t(p)}- Y}{t} =[X,Y]\]
\[\mathcal{L}_{X}f = X(f)\]\\
\textbf{Useful Formulas 3}\\
\[\mathcal{L}_{X}(Y + \lambda Z) = \mathcal{L}_{X}Y + \lambda\mathcal{L}_{X}Z\]
\[\mathcal{L}_{X}(f \cdot Y) = \mathcal{L}_{X}(f) \cdot Y + f\mathcal{L}_{X}Y\]
\[\mathcal{L}_{X}([Y,Z]) = [\mathcal{L}_{X}Y,Z]+ [Y,\mathcal{L}_{X}Z]\]\\ \\
\textbf{Theorem 4} Let $M$ be a n-dimensional smooth manifold and $X \in \mathcal{H}(M)$. If $p \in M$ and $X_p \neq 0$, $\exists (V,x^i)$ and $p \in V$ that $X|_V = \frac{\partial}{\partial y^1 }$.\\ \\
\href{https://en.wikipedia.org/wiki/Distribution_(differential_geometry)}{\textbf{Distribution}} Let $M$ be a $C^{\infty }$  manifold of dimension $m$ , and let $n \leq m$ . Suppose that for each $x\in M$ , we assign an $n$-dimensional subspace $\Delta _{x}\subset T_{x}(M)$ of the tangent space in such a way that for a neighbourhood $N_{x}\subset M$ of $x$ there exist $n$ linearly independent smooth vector fields $X_{1},\ldots ,X_{n}$ such that for any point $y\in N_{x}$, $X_{1}(y),\ldots ,X_{n}(y)$ span $\Delta _{y}$. We let $\Delta$  refer to the collection of all the $\Delta_{x}$ for all $ x\in M$ and we then call $\Delta$ a distribution of dimension $n$ on $M$ , or sometimes a $C^{\infty }$ $n$-plane distribution on $M$. The set of smooth vector fields $\{X_{1},\ldots ,X_{n}\}$ is called a local basis of $\Delta$.\\ \\
\textbf{Involutive distributions} We say that a distribution $\Delta$ on $M$ is involutive if for every point $x \in M$ there exists a local basis $\{X_{1},\ldots ,X_{n}\}$ of the distribution in a neighbourhood of $x$ such that for all $1\leq i,j\leq n$ , $[X_{i},X_{j}]$  is in the span of $\{X_{1},\ldots ,X_{n}\}$.That is, if $[X_{i},X_{j}]$ is a linear combination of $\{X_{1},\ldots ,X_{n}\}$. Normally this is written as $ [\Delta ,\Delta ]\subset \Delta $.\\
Involutive distributions are the tangent spaces to foliations. Involutive distributions are important in that they satisfy the conditions of the Frobenius theorem, and thus lead to integrable systems.
A related idea occurs in Hamiltonian mechanics: two functions f and g on a symplectic manifold are said to be in mutual involution if their Poisson bracket vanishes.\\ \\
\textbf{Frobenius Theorem} If distribution $\Delta$ on $M$ is involutive, then $\forall p \in M$, $\exists (V,x^i)$ and $p \in V$ that $\Delta|_{V} = \mathrm{Span} \{ \frac{\partial}{\partial y^1} , \cdots, \frac{\partial}{\partial y^h}\}$.\\ \\
\textbf{Integrable manifold} Let $L^{h}$ be a smooth distribution on $M$. If $\phi:N \to M$ is an injective immersion manifold, and $\forall p \in N$, $\phi_{*}(T_pN) \subset L^h(\phi(p))$, then $(\phi,N)$ is called an integrable manifold of $L^h$.\\
If $\forall q \in M$, there is an integrable manifold of $L^h$ through it, we say that $L^h$ is completely integrable.\\ \\
\textbf{Theorem 5} Let
\[\tau: \underbrace{A^1(M) \times \cdots \times A^1(M)}_p \times \underbrace{\mathcal{H}(M) \times \cdots \times \mathcal{H}(M)}_q \to c^{\infty}(M)\] 
be a $p+q$ multi-linear mapping, if $\forall 1 \leq a \leq p,1 \leq b \leq q$ and $\mu \in C^{\infty}(M)$,
\begin{eqnarray}
&&\tau(\alpha^1,\cdots,\mu \alpha^{a},\cdots,\alpha^p,v_1,\cdots,v_q)\nonumber \\
&=&\tau(\alpha^1,\cdots,\alpha^p,v_1,\cdots,\mu v_{b},\cdots,v_q)\nonumber \\
&=&\mu \cdot \tau(\alpha^1,\cdots,\alpha^p,v_1,\cdots,v_q)\nonumber
\end{eqnarray}
then the mapping $\tau$ define a $(p,q)$ tensor for all $x \in M$ smoothly.\\ \\
\textbf{Lie derivatives(2)} Let $X$ be a smooth tangent vector field on $M$ and $\phi_t$ the one parameter differentiable transformation group inducing it. Denote the trajectory of $\phi_t$ through $x$ by $\gamma_x(t)$. So we have linear isomorphism
\[(\phi_t^{-1})_{*} = (\phi_{-t})_{*} : T_{\gamma_x(t)}M \to T_xM\]
\[(\phi_t)^* : T_{\gamma_x(t)}^* \to T_xM\]
So we can induce the linear isomorphism
\[\Phi_t: T^p_q(\gamma_x(t)) \to T^p_q(x)\]
If $S$ and $T$ are smooth tensor fields on $M$,\\
(1) for all $t$ which is small enough, $\Phi_tS$ is a smooth tensor field on $M$ which has the same type as $S$ ,and $\lim_{t \to 0} \Phi_t(S(\gamma_p(t))) = S(p),\forall p \in M$.\\
(2)$\Phi_t(S \otimes T) = \Phi_tS \otimes \Phi_tT$.\\
(3)$\Phi_t(C^a_b(S)) = C^a_b(\Phi_t(S))$, $C^a_b$ is a tag for contraction.\\
So,we can define the Lie derivative for smooth tensor field $\tau$ on $M$ as
\[\mathcal{L}_{X}(\tau) = \lim_{t \to 0}\frac{\Phi_t(\tau)-\tau}{t}\]\\
\textbf{Useful Formulas 4}
\[\mathcal{L}_X(\tau_1+\lambda \tau_2) = \mathcal{L}_X \tau_1 + \lambda \mathcal{L}_X \tau_2\]
\[\mathcal{L}_X(\tau_1 \otimes \tau_2) = \mathcal{L}_X\tau_1 \otimes \tau_2 + \tau_1 \otimes \mathcal{L}_X \tau_2\]
\[C^r_s(\mathcal{L}_X \tau) = \mathcal{L}_X(C^r_s(\tau))\]
\[(\mathcal{L}_X \omega)(Y) = X(\omega(Y)) - \omega([X,Y])\]
\[((\mathcal{L}_X \tau)|_U)^{\mu_1,\cdots,\mu_p}_{v_1,\cdots,v_q} = X^{\alpha} \partial_{\alpha} \tau^{\mu_1,\cdots,\mu_p}_{v_1,\cdots,v_q} - \sum_{i=1}^{p} \tau^{\mu_1,\cdots,\alpha,\cdots,\mu_p}_{v_1,\cdots,v_q} \partial_{\alpha} X^{\mu_i} + \sum_{j=1}^{q}\tau^{\mu_1,\cdots,\mu_p}_{v_1,\cdots,\alpha,\cdots,v_q} \partial_{v_j}X^{\alpha}\]
\[\mathcal{L}_{[X,Y]} = \mathcal{L}_X \circ \mathcal{L}_Y - \mathcal{L}_Y \circ \mathcal{L}_X \]
\[\mathcal{L}_{X+Y} = \mathcal{L}_{X} + \mathcal{L}_{Y}\]

\subsection{Exterior differential}
\textbf{Exterior form space} $A(M) = \sum_{r=0}^{m} A^{r}(M)$\\
For $\tau \in A^r(M)$,
\[\tau|_{U} = \frac{1}{r!} \tau_{i_1\cdots i_r} dx^{i_1} \wedge \cdots \wedge dx^{i_r} = \tau_{|i_1\cdots i_r|} dx^{i_1} \wedge \cdots \wedge dx^{i_r}\] 
\[ \tau_{i_1\cdots i_r} = \tau(\frac{\partial}{\partial x^{i_1}},\cdots,\frac{\partial}{\partial x^{i_r}}) \]
\begin{eqnarray}
\tau(v_1,\cdots,v_r)|_{U} &=& \tau_{|i_1\cdots i_r|}dx^{i_1} \wedge \cdots \wedge dx^{i_r} (v_1,\cdots,v_r) \nonumber \\
&=& \tau_{|i_1\cdots i_r|} \left| \begin{matrix} v_1^{i_1}& \cdots & v_r^{i_1}\\ \vdots & & \vdots \\ v_1^{i_r} & \cdots & v_r^{i_r} \end{matrix} \right| \nonumber
\end{eqnarray}
It is a $r$ multi-linear mapping, and for every variable, it is $C^{\infty}(M)$ linear.\\ \\
\textbf{Pull back mapping}\\
$f:M \to N \Rightarrow f_{*}:T_{p}M \to T_{f(p)}N \Rightarrow f^{*}:\wedge^{r}(T_{f(p)}^{*}N) \to \wedge^{r}(T_p^*M) $\\
\[f^* \phi(v_1,\cdots,v_r) = \phi(f_*v_1,\cdots,f_*v_r)\]
\[f^*\phi|_U = \frac{1}{r!}(\phi_{\alpha_1\cdots\alpha_r} \circ f) \cdot \frac{\partial f^{\alpha_1}}{\partial x^{i_1}} \cdots \frac{\partial f^{\alpha_r}}{\partial x^{i_r}} dx^{i_1} \wedge \cdots \wedge dx^{i_r}\]
\[f^*(\phi \wedge \psi) = f^*\phi \wedge f^* \psi\]\\
\textbf{Exterior differential} Let $M$ be a m-dimensional smooth manifold. Then $\exists$ a unique mapping $d:A(M) \to A(M)$ satisfy that\\
(1) $d(A^r(M)) \subset A^{r+1}(M)$\\
(2) $\forall \omega_1,\omega_2 \in A(M),d(\omega_1+\omega_2) = d\omega_1 +d\omega_2$\\
(3) if $\omega_1 \in A^r(M)$,then $d(\omega_1 \wedge \omega_2)=d\omega_1 \wedge \omega_2 +(-1)^r \omega_1 \wedge d\omega_2$\\
(4) $f \in A^0(M)$,$df$ is just the differential of $f$\\
(5) $\forall f \in A^0(M)$,$d(df)=0$\\
$d$ is called exterior differential.\\ \\
\textbf{Theorem 1}\\
$\forall \omega \in A^1(M),X,Y \in \mathcal{H}(M)$,
\[d\omega(X,Y) = X \langle Y,\omega \rangle -Y \langle X,\omega \rangle -\langle [X,Y],\omega \rangle \] 
$\forall \omega \in A^r(M)$, $X_1,\cdots,X_{r+1} \in \mathcal{H}(M)$,
\begin{eqnarray}
d\omega(X_1,\cdots,X_{r+1}) &=& \sum_{i=1}^{r+1}(-1)^{i+1} X_{i}(\langle X_1 \wedge \cdots \wedge \hat{X_i} \wedge \cdots \wedge X_{r+1},\omega \rangle) \nonumber \\
&+& \sum_{1 \leq i < j \leq r+1}(-1)^{i+j} \langle [X_i,X_j] \wedge \cdots \wedge \hat{X_i} \wedge \cdots \wedge \hat{X_j} \wedge \cdots X_{r+1},\omega \rangle
\end{eqnarray}\\
\textbf{Theorem 2} $f^{*}(d\omega) = d(f^* \omega)$\\ \\
\textbf{Poincare Lemma(1)} $d^2=0$ \\
\textbf{Poincare Lemma(2)} Let $U=B_0(r)$ be a spherical neighbourhood with center origin $O$ and radius $r$ in $R^n$. $\forall \omega \in A^r(U)$ and $d\omega =0$, $\exists \tau \in A^{r-1}(U)$,satisfy that $\omega = d\tau$.\\ \\
\textbf{Pfaff equation set} Let $\omega^{\alpha}(1 \leq \alpha \leq r) \in A^1(U)$ and $U$ is an open set of $m$-dimensional smooth manifold $M$. Differential equation set $\omega^{\alpha} = 0$ is called Pfaff equation set.\\ \\
\textbf{Integral manifold of Pfaff equation set} If there is an injective immersion submanifold $\phi:N \to U$ satisfying that $\phi^{*} \omega^{\alpha} = 0$,$(\phi,N)$ is called an integral manifold of Pfaff eqation set.\\
There is a set of first order partial differential equations
\[\frac{\partial y^{\alpha}}{\partial x^i} = f^{\alpha}_{i}(x^1,\cdots,x^{m},y^1,\cdots,y^{n}) \ \ \ (1 \leq i \leq m,1 \leq \alpha \leq n)\]
$f^{\alpha}_{i}(x,y)$ is a smooth function on the open set $U \times V \subset R^m \times R^n$. The equations sets can be written as Pfaff equations on $U \times V$
\[\omega^{\alpha} \equiv dy^{\alpha} - f^{\alpha}_{i}(x,y)dx^i = 0\]
If the partial differential equations have solution
\[y^{\alpha} = g^{\alpha}(x^1,\ldots,x^m)\]
then the submanifold $\phi:U \to U \times V$,
\[\phi(x^1,\ldots,x^m) = (x^1,\ldots,x^m,g^1(x),\ldots,g^n(x))\]
is an integral manifold of the Pfaff equations , i.e. $\phi^* \omega^{\alpha} =0$\\ \\
\textbf{Pfaff equation sets and distribution} Pfaff equation set $\omega^{\alpha}$ on open set $V \in M$ with rank $r$ is equivalent to a $h=m-r$ dimensional smooth distribution locally.
\[\Delta^h(p) = \{v \in T_pM:\omega^{\alpha}(v)=0,1 \le \alpha \le r \}\]
If $\phi:N \to V$ is an integral manifold of $\omega^{\alpha}$,$\forall X \in T_pN$,$\omega^{\alpha}(\phi_{*}X) = \phi^* \omega_{\alpha}(X) =0$. So $\phi_* X \in \Delta^h(p)$, and so $\phi:N \to V$ is an integral manifold of $\Delta^h$.\\ \\
\textbf{Frobenius condition} Frobenius condition for Pfaff equations $\omega^{\alpha} =0(1 \le \alpha \le r)$ is that
\[d\omega^{\alpha} \equiv 0(\mathrm{mod}(\omega^1,\ldots,\omega^r))\]
Pfaff equations satisfying Frobenius condition is completely integrable.\\ \\
\textbf{Frobenius Theorem} If Pfaff equations $\omega^{\alpha} = 0$, $\forall p \in U$, there exists a local coordinate $(V,x^i)$ that make Pfaff equations equivalent to
\[dx^{\alpha}=0 (1 \le \alpha \le r)\]\\ \\
\textbf{Orientation of manifold} Let $\alpha:[0,1] \to M$ be a path on $M$. $\forall t \in [0,1]$, assign an orientation for $T_{\alpha(t)}M$, denoted by $\mu_t$. If for $t_0 \in [0,1]$, there is a local coordinate $(U;x_i)$ of $\alpha(t_0)$ and a neighbourhood $[t_0-\delta_1,t_0+\delta_2]$ of $t_0$ that
\[\alpha([t_0-\delta_1,t_0+\delta_2]) \subset U\] 
and
\[\left\{ \frac{\partial}{\partial x^1},\ldots,\frac{\partial}{\partial x^m}\right\}|_{\alpha(t)} \in \mu_t,\forall t \in [t_0-\delta_1,t_0+\delta_2],\] 
$\mu$ is called a continuous topological orientation of $\alpha$.\\ \\
\textbf{The propagation of orientation} Let $p,q \in M$ and $\alpha:[0,1] \to M$ a path connecting $p,q$. Assign an orientation $\lambda$ of $T_pM$. If there is a continuous topological orientation of $\alpha$ $\mu$ satisfying that $\mu_0 = \lambda$, then orientation $\mu_1$ of $T_qM$ is called the propagation of orientation $\lambda$ along $\alpha$. The orientation of $\mu_1$ is unique.\\ \\
\textbf{Orientable manifold} Let $M$ be a $m$ dimensional smooth manifold. If there is an atlases $(\mathcal{A_0} = \{(U_{\alpha},\phi_{\alpha})\})$, making that if $U_{\alpha} \cap U_{\beta} \neq \emptyset$,the Jacobian of
\[\phi_{\beta} \circ \phi_{\alpha}^{-1} : \phi_{\alpha}(U_{\alpha} \cap U_{\beta}) \to \phi_{\beta}(U_{\alpha} \cap U_{\beta})\]
is positive. Then $M$ is called orientable manifold.\\ \\
\textbf{Theorem 3} Let $M$ be a orientable connected manifold. $\forall p \in M$,assign an orientation $\lambda$ for $T_pM$, then for all point $q \in M$, the propagation of $\lambda$ along an arbitrary path define a unique orientation $\mu$ for $T_q M$.\\ \\
\textbf{Manifold with boundary} A topological manifold with boundary is a Hausdorff space in which every point has a neighbourhood homeomorphic to an open subset of Euclidean half-space (for a fixed $n$):
\[\mathbb {R} _{+}^{n}=\{(x_{1},\ldots ,x_{n})\in \mathbb {R} ^{n}:x_{n}\geq 0\}\]\\ \\
\textbf{Boundary and interior} Let M be a manifold with boundary. The interior of $M$, denoted $Int \ M$, is the set of points in $M$ which have neighbourhoods homeomorphic to an open subset of $\mathbb {R} ^{n}$. The boundary of $M$, denoted $\partial M$, is the complement of $Int \ M$ in $M$. The boundary points can be characterized as those points which land on the boundary hyperplane ($x^n=0$) of $ \mathbb {R} _{+}^{n}$ under some coordinate chart.\\
If $M$ is a manifold with boundary of dimension $n$, then $Int \ M$ is a manifold (without boundary) of dimension $n$ and $\partial M$ is a manifold (without boundary) of dimension $n-1$.\\ \\
\textbf{Theorem 4} Let $M$ be a smooth manifold with boundary and $\partial M \neq \emptyset$. The differential structure of $\partial M$ can be deduced from the $M$, making $\partial M$ a $m-1$ dimensional smooth manifold and the inclusion map $i:\partial M \to M$ is embedding map. If $M$ is orientable, then $\partial M$ is also orientable.\\ \\
\textbf{Induced orientation} Let $M$ be an orientable $m$ dimensional smooth manifold with boundary and $\partial M \neq \emptyset$. $\mathcal{A}$ is the orientation of $M$. For local coordinates $(U;x^i) \in \mathcal{A}$, when 
\[\tilde{U} = U \cap \partial M = \{(x^1,\ldots,x^m) \in U:x^m=0 \} \neq \emptyset\]
assign a local coordinate system $((-1)^m \cdot x^1,x^2,\ldots,x^{m-1})$ on $\tilde{U}$. The orientation defined by this local coordinate system is called induced orientation of $\partial M$.\\ \\
\textbf{Support set} Let $M$ be a $m$ dimensional orientable smooth manifold. $\omega \in A^r(M)$, the support set of $\omega$ can be defined as
\[Supp \ \omega = \overline {\{ p \in M :\omega(p) \neq 0\}}\]\\
All the $r$-form with compact support set is denoted as $A^r_0(M)$.\\ \\
\textbf{Partition of unity} Let $\Sigma$ be an open cover of $M$. Then there is a family of smooth function $g_{\alpha}$ on $M$ that\\
(1) $\forall \alpha$,$0 \leq g_{\alpha} \leq 1$,$\mathrm{supp} \ g_{\alpha}$ is compact and there is an open set $W_i \in  \Sigma$ that $\mathrm{supp} \  g_{\alpha} \subset W_i$.\\
(2) $\forall p \in M$, it has a neighbourhood $U$ which intersect finite $\mathrm{supp} \ g_{\alpha}$.\\
(3) $\sum_{\alpha} g_{\alpha} = 1$\\ \\
\textbf{Integral of differential form with compact support on orientable manifold}\\
\[\phi = (\sum_{\alpha} g_{\alpha})\cdot \phi = \sum_{\alpha} (g_{\alpha} \cdot \phi)\]
\[\int_{M} g_{\alpha} \cdot \phi = \int_{W_i} g_{\alpha} \cdot \phi = \int_{W_i} f(u^1,\cdots,u^m) du^1 \wedge \cdots \wedge du^m = \int_{W_i} f(u^1,\cdots,u^m) du^1 \cdots du^m\]
\[\int_{M} \phi = \sum_{\alpha} \int_{M} g_{\alpha} \cdot \phi\]
\textbf{Stokes Theorem} Let $M$ be an orientable $m$ dimensional smooth manifold with boundary and $\omega \in A_0^{m-1}(M)$,then
\[\int_{M} d\omega = \int_{\partial M} i^* \omega\]
Here, $\partial M$ has an orientation induced by $M$ and $i$ is  embedding mapping.

\subsection{Connection}
\textbf{Connection} Let $M$ be a smooth manifold and $E$ a $q$ dimensional real vector bundle on $M$. $\Gamma(E)$ is the set of all smooth sections of $E$ on $M$. The connection on $E$ is a mapping:
\[D:\Gamma(E) \to \Gamma(T^*(M) \otimes E)\]
it satisfies that\\
(1) $\forall s_1 ,s_2 \in \Gamma(E)$,$D(s_1+s_2) = Ds_1+Ds_2$\\
(2) $\forall s \in \Gamma(E)$ and $\alpha \in C^{\infty}(M)$, $D(\alpha s) = d\alpha \otimes s + \alpha Ds$\\
If $X$ is a smooth tangent vector field on $M$, $s \in \Gamma(E)$,then
$D_{X}s = \langle X, Ds \rangle$, called absolute derivative of $s$ along $X$.\\ \\
\textbf{Local representation}\\
\[Ds_{\alpha} = \sum_{1 \leq i \leq m,1 \leq \beta \leq q} \Gamma^{\beta}_{\alpha i} du^i \otimes s_{\beta}\]
\[\omega^{\beta}_{\alpha} = \sum_{1 \leq i \leq m} \Gamma^{\beta}_{\alpha i} du^i\]
\[Ds_{\alpha} = \sum_{\beta=1}^{q} \omega^{\beta}_{\alpha} \otimes s_{\beta}\]
\[DS = \omega \otimes S\]\\
\textbf{Transformation of frame $s_{\alpha}$}\\
\[S' = A \cdot S\] 
\begin{eqnarray}
DS' &=& dA \otimes S + A \cdot DS \nonumber \\
&=& (dA + A \cdot \omega) \otimes S \nonumber \\
&=& (dA \cdot A^{-1} + A \cdot \omega \cdot A^{-1}) \otimes S' \nonumber
\end{eqnarray}
\[\omega' = dA \cdot A^{-1} + A \cdot \omega \cdot A^{-1}\]\\
\textbf{Theorem 1} For an arbitrary vector bundle, connection always exists.\\ \\
\textbf{Theorem 2} Let $D$ be a connection of vector bundle $E$. $\forall p \in M$, there exists a local frame field $S$ on the neighbourhood of $p$ that $\omega(p) =0$.\\ \\
\textbf{Curvature matrix}\\
\[\Omega = d\omega - \omega \wedge \omega\]
\[\Omega' = A \cdot \Omega \cdot A^{-1}\]\\
\textbf{Curvature operator}\\
\[s = \sum_{\alpha=1}^{q} \lambda^{\alpha} s_{\alpha|p}\]
\[R(X,Y)s = \sum_{\alpha,\beta=1}^{q} \lambda^{\alpha} \Omega^{\beta}_{\alpha}(X,Y)s_{\beta|p}\]
\[(R(X,Y)s)(p) = R(X_p,Y_p)s_p\]
\textbf{Theorem 3}\\
\[R(X,Y) = D_X D_Y - D_Y D_X -D_{[X,Y]}\]\\
\textbf{Bianchi equation}\\
\[d\Omega = \omega \wedge \Omega - \Omega \wedge \omega\]\\
\textbf{Induced connection}\\
\[d\langle s,s^* \rangle = \langle Ds,s^* \rangle + \langle s,Ds^* \rangle\]
\[\langle s_{\alpha},s^{*\beta} \rangle  = \delta^{\beta}_{\alpha}  \Rightarrow Ds^{*\beta} = -\sum_{\alpha=1}^{q} \omega^{\beta}_{\alpha} \otimes s^{*\alpha}\]
\[D(s_1 \oplus s_2) = Ds_1 \oplus Ds_2\]
\[D(s_1 \otimes s_2) = Ds_1 \otimes s_2 + s_1 \otimes Ds_2\]\\
\textbf{Affine connection} Connection on $T(M)$.\\
\[D\frac{\partial}{\partial u^i} = \omega^j_i \otimes \frac{\partial}{\partial u^j} = \Gamma^{j}_{ik} du^k \otimes \frac{\partial}{\partial u^j}\]
\[\Gamma'^{j}_{ik} = \Gamma^{q}_{pr} \frac{\partial w^j}{\partial u^q} \frac{\partial u^p}{\partial w^i} \frac{\partial u^r}{\partial w^k} + \frac{\partial^2 u^p}{\partial w^i \partial w^k} \frac{\partial w^j}{\partial u^p}\]
\[DX = (dx^i + x^j \omega^i_j)\otimes \frac{\partial}{\partial u^i} = (x^i_{,j} + x^k \Gamma^{i}_{kj}) du^j \otimes \frac{\partial}{\partial u^i} = x^i_{;j}du^j \otimes \frac{\partial}{\partial u^i} \]
\[D \alpha = (d\alpha_i - \alpha_j \omega^j_i)\otimes du^i = (\alpha_{i,j} -\alpha_k \Gamma^k_{ij})du^j \otimes du^i = \alpha_{i;j} du^j \otimes du^i \]\\
\textbf{Geodesic}
\[\frac{D(\frac{du^i(t)}{dt} \frac{\partial}{\partial u^i})}{dt} = 0 \]
\[\frac{d^2 u^i}{dt^2} + \Gamma^i_{jk} \frac{du^j}{dt} \frac{du^k}{dt} = 0\]\\
\textbf{Curvature Tensor}\\
\[\Omega^j_i = \frac{1}{2}R^j_{ikl} du^k \wedge du^l \]
\[R^j_{ikl} = \frac{\partial \Gamma^j_{il}}{\partial u^k} - \frac{\partial \Gamma^j_{ik}}{\partial u^l} + \Gamma^h_{il} \Gamma^j_{hk} - \Gamma^h_{ik} \Gamma^j_{hl}\] 
\[R'^j_{ikl} = R^q_{prs} \frac{\partial w^j}{\partial u^q} \frac{\partial u^p}{\partial w^i} \frac{\partial u^r}{\partial w^k}\frac{\partial u^s}{\partial w^l}\]
\[R^{j}_{ikl} = \langle R(\frac{\partial}{\partial u^k} \frac{\partial}{\partial u^l})\frac{\partial}{\partial u^i},du^j \rangle\]\\
\textbf{Torsion tensor}
\[T^j_{ik} = \Gamma^j_{ki} - \Gamma^j_{ik}\]
\[T = T^j_{ik} \frac{\partial}{\partial u^j} \otimes du^i \otimes du^k\]
\[T(X,Y) = T^k_{ij}X^i Y^j \frac{\partial}{\partial u^k}\]
\[T(X,Y) = D_X Y - D_Y X - [X,Y]\]
\textbf{Theorem 4} Let $D$ be an affine connection without torsion on $M$. $\forall p \in M$, there is a local coordinate system that $\Gamma^j_{ik}(p)$ vanishes.\\ \\
\textbf{Theorem 5} Let $D$ be an affine connection without torsion on $M$. Then we have Bianchi equation
\[R^j_{ikl;h} + R^j_{ihk;l} + R^j_{ilh;k} = 0\]

\subsection{Riemann manifold}
\textbf{Riemann manifold} Let $M$ be a smooth manifold equipped with a smooth non-degenerate symmetric second order covariant tensor field $G$, then $M$ is called general Riemann manifold and $G$ is called the metric tensor of $M$.\\
If $G$ is positive definite, then $M$ is called Riemann manifold.\\ \\
\textbf{Theorem 1} There must be a Riemann metric on $m$ dimensional manifold $M$.\\ \\
\textbf{Index lifting}\\
\[f:T_p(M) \to T^*_p(M) \ \ \alpha_X(Y) = G(X,Y)\]\\
\textbf{Adapted connection} Let $(M,G)$ be a general Riemann manifold and $D$ a connection on $M$. If $DG=0$,then $D$ is called adapted connection on $M$. \\ \\
\textbf{Christoffel-Levi-Civita connection} Let $M$ be a general Riemann manifold, then there is a unique adapted connection without torsion on $M$, called Christoffel-Levi-Civita connection.
\[\Gamma^{k}_{ij} = \frac{1}{2} g^{kl}(\frac{\partial g_{il}}{\partial u^j} + \frac{\partial g_{jl}}{\partial u^i} - \frac{\partial g_{ij}}{\partial u^l})\]\\
\textbf{Curvature tensor}\\
\[R_{ijkl} = -R_{jikl} = -R_{ijlk}\]
\[R_{ijkl}+R_{iklj}+R_{iljk}=0\]
\[R_{ijkl} = R_{klij}\]\\
\href{https://en.wikipedia.org/wiki/Normal_coordinates}{\textbf{Normal coordinates}}  In differential geometry, normal coordinates at a point $p$ in a differentiable manifold equipped with a symmetric affine connection are a local coordinate system in a neighbourhood of $p$ obtained by applying the exponential map to the tangent space at $p$. In a normal coordinate system, the Christoffel symbols of the connection vanish at the point $p$, thus often simplifying local calculations. In normal coordinates associated to the Levi-Civita connection of a Riemann manifold, one can additionally arrange that the metric tensor is the Kronecker delta at the point $p$, and that the first partial derivatives of the metric at $p$ vanish.\\
The properties of normal coordinates often simplify computations. In the following, assume that $U$ is a normal neighbourhood centred at $p$ in $M$ and ($x_i$) are normal coordinates on $U$.\\
Let $V$ be some vector from $T_pM$ with components $V^i$ in local coordinates, and $\gamma_V$ be the geodesic with starting point $p$ and velocity vector $V$, then $\gamma_V$ is represented in normal coordinates by $\gamma _{V}(t)=(tV^{1},...,tV^{n})$ as long as it is in $U$.\\
    The coordinates of $p$ are $(0, \cdots, 0)$\\
    In Riemann normal coordinates at p the components of the Riemann metric $g$ simplify to $\delta _{ij}$.\\
    The Christoffel symbols vanish at $p$. In the Riemann case, so do the first partial derivatives of $g_{ij}$.\\ \\
\textbf{Theorem 2} Let $M$ be a differentiable manifold equipped with a symmetric affine connection. $\forall x_0 \in M$, there is a neighbourhood $W$ that for every point in $W$, there is a neighbourhood equipped with a normal coordinate system which contains $W$. \\ \\
\textbf{Theorem 3} Let $M$ be a Riemann manifold. $\forall O \in M$,there is a neighbourhood with normal coordinates $W$ that:\\
(1) For every point in $W$, there is a neighbourhood equipped with a normal coordinates which contains $W$.\\
(2) The geodesic connecting $O$ and $p \in W$ is the only shortest path connecting these two points in $W$.\\ \\
\textbf{Theorem 4} Let $U$ be the neighbourhood with normal coordinates of $O$. $\exists \epsilon >0, \forall \delta \in (0,\epsilon)$,the surface
\[\Sigma_{\delta} = {p \in U | \sum_{i=1}^{m}} (u^i(p))^2 = \delta^2\]
has following properties:\\
(1) $\forall p \in \Sigma_{\delta}$, there is a unique shortest geodesic connecting $p$ and $O$ in $U$.\\
(2) For all geodesics tangent to $\Sigma_{\delta}$, there is a neighbourhood of the cut point in which the geodesics lies outside of $\Sigma_{\delta}$ \\ \\
\textbf{Theorem 5} Let $M$ be a Riemann manifold and $\forall p \in M$, there is a $\eta$-spherical neighbourhood $W$ that for arbitrary two points in $W$, there is a unique geodesic connecting these two points.\\ \\
\textbf{Cross section curvature} \\
\[R(X,Y,Z,W) = R_{ijkl}X^iY^jZ^kW^l\]
\[R(X,Y,Z,W) = (R(Z,W)X) \cdot Y\]
\[G(X,Y,Z,W) = G(X,Z)G(Y,W) - G(X,W)G(Y,Z)\]
Let $E$ be a two dimensional subspace of $T_p(M)$ and $X,Y$ two linearly independent tangent vector of $E$,then
\[K(E) = -\frac{R(X,Y,X,Y)}{G(X,Y,X,Y)}\]
is a function of $E$, which is independent of the choice of $X,Y$,called cross section curvature.\\ \\
\textbf{Theorem 6} Let $M$ be a Riemann space. The curvature tensor of $p \in M$ is uniquely determined by the cross section curvature of all the two dimensional subspace of $T_p(M)$.\\ \\ 
\textbf{Constant curvature Riemann manifold} \\
Let $M$ be a Riemann manifold. If all of $K(E)$ on $p$ is constant, then $M$ is called isotropic on $p$. \\
If $M$ is isotropic every where and $K(p)$ is constant over $M$, then $M$ is called constant curvature Riemann manifold.\\ \\
\textbf{F.Schur Theorem} Let $M$ be a $m$-dimensional connected Riemann manifold that is isotropic every where. If $m \geq 3$, then $M$ is constant curvature Riemann manifold.

\section{A Geometrical Description of Newton Theory}
\subsection{Introduction}
We choose Euclidean coordinates for our absolute space and an absolute time $t$, than the equation of motion can be written as
\[\frac{d^2 t}{d\lambda^2} = 0\]
\[\frac{d^2 x^i}{d\lambda^2} + \frac{\partial \Phi}{\partial x^i} (\frac{d\lambda}{dt})^2=0\]
It is convenient to define that $\Gamma^i_{00} = \frac{\partial \Phi}{\partial x^i}$,and all other $\Gamma^{\alpha}_{\beta \gamma}$ vanish.Then we can write the equation of motion as
\[\frac{d^2 x^{\alpha}}{d\lambda^2} + \Gamma^{\alpha}_{\beta \gamma} \frac{dx^{\beta}}{d \lambda} \frac{dx^{\gamma}}{d \lambda}=0\]
Next, we can get the Riemann tensor given the connection above\[
R^i_{0j0} = -R^i_{00j} = \frac{\partial \Phi}{\partial x^i \partial x^j},
\]and all other terms vanish. It is straight forward to derive the expression of Ricci tensor,
\[R_{00} = \Phi_{ii} = \nabla^2\Phi,\]and all other terms vanish.
So, newton gravity law can be written as
\[R_{00} = 4\pi\rho\]

\subsection{Geometry structure of Newtonian Space-time}
\subsubsection*{Stratification of space-time}
Regard absolute time $t$ as a scalar field defined once and for all in Newtonian space-time $t=t(\mathcal{P})$.The layers of space-time are the slices of constant $t$-the "space slices"-each of which has an identical geometric structure: the old "absolute space."
\subsubsection*{Flat Euclidean space}
A given space slice is endowed with basis vectors $\mathbf{e}_i = \frac{\partial}{\partial x^i}$; and this basis has vanishing connection coefficients,$\Gamma^i_{jk} = 0$. Consequently,the geometry of each space slice is completely flat. Absolute space is Euclidean in its geometry.Each space slice is endowed with a three-dimensional metric, and its Galilean coordinate basis is orthonormal,$\mathbf{e}_i \cdot \mathbf{e}_j = \delta_{ij}$.
\subsubsection*{Curvature of space-time}
Parallel transport a vector around a closed curve lying entirely in a space slice; it will return to its starting point unchanged.But transport it forward in time by $\Delta t$, northerly in space by $\Delta x_k$, back in time by $-\Delta t$, and southerly by $-\Delta x_k$ to its starting point; it will return changed by
\[\delta \mathbf{A} = -\mathcal{R}(\Delta t \frac{\partial}{\partial t},\Delta x_k \frac{\partial}{\partial x_k}) \mathbf{A}\]
Geodesics of a space slice (Euclidean straight lines) that are initially parallel remain always parallel. But geodesics of space-time (trajectories of freely falling particles) initially parallel get pried apart or pushed together by space-time curvature,
\[\nabla_{\mathbf{u}} \nabla_{\mathbf{u}} \mathbf{n} + \mathcal{R}(\mathbf{n},\mathbf{u})\mathbf{u} = 0\]

\subsection{Geometry formulation of Newtonian gravity}
(1) There exists a function $t$ called "universal time", and a symmetric covariant derivative $\nabla$.\\
(2) The 1-form $\mathbf{d}t$ is covariant constant, i.e.,
\[\nabla_{\mathbf{u}} \mathbf{d}t = 0  \mbox{ for all } \mathbf{u}.
\]
[Consequence: if $\mathbf{w}$ is a spatial vector field, then $\nabla_{\mathbf{u}}\mathbf{w}$ is also spatial for every $\mathbf{u}$.]\\
(3) Spatial vectors are unchanged by parallel transport around infinitesimal closed curves; i.e.,
\[\mathcal{R}(\mathbf{n},\mathbf{u}) \mathbf{w}=0 \mbox{ if }\mathbf{w}\mbox{ is spatial, for every } \mathbf{u} \mbox{ and } \mathbf{n}.\]
(4) All vectors are unchanged by parallel transport around infinitesimal, spatial, closed curves; i.e.,
\[\mathcal{R}(\mathbf{v},\mathbf{w}) =0 \mbox{ for every spatial }\mathbf{v} \mbox{ and } \mathbf{w}.\]
(5) The Ricci curvature tensor has the form
\[\mathbf{Ricci} = 4 \pi \rho \mathbf{d}t \otimes \mathbf{d}t\]
where $\rho$ is the density of mass.\\
(6) There exists a metric $\mathbf{\cdot}$ defined on spatial vectors only, which is compatible with the covariant derivative in this sense: for any spatial $\mathbf{w}$ and $\mathbf{v}$, and for any $\mathbf{u}$ whatsoever,
\[\nabla_{\mathbf{u}}(\mathbf{w} \cdot \mathbf{v}) = (\nabla_{\mathbf{u}} \mathbf{w}) \cdot \mathbf{v} + \mathbf{w} \cdot (\nabla_{\mathbf{u}} \mathbf{v}).\]
[Note: axioms (1), (2), and (3) guarantee that such a spatial metric can exist.]\\
(7) The Jacobi curvature operator $\mathcal{J}(\mathbf{u},\mathbf{v})$, defined for any vectors $\mathbf{u},\mathbf{n},\mathbf{p}$ by
\[\mathcal{J}(\mathbf{u},\mathbf{n})\mathbf{p} = \frac{1}{2}[\mathcal{R}(\mathbf{p},\mathbf{n})\mathbf{u} + \mathcal{R}(\mathbf{p},\mathbf{u})\mathbf{n}]\]
is "self-ad-joint" when operating on spatial vectors,i.e.,
\[\mathbf{v} \cdot [\mathcal{R}(\mathbf{u},\mathbf{n})\mathbf{w}] = 
\mathbf{w} \cdot [\mathcal{R}(\mathbf{u},\mathbf{n})\mathbf{v}] \mbox{ for all spactial }\mathbf{v},\mathbf{w};\mbox{ and for any }\mathbf{u},\mathbf{n}. \]
(8) "Ideal rods" measure the lengths that are calculated with the spatial metric; "ideal clocks" measure universal time $t$ ( or some multiple thereof); and "freely falling particles" move along geodesics of $\nabla$.

\subsection{Standard formulation of Newtonian gravity}
(1) There exist a universal time $t$, a set of Cartesian space coordinates $x_i$ (called "Galilean coordinates"), and a Newtonian gravitational potential $\Phi$.\\
(2) The density of mass $\rho$ generates the Newtonian potential by Poisson's equation,
\[\nabla^2\Phi = \frac{\partial \Phi}{\partial x^i \partial x^i} = 4\pi\rho.\]
(3) The equation of motion for a freely falling particle is
\[\frac{d^2 x^i}{dt^2} + \frac{\partial \Phi}{\partial x^i} =0 .\]
(4) "Ideal rods" measure the Galilean coordinate lengths; "ideal clocks" measure universal time.

\subsection{Galilean coordinate system}
The features of Galilean coordinate systems are
\[x^0(\mathcal{P}) = t(\mathcal{P})\]
\[\frac{\partial}{\partial x^i} \cdot \frac{\partial}{\partial x^j} = \delta_{ij}\]
\[\Gamma^j_{00} = \Phi_{,j} \mbox{ for some scalar field }\Phi,\mbox{ and all other } \Gamma^{\alpha}_{\beta \gamma} \mbox{vanish.}\]
Consider following coordinate transformation:\\
(1)$x^{0'}=x^0=t$,both time coordinates must be universal time.\\
(2)at fixed $t$,both sets of space coordinates must be Euclidean, so they must be related by a rotation and a translation:
\[\bar{x}^{i'}(t) = A_{i'j}(t)x^j + \bar{a}^{i'}(t)\]
We can get
\[\bar{\Gamma}^{i'}_{0j'} = \bar{\Gamma}^{i'}_{j'0} = A_{i'l}\dot{A}_{j'l}\]
\[\bar{\Gamma}^{i'}_{00} = \Phi_{,i'} + A_{i'j}(\ddot{A}_{l'j}\bar{x}^{l'}-\ddot{a^{j}}), \mbox{   here, }a^{j} = \bar{a}^{l'}A_{l'j}\]
and all other terms vanish.So, new coordinates have the standard Galilean form  if and only if
\[\dot{A}_{i'j}=0,\ \ \Phi'=\Phi-\ddot{\bar{a}}^{i'}x^{i'}+C\]
Were all the matter in the universe concentrated in a finite region of space and surrounded by emptiness ("island universe"), then one could impose the global boundary condition $\Phi \to 0$ as $r \equiv (x^ix^i)^{\frac{1}{2}} \to \infty$. This would single out a subclass of Galilean coordinates ("absolute" Galilean coordinates), with a unique, common Newtonian potential. The transformation from one
absolute Galilean coordinate system to any other is called Galilean transformation.

\subsection{Coordinate transformation in space}
We now consider a coordinate transformation of Galilean coordinate system purely in space without any terms related with time. That means that $\bar{x}^{i'} = y^{i'}(x^{i})$ and $t'=t$. We can calculate the connection term in the new coordinate system.
\[\bar{\Gamma}^{i'}_{00} = \Gamma^{i}_{00} \frac{\partial y^{i'}}{\partial x^{i}}\]
\[\bar{\Gamma}^{i'}_{j'k'} = \frac{\partial^2 x^{m}}{\partial y^{i'}\partial y^{k'}} \frac{\partial y^{i'}}{\partial x^{m}}\]
The equation of motion of free fall body is that
\[\frac{d^2 t'}{d\lambda^2} = 0\]
\[\frac{d^2 \bar{x}^{i'}}{d\lambda^2} + \bar{\Gamma}^{i'}_{j' k'} \frac{d\bar{x}^{j'}}{d \lambda} \frac{d\bar{x}^{k'}}{d \lambda} + \bar{\Gamma}^{i'}_{0 0} \frac{dt'}{d \lambda} \frac{dt'}{d \lambda}=0\]
We can write it compactly as 
\[\frac{d^2 \bar{x}^{i'}}{dt^2} + \bar{\Gamma}^{i'}_{j' k'} \frac{d\bar{x}^{j'}}{dt} \frac{d\bar{x}^{k'}}{dt} + \bar{\Gamma}^{i'}_{0 0}=0\]
We can demonstrate that
\[\bar{\Gamma}^{i'}_{j'k'} = \frac{1}{2} \bar{g}^{i'p'}(\partial_{k'}\bar{g}_{j'p'}+\partial_{j'}\bar{g}_{k'p'}-\partial_{p'}\bar{g}_{j'k'})\]
and
\[\bar{\Gamma}^{i'}_{00} = \bar{g}^{i'j'}\partial_{j'}\Phi\]
Here,$\bar{g}$ is the metric of the space in new coordinate system.

\section{The Formulation of General Relativity}
\subsection{More on the manifold of space-time}
\textbf{Hodge dual}\\
The Hodge star operator on a vector space $V$ with a non-degenerate symmetric bilinear form (herein referred to as the inner product) is a linear operator on the exterior algebra of $V$, mapping $k$-vectors to ($n-k$)-vectors where $n = \text{dim} \ V$, for $ 0 \leq k \leq n$. It has the following property, which defines it completely: given two $k$-vectors $\alpha,\beta$,
\[\alpha \wedge (\star \beta )=\langle \alpha ,\beta \rangle \omega\]
where $\langle \cdot ,\cdot \rangle$ denotes the inner product on $k$-vectors and $\omega$ is the preferred unit n-vector.
The inner product $\langle \cdot ,\cdot \rangle$ on $k$-vectors is extended from that on V by requiring that
\[\langle \alpha ,\beta \rangle =\det \left[\left\langle \alpha _{i},\beta _{j}\right\rangle \right]\]
for any decomposable $k$-vectors $\alpha =\alpha _{1}\wedge \dots \wedge \alpha _{k}$ and $\beta =\beta _{1}\wedge \dots \wedge \beta _{k}$. The unit $n$-vector $\omega$ is unique up to a sign. The preferred choice of $\omega$ defines an orientation on $V$.\\ \\
Given an orthonormal basis $(e_{1},\cdots ,e_{n})$ ordered such that $\omega =e_{1}\wedge \cdots \wedge e_{n}$, we see that
\[\star (e_{i_{1}}\wedge e_{i_{2}}\wedge \cdots \wedge e_{i_{k}})=e_{i_{k+1}}\wedge e_{i_{k+2}}\wedge \cdots \wedge e_{i_{n}}\]
where $(i_{1},i_{2},\cdots ,i_{n})$ is an even permutation of $\{1,2,\cdots,n\}$.Of these $\frac{n!}{2}$, only $n \choose k$ are independent. The first one in the usual lexicographical order reads
\[\star (e_{1}\wedge e_{2}\wedge \cdots \wedge e_{k})=e_{k+1}\wedge e_{k+2}\wedge \cdots \wedge e_{n}\]\\ \\
\textbf{Levi-Civita tensor}\\
Let $\epsilon$ be a Levi-Civita symbol. We can define Levi-Civita tensor as 
\[\tilde{\epsilon}_{i_1,\cdots,i_n} = |g|^{\frac{1}{2}} \epsilon_{i_1,\cdots,i_n}\]
\[\tilde{\epsilon}^{i_1,\cdots,i_n} = g^{i_1j_1} \cdots g^{i_nj_n} \tilde{\epsilon}_{j_1,\cdots,j_n} = \frac{|g|^{\frac{1}{2}}}{g} \epsilon^{i_1,\cdots,i_n} = sgn(g) \frac{1}{|g|^{\frac{1}{2}}} \epsilon^{i_1,\cdots,i_n}\]\\
Using tensor index notation, the Hodge dual is obtained by contracting the indices of a k-form with the n-dimensional completely antisymmetric Levi-Civita tensor.Thus one writes
\[(\star \eta )_{i_{1},i_{2},\ldots ,i_{n-k}}={\frac {1}{(n-k)!}}\eta ^{j_{1},\ldots ,j_{k}} \tilde{\epsilon} _{j_{1},\ldots ,j_{k},i_{1},\ldots ,i_{n-k}}\]
where $\eta$ is an arbitrary antisymmetric tensor in $k$ indices. It is understood that indices are raised and lowered using the same inner product g as in the definition of the Levi-Civita tensor. Although one can apply this expression to any tensor, the result is antisymmetric, since the symmetric components of the tensor completely cancel out when contracted with the completely anti-symmetric Levi-Civita symbol.\\ \\
\textbf{Metric-induced properties of Riemann curvature tensor}\\
(1) In a $n$ dimensional manifold with torsion-free affine connection, the number of independent components of Riemann tensor is \[\frac{n^3(n-1)}{2} - \frac{n^2(n-1)(n-2)}{6} = \frac{(n^2-1)n^2}{3}\]
In a $n$ dimensional Riemann manifold,  the number of independent components of Riemann tensor is \[(\frac{n(n-1)}{2})^2 -\frac{n^2(n-1)(n-2)}{6} = \frac{(n^2-1)n^2}{12} \]
(2) The double dual of Riemann tensor
\[\bar{G}^{\alpha \beta}_{\phantom{\alpha \beta} \gamma \delta} = \frac{1}{2} \tilde{\epsilon}^{\alpha \beta \mu \nu} R_{\mu \nu}^{\phantom{\mu \nu} \rho \sigma} \frac{1}{2} \tilde{\epsilon}_{\rho \sigma \gamma \delta} = -\frac{1}{4} \delta^{\alpha \beta \mu \nu}_{\rho \sigma \gamma \delta} R_{\mu \nu}^{\phantom{\mu \nu} \rho \sigma} \]
contains precisely the same amount of information as
Riemann tensor, and satisfies precisely the same set of symmetries.\\
(3) The Einstein curvature tensor, which is symmetric
\[G^{\beta}_{\phantom{\beta}\delta} = \bar{G}^{\mu \beta}_{\phantom{\mu \beta} \mu \delta}; \ \ \ \ G_{\beta \delta}=G_{\delta \beta}\]
(4) The Bianchi identity takes a particularly simple form when rewritten in Bianchi identities terms of the double dual $\bar{G}$:
\[\bar{G}^{\alpha \beta \phantom{\gamma \delta};\delta}_{\phantom{\alpha \beta} \gamma \delta} = 0\]
and it has the obvious consequence
\[G_{\beta \delta}^{\phantom{\beta \delta};\delta} = 0\]
(5) The Ricci curvature tensor, which is symmetric, and the curvature scalar
\[R^{\beta}_{\phantom{\beta}\delta} = R^{\mu \beta}_{\phantom{\mu \beta} \mu \delta}; \ \ \ \ R_{\beta \delta}=R_{\delta \beta}; \ \ \  \ R = R^{\beta}_{\phantom{\beta}\beta}\]
which are related to the Einstein tensor by 
\[G^{\beta}_{\phantom{\beta}\delta} =  R^{\beta}_{\phantom{\beta}\delta} - \frac{1}{2} \delta^{\beta}_{\delta}R\]
(6) The Weyl conformal tensor
\[C^{\alpha \beta}_{\phantom{\alpha \beta} \gamma \delta} = R^{\alpha \beta}_{\phantom{\alpha \beta} \gamma \delta} -2 \delta^{[\alpha}_{\phantom{[\alpha}[\gamma} R^{\beta]}_{\phantom{\beta]}\delta]} + \frac{1}{3} \delta^{[\alpha}_{\phantom{[\alpha}[\gamma} \delta^{\beta]}_{\phantom{\beta]}\delta]} R \]
possesses the same symmetries as the Riemann tensor. 
Weyl tensor is completely "trace-free"; i.e., that
contraction of $C_{\alpha \beta \gamma \delta}$ on any pair of slots vanishes. Thus, $C_{\alpha \beta \gamma \delta}$ can be regarded as the trace-free part of Riemann, and $R_{\alpha \beta}$ can be regarded as the trace of Riemann. Riemann is determined entirely by its trace-free part $C_{\alpha \beta \gamma \delta}$ and
its trace $R_{\alpha \beta}$.\\ \\
\textbf{Riemann normal coordinates}
\[\Gamma^{\alpha}_{\phantom{\alpha} \beta \gamma}(\mathcal{P}_0) = 0\]
\[\Gamma^{\alpha}_{\phantom{\alpha} \beta \gamma,\mu}(\mathcal{P}_0) = -\frac{1}{3} (R^{\alpha}_{\phantom{\alpha} \beta \gamma \mu} + R^{\alpha}_{\phantom{\alpha} \gamma \beta \mu})\]
\[g_{\alpha \beta}(\mathcal{P}_0) = \eta_{\alpha \beta}\]
\[g_{\alpha \beta,\gamma}(\mathcal{P}_0) = 0\]
\[g_{\alpha \beta,\mu \nu}(\mathcal{P}_0) = -\frac{1}{3} (R_{\alpha \mu \beta \nu} +R_{\alpha \nu \beta \mu})\]
\[R_{\alpha \beta \gamma \delta}(\mathcal{P}_0) = g_{\alpha \delta, \beta \gamma} - g_{\alpha \gamma , \beta \mu}\]\\
\textbf{The proper reference frame of an accelerated observer}\\
(1) Let $\tau$ be proper time as measured by the accelerated observer's clock .Let $\mathcal{P} = \mathcal{P}_0(\tau)$ be the observer's world line.\\
(2) The observer carries with himself an orthonormal tetrad $\{\mathbf{e}_{\hat{\alpha}}\}$
with
\[\mathbf{e}_{\hat{0}} = \mathbf{u} = \frac{d \mathcal{P}_0}{d \tau}\]
and with
\[\mathbf{e}_{\hat{\alpha}} \cdot \mathbf{e}_{\hat{\beta}} = \eta_{\alpha \beta}\]
(3) The tetrad changes from point to point along the observer's world line, relative to parallel transport:
\[\nabla_{\mathbf{u}} \mathbf{e}_{\hat{\alpha}} = - \mathbf{\Omega} \cdot \mathbf{e}_{\hat{\alpha}} \]
\[\Omega^{\mu \nu} = a^{\mu} u^{\nu} - u^{\mu} a^{\nu} + u_{\alpha} a_{\beta}\epsilon^{\alpha \beta \mu \nu}\]
This transport law has the same form in curved space-time as in flat because curvature can only be felt over finite distances, not over
the infinitesimal distance involved in the "first time-rate of change of a vector" (equivalence principle).
\[\mathbf{a} = \nabla_{\mathbf{u}} \mathbf{u}\]
\[\mathbf{u} \cdot \mathbf{a} = \mathbf{u} \cdot \mathbf{\omega} = 0\]
If $\omega$ were zero, the observer would be Fermi-Walker-transporting his tetrad (gyroscope-type transport). If both $\mathbf{a}$ and $\mathbf{\omega}$ were zero, he would be freely falling (geodesic motion) and would be parallel-transporting his tetrad.\\
(4) The observer constructs his proper reference frame (local coordinate system) in a manner analogous to the Riemann-normal construction. From each event $\mathcal{P}_0(\tau)$ on his world line, he sends out purely spatial geodesics (geodesics orthogonal to $\mathbf{u}$), with affine parameter equal to proper length. The tangent vector has unit length, because the chosen affine
parameter is proper length.\\
(5) Each event near the observer's world line is intersected by precisely one of the geodesics $\mathcal{G}[\tau,\mathbf{n},s]$. [Far away, this is not true; the geodesics may cross, either
because of the observer's acceleration].\\
(6) Pick an event $\mathcal{P}$ near the observer's world line. The geodesic through it originated on the observer's world line at a specific time $\tau$, had original direction $\mathbf{n} = n^{\hat{j}} \mathbf{e}_{\hat{j}}$; and needed to extend a distance $s$ before reaching $\mathcal{P}$. Hence, the four numbers
\[(x^{\hat{0}},x^{\hat{1}},x^{\hat{2}},x^{\hat{3}}) \equiv (\tau,s n^{\hat{1}},s n^{\hat{2}},s n^{\hat{3}})\]
are a natural way of identifying the event $\mathcal{P}$. These are the coordinates of $\mathcal{P}$ in the observer's proper reference frame.\\ \\
Along the world line of observer:
\[\frac{\partial}{\partial x^{\hat{\alpha}}} = \mathbf{e}_{\hat{\alpha}}\]
\[g_{\hat{\alpha} \hat{\beta}} = \mathbf{e}_{\hat{\alpha}} \cdot \mathbf{e}_{\hat{\beta}} = \eta_{\hat{\alpha}\hat{\beta}}\]
\[\Gamma^{\hat{\beta}}_{\hat{\alpha} \hat{0}} = - \Omega^{\hat{\beta}}_{\hat{\alpha}} \]
\[\Gamma^{\hat{0}}_{\hat{0} \hat{0}} =0\]
\[\Gamma^{\hat{0}}_{\hat{j} \hat{0}} = \Gamma^{\hat{j}}_{\hat{0} \hat{0}} = a^{\hat{j}} \]
\[\Gamma^{\hat{j}}_{\hat{k} \hat{0}} = -\omega^{\hat{i}} \epsilon_{0 \hat{i} \hat{j} \hat{k}}\]
\[\Gamma^{\hat{\alpha}}_{\hat{j} \hat{k}} = 0 \]
\[g_{\hat{\alpha} \hat{\beta},\hat{0}} = 0\]
\[g_{\hat{j} \hat{k},\hat{l}} = 0\]
\[g_{\hat{0} \hat{0},\hat{j}} = -2 a^{\hat{j}}\]
\[g_{\hat{0} \hat{j},\hat{k}} = -\epsilon_{0 \hat{j} \hat{k} \hat{l}} \omega^{\hat{l}}\]
\subsection{Basic assumptions of general relativity}
(1) Space-time is a four dimensional pseudo-Riemann manifold.\\
(2) The metric of the manifold is governed by the Einstein field equation
\[\mathbf{G} = 8\pi\mathbf{T}.\]
(3) All special relativistic laws of physics are valid in local Lorentz frames of metric.\\
\end{document}



\part{Quantum Mechanics}
\documentclass[cyan]{elegantnote}
\author{Yuyang Songsheng}
\email{songshengyuyang@gmail.com}
\zhtitle{物理}
\entitle{Physics}
\version{1.00}
\myquote{Do not ask what it is. Ask what you can say about it.}
\logo{logo.jpg}
\cover{cover.pdf}
%green color
   \definecolor{main1}{RGB}{210,168,75}
   \definecolor{seco1}{RGB}{9,80,3}
   \definecolor{thid1}{RGB}{0,175,152}
%cyan color
   \definecolor{main2}{RGB}{239,126,30}
   \definecolor{seco2}{RGB}{0,175,152}
   \definecolor{thid2}{RGB}{236,74,53}
%cyan color
   \definecolor{main3}{RGB}{127,191,51}
   \definecolor{seco3}{RGB}{0,145,215}
   \definecolor{thid3}{RGB}{180,27,131}


\usepackage{makecell}
\usepackage{lipsum}
\usepackage{amssymb}
\usepackage{float}
\usepackage{wrapfig}
\usepackage{latexsym}
\usepackage{hyperref}
\usepackage{feynmf}
\usepackage{exscale}
\usepackage{relsize}
\usepackage{bm}%bold math, for vector


\begin{document}
\maketitle
\tableofcontents

\section{Eigenvalues of angular momentum operator}
The commutation relations among the angular momentum operators are
\[[J_i,J_j] = \epsilon_{ijk} J_k\]
And these three operators are self-adjoint. We first introduce the operator $J^2 = J_x^2 + J_y^2 + J_z^2$. We can verify that $[J^2,\bm{J}] = 0$. Thus there exists a complete set of common eigenvectors of $J^2$ and any one component of $\bm{J}$. Particularly, we have the pair of eigenvalue equations
\[J^2 | \beta,m\rangle = \beta | \beta,m\rangle \quad J_z | \beta,m\rangle = m | \beta,m\rangle\]
Since
\[\langle \beta,m | J^2 | \beta,m \rangle = \langle \beta,m | J_x^2 | \beta,m \rangle + \langle \beta,m | J_y^2 | \beta,m \rangle + \langle \beta,m | J_z^2 | \beta,m \rangle\]
we have $m^2 \leq \beta$. Thus for a fixed value of $\beta$ there must be maximum and minimum values for $m$.\\
Define
\[J_+ \equiv J_x + iJ_y \quad J_- \equiv J_x - iJ_y\]
we have the commutation relations
\[[J_z,J_+] = J_+ \quad [J_z,J_-] = -J_- \quad [J_+,J_-] = 2J_z\]
So
\[J_z J_+ | \beta,m\rangle = J_+(J_z + 1)| \beta,m\rangle = (m+1)J_+| \beta,m\rangle\]
Therefore, either $J_+ | \beta,m\rangle$ is an eigenvector of $J_z$ with the raised eigenvalue $m+1$, or $J_+ | \beta,m\rangle = 0$. Now for fixed $\beta$ there is a maximum value of $m$, which we shall denote as $j$. It must be the case that
\[J_+ |\beta,j\rangle = 0\]
Since
\[J_- J_+ = J^2 - J_z^2 - J_z\]
it is obvious that $\beta = j(j+1)$. By similar method, we can show the minimum eigenvalue of $J_z$ for fixed $\beta$ satisfy that $\beta = k(k-1)$. So, we have $k = -j$. \\
We have thus shown the existence of a set of eigenvectors corresponding to integer spaced $m$ values in the range $-j \leq m \leq j$. Since the difference between the maximum value $j$ and the minimum value $-j$ must be an integer, it follows that $j = \mbox{ integer} / 2$. Henceforth we shall adopt the common and more convenient notation of labeling the eigenvectors by $j$ instead of by $\beta$. Thus the vector that was previously denoted as $|\beta,m\rangle$ will now be denoted as $|j,m\rangle$. \\ \\
To find the matrix element of angular momentum operator, we notice that
\[\langle j,m| J_-J_+|j,m\rangle = j(j+1)-m(m+1)\]
so, we can get
\[J_+ |j,m\rangle = \sqrt{(j+m+1)(j-m)} |j,m+1\rangle\]
Similarly, we have
\[J_- |j,m\rangle = \sqrt{(j-m+1)(j+m)} |j,m-1\rangle\]
The matrix element of $J_+$, $J_-$ and $J_z$ are
\[\langle j',m'| J_+ | j,m\rangle = \sqrt{(j+m+1)(j-m)} \delta_{jj'}\delta_{m',m+1}\]
\[\langle j',m'| J_- | j,m\rangle = \sqrt{(j-m+1)(j+m)} \delta_{jj'}\delta_{m',m-1}\]
\[\langle j',m'| J_z | j,m\rangle = m \delta_{jj'}\delta_{m',m}\]

\section{Orbital Angular Momentum and Spin}
Let $\psi(\bm{x})$ be a one-component state function in coordinate representation. When it is subjected to a rotation it is transformed into
\[\bm{R}\psi(\bm{x}) = \psi(R^{-1}\bm{x})\]
where $\bm{R}$ is the rotation operator generated by $\bm{R} = \exp(-i\bm{J}\cdot\bm{n}\theta)$. For a rotation through infinitesimal angle $\epsilon$ about the $z$ axis, we have
\[\bm{R}_z(\epsilon) \psi(x,y,z) = \psi(x+\epsilon y,y-\epsilon x,z) = \psi(x,y,x) + \epsilon (y \frac{\partial \psi}{\partial x} - x \frac{\partial \psi}{\partial y})\]
On the other hand, 
\[\bm{R}_z(\epsilon) = I - i\epsilon J_z\]
so, we have $J_z = -i(x \frac{\partial}{\partial y} - y \frac{\partial}{\partial x})$. This is just the $z$ component of the orbital angular momentum operator $L = \bm{X} \times \bm{P}$.\\
For a multicomponent state function, we have
\[\bm{R} \left( \begin{matrix} \psi_1(\bm{x}) \\ \psi_2(\bm{x}) \\  \vdots \end{matrix} \right) = D \left( \begin{matrix} \psi_1(R^{-1}\bm{x}) \\ \psi_2(R^{-1}\bm{x}) \\  \vdots \end{matrix} \right)\]
Thus the general form of the rotation operator will be
\[\bm{R}_{n}(\theta) = e^{-i\bm{L}\cdot\bm{n}\theta}D_n(\theta)\]
The two factors commute because the first acts only on the coordinate and the second acts only on the components of the column vector. The matrix $D$ must be unitary, and so it can be written as
\[D_n(\theta) = e^{-i\bm{S}\cdot\bm{n}\theta}\]
The angular momentum operator $\bm{J}$ has the form
\[\bm{J} = \bm{L} + \bm{S}\]
with $\bm{L} = \bm{X} \times \bm{P}$ and $[L_{\alpha},S_{\beta}] = 0$. In the particular representation used in this section, we have $\bm{L} = -i\bm{x}\times\bm{\nabla}$, and the components of $S$ are
discrete matrices. The operators $\bm{L}$ and $\bm{S}$ are called the orbital and spin parts of the angular momentum.
\subsubsection{Orbital angular momentum}
The form of the gradient operator in spherical coordinates is
\[\bm{\nabla} = \bm{e}_r \frac{\partial}{\partial r} + \bm{e}_{\theta} \frac{1}{r} \frac{\partial}{\partial \theta} + \bm{e}_{\phi} \frac{1}{r\sin\theta} \frac{\partial}{\partial \phi}\]
The orbital angular momentum operator then has the form
\[\bm{L} = r\bm{e}_r \times (-i\bm{\nabla}) = (-i) \left[ \bm{e}_{\phi} \frac{\partial}{\partial \theta} - \bm{e}_{\theta} \frac{1}{\sin\theta} \frac{\partial}{\partial \phi} \right]\]
So,we have
\[L_z = \bm{L}\cdot\bm{e}_z = -i\frac{\partial}{\partial \phi}\]
\[L^2 = \bm{L}\cdot\bm{L} = - \left [ \frac{1}{\sin\theta} \frac{\partial }{\partial \theta} (\sin\theta \frac{\partial }{\partial \theta}) + \frac{1}{\sin^2\theta} \frac{\partial^2}{\partial \phi^2}  \right ]\]
We must now solve the two coupled differential equations,
\[L_z Y(\theta,\phi) = m Y(\theta,\phi) \quad L^2 Y(\theta,\phi) = l(l+1) Y(\theta,\phi)\]
Apart from normalization, we have $Y(\theta,\phi) = e^{im\phi}P_l^m(\cos\theta)$. Here, $P_l^m$ is the \href{https://en.wikipedia.org/wiki/Associated_Legendre_polynomials}{associated Legendre polynomials}. If we assume that the solution must be single-valued under rotation, then it will follow that $m$ must be an integer. If we further assume that it must be nonsingular at $\theta = 0$ and $\theta = \pi$, then from the standard theory of the Legendre equation it will follow that $l$ must be a nonnegative integer in the range $l \geq |m|$. The normalized solutions that result from these assumptions are the well-known \href{https://en.wikipedia.org/wiki/Spherical_harmonics}{spherical harmonics}
\[Y_l^m(\theta,\phi) = (-1)^{(m+|m|)/2} \left [\frac{(2l+1)(l-|m|)!}{4\pi(l+|m|)!}  \right ]^{1/2}e^{im\phi} P_l^{|m|}(\cos\theta)\]
\subsubsection{Spin}
A particular species of particle is characterized by a set of quantum numbers that includes the value of its spin s, it is often sufficient to treat the spin operators $\bm{S}$ as acting on the space of dimension
$2s+1$ that is spanned by the eigenvectors of for a fixed value of $s$. \\
If $s= \frac{1}{2}$, we have
\[S_x = \frac{1}{2} \left[ \begin{matrix} 0& 1\\ 1& 0\end{matrix} \right] \quad S_y = \frac{1}{2} \left[ \begin{matrix} 0& -i\\ i& 0\end{matrix} \right] \quad S_z = \frac{1}{2} \left[ \begin{matrix} 1& 0\\ 0& -1\end{matrix} \right]\]
The spin operator in direction $\bm{n} = (\sin\theta\cos\phi,\sin\theta\sin\phi,\cos\theta)$ is
\[\bm{S}_{n} = \frac{1}{2} \left[ \begin{matrix} \cos\theta& e^{-i\phi}\sin\theta\\ e^{i\phi}\sin\theta& -\cos\theta\end{matrix} \right]\]
The eigenvectors are
\[\left[ \begin{matrix} e^{-i\phi/2}\cos\frac{\theta}{2}\\ e^{i\phi/2}\sin\frac{\theta}{2} \end{matrix} \right] \quad \left[ \begin{matrix} -e^{-i\phi/2}\sin\frac{\theta}{2}\\ e^{i\phi/2}\cos\frac{\theta}{2} \end{matrix} \right]\]
corresponding to eigenvalues $\frac{1}{2}$ and $-\frac{1}{2}$.\\
If $s= 1$, we have
\[S_x = \sqrt{\frac{1}{2}} \left[ \begin{matrix} 0& 1& 0\\ 1& 0& 1\\ 0& 1& 0\end{matrix} \right]  \quad S_y = \sqrt{\frac{1}{2}} \left[ \begin{matrix} 0& -i& 0\\ i& 0& -i\\ 0& i& 0\end{matrix} \right] \quad S_z = \sqrt{\frac{1}{2}} \left[ \begin{matrix} 1& 0& 0\\ 0& 0& 0\\ 0& 0& -1\end{matrix} \right]\]
The spin operator in direction $\bm{n} = (\sin\theta\cos\phi,\sin\theta\sin\phi,\cos\theta)$ is
\[\bm{S}_{n} =  \left[ \begin{matrix} \cos\theta& \sin\theta e^{-i\phi} \sqrt{\frac{1}{2}}& 0\\ \sin\theta e^{i\phi}\sqrt{\frac{1}{2}}& 0& \sin\theta e^{-i\phi} \sqrt{\frac{1}{2}}\\ 0& \sin\theta e^{i\phi} \sqrt{\frac{1}{2}}& -\cos\theta\end{matrix} \right]\]
The eigenvectors are
\[\left[ \begin{matrix} \frac{1}{2}(1+\cos\theta)e^{-i\phi}\\ \sqrt{\frac{1}{2}}\sin\theta \\ \frac{1}{2}(1-\cos\theta)e^{i\phi} \end{matrix} \right] \quad \left[ \begin{matrix} -\sqrt{\frac{1}{2}}\sin\theta e^{-i\phi}\\ \cos\theta \\ \sqrt{\frac{1}{2}}\sin\theta e^{i\phi} \end{matrix} \right] \quad \left[ \begin{matrix} \frac{1}{2}(1-\cos\theta)e^{-i\phi}\\ -\sqrt{\frac{1}{2}}\sin\theta \\ \frac{1}{2}(1+\cos\theta)e^{i\phi} \end{matrix} \right]\]
corresponding to eigenvalues $1$, $0$ and $-1$.

\section{Rotation operator}
Three parameters are required to describe an arbitrary rotation. A common parameterization is by the Euler angles. From the fixed system of axes $Oxyz$, a new rotated set of axes $Ox'y'z'$ is produced in three steps:
\begin{itemize}
\item Rotate through angle $\alpha$ about $Oz$, carrying $Oy$ into $Ou$
\item Rotate through angle $\beta$ about $Ou$, carrying $Oz$ into $Oz'$
\item Rotate through angle $\gamma$ about $Oz'$ , carrying $Ou$ into $Oy'$
\end{itemize}
\begin{figure}[!h]
	\centering
	\includegraphics[height=4.5cm ,width=5cm]{QM/rotation.png}
	\caption{Euler angles}
\end{figure}
The net rotation is
\[\bm{R}(\alpha,\beta,\gamma) = \bm{R}_{z'}(\gamma) \bm{R}_{u}(\beta) \bm{R}_{z}(\alpha) = e^{-i\gamma J_{z'}} e^{-i\beta J_{u}} e^{-i\alpha J_{z}}\]
Since $J_u = \bm{R}_z(\alpha) J_y \bm{R}_z(-\alpha)$, we have $\bm{R}_u(\beta) = \bm{R}_z(\alpha) \bm{R}_y(\beta) \bm{R}_z(-\alpha)$. Similarly, we can obtain $\bm{R}_{z'}(\gamma) = \bm{R}_{u}(\beta) \bm{R}_z(\gamma) \bm{R}_u(-\beta)$. So, the rotation operator is
\[\bm{R}(\alpha,\beta,\gamma) = \bm{R}_{z}(\alpha) \bm{R}_{y}(\beta) \bm{R}_{z}(\gamma) = e^{-i\alpha J_{z}} e^{-i\beta J_{y}} e^{-i\gamma J_{z}}\]
The matrix representation of the rotation operator in the basis $|j,m\rangle$
\[\langle j',m' | \bm{R}(\alpha,\beta,\gamma) | j,m \rangle = \delta_{jj'} D_{m'm}^{(j)}(\alpha,\beta,\gamma)\]
gives rise to the rotation matrices,
\[D_{m'm}^{(j)}(\alpha,\beta,\gamma) = \langle j',m' | e^{-i\alpha J_{z}} e^{-i\beta J_{y}} e^{-i\gamma J_{z}} | j,m \rangle = e^{-i(\alpha m' + \gamma m)} d_{mm'}^{(j)}(\beta)\]
where
\[ d_{mm'}^{(j)}(\beta) = \langle j',m' | e^{-i\beta J_{y}} | j,m \rangle\]
For the case of $j = \frac{1}{2}$, we have $J_y = \frac{1}{2}\sigma_y$ and $\sigma_y^2 = I$. We can obtain
\[d^{(1/2)}(\beta) = \left[ \begin{matrix} \cos \frac{\beta}{2} & -\sin \frac{\beta}{2} \\ \cos \frac{\beta}{2}& \sin \frac{\beta}{2}\end{matrix} \right] \]
Notice that this matrix is periodic in $\beta$ with period $4\pi$, but it changes sign when $2\pi$ is added to $\beta$. This double-valuedness under rotation by $2\pi$ is a characteristic of the full rotation matrix whenever $j$ is a half odd-integer. The matrix is single-valued under rotation by $2\pi$ whenever $j$ is an integer.\\
Rotation of angular momentum eigenvectors now can be written as
\[\bm{R}(\alpha,\beta,\gamma)|j,m\rangle = \sum_{m'} D_{m'm}^{(j)}(\alpha,\beta,\gamma) |j,m'\rangle\]
When it comes to spherical harmonics, we have
\[Y_l^m(\theta',\phi') = \bm{R}^{-1}((\alpha,\beta,\gamma)) Y_l^m(\theta,\phi) = \sum_{m'} Y_{l}^{m'}(\theta,\phi) [D_{mm'}^{(j)}((\alpha,\beta,\gamma))]^*\]
By putting $\beta = \gamma = 0$ we obtain
\[Y_l^m(\theta,\phi+\alpha) = \sum_{m'} Y_{l}^{m'}(\theta,\phi) [D_{mm'}^{(j)}((\alpha,0,0))]^* = e^{i\alpha m} Y_{l}^{m}(\theta,\phi)\]
Setting $\phi=0$ then yields
\[Y_l^m(\theta,\alpha) = e^{i\alpha m} Y_{l}^{m}(\theta,0)\]
Since the direction $\theta = 0$ is the polar axis, continuity of the spherical harmonic requires that $Y_l^m(0,\alpha)$ be independent of $\alpha$. Therefore we must
have $Y_l^m(0,0) = 0$ for $m \neq 0$, and so we can write\\
\[Y_{l}^{m}(\theta,0) = c_{l}\delta{m0}\]
The we have
\[Y_l^m(\theta,\phi) = \sum_{m'} Y_{l}^{m'}(0,0) [D_{mm'}^{(j)}((\phi,\theta,\gamma))]^* = c_l [D_{m0}^{(j)}((\phi,\theta,\gamma))]^*\]
for arbitrary $\gamma$, thus obtaining a simple relation between the spherical harmonics and the rotation matrices. Conventional normalization is obtained if we put
\[c_l = \left( \frac{2l+1}{4\pi} \right) ^{1/2}\]
\\
The operator for a rotation through $2\pi$ about an axis
along the unit vector $\bm{n}$ is $\bm{R}_n(2\pi) = e^{-2\pi i\bm{n}\cdot\bm{J}}$. Its effect on the standard angular momentum eigenvectors is
\[\bm{R}_n(2\pi) = (-1)^{2j}|j,m\rangle \]
We assume a rotation through $2\pi$ as a trivial operation that leaves everything unchanged, i.e. all dynamical variables are invariant under $2\pi$ rotation:
\[\bm{R}(2\pi) A \bm{R}^{-1}(2\pi) = A\]
where $A$ may represent any physical observable. \\
the operator $\bm{R}_{2\pi}$ divides the vector space into two subspaces. A typical vector in the first subspace,
denoted as $|+\rangle$, has the property $\bm{R}(2\pi)|+\rangle = |+\rangle$, whereas a typical vector in the second subspace, denoted as $|-\rangle$, has the property $\bm{R}(2\pi)|-\rangle = -|-\rangle$. Now, if $A$ represents any physical observable, we have $\langle + | \bm{R}(2\pi) A| - \rangle = \langle + | A\bm{R}(2\pi)| - \rangle$, leading to 
\[\langle + | A | - \rangle = 0\] 
No physical observable can have nonvanishing matrix elements between states with integer angular momentum and states with half odd-integer angular momentum. This fact forms the basis of a superselection rule: There is no observable distinction among the state vectors of the form\\
\[|\Psi_{\omega}\rangle = |+\rangle + e^{i\omega}|-\rangle\]
for different values of the phase $\omega$.

\section{Addition of angular momentum}
Let us consider a two-component system, each component of which has angular momentum degrees of freedom. Basis vectors for the composite system can be formed from the basis vectors of the components by taking all binary products of a vector from each set
\[|j_1,j_2,m_1,m_2\rangle = |j_1,m_1\rangle ^{(1)} |j_2,m_2\rangle ^{(2)}\]
These vectors are common eigenvectors of the four commutative operators $\bm{J}^{(1)}\cdot\bm{J}^{(1)}$, $\bm{J}^{(2)}\cdot\bm{J}^{(2)}$, $\bm{J}_z^{(1)}$, and $\bm{J}_z^{(2)}$.
It is often desirable to form eigenvectors of the total angular momentum operators, $\bm{J}\cdot\bm{J}$ and $\bm{J}_z$, where the total angular momentum vector operator is
\[\bm{J} = \bm{J}^{(1)}\otimes\bm{1} + \bm{1}\otimes\bm{J}^{(2)}\]
This is useful when the system is invariant under rotation as a whole, but not under rotation of the two components separately. 
The eigenvectors of $\bm{J}\cdot\bm{J}$ and $\bm{J}_z$ may be denoted as $|\alpha, J, M \rangle$. It is easy to verify that the four operators $\bm{J}^{(1)}\cdot\bm{J}^{(1)}$, $\bm{J}^{(2)}\cdot\bm{J}^{(2)}$, $\bm{J}\cdot\bm{J}$ and $\bm{J}_z$ are mutually commutative, and hence they possess a complete set of common eigenvectors. 
Since the set of product vectors and the new set of total angular momentum eigenvectors are both eigenvectors of $\bm{J}^{(1)}\cdot\bm{J}^{(1)}$ and $\bm{J}^{(2)}\cdot\bm{J}^{(2)}$, the eigenvalues $j_1$ and $j_2$ will be constant in both sets. Therefore
we may confine our attention to the vector space of dimension $(2j_1+1)(2j_2+1)$ that is spanned by product vectors with fixed values of $j_1$ and $j_2$.\\
Now the $2J+1$ vectors $|\alpha, J, M \rangle$, with $M$ in the range $-J \leq M \leq J$, span an irreducible subspace.
Therefore if the vector $|\alpha, J, M \rangle$, for a particular value of $M$, can be constructed in the space under consideration, then so can the entire set of $2J+1$
such vectors with $M$ in the range $-J \leq M \leq J$.
For a particular value of $J$, it might be possible to construct one such set of vectors, two or more linearly independent sets, or none at all. \\
Let $N(J)$ denote the number of independent sets that can be constructed. Let $n(M)$ be the degree of degeneracy, in this space, of the eigenvalue $M$ . The relation between these two quantities is
\[n(M) = \sum_{J \geq |M|} N(J)\]
and hence
\[N(J) = n(J) - n(J+1)\]
The product vectors $|j_1,m_1\rangle |j_2,m_2\rangle$ are eigenvectors of the operator $\bm{J}_z$, with eigenvalue $m_1+m_2$, and the degree of degeneracy $n(M)$ is equal to the number of pairs $(m_1,m_2)$ such that $M=m_1+m_2$.
\begin{figure}[!h]
	\centering
	\includegraphics[height=4.14cm ,width=6.4cm]{QM/angular_momentum.png}
	\caption{Possible values of $M=m_1+m_2$ , illustrated for $j_1=3$, $j_2=2$}
\end{figure}
Therefore,
\[n(M)=\begin{cases} 0 \quad |M| > j_1 + j_2\\ j_1+j_2+1-|M|\quad |j_1-j_2| \leq M \leq |j_1+j_2|\\ 2j_{min}+1 \quad 0 \leq |M|\leq |j_1-j_2|\end{cases} \]
It then follows that
\[N(J)=\begin{cases} 1 \quad |j_1-j_2| \leq J \leq |j_1+j_2|\\ 0 \quad \mbox{otherwise}\end{cases} \]
It has turned out that $N(J)$ is never greater that $1$, and so the vectors $|\alpha, J, M \rangle$ can be uniquely labelled by the eigenvalues of the four operators $\bm{J}^{(1)}\cdot\bm{J}^{(1)}$, $\bm{J}^{(2)}\cdot\bm{J}^{(2)}$, $\bm{J}\cdot\bm{J}$ and $\bm{J}_z$. Henceforth these total angular momentum
eigenvectors will be denoted as $|j_1,j_2,J,M\rangle$. And we have the unitarity transformation 
\[|j_1,j_2,J,M\rangle = \sum_{m_1,m_2} |j_1,j_2,m_1,m_2\rangle \langle j_1,j_2,m_1,m_2 | j_1,j_2,J,M\rangle\]
The coefficients of this transformation are called the Clebsch–Gordan coefficients, denoted as $(j_1,j_2,m_1,m_2|J,M)$.
The phases of the CG coefficients are not yet defined because of the indeterminacy of the relative phases of the vectors $|j_1,j_2,J,M\rangle$. For different values of $M$ but fixed $J$ we adopt the usual phase convention that led to
\[J_+ |j_1,j_2,J,M\rangle = \sqrt{(J+M+1)(J-M)} |j_1,j_2,J,M+1\rangle\]
This leaves one arbitrary phase for each $J$ value, which we fix by requiring that $(j_1,j_2,j_1,J-j_1|J,J)$ be real and positive. It can be shown that all of the CG coefficients are now real.\\
We can also prove that the CG coefficient vanishes unless the following conditions are satisfied:
\begin{itemize}
\item $m_1+m_2=M$
\item $|j_1-j_2| \leq J \leq |j_1+j_2|$
\item $j_1+j_2+J =$ an integer
\end{itemize}
It is possible to calculate the values of the CG coefficients by successive application of the raising or lowering operator to
\[|j_1,j_2,J,M\rangle = \sum_{m_1,m_2} |j_1,j_2,m_1,m_2\rangle (j_1,j_2,m_1,m_2|J,M)\]
The details of the calculation can be found in section 7.7 of \emph{Quantum mechanics - a modern development(Leslie E. Ballentine)}.And we have \href{https://en.wikipedia.org/wiki/Table_of_Clebsch-Gordan_coefficients}{Table of CG coefficients} and \href{http://www.wolframalpha.com/input/?i=CG+coefficient}{Calculator of CG coefficients} on the internet. A special case of angular momentum addition is spin–orbit coupling of spin $\frac{1}{2}$ particles, and we list the corresponding CG coefficients $(l,\frac{1}{2}, M-m_s, m_s|J,M)$ in the table 1.1.
\begin{table}[!th]
\centering
\begin{tabular}{|c|c|c|}
\hline
 & $J=l+\frac{1}{2}$ & $J=l-\frac{1}{2}$ \\
 \hline
 $m_s = \frac{1}{2}$ & $\left[ \frac{l+M+\frac{1}{2}}{2l+1}\right ]^{\frac{1}{2}} $ & $-\left[ \frac{l-M+\frac{1}{2}}{2l+1}\right ]^{\frac{1}{2}} $ \\
 \hline
 $m_s = -\frac{1}{2}$ & $\left[ \frac{l-M+\frac{1}{2}}{2l+1}\right ]^{\frac{1}{2}} $ & $\left[ \frac{l+M+\frac{1}{2}}{2l+1}\right ]^{\frac{1}{2}} $ \\
\hline
\end{tabular}
\caption{Spin-Orbit coupling}
\end{table}\\
Now let us consider the relation between CG coefficients and rotation matrices. On the one hand, we have
\[\langle j_1,j_2,m_1,m_2 | \bm{R} | j_1,j_2,m'_1,m'_2\rangle = D_{m_1m'_1}^{(j_1)}(R) D_{m_2m'_2}^{(j_2)}(R)\]
On the other hand, we have
\begin{eqnarray}
&\phantom{=}& \langle j_1,j_2,m_1,m_2 | \bm{R} | j_1,j_2,m'_1,m'_2\rangle \nonumber \\
&=& \sum_{J,M,J',M'} (j_1,j_2,m_1,m_2|J,M) (j_1,j_2,m'_1,m'_2|J',M')  \langle j_1,j_2,J,M | \bm{R} | j_1,j_2,J',M'\rangle \nonumber \\
&=& \sum_{J,M,M'} (j_1,j_2,m_1,m_2|J,M) (j_1,j_2,m'_1,m'_2|J,M')D_{MM'}^{(J)}(R) \nonumber
\end{eqnarray}
So, we can get
\[D_{m_1m'_1}^{(j_1)}(R) D_{m_2m'_2}^{(j_2)}(R) = \sum_{J,M,M'} (j_1,j_2,m_1,m_2|J,M) (j_1,j_2,m'_1,m'_2|J,M')D_{MM'}^{(J)}(R)\]
It is called Clebsch-Gordan series.\\
Recall that
\[Y_l^m(\theta,\phi) =  \left( \frac{2l+1}{4\pi} \right) ^{1/2} [D_{m0}^{(j)}((\phi,\theta,0))]^* \]
so, we have
\[Y_{l_1}^{m_1}(\theta,\phi) Y_{l_2}^{m_2}(\theta,\phi) = \sum_{l,m} \sqrt{\frac{(2l_1+1)(2l_2+1)}{4\pi(2l+1)}} (l_1,l_2,m_1,m_2|l,m) (l_1,l_2,0,0|l,0) Y_l^m(\theta,\phi)\]
The orthogonal relation of spherical harmonics then would imply that
\[\int d\Omega Y_l^{m*}(\theta,\phi)Y_{l_1}^{m_1}(\theta,\phi) Y_{l_2}^{m_2}(\theta,\phi) = \sqrt{\frac{(2l_1+1)(2l_2+1)}{4\pi(2l+1)}} (l_1,l_2,m_1,m_2|l,m) (l_1,l_2,0,0|l,0)\]

\section{Tensor operators}
Suppose the state of the system is $|\psi\rangle$, then the state after rotation $R$ is $U(R)|\psi\rangle$, denoted as $|\psi'\rangle$. An operator $K$ is called scalar operator if and only if
\[\langle \psi' | K | \psi' \rangle = \langle \psi | K | \psi\rangle\]
i.e.
\[U^{-1}(R)KU(R) = K\]
Taking the case of infinitesimal rotation, we can derive that
\[[\bm{J},K] = 0\]
A group of operators $\bm{V}$ is called vector operator if and only if
\[\langle \psi' | V_i | \psi' \rangle = R_{ii'} \langle \psi | V_{i'} | \psi\rangle\]
i.e.
\[U^{-1}(R)V_{i}U(R) = \sum_{i'} R_{ii'} V_{i'}\]
Taking the case of infinitesimal rotation, we can derive that
\[[J_i, V_j] = i\epsilon_{ijk}V_k\]
If $\bm{V}$ and $\bm{W}$ are vector operators, we can prove that $\bm{V} \cdot \bm{W}$ is scalar operator and $\bm{V} \times \bm{W}$ is vector operator.\\
Similarly, tensor operators are defined as
\[U^{-1}(R) T_{ij\cdots k} U(R) = \sum_{i'\cdots} R_{ii'} R_{jj'} \cdots R_{kk'} T_{i'j'\cdots k'}\]
Such a tensor is known as a Cartesian tensor.\\
The trouble with a Cartesian tensor is that it is reducible, i.e. it can be decomposed into objects that transform independently under rotations. For example, the trace of a tensor transform like a scalar under rotations. 
So we now define spherical tensor operators which are irreducible under rotations.  We define a spherical tensor operator of rank $k$ with $(2k+1)$ components as
\[U^{-1}(R) T^{(k)}_q U(R) = \sum_{q'=-k}^{k} [D^{(k)}_{qq'}(R)]^* T^{(k)}_{q'}\]
or equivalently
\[U(R) T^{(k)}_q U^{-1}(R) = \sum_{q'=-k}^{k} D^{(k)}_{q'q}(R) T^{(k)}_{q'}\]
where $D^{(k)}_{qq'}$ is the rotation matrix.\\
Taking the case of infinitesimal rotation, we can derive that
\[[J_{\pm},T^{(k)}_{q}] = \sqrt{(k \mp q)(k \pm q +1)} T^{(k)}_{q \pm 1}\]
\[[J_z, T^{(k)}_{q}] = q T^{(k)}_{q}\]
For example, spherical components of a vector operator $\bm{V}$, 
\[V_{-1} = \frac{V_x - i V_y}{\sqrt{2}} \quad V_0 = V_z \quad V_{1} = -\frac{V_x + i V_y}{\sqrt{2}}\]
satisfy the commutation relation above, so they are spherical tensor of rank $1$.
Spherical tensors can be formed as products of other spherical  tensors, we have following theorem.
\newpage
\begin{newthem}
Let $X^{(k_1)}_{q_1}$ and $Z^{(k_2)}_{q_2}$ be irreducible spherical tensors of rank $k_1$ and $k_2$. Then
\[T^{(k)}_{q} = \sum_{q_1.q_2} (k_1,k_2,q_1,q_2|k,q) X^{(k_1)}_{q_1} Z^{(k_2)}_{q_2}\]
is a irreducible spherical tensor of rank $k$. 
\end{newthem}
\noindent
The proof can be found in section 3.10 of \emph{Modern Quantum Mechanics(J.J.Sakurai)}.\\
For example, suppose $\bm{V}$ and $\bm{U}$ are spherical tensor
\[T^{(0)}_{0} = \sqrt{\frac{1}{3}} (U_{-1}V_{1} + U_{1}V_{-1}-U_{0}U_{0}) = - \sqrt{\frac{1}{3}} (U_x^2 + U_y^2 + U_z^2)\]
is tensor of rank $1$, then is a spherical tensor of rank $0$. Another important theorem is Wigner-Eckart theorem.
\begin{newthem}
The matrix elements of tensor operators with respect to angular-momentum eigenstates satisfy
\[\langle \tau',j',m'| T^{(k)}_{q}| \tau,j,m\rangle = (j,k,m,q|j',m') \frac{\langle \tau',j' || T^{(k)} || \tau,j\rangle}{\sqrt{2j+1}}\]
where the double-bar matrix element is independent of $m$ and $m'$ and $q$.
\end{newthem}
\noindent
The proof can be found in section 3.10 of \emph{Modern Quantum Mechanics(J.J.Sakurai)}.\\
So, for scaler operator $K$, we have
\[\langle \tau',j',m'| S| \tau,j,m\rangle = \delta_{jj'}\delta_{mm'} \frac{\langle \tau',j' || S || \tau,j\rangle}{\sqrt{2j+1}}\]
For spherical tensor of rank $1$, we have
\[\langle \tau',j',m'| V_{q} | \tau,j,m\rangle = (j,1,m,q|j',m') \frac{\langle \tau',j' || V_q || \tau,j\rangle}{\sqrt{2j+1}}\]
It would vanish unless
\[m'-m = q \quad j'-j = 0,1,-1 \quad j \mbox{ and } j' \mbox{ are not both  } 0\]
For $j=j'$, Wigner-Eckart theorem - when applied to the vector operator- takes a particularly simple form. We can derive that
\[\langle \tau',j,m' | V_q | \tau, j ,m \rangle = \frac{\langle \tau',j,m | \bm{J}\cdot\bm{V} | \tau, j ,m \rangle}{j(j+1)} \langle j,m' | J_q | j ,m \rangle\]
For example, the magnetic moment operator for an atom has the form
\[\bm{\mu} = \frac{-e}{2m_e c} (g_L\bm{L} + g_s \bm{S})\]
The parameters $g_L$ and $g_S$ have approximately the
values $g_L = 1$ and $g_S = 2$. The former is an generalization of the magnetic moment we calculated in classical electrodynamics for a system of charged particles. The latter will be discussed in quantum field theory.
We define the effective Lande factor as
\[\langle \tau,J,M' | \bm{\mu} | \tau,J,M \rangle = \frac{-e}{2m_e c} g_{eff} \langle J,M' | \bm{J} | J,M \rangle\]
Then, we have
\[g_{eff} = \frac{\langle \tau,J,M | g_L\bm{L}\cdot\bm{J} + g_s \bm{S}\cdot\bm{J} | \tau,J,M \rangle}{J(J+1)} = 1 + \frac{J(J+1)- L(L+1) + S(S+1)}{2J(J+1)}\]

\section{Spherical potential well}
The stationary states of a particle in the spherical potential well are determined by
\[-\frac{1}{2M}\nabla^2 \Psi  + W(r) \Psi = E\Psi \]
In spherical coordinates, 
\[\nabla^2 = \frac{1}{r^2} \frac{\partial}{\partial r} \left [ r^2\frac{\partial}{\partial r} \right ] + \frac{1}{r^2\sin\theta} \frac{\partial}{\partial \theta} \left [\sin\theta \frac{\partial}{\partial \theta} \right ] + \frac{1}{r^2\sin^2\theta} \frac{\partial^2}{\partial\phi^2}\]
So, the eigenvalue equation becomes
\[-\frac{1}{2M} \frac{1}{r^2} \frac{\partial}{\partial r} \left [ r^2\frac{\partial \Psi}{\partial r}\right ]  + \frac{L^2}{2Mr^2} \Psi + W(r)\Psi = E\Psi\]
Suppose the eigenfunctions have the factored form
\[\Psi(r,\theta,\phi) = Y_l^m(\theta,\phi) \frac{u(r)}{r}\]
The radial function then satisfies the equation
\[-\frac{1}{2M} \frac{d^2 u(r)}{dr^2} + \left[ \frac{l(l+1)}{2Mr^2} + W(r)\right]u(r) = Eu(r)\]
The radial function must satisfy the boundary condition $u(0) = 0$ since $\Psi(r,\theta,\phi)$ would otherwise have an $r^{-1}$ singularity at the origin. The normalization $\langle \Psi | \Psi \rangle = 1$ implies that
\[\int_0^{\infty} = |u(r)|^2 dr = 1\]

\subsubsection{The hydrogen atom}
The hydrogen atom is a two-particle system consisting of an electron and a proton. The Hamiltonian is
\[H = \frac{P_e^2}{2M_e} + \frac{P_p^2}{2M_p} - \frac{e^2}{4\pi|\bm{Q}_e-\bm{Q}_p|}\]
We take as independent variables the center of mass and relative coordinates of the particles
\[\bm{Q}_c = \frac{M_e\bm{Q}_e + M_p\bm{Q}_p}{M_e+M_p} \quad \bm{Q}_r = \bm{Q}_e-\bm{Q}_p\]
The corresponding momentum operators are
\[\bm{P}_c = \bm{P}_e + \bm{P}_p \quad \bm{P}_r = \frac{M_p\bm{P}_e-M_e\bm{P}_p}{M_e + M_p}\]
We can verify that
\[[Q_{c\alpha},P_{c\beta}] = [Q_{r\alpha},P_{r\beta}] = i\delta_{\alpha\beta} \quad [Q_{c\alpha},P_{r\beta}] = [Q_{r\alpha},P_{c\beta}] = 0\]
The Hamiltonian becomes
\[H = \frac{P_c^2}{2(M_e+M_p)} + \frac{p_r^2}{2\mu} - \frac{e^2}{4\pi|\bm{Q}_r|}\]
where $\mu$ is called the reduced mass, and is defined by $\mu \equiv \frac{M_eM_p}{M_e+M_p}$.
The center of mass behaves as a free particle, and its
motion is not coupled to the relative coordinate. We shall confine our attention to the internal degrees of freedom described by the relative coordinate $\bm{Q}_r$. The energy eigenvalue equation in coordinate representation is
\[-\frac{1}{2\mu}\nabla^2 \Psi(\bm{r})  -\frac{e^2}{4\pi r} \Psi(\bm{r}) = E\Psi(\bm{r})\]
Suppose $\Psi(r,\theta,\phi) = Y_l^m(\theta,\phi) \frac{u(r)}{r}$, we have
\[-\frac{1}{2\mu} \frac{d^2 u(r)}{dr^2} + \left[ \frac{l(l+1)}{2\mu r^2} - \frac{e^2}{4\pi r}\right]u(r) = Eu(r)\]
Define
\[\rho \equiv \alpha r \quad \alpha \equiv \sqrt{8\mu|E|} \quad \lambda \equiv \frac{e^2}{4\pi} \sqrt{\frac{\mu}{2|E|}}\]
we have
\[\frac{d^2 u}{d\rho^2} + \left[ -\frac{1}{4} + \frac{\lambda}{\rho} - \frac{l(l+1)}{\rho^2} \right] u = 0\]
As $\rho \to \infty$, we have $u \sim e^{-\rho/2}$. And as $\rho \to 0$, we have $u \sim \rho^{l+1}$. So, we can suppose
\[u(\rho) = \rho^{l+1} e^{-\rho/2} v(\rho)\]
And we can get
\[\rho \frac{d^2 v}{d\rho^2} + (2l+2-\rho)\frac{dv}{d\rho} + (\lambda-l-1)v = 0\]
It is the so-called \href{http://mathworld.wolfram.com/ConfluentHypergeometricDifferentialEquation.html}{Confluent Hypergeometric Differential Equation}.
When $\lambda - 1- l = n_r$, we have regular solutions. Solutions are \href{https://en.wikipedia.org/wiki/Laguerre_polynomials#Generalized_Laguerre_polynomials}{Associated Laguerre Polynomial}, and will be denoted as $L_{n-l-1}^{2l+1}(\rho) (n=n_r+l+1)$. The energy levels are
\[E_n = -\frac{\mu e^4}{32\pi^2 n^2}\]
The degeneracy of an eigenvalue $E_n$ is
\[\sum_{l=0}^{n-1} (2l+1) = n^2\]
\begin{note}
The degeneracy of an energy level of a hydrogen atom is greater than this by a factor of $4$, which arises from the two-fold orientational degeneracies of the electron and proton spin states. This four-fold degeneracy is modified by the hyperfine interaction between the magnetic moments of the electron and the proton.
\end{note}
The orthonormal energy eigenfunctions for the hydrogen atom are
\[\Psi_{nlm}(r,\theta,\phi) =  \left[ \frac{4(n-l-1)!}{(na_0)^3 n[(n+l)!]^3} \right]^{\frac{1}{2}} \rho^l L_{n-l-1}^{2l+1}(\rho) e^{-\rho/2} Y_l^m (\theta,\phi)\]
where $\rho = \alpha r = \frac{2r}{na_0}$, and $a_0 \equiv \frac{4\pi}{\mu e^2}$ is a characteristic length for the
atom, known as the Bohr radius.The ground state wave
function is
\[\Psi_{000} = (\pi a_0^3)^{-\frac{1}{2}} e^{-\frac{r}{a_0}}\]
A measure of the spatial extent of the bound states of hydrogen is given by the averages of various powers of the distance $r$.
\begin{eqnarray}
\langle r \rangle &=& n^2a_0 \left \{ 1 + \frac{1}{2} \left [ 1 - \frac{l(l+1)}{n^2} \right] \right\} \nonumber \\
\langle r^2 \rangle &=& n^4a_0^2 \left \{ 1 + \frac{3}{2} \left [ 1 - \frac{l(l+1)-1/3}{n^2} \right] \right\} \nonumber \\
\langle \frac{1}{r} \rangle &=& \frac{1}{n^2 a_0} \nonumber
\end{eqnarray}
\end{document}

\chapter{Approximation method}
\section{Time independent perturbation theory}
\subsection{Brillouin-Wigner perturbation theory}
We consider an unperturbed Hamiltonian $H_0$ with eigenvalues $\epsilon_k$ and eigenstates $|k\alpha\rangle$, where $\alpha$ is an index introduced to resolve degeneracies, so that
\[H_0 |k\alpha\rangle = \epsilon_k |k\alpha\rangle\]
We pick one of these levels $\epsilon_n$ for study, so the index $n$ will be fixed for the following discussion. We denote the eigenspace of the unperturbed system corresponding to eigenvalue  $\epsilon_n$ by $\mathcal{H}$, so that the unperturbed eigenkets
$\{ |n\alpha\rangle, \alpha = 1,2,\cdots\}$ form a basis in this space.\\
We take the perturbed Hamiltonian to be $H = H_0 + \lambda H_1$, where $\lambda$ is a formal expansion parameter that we allow to vary between $0$ and $1$ to interpolate between the unperturbed and perturbed system. When the perturbation is turned on, the unperturbed energy level $\epsilon_n$ may split and shift. We denote one of the exact energy levels that grows out of $\epsilon_n$ by $E$. We let $|\psi\rangle$ be an exact energy eigenket corresponding to energy $E$, so that
\[H|\psi\rangle = (H_0 + \lambda H_1)|\psi\rangle = E|\psi\rangle\]
Both $E$ and $|\psi\rangle$ are understood to be functions of $\lambda$; as $\lambda \to 0$, $E$ approaches $\epsilon_n$ and $|\psi\rangle$ approaches some state lying in $\mathcal{H}_n$. We break the Hilbert space into the subspace $\mathcal{H}_n$ and its orthogonal complement which we denote by $\mathcal{H}_n^{\bot}$. The components of $|\psi\rangle$ parallel and perpendicular to $\mathcal{H}_n$ are conveniently expressed in terms of the projector $P$ onto the subspace $\mathcal{H}_n$ and the orthogonal projector $Q$, defined by
\[P = \sum_{\alpha} |n\alpha\rangle \langle n\alpha| \quad Q = \sum_{k \neq n,\alpha} |k\alpha\rangle \langle k\alpha|\]
These projectors satisfy
\[P^2=P \quad Q^2=Q \quad PQ=QP=0 \quad P+Q=I \quad [P,H_0]=[Q,H_0]=0\]
The component $P|\psi\rangle$ is a linear combination of the known unperturbed eigenstates $\{ |n\alpha\rangle, \alpha = 1,2,\cdots\}$, and is easily characterized. 
The orthogonal component $Q|\psi\rangle$ is harder to find. It turns out it is possible to write a neat power series expansion for this solution. Firstly, we have
\[(E-H_0)|\psi\rangle = \lambda H_1 |\psi\rangle\]
Now we define a new operator $R$
\[R \equiv \sum_{k \neq n,\alpha} \frac{|k\alpha\rangle \langle k\alpha|}{E-\epsilon_k}\]
\begin{note}
If there are other unperturbed energy levels $\epsilon_k$ lying close to $\epsilon_n$, then the perturbation could push
the exact energy $E$ near to or past some of these other levels, and then other small denominators would make $R$ ill defined.
This will certainly happen if the perturbation is large enough. For the time being we will assume this does not happen, so that $R$ is free of small denominators. When this is not the case we shall refer to "nearly degenerate perturbation theory", which is discussed later.
\end{note}
\noindent
The operator $R$ satisfies
\[PR=RP=0 \quad QR=RQ=R \quad R(E-H_0) = (E-H_0)R = Q\]
Then we have
\[R(E-H_0)|\psi\rangle = Q|\psi\rangle = \lambda RH_1 |\psi\rangle\]
and
\[|\psi\rangle = P|\psi\rangle + \lambda R H_1 |\psi\rangle\]
$|\psi\rangle$ can be solved as a series of $P|\psi\rangle$:
\[|\psi\rangle = \frac{1}{1-\lambda RH_1} P|\psi\rangle = P|\psi\rangle + \lambda RH_1P|\psi\rangle + \lambda^2 RH_1RH_1 P|\psi\rangle + \cdots\]

\subsection{Nondegenerate perturbation theory}
In nondegenerate perturbation theory the level $\epsilon_n$ of $H_0$ is nondegenerate. Then the index $\alpha$ is not needed for the level $\epsilon_n$, and we can write simply $|n\rangle$ for the corresponding eigenstate. We assume that $P|\psi\rangle$ is normalized rather than $\psi\rangle$ so that
\[P|\psi\rangle  = |n\rangle\]
With this normalization convention, we have
\[\langle n | \psi \rangle = 1\]
Now the series becomes
\[|\psi\rangle = |n\rangle + \lambda \sum_{k\neq n,\alpha} |k\alpha\rangle \frac{\langle k\alpha | H_1 | n \rangle}{E-\epsilon_k} + \lambda^2 \sum_{k\neq n,\alpha} \sum_{k'\neq n,\alpha'} |k\alpha\rangle \frac{\langle k\alpha | H_1 | k'\alpha' \rangle \langle k'\alpha' | H_1 | n \rangle}{(E-\epsilon_k)(E-\epsilon_{k'})}\]
To find an equation for $E$, we have
\[\langle n | E-H_0 | \psi\rangle = E-\epsilon_n = \lambda \langle n | H_1 | \psi\rangle\]
then we can get
\begin{eqnarray}
E &=& \epsilon_n + \lambda \langle n | H_1 | n\rangle + \lambda^2 \langle n | H_1RH_1 | n\rangle + \lambda^3 \langle n | H_1RH_1RH_1 | n\rangle + \cdots \nonumber \\
&=& \epsilon_n 
+ \lambda \langle n | H_1|n\rangle 
+ \lambda^2 \sum_{k\neq n,\alpha}  \frac{\lambda \langle n | H_1|k\alpha\rangle \langle k\alpha | H_1 | n \rangle}{E-\epsilon_k} \nonumber \\
&+& \lambda^3 \sum_{k\neq n,\alpha} \sum_{k'\neq n,\alpha'} \frac{\langle n | H_1 |k\alpha\rangle \langle k\alpha | H_1 | k'\alpha' \rangle \langle k'\alpha' | H_1 | n \rangle}{(E-\epsilon_k)(E-\epsilon_{k'})} + \cdots \nonumber
\end{eqnarray}
It is easy to get $E$ up to $O(\lambda^3)$,
\[E = \epsilon_n  + \lambda \langle n | H_1|n\rangle  + \lambda^2 \sum_{k\neq n,\alpha}  \frac{\lambda \langle n | H_1|k\alpha\rangle \langle k\alpha | H_1 | n \rangle}{\epsilon_n-\epsilon_k} + O(\lambda^3)\]
and $|\psi\rangle$ up to $O(\lambda^2)$,
\[|\psi\rangle = |n\rangle + \lambda \sum_{k\neq n,\alpha} |k\alpha\rangle \frac{\langle k\alpha | H_1 | n \rangle}{\epsilon_n-\epsilon_k} + O(\lambda^2)\]
\href{https://en.wikipedia.org/wiki/Perturbation_theory_(quantum_mechanics)#Second-order_and_higher_corrections}{Higher corrections} can be found on the internet.

\subsection{Degenerate perturbation theory}
In the case that the unperturbed energy level $\epsilon_n$ is degenerate, we have
\[P|\psi\rangle = \sum_{\alpha} |n\alpha\rangle c_{\alpha}\]
and
\[\langle n\alpha | P |\psi\rangle = \langle n\alpha  |\psi\rangle = c_{\alpha}\]
Then we can obtain an equation for the $c_{\alpha}$,
\[\langle n\alpha | E-H_0 | \psi\rangle = c_{\alpha}(E-\epsilon_n) = \lambda \langle n\alpha | H_1 | \psi\rangle\]
then we can get
\begin{eqnarray}
(E-\epsilon_{n})c_{\alpha} &=& \lambda \sum_{\beta} \langle n\alpha | H_1 | n\beta\rangle c_{\beta} + \lambda^2 \sum_{\beta} \langle n\alpha | H_1RH_1 | n\beta\rangle c_{\beta} + \cdots \\
&=& \lambda \sum_{\beta} \langle n\alpha | H_1 | n\beta\rangle c_{\beta}
+ \lambda^2 \sum_{\beta} \sum_{k\neq n,\gamma}  \frac{\lambda \langle n\alpha | H_1|k\gamma\rangle \langle k\gamma | H_1 | n\beta \rangle}{E-\epsilon_k}c_{\beta} + \cdots \nonumber
\end{eqnarray}
This equation must be solved simultaneously for the eigenvalues $E$ and the unknown expansion coefficients $c_{\alpha}$.\\
If we truncate the series at first order, we see that the corrections $E-\epsilon_{n}$ to the energies are determined as the eigenvalues of the matrix $\langle n\alpha | H_1 | n\beta\rangle$, and the coefficients $c_{\alpha}$ are the corresponding eigenvectors.
This determines the energies to first order, but the coefficients $c_{\alpha}$ only to zeroth order. Then $P|\psi\rangle$ becomes known to zeroth order and $Q|\psi\rangle$ to first order.\\
The first order matrix may or may not have degeneracies itself. If it does not, then all degeneracies are lifted at first order; if it does, the remaining degeneracies may be lifted at a higher order, or may persist to all orders. Degeneracies that persist to all orders are almost always due to some symmetry of the system, which can usually be recognized at the outset.\\
The higher order corrections can be calculated step by step, which will not be listed here.\\ \\
Now let us consider the case in which the unperturbed levels of $H_0$ , while not technically degenerate, are close to one another. Suppose to be specific that two levels, say, $\epsilon_n$ and $\epsilon_m$, are close enough to one another that first order perturbations will push the exact level $E$ close to or onto the unperturbed level $\epsilon_m$.\\
In this case we choose some energy, call it $\bar{\epsilon}$, which is close to $\epsilon_n$ and $\epsilon_m$. Then let us take the original unperturbed Hamiltonian and perturbation and rearrange them in the form,
\[H = H_0 + H_1 = H'_0 + H'_1\]
where
\begin{eqnarray}
H_0 &=& \sum_{k\alpha} \epsilon_k |k\alpha\rangle\langle k\alpha| \nonumber \\
H'_0 &=& \sum_{k\neq m,n; \alpha} \epsilon_k |k\alpha\rangle\langle k\alpha| + \sum_{k= m,n; \alpha} \bar{\epsilon} |k\alpha\rangle\langle k\alpha| \nonumber \\
H'_1 &=& H_1 + \sum_{k= m,n; \alpha} (\epsilon_k - \bar{\epsilon} )|k\alpha\rangle\langle k\alpha| \nonumber
\end{eqnarray}
Then standard degenerate perturbation theory may be applied.
We will call this procedure "nearly degenerate perturbation theory."

\section{Application of time independent perturbation theory}
\subsection{Stark effect in hydrogen atom}
The Stark effect concerns the behaviour of atoms in external electric fields. We choose hydrogen atom because it is single-electron atoms.
The hydrogen atom will be modelled with the central force Hamiltonian
\[H_0 = \frac{p^2}{2m} - \frac{e^2}{4\pi r}\]
In this Hamiltonian we ignore spin and other small effects such as relativistic corrections, hyperfine effects and the Lamb shift. These effects cause a splitting and shifting of the energy levels of our simplified model, as well as the introduction of new quantum numbers and new degrees of freedom.
But these effects are all small, and if the applied electric field is strong enough, it will overwhelm them and the physical consequences will be much as we shall describe them with our simplified model. \\
The unperturbed energy levels in hydrogen are given by
\[E_n = -\frac{1}{2n^2} \frac{e^2}{4\pi a_0}\]
where $a_0$ is the Bohr radius. These levels are $n^2$ degenerate.\\
As for the perturbation, let us write $\bm{F}$ for the external electric field , and let us take it to lie in the z-direction. Thus, the perturbing potential has the form
\[V_1 = -(-e)\bm{F}\cdot\bm{x} = eFz\]
For small $z$, the attractive Coulomb field dominates the total potential and we have the usual Coulomb well that supports atomic bound states. However, for large negative $z$, the unperturbed potential goes to zero, while the perturbing
potential becomes large and negative. At intermediate values of negative $z$, the competition between the two potentials gives a maximum in the total potential. The electric force on the electron is zero at the maximum of the potential. 
Given the relative weakness of the applied field, the maximum must occur at a distance from the nucleus that is large in comparison to the Bohr radius $a_0$. Atomic states with small principal quantum numbers $n$ lie well inside this radius. The perturbation analysis we shall perform applies to these states.\\
The bound states of the unperturbed system are able to tunnel through the potential barrier. When an external electric field is turned on, the bound states of the atom cease to be bound in the strict sense, and become resonances. 
Electrons that tunnel through the barrier and emerge into the classically allowed region at large negative $z$ will accelerate in the external field, leaving behind an ion. This is the phenomenon of field ionization. This effect can be neglected if the external field is weak enough and the lifetime of the "bound state" is long enough.\\ \\
In the case of hydrogen, the ground state is $|100\rangle$. The first order shift in the ground state energy level is given by
\[\Delta E^{(1)}_{gnd} = \langle 100 | eFz | 100\rangle = 0\]
which vanishes because the parity of $z$ is odd, but $\langle 100 |$ and $| 100\rangle$ have the same parity.\\
For the excited states of hydrogen, according to first order degenerate perturbation theory, the shifts in the energy levels $E_n$ are given by the eigenvalues of the $n^2\times n^2$ matrix,
\[\langle nlm | eFz | nl'm'\rangle\]
According to the Wigner-Eckart theorem and parity, the matrix elements vanish unless $l -l' = \pm 1$ and $m = m'$. Consider, for example, the case $n=2$. The four degenerate states are  $|2,0,0\rangle$, $|2,1,-1\rangle$, $|2,1,0\rangle$ and $|2,1,1\rangle$. Only the states $|2,0,0\rangle$ and $|2,1,0\rangle$ are connected by the perturbation. Therefore of the $16$ matrix elements, the only nonvanishing ones are
\[\langle 2,0,0 | eFz | 2,1,0\rangle = -W = -3eFa_0\]
and its complex conjugate. The matrix connecting the two states $|2,0,0\rangle$ and $|2,1,0\rangle$ is
\[\begin{pmatrix}0 & -W \\ -W & 0\end{pmatrix} \]
and its eigenvalues are the first order energy shifts in the $n=2$ level,
\[\Delta E_2^{(1)} = \pm W\]
In addition, the two states $|2,1,-1\rangle$ and $|2,1,1\rangle$ do not shift their energies at first order. The perturbed eigenfunctions are
\[|+W\rangle = \frac{|2,0,0\rangle - |2,1,0\rangle}{\sqrt{2}} \quad |-W\rangle = \frac{|2,0,0\rangle + |2,1,0\rangle}{\sqrt{2}}\]
This is zeroth order part of the exact eigenstates.\\
Now let us look at the exact symmetries of the full, perturbed Hamiltonian $ H = H_0 + H_1$, without doing perturbation theory at all. Since $[H,L_z]$ the exact eigenstates of $H$ can be chosen to be eigenstates of $L_z$ as well.
Denote these by $|\gamma m \rangle$, where $\gamma$ is an additional index needed to specify an energy
eigenstate. Thus, we have
\[L_z |\gamma m \rangle = m |\gamma m \rangle \quad H |\gamma m \rangle = E_{\gamma m} |\gamma m \rangle\]
where $E_{\gamma m}$ is allowed to depend on $m$ since the full rotational symmetry is broken. \\
As for time reversal, the state $T|\gamma m\rangle$ must be an eigenstate of energy with eigenvalue $E_{\gamma m}$ since $TH = HT$. But because $T^{-1}L_zT = -L_z$, it also follows that $T|\gamma m\rangle$ is an eigenstate of $L_z$ with eigenvalue $-m$ . If $m \neq 0$, we must have a degeneracy of at least two. 
The only energy levels that can be nondegenerate are those with $m=0$. In the example above, even higher order corrections cannot separate $|2,1,-1\rangle$ and $|2,1,1\rangle$.

\subsection{Fine structure of hydrogen atom}
Fine structure of atoms concerns the effects of relativity and spin on the dynamics of the electron. Both these effects are of the same order of magnitude, and must be treated together in any realistic treatment of the atomic structure.\\
The fine structure terms account for relativistic effects through order $v^2$ , and have the effect of enlarging the Hilbert space by the inclusion of the spin degrees of freedom, introducing new quantum numbers, and shifting and splitting the energy levels of the electrostatic model. 
The splitting in particular means that spectral lines that appear a singlets under low resolution become closely spaced multiplets under higher resolution.\\
Derivation of the exact form of relativistic corrections of Hamiltonian in quantum mechanics can be very rigorous and needs some reasonable guess. The details of derivation can be found in 
\href{http://bohr.physics.berkeley.edu/classes/221/1112/notes/finestruc.pdf}{lecture notes on fine structure} by Robert G. Littlejohn. Here we just list the result.
\[H_{FS} = H_{RKE} + H_{D} + H_{SO}\]
The term $H_{RKE}$ is due to the second order term of the expansion series of $E = \sqrt{p^2+m^2}$. (The first order term is just the kinetic energy in non relativistic quantum mechanics). We have
\[H_{RKE} = - \frac{p^4}{8m^3}\]
The term $H_{D}$ comes out as a result of virtual process $e^{-} \to e^{-} + e^{-} + e^{+}$ in the region whose is scale is smaller than the Compton length $\lambda_C = \frac{1}{m} = \alpha a_0$ of electrons. Such virtual states appear in perturbation theory when one sums over intermediate states, which derive ultimately from a resolution of the identity. The effect is to smear out the position of the atomic electron
over a distance of order $\lambda_C$. We have
\[H_D = \frac{1}{8m^2} \nabla^2 V\]
The term $H_{SO}$ arises because the electric field of nuclei generates a magnetic field in the rest frame of electron. We have
\[H_{SO} = \frac{1}{2m^2} \frac{1}{r} \frac{dV}{dr} \bm{L}\cdot\bm{S}\]
The unperturbed energy levels in hydrogen are given by
\[E_n = -\frac{1}{2n^2} \frac{e^2}{4\pi a_0}\]
When spin of electron is taken into account,  these levels are $2n^2$ degenerate. One choice of base is $|nlm_{l}s\rangle$. It is the eigenvector of operator $L^2$, $L_z$ and $S_z$. However, $L_z$ and $S_z$ do not commute with $H_{SO}$. A better choice of base is $|nljm_j\rangle$. It is the eigenvector of operator $L^2$, $J^2$ and $J_z$. $H_{SO}$, $H_{RKE}$ and $H_{SO}$ are all commute with $L^2$, $J^2$ and $J_z$. So
\[\langle n l' j' m'_j | H | n l j m_j\rangle\]
vanishes unless $l'=l$, $j=j'$ and $m'_j = m_j$. The final results are
\[\langle n l j m_j | H_{RKE} | n l j m_j\rangle = -\alpha^2 E_n \frac{1}{n^2} \left(\frac{3}{4} - \frac{n}{l+\frac{1}{2}} \right)\]
\[\langle n l j m_j | H_{D} | n l j m_j\rangle = -\alpha^2 E_n \frac{1}{n}\delta_{l0}\]
\[\langle n l j m_j | H_{SO} | n l j m_j\rangle = -\alpha^2 E_n \frac{1}{2n} \frac{j(j+1)-l(l+1)-\frac{3}{4}}{l(l+\frac{1}{2})(l+1)}\]
When we add them up to get the total energy shift due to the fine structure we find
\[\Delta E_{FS} = -\alpha^2 E_n \frac{1}{n^2} \left(\frac{3}{4} - \frac{n}{j+\frac{1}{2}} \right)\]
It is independent of the orbital angular momentum quantum number $l$, although each of the individual terms does depend on $l$. However, the total energy shift does depend on $j$ in addition to the principal quantum number $n$, so when we take into account the fine structure corrections, the energy levels of hydrogen atom have the form $E_{nj}$.

\section{Time dependent perturbation theory}
\section{Atomic Radiation}
\section{The classical limit}

\chapter{Many body problem}
\section{Identical particles}
\section{Non-relativistic quantum field theory}

\chapter{Scattering theory}
\section{Lippmann–Schwinger equation}
Imagine a particle coming in and getting scattered by a short-ranged potential $V(x)$ located around the origin $x \sim 0$. The time-independent Schr\"{o}dinger equation is simply
\[(H_0 + V)|\psi\rangle = E |\psi\rangle\]
Here, $H_0 = \frac{p^2}{2m}$ is the free-particle Hamiltonian operator. We can write the solution as
\[|\psi^{(\pm)}\rangle = \frac{1}{E-H_0 \pm i\epsilon}V|\psi^{(\pm)}\rangle + |\phi\rangle\]
Here, $H_0 |\phi\rangle = E |\phi\rangle$. In coordinate representation,
\[\psi^{(\pm)}(\mathbf{x}) = \phi(\mathbf{x}) + \int d^3x' \langle \mathbf{x} | \frac{1}{E-H_0 \pm i\epsilon} | \mathbf{x}' \rangle V(\mathbf{x}') \psi^{(\pm)}(\mathbf{x}')\]
Here, $\phi(\mathbf{x}) = \frac{e^{i\mathbf{k}\cdot\mathbf{x}}}{(2\pi)^{\frac{3}{2}}}$. Define the Green function as
\[G_{\pm}(\mathbf{x},\mathbf{x}') \equiv \frac{1}{2m} \langle \mathbf{x} | \frac{1}{E-H_0 \pm i\epsilon} | \mathbf{x}' \rangle\]
We can derive that
\[G_{\pm}(\mathbf{x},\mathbf{x}') = -\frac{1}{4\pi} \frac{e^{\pm ik|\mathbf{x}-\mathbf{x}'|}}{|\mathbf{x}-\mathbf{x}'|}\]
where $k = \sqrt{2mE}$. And it is easy to show that
\[(\nabla^2 + k^2)G_{\pm}(\mathbf{x},\mathbf{x}') = \delta(\mathbf{x}-\mathbf{x}')\]
So, we have
\[\psi^{(\pm)}(\mathbf{x}) = \frac{e^{i\mathbf{k}\cdot\mathbf{x}}}{(2\pi)^{\frac{3}{2}}} - 2m \int d^3x' \frac{1}{4\pi} \frac{e^{\pm ik|\mathbf{x}-\mathbf{x}'|}}{|\mathbf{x}-\mathbf{x}'|} V(\mathbf{x}') \psi^{(\pm)}(\mathbf{x}')\]
We now can interpret $\psi^{+}(\mathbf{x})$ as a superposition of incident plane wave and scattered wave which propagate from scatterer to outside region. From now on, we will denote it as $\psi(\mathbf{x})$.

The experiment is done typically by placing the detector far away from the scatterer $|\mathbf{x}| \ll a$ where $a$ is the "size" of the scatterer. The integration over $\mathbf{x}'$, on the other hand, is limited within the "size" of the scatterer because of the $V(\mathbf{x}')$ factor. Therefore, we are in the situation $|\mathbf{x}| \ll |\mathbf{x}'|$, and hence can use the approximation
\[|\mathbf{x}-\mathbf{x}'| \approx |\mathbf{x}| - \frac{\mathbf{x}' \cdot \mathbf{x}}{|\mathbf{x}|}\]
Under this limit,
\[\psi(\mathbf{x}) = \frac{e^{i\mathbf{k}\cdot\mathbf{x}}}{(2\pi)^{\frac{3}{2}}} - 2m \frac{e^{ikr}}{4\pi r} \int d^3x' e^{-\mathbf{k}' \cdot \mathbf{x}'} V(\mathbf{x}') \psi(\mathbf{x}')\]
Here, $r = |\mathbf{x}|$ and $\mathbf{k}' = k \frac{\mathbf{x}}{r}$. It is customary to write this equation in the form
\[\psi(\mathbf{x}) = \frac{1}{(2\pi)^{\frac{3}{2}}}\left( e^{i\mathbf{k}\cdot\mathbf{x}} +  f(\mathbf{k},\mathbf{k}') \frac{e^{ikr}}{r} \right) \]
Here,
\[f(\mathbf{k},\mathbf{k}') \equiv - \frac{m}{2\pi} (2\pi)^3  \langle \mathbf{k}'| V | \psi\rangle \]
Recall the definition of cross section
\[\sigma \equiv \frac{\mbox{Number of Events}}{\mbox{Time} \times \mbox{Incident Flux}}\]
So, the differential cross section for particles being scattered into the solid angle is
\[d\sigma = \frac{|\mathbf{j}_{\mathrm{scatt}}| r^2 d\Omega}{|\mathbf{j}_{\mathrm{inc}}|} = |f(\mathbf{k},\mathbf{k}')|^2 d\Omega\]

In a more realistic situation, we should use wave packets to describe the scattering process. The basic picture is a free wave packet approaches the scattering center. After a long time, we have both the original wave packet moving in the original direction plus a spherical wave front that moves outward. The details can be found in the section 3 of the lecture notes 
\href{http://hitoshi.berkeley.edu/221B/index.html}{\emph{Scattering Theory I (Hitoshi Murayama)}}.

Furthermore, if we require that the normalization of the wave function should always satisfy $\int dx^3 |\psi(\mathbf{x})|^2$ for any $t$, as guaranteed by the unitarity of time evolution operator. This requirement leads to a special requirement on the scattered wave, and hence $f(\mathbf{k},\mathbf{k}')$, from witch we can derive the optical theorem.

\begin{newthem}[Optical theorem]
\[\mathrm{Im} f(\theta = 0) = \frac{k\sigma_{\mathrm{tot}}}{4\pi}\]
where
\[f(\theta = 0) \equiv f(\mathbf{k},\mathbf{k}),\]
the setting of $\mathbf{k} \equiv \mathbf{k}'$ imposes scattering in the forward direction, and
\[\sigma_{\mathrm{tot}} = \int \frac{d\sigma}{d\Omega} d\Omega\]
\end{newthem}
The meaning of this theorem is clear. Because the scattered wave takes the probability away to different directions, the total probability for the particle to go to the forward direction (unscattered) should decrease. This decrease is caused by the interference between the unscattered and scattered waves and hence is proportional to $f(0)$. On the other hand, the amount of decrease in the forward direction should equal the total probability at other directions, which is proportional to the total cross section. The proof can be found in the section 4 of the lecture notes \href{http://hitoshi.berkeley.edu/221B/index.html}{\emph{Scattering Theory I (Hitoshi Murayama)}}.

\section{Born approximation}
If $|\psi\rangle = |\phi\rangle + O(V)$ is close to $|\phi\rangle$, we can solve the Lippmanmn-Schwinger equation by perturbation theory. The lowest order approximation in $V$ is
\[|\psi\rangle = \frac{1}{E-H_0 + i\epsilon} V|\phi\rangle + |\phi\rangle\]
This is called Born approximation. In coordinate representation,
\[f^{(1)}(\mathbf{k},\mathbf{k}') = - \frac{m}{2\pi} \int d^3x V(\mathbf{x}) e^{i\mathbf{q}\cdot\mathbf{x}}\]
Here, $\mathbf{q} = |\mathbf{k} - \mathbf{k}'|$. If the potential is central, we can derive that
\[f^{(1)}(\mathbf{k},\mathbf{k}') = - \frac{2m}{q} \int_0^{\infty} dr \: r V(r) \sin(qr)\]

\subsubsection{Yukawa potential}
\[V = \frac{\alpha}{r}  e^{-\mu r}\]
So, we can derive
\[f(\theta) = - \frac{2m\alpha}{q^2 + \mu^2}\]
Different cross section is therefore given by
\[\frac{d\sigma}{d\Omega} = (2m\alpha)^2 \frac{1}{[2k^2(1-\cos\theta) + \mu^2]^2}\]
The total cross section is obtained by integrating over $d\Omega$,
\[\sigma = (2m\alpha)^2 \frac{4\pi}{4k^2\mu^2 + \mu^4}\]

\subsubsection{Coulomb potential}
\[V = \frac{\alpha}{r}\]
Take the limit $\mu \to 0$, we can get
\[f(\theta) = - \frac{2m\alpha}{q^2}\]
Different cross section is given by
\[\frac{d\sigma}{d\Omega} = (\frac{\alpha}{4E})^2 \frac{1}{\sin^4{\frac{\theta}{2}}}\]
The total cross section diverges. The divergence is in the $\cos\theta$ integral when $\theta \to 0$. In other words, the divergence occurs for the small momentum transfer $q \to 0$, which corresponds to large distances.
The reason why the total cross section diverges is because the Coulomb potential is actually a long-range force. No matter how far the incident particles are from the charge, there is always an effect on the motion of the particles and they get scattered.

\subsubsection{Form factor}
\noindent
If the source of Coulomb potential has an distribution $\rho_N(\mathbf{x})$, then
\[V(\mathbf{x}) = \int d^3x \frac{\alpha}{|\mathbf{x}-\mathbf{x}'|} \rho(\mathbf{x}')\]
Note that the potential is mathematically a convolution of the Coulomb potential and the probability density. Since the first Born amplitude is nothing but the Fourier transform of the potential, the convolution becomes a product of Fourier transforms, one for the Coulomb potential and the other for the probability density. So
\[f(\theta) = f(\theta)_{\mathrm{pointlike}} F(q)\]
Here,
\[F(q) \equiv \int d^3x \rho_N(\mathbf{x}) e^{i \mathbf{q} \cdot \mathbf{x}},\]
being called form factor.

\subsubsection{Born expansion}
\noindent
Define T-matrix by
\[V | \psi \rangle = T |\phi\rangle\]
Using the definition of the T-matrix, we find
\[f(\mathbf{k},\mathbf{k}') = - \frac{m}{2\pi} (2\pi)^3  \langle \mathbf{k}'| T | \mathbf{k}\rangle \]
Using the Lippmann–Schwinger equation and multiplying the
both sides by $V$ from left, we find
\[ T |\phi\rangle = V \frac{1}{E-H_0 + i\epsilon}T|\phi\rangle + V|\phi\rangle\]
A formal solution to the T-matrix is
\[T = \frac{1}{1-V\frac{1}{E-H_0 + i\epsilon}}V\]
By Taylor expanding this operator in geometric series, we find
\[T = V + V \frac{1}{E-H_0 + i\epsilon} V + V \frac{1}{E-H_0 + i\epsilon} V \frac{1}{E-H_0 + i\epsilon} V + \cdots\]
So,
\[|\psi\rangle = \left( 1 +  \frac{1}{E-H_0 + i\epsilon} V +  \frac{1}{E-H_0 + i\epsilon} V \frac{1}{E-H_0 + i\epsilon} V + \cdots \right) | \phi \rangle\]
The first term is the wave which did not get scattered.
The second term is the wave that gets scattered at a point in the potential and then propagates outwards by the propagator. 
In the third term, the wave gets scattered at a point in the potential, propagates for a while, and gets scattered again at another point in the potential, and propagates outwards. 
In the $n+1$-th term, there are $n$ times scattering of the wave before it propagates outwards.

\section{Partial wave analysis}
\subsubsection{Partial wave expansion}
When the potential is \textbf{central}, angular momentum is conserved due to Noether's theorem. Therefore, we can expand the wave function in the eigenstates of the angular momentum. Obtained waves with definite angular momenta are called partial waves. We can solve the scattering problem for each partial wave separately, and then in the end put them together to obtain the full scattering amplitude.
The plane wave can be expanded as follows.
\[e^{ikz} = \sum_{l=0}^{\infty}(2l+1)i^l j_l(kr) P_l(\cos \theta)\]
Here, $j_l(kr)$ is spherical Bessel functions of first kind. The asymptotic behaviour of $j_l(kr)$ at large $r$ can be written as
\[j_l(kr) \sim \frac{\sin(kr-\frac{l\pi}{2})}{kr}\]
so,
\[e^{ikz} \sim \frac{1}{2ikr} \sum_{l=0}^{\infty} (2l+1) (e^{ikr} - (-1)^l e^{-ikr})P_l(\cos \theta)\]
Meanwhile, the $f$ factor can be expanded as
\[f(\theta) = \sum_{l=0}^{\infty} f_l (2l+1)P_l(\cos \theta)\]

\subsubsection{Optical theorem constraint}
\noindent
The cross section can be represented by expansion coefficient of $f$ factor as
\[\sigma = 4\pi \sum_l (2l+1)|f_l|^2\]
On the other hand, 
\[\mathrm{Im} f(0) = \sum_l (2l+1) \mathrm{Im} f_l\]
From optical theorem we can derive that
\[|f_l|^2 = \frac{1}{k} \mathrm{Im} f_l\]
This constraint can be rewritten as
\[|1+2ikf_l|^2 = 1\]
So we can define a phase $\delta_l$ as 
\[1+2ikf_l = e^{i\delta_l}\]
or equivalently,
\[f_l = \frac{1}{k} e^{i\delta_l} \sin(\delta_l)\]

\subsubsection{Phase shifts}
\noindent
We can derive the asymptotic behaviour of the wave function as
\[\psi(\mathbf{x}) \sim \frac{1}{2ikr} \sum_{l} (2l+1)P_l(\cos \theta) [e^{ikr}e^{2i\delta_l} - (-1)^l e^{-ikr}]\]
Compare it to the case of the plane wave without scattering. What this equation says is that the wave converging on the scatterer
has the well-defined phase factor $-(-1)^l$, the same as in the case without scattering. On the other hand, the wave that emerges from the scatterer has an additional phase factor $e^{2i\delta_l}$. All what scattering did is to shift the phase of the emerging wave by $2\delta_l$. The reason why this is merely a phase factor is
the conservation of probability. What converged to the origin must come out with the same strength. But this shift in the phase causes the interference among all partial waves different from the case without the phase shifts, and the result is not a plane wave but contains the scattered wave.\\
In terms of the phase shifts, the cross section is given by
\[\sigma = \frac{4\pi}{k^2} \sum_l (2l+1) \sin^2\delta_l\]
Actual calculation of phase shifts is basically to solve the Schr\"{o}dinger equation for each partial waves,
\[\left[-\frac{1}{r}\frac{d^2}{dr^2}r+\frac{l(l+1)}{r^2}+2mV(r)\right]R_l(r) = k^2 R_l(r)\]
After solving the equation, we take the asymptotic limit $r \to \infty$, and write $R_l(r)$ as a linear combination of $j_l(kr)\cos \delta_l + n_l(kr) \sin \delta_l $. The relative coefficients of $j_l$ and $n_l$ determines the phase shift $\delta_l$, and hence the cross section.
\part{Quantum Field Theory}
\include{QFT1}
\chapter{Vector Field}
\section{Vector field}
Consider a vector field $A^{\mu}(x)$. Here the index $\mu$ is a vector index that takes on four possible values. Under a Lorentz transformation, we have
\[U(\Lambda)^{-1} A^{\mu}(x) U(\Lambda) = \Lambda^{\mu}_{\phantom{\mu}\nu} A^{\nu}(\Lambda^{-1}x)\]
For an infinitesimal transformation, we can write
\[\delta^{\mu}_{\phantom{\mu}\nu}+\delta \omega ^{\mu}_{\phantom{\mu}\nu} = \delta^{\mu}_{\phantom{\mu}\nu} + \frac{i}{2} \delta \omega_{\rho \sigma} (S_V^{\rho \sigma})^{\mu}_{\phantom{\mu}\nu}\]
Here
\[(S_V^{\rho \sigma})^{\mu}_{\phantom{\mu}\nu} = -i(\eta^{\rho \mu}\delta ^{\sigma}_{\phantom{\sigma}\nu} - \eta^{\sigma \mu}\delta^{\rho}_{\phantom{\rho}\nu})\]
It is obvious that $A^{\dagger \mu}$ is also a vector field. 
We know that $\eta^{\mu \nu}$ is invariant under Lorentz transformation, i.e.
\[\Lambda^{\mu}_{\phantom{\mu}\rho} \Lambda^{\nu}_{\phantom{\mu}\sigma} \eta^{\rho \sigma} = \eta^{\mu \nu} \]
We can use $\eta^{\mu \nu}$ and and its inverse $\eta_{\mu\nu}$ to raise and lower vector indices of the vector field,
\[A_{\mu} \equiv \eta_{\mu \nu} A^{\nu}\]
And we can verify the following equations
\[\Lambda^{\mu}_{\phantom{\mu}\nu} \Lambda_{\mu}^{\phantom{\mu}\rho} = \delta^{\rho}_{\nu} \]
\[A^{\mu}(x) = \eta^{\mu \nu} A_{\nu}(x)\]
\[\Lambda_{\mu}^{\phantom{\mu}\rho} \Lambda_{\nu}^{\phantom{\nu}\sigma} \eta_{\rho \sigma} = \eta_{\mu \nu}\]
\[U(\Lambda)^{-1} A_{\mu}(x) U(\Lambda) = \Lambda_{\mu}^{\phantom{\mu}\nu} A_{\nu}(\Lambda^{-1}x)\]
Define $C_i \equiv \frac{1}{2}\epsilon_{ijk}S_V^{jk}$,$D_i \equiv S_V^{i0}$. For example, we have
\[(C_3)_{\mu}^{\phantom{\mu}\nu} = \left(\begin{array}{rrrr}
0 & 0 & 0 & 0 \\
0 & 0 & -i & 0 \\
0 & i & 0 & 0 \\
0 & 0 & 0 & 0
\end{array}\right)\]
The eigenvectors of $C_3$ are
\[\left[\left(-1, \left[\left(0,\,1,\,-i,\,0\right)\right], 1\right),
\left(1, \left[\left(0,\,1,\,i,\,0\right)\right], 1\right), \left(0,
\left[\left(1,\,0,\,0,\,0\right), \left(0,\,0,\,0,\,1\right)\right],
2\right)\right]\]
We further define $N_i \equiv \frac{1}{2}(C_i-iD_i)$ and $N^{\dagger}_i \equiv \frac{1}{2}(C_i + i D_i)$. For example, we have
\[(N_1)_{\mu}^{\phantom{\mu}\nu} = \left(\begin{array}{rrrr}
0 & -\frac{1}{2} & 0 & 0 \\
-\frac{1}{2} & 0 & 0 & 0 \\
0 & 0 & 0 & -\frac{1}{2} i \\
0 & 0 & \frac{1}{2} i & 0
\end{array}\right)\]
The eigenvectors of $N_1$ are
\[\left[\left(-\frac{1}{2}, \left[\left(1,\,1,\,0,\,0\right),
\left(0,\,0,\,1,\,-i\right)\right], 2\right), \left(\frac{1}{2},
\left[\left(1,\,-1,\,0,\,0\right), \left(0,\,0,\,1,\,i\right)\right],
2\right)\right]\]
And we can conclude that vector is in the $(2,2)$ representation of the Lie algebra of the Lorentz group.

\section{Electromagnetic field and gauge invariance}
\noindent
The Lagrangian of EM field is
\[\mathcal{L} = -\frac{1}{4}F_{\mu\nu}F^{\mu\nu}\]
Here,
\[F_{\mu\nu} = \partial_{\mu} A_{\nu} - \partial_{\nu} A_{\mu} \quad \mbox{and} \quad A^{\mu} = (\phi,\bm{A})\]
So,
\[F_{0i} = \dot{A}^i + \nabla_i \phi \equiv -E^i \quad \mbox{and} \quad F_{ij} = \nabla_i A^j - \nabla_j A^i \equiv \epsilon_{ijk}B^k\]
We can derive the equation of motion of the EM field by variation method,
\[\partial_{\mu}F^{\mu \nu} = 0\]
It can be rewritten in terms of $\bm{E}$ and $\bm{B}$, i.e. Maxwell equations:
\begin{eqnarray}
&\phantom{=}&\bm{\nabla} \cdot \bm{E} = 0 \quad \frac{\partial \bm{E}}{\partial t} = \bm{\nabla} \times \bm{B} \nonumber \\
&\phantom{=}& \bm{\nabla} \cdot \bm{B} = 0  \quad \frac{\partial \bm{B}}{\partial t} = - \bm{\nabla} \times \bm{E}\nonumber
\end{eqnarray}

The massless vector field $A_{\mu}$ has 4 components, which would naively seem to tell us that the gauge field has 4 degrees of freedom.But there are two related comments which will ensure that quantizing the gauge field $A_{\mu}$ gives rise to 2 degrees of freedom, rather than 4.
\begin{itemize}
\item The field $A_0$ has no kinetic term $\dot{A_0}$ in the Lagrangian: it is not dynamical. This means that if we are given some initial data $A_i$ and $\dot{A_i}$ at a time $t_0$, then the field $A_0$ is fully determined by the equation of motion $\bm{\nabla} \cdot \bm{E} = 0$,which, expanding out,
reads
\[\nabla^2 A_0 = \bm{\nabla} \cdot \frac{\partial \bm{A}}{\partial t}\]
So $A_0$ is not independent: we don't get to specify $A_0$ on the initial time slice.
\item If we transform the EM field as
\[A_{\mu} \to A_{\mu} + \partial_{\mu}\lambda(x) \]
we can derive that
\[F_{\mu\nu} \to F_{\mu \nu} \quad \mathcal{L} \to \mathcal{L}\]
The seemed infinite number of symmetries, one for each function $\lambda(x)$, is to be viewed as a redundancy in our description. That is, two states related by a gauge symmetry are to be identified: they are the same physical state. One way to see that this interpretation is necessary is to notice that Maxwell’s equations are not sufficient to specify the evolution of $A_{\mu}$.The equations read,
\[(\eta_{\mu\nu} \partial^2 - \partial_{\mu} \partial_{\nu}) A^{\nu} = 0\]
But the operator $(\eta_{\mu\nu} \partial^2 - \partial_{\mu} \partial_{\nu})$ is not invertible: it annihilates any function of
the form $\partial_{\mu} \lambda$. This means that given any initial data, we have no way to uniquely determine $A_{\mu}$ at a later time since we can't distinguish between $A_{\mu}$ and $A_{\mu} + \partial_{\mu} \lambda$. This would be problematic if we thought that $A_{\mu}$ is a physical object. However, if we're happy to identify $A_{\mu}$ and $A_{\mu} + \partial_{\mu} \lambda$ as corresponding to the same physical state, then our problems disappear. 
\end{itemize}

The picture that emerges for the theory of electromagnetism is of an enlarged phase space, foliated by gauge orbits. All states that lie along a given gauge orbit can be reached by a gauge transformation and are identified. To make progress, we pick a representative from each gauge orbit. It doesn't matter which representative we pick after all, they're all physically equivalent. But we should make sure that we pick a "good" gauge, in which we cut the orbits. Here we'll look at two different gauges:
\begin{itemize}
\item Coulomb Gauge: $\bm{\nabla} \cdot \bm{A} = 0$\\
We can make use of the residual gauge transformations in Lorentz gauge to pick $\bm{\nabla} \cdot \dot{\bm{A}} = 0$. We
have as a consequence $A_0 = 0$. Coulomb gauge is sometimes called radiation gauge.
\item Lorentz Gauge: $\partial^{\mu} A_{\mu} = 0$\\
In fact this condition doesn't pick a unique representative from the gauge orbit. We're always free to make further gauge transformations with $\partial^{\mu}\partial_{\mu} \lambda = 0$, which also has non-trivial solutions. As the name suggests, the Lorentz gauge has the advantage that it is Lorentz invariant.
\end{itemize}

\section{Canonical quantization of EM field}
\subsection{Canonical quantization in Coulomb gauge}

\subsubsection{Canonical momentum and Hamiltonian}
\[\pi^0 = \frac{\partial \mathcal{L}}{\partial \dot{A_0}} = 0 \quad  \pi^{i} = \frac{\partial \mathcal{L}}{\partial(\partial_0 A_i)} = \dot{A}^i + \nabla_i \phi = -E^i\]
\[\mathcal{H} = \frac{1}{2}(\bm{\pi}^2 + \bm{B}^2) + (\bm{\pi} \cdot \bm{\nabla}) A_0\]
Integration by parts can give
\[H = \int d^3x \frac{1}{2}(\bm{\pi}^2 + \bm{B}^2)\]

\subsubsection{Momentum and angular momentum}
\[P^0 = H \quad \vec{P} = \int - \bm{\pi} \vec{\nabla} \bm{A} d^3x\]
\[\vec{J} = - \int \bm{\pi} (\vec{x}\times \vec{\nabla} + i \vec{C})\bm{A} \; d^3x \quad \vec{S} = -i \int \bm{\pi} \vec{C} \bm{A} \; d^3x\]

\subsubsection{Canonical quantization}
\noindent
In Coulomb gauge, we have
\[A_0 = \pi^0 = 0 \quad \pi^i = \dot{A}^i \]
Three pairs of $A_i$ and $\pi^i$ are not independent from each other. They must satisfy the constraint equations
\[\nabla \cdot \bm{A} = 0 \quad \nabla \cdot \bm{\pi} = 0\]
A reasonable quantization condition can be written as
\[[A_i(\bm{x},t),A_j(\bm{x}',t)] = 0 \quad [\pi^i(\bm{x},t),\pi^j(\bm{x}',t)] = 0\]
\[[A_i(\bm{x},t),\pi^j(\bm{x}',t)] = i \left( \delta^{j}_i - \frac{\partial_i \partial^j}{\nabla^2} \right) \delta(\bm{x}-\bm{x}') \equiv i \int \frac{d^3k}{(2\pi)^3} \; (\delta^j_i - \frac{k_ik^j}{\bm{k}^2})e^{i\bm{k}\cdot(\bm{x}-\bm{x}')}\]
In this case, we can verify that
\[\dot{A}_i = -i[A_i(\bm{x},t),H] = \pi_i (\bm{x},t)\]
\[\dot{\pi}^i = -i[\pi^i(\bm{x},t),H] = \nabla^2 A^i(\bm{x},t)\]
It is constant with the field equation we derive from Euler-Lagrange equation.

\subsubsection{Fourier expansion}
\[\bm{A}(x) = \sum_{r = \pm} \int \widetilde{dp} [a_{r}(\bm{p}) \bm{\epsilon}_r(\bm{p})e^{ipx} + a^{\dagger}_{r}(\bm{p}) \bm{\epsilon}^*_r(\bm{p})e^{-ipx}]\]
And we can derive from constraint condition that
\[\bm{\epsilon} \cdot \bm{p} = 0\]
We will choose $\bm{\epsilon}$ to satisfy that
\[\bm{\epsilon}_r \cdot \bm{\epsilon}^*_s = \delta_{rs}\]
So, the completeness relation for the polarization vectors is
\[\sum_{r=\pm} \epsilon_r^i(\bm{p}) \epsilon_r^{*j}(\bm{p}) = \delta^{ij} - \frac{p^ip^j}{|\bm{p}|^2}\]
\begin{example}
If $\bm{p} = (0,0,p)$, we usually choose 
\[\bm{\epsilon}_{+} = \frac{1}{\sqrt{2}}(1,i,0) \quad \bm{\epsilon}_{-} = \frac{1}{\sqrt{2}}(1,-i,0) \]
$\bm{\epsilon}_{+}$ corresponds to left-handed rotation and it is the eigenvectors of the space-part of $C_3$ with eigenvalue $+1$. $\bm{\epsilon}_{-}$ corresponds to right-handed rotation and it is eigenvector of the space-part of $C_3$ with eigenvalue $- 1$.
\end{example}
\noindent
We can further derive from above discussion that
\[\bm{\pi}(x) = -i \sum_{r = \pm} \int \widetilde{dp} \omega [a_{r}(\bm{p}) \bm{\epsilon}_r(\bm{p})e^{ipx} - a^{\dagger}_{r}(\bm{p}) \bm{\epsilon}^*_r(\bm{p})e^{-ipx}]\]
\[a_r(\bm{p}) = \bm{\epsilon}^*_r \int d^3x e^{-ikx}(i\bm{\pi}+\omega\bm{A})\]
\[a^{\dagger}_r(\bm{p}) = \bm{\epsilon}_r \int d^3x e^{ikx}(-i\bm{\pi}+\omega\bm{A})\]
\[[a_r(\bm{p}),a_{r'}(\bm{p'})] = 0 \quad [a^{\dagger}_r(\bm{p}),a^{\dagger}_{r'}(\bm{p'})] = 0 \quad [a_r(\bm{p}),a^{\dagger}_{r'}(\bm{p'})] = (2\pi)^3 2\omega \delta_{rr'} \delta(\bm{p} - \bm{p}')\]

\subsubsection{Operator represented by $a$ and $a^{\dagger}$}
\noindent
Define that
\[N(\bm{p},r) \equiv a^{\dagger}_{r}(\bm{p}) a_r(\bm{p})\]
So, we can derive
\[H = \sum_{r = \pm} \int \widetilde{dp} \; \omega N(\bm{p},r) + 2\mathcal{E}_0V\]
\[\vec{P} = \sum_{r = \pm} \int \widetilde{dp} \; \vec{p} N(\bm{p},r) \]
\[\vec{S} = \sum_{r,s = \pm} \int \widetilde{dp} \; \frac{1}{2}(\bm{\epsilon}^*_{s} \vec{C}\bm{\epsilon}_{r} - \bm{\epsilon}_{r} \vec{C}\bm{\epsilon}^*_{s}) a^{\dagger}_{s}(\bm{p}) a_r(\bm{p})\]
From above equation, we can say that $a^{\dagger}_r(\bm{p})$ create an photon with energy $\omega$, momentum $\bm{p}$ and spin angular momentum along the direction of momentum $r$.

\subsubsection{Propagator}
\[G_F(x-y)_{ij} \equiv \langle 0 |T A_i(x) A_j(y) | 0 \rangle = \int \frac{d^4p}{(2\pi)^4} \frac{-i}{p^2-i\epsilon} \left(\delta_{ij} - \frac{p_ip_j}{|\bm{p}|^2}\right) e^{ip(x-y)}\]

\subsection{Canonical quantization in Lorentz gauge}
\subsubsection{Undefined metric formalism}
\noindent
Modify the Maxwell Lagrangian introducing a new term
\[\mathcal{L} = -\frac{1}{4} F_{\mu\nu}F^{\mu\nu} - \frac{1}{2\xi} (\partial_{\mu} A^{\mu})^2\]
The equations of motion are now
\[\partial^2 A_{\mu} - (1-\frac{1}{\xi})\partial^{\mu}(\partial \cdot A) = 0\]
Canonical momentums are
\[\pi^0 = \frac{1}{\xi} \partial \cdot A = \frac{1}{\xi}(-\dot{A}_0 + \partial_i A^i) \quad \pi^i = \dot{A}^i + \nabla^i A^0 = -E^i\]
Hamiltonian is
\[\mathcal{H} = \frac{1}{2}(\bm{\pi}^2 + \bm{B}^2 - \xi \pi^0 \pi^0) + (\bm{\pi} \cdot \bm{\nabla}) A_0 + \pi^0 (\bm{\nabla} \cdot \bm{A})\]
\[H =  \int \left[ \frac{1}{2}(\bm{\pi}^2 + \bm{B}^2 - \xi \pi^0 \pi^0) -A_0(\bm{\nabla} \cdot \bm{\pi}) + \pi^0 (\bm{\nabla} \cdot \bm{A}) \right] d^3x \]
We remark that the above Lagrangian and the equations of motion, reduce to Maxwell theory in the gauge $\partial \cdot A = 0$. This why we say that our choice corresponds to a class of Lorenz gauges with parameter $\xi$. With this abuse of language (in fact we are not setting $\partial \cdot A = 0$, otherwise the problems would come back) the value of $\xi=1$ is known as the Feynman gauge and $\xi=0$ as the Landau gauge. From now on we will take the case of the so-called Feynman gauge, where $\xi=1$. Then the equation of motion coincide with the Maxwell theory in the Lorenz gauge. In Feymann gauge,the canonical quantization conditions can be written as
\[[A_{\mu}(\bm{x},t),A_{\nu}(\bm{x}',t)] = 0 \quad [\pi^{\mu}(\bm{x},t),\pi^{\nu}(\bm{x}',t)] = 0 \quad [A_{\mu}(\bm{x},t),\pi^{\nu}(\bm{x}',t)] = i\delta^{\nu}_{\mu} \delta(\bm{x}-\bm{x}')\]
we can also derive that
\[[\dot{A}_{\mu}(\bm{x},t),\dot{A}_{\nu}(\bm{x}',t)] = 0 \quad [A_{\mu}(\bm{x},t),\dot{A}_{\nu}(\bm{x}',t)] = i\eta_{\mu\nu} \delta(\bm{x}-\bm{x}')\]

\subsubsection{Fourier expansion}
\[A(x) = \sum_{\lambda=0}^{3} \int \widetilde{dp} [a_{\lambda}(\bm{p}) \epsilon_{\lambda}(\bm{p})e^{ipx} + a^{\dagger}_{\lambda}(\bm{p}) \epsilon^*_{\lambda}(\bm{p})e^{-ipx}]\]
where $\epsilon_{\lambda \mu}$ are a set of four independent 4-vectors.  We will now make a choice for these 4-vectors. We choose $\epsilon_{1\mu}$ and $\epsilon_{2\mu}$ orthogonal to $k^{\mu}$ and $n^{\mu}$, such that
\[\epsilon_{\lambda \mu} \epsilon^{*\mu}_{\lambda} = \delta_{\lambda \lambda'} \quad \lambda,\lambda' = 1,2\]
After, we choose $\epsilon_{3\mu}$ in the plane $(k^{\mu},n^{\mu})$ and perpendicular to $n^{\mu}$ such that
\[\epsilon_{3\mu} n^{\mu} = 0 \quad \epsilon_{3\mu} \epsilon^{*\mu}_{3} = 1\]
Finally we choose $\epsilon_{0\mu} = n_{\mu}$. The vectors $\epsilon_{1\mu}$ and $\epsilon_{2\mu}$ are called transverse polarizations, while $\epsilon_{3\mu}$ and $\epsilon_{0\mu}$ longitudinal and scalar polarizations, respectively.\\
In general we can show that
\[\epsilon_{\lambda} \cdot \epsilon^*_{\lambda'} = \eta_{\lambda \lambda'} \quad \eta^{\lambda \lambda'} \epsilon_{\lambda \mu} \epsilon^*_{\lambda' \nu} = \eta_{\mu \nu} \]
We can further derive from above discussion that
\[\dot{A}(x) = -i \sum_{\lambda=0}^{3} \int \widetilde{dp} \omega [a_{\lambda}(\bm{p}) \epsilon_{\lambda}(\bm{p})e^{ipx} - a^{\dagger}_{\lambda}(\bm{p}) \epsilon^*_{\lambda}(\bm{p})e^{-ipx}]\]
\[ a_{\lambda}(\bm{p}) =  \eta_{\lambda \lambda'} \epsilon^*_{\lambda'} \cdot \int d^3x e^{-ipx}(i\dot{A}+\omega A)\]
\[ a^{\dagger}_{\lambda}(\bm{p}) =  \eta_{\lambda \lambda'} \epsilon_{\lambda'} \cdot \int d^3x e^{ipx}(-i\dot{A}+\omega A)\]
\[[a_{\lambda}(\bm{p}),a_{\lambda'}(\bm{p'})] = 0 \quad [a^{\dagger}_{\lambda}(\bm{p}),a^{\dagger}_{\lambda'}(\bm{p'})] = 0 \quad [a_{\lambda}(\bm{p}),a^{\dagger}_{\lambda'}(\bm{p'})] = (2\pi)^3 2\omega \eta_{\lambda \lambda'} \delta(\bm{p} - \bm{p}')\]

\subsubsection{Indefinite metric problem}
We Introduce the vacuum state defined by
\[a_{\lambda}(\bm{p}) | 0 \rangle = 0\]
To see the problem with the sign we construct the one-particle state with scalar polarization, that is
\[|1\rangle = \int \widetilde{dp} a^{\dagger}_{0}(\bm{p})|0\rangle
\]
and calculate its norm
\[\langle 1 | 1 \rangle = -\langle 0 | 0 \rangle \int \widetilde{dp} |f(p)|^2\]
The state $| 1 \rangle$ has a negative norm.

To solve this problem we note that we are not working anymore with the classical Maxwell theory because we modified the Lagrangian. What we would like to do is to impose the condition $\partial \cdot A = 0$, but that is impossible as an equation for operators. We can, however, require that condition on a weaker form, as a condition only to be verified by the physical states.

More specifically, we require that the part of $\partial \cdot A$ that contains the annihilation operator (positive frequencies) annihilates the physical states,
\[\partial^{\mu} A^{+}_{\mu} | \psi \rangle = 0\]
The states $| \psi \rangle$ can be written in the form
\[| \psi \rangle = | \psi_T \rangle | \phi \rangle\]
where $| \psi_T \rangle$  is obtained from the vacuum with creation operators with transverse polarization and $| \phi \rangle$ with scalar and longitudinal polarization.

$\partial^{\mu} A^{+}_{\mu}$ contains only scalar and longitudinal polarizations
\[\partial^{\mu} A^{+}_{\mu} = i\sum_{\lambda=0,3} \int \widetilde{dp} a_{\lambda}(\bm{p}) (p \cdot \epsilon_{\lambda}(\bm{p}) ) e^{ipx} \]
Therefore the previous condition becomes
\[i\sum_{\lambda=0,3} (p \cdot \epsilon_{\lambda}(\bm{p})) a_{\lambda}(\bm{p})  | \phi \rangle = 0\]
The condition is equivalent to,
\[(a_{0}(\bm{p}) - a_{3}(\bm{p})) | \phi \rangle = 0\]
We can construct $| \phi \rangle$ as a linear combination of states $| \phi \rangle$ with $n$ scalar or longitudinal photons:
\[| \phi \rangle = C_0 | \phi_0 \rangle + C_1 | \phi \rangle + \cdots \quad \mbox{Here,} | \phi_0 \rangle \equiv | 0 \rangle\]
The states $|\phi_n\rangle$ are eigenstates of the operator number for scalar or longitudinal photons
\[N' | \phi_n \rangle = n | \phi_n \rangle\]
where,
\[N' = \int \widetilde{dp} [a^{\dagger}_{3}(\bm{p})a_{3}(\bm{p})-a^{\dagger}_{0}(\bm{p})a_{0}(\bm{p})] \]
Then
\[n \langle \phi_n | \phi_n \rangle = \langle \phi_n |N'| \phi_n \rangle = 0\]
This means that
\[\langle \phi_n | \phi_n \rangle = \delta_{n0}\]
that is, for $n \neq 0$, the state $| \phi_n \rangle$ has zero norm. We have then for the general state $| \phi \rangle$,
\[\langle \phi | \phi \rangle = |C_0|^2 \geq 0\]
and the coefficients $C_i(i=1,2,\cdots)$ are arbitrary.

\subsubsection{Operator represented by $a$ and $a^{\dagger}$}
\noindent
Define that
\[N'(\bm{p}) \equiv a^{\dagger}_{3}(\bm{p})a_{3}(\bm{p})-a^{\dagger}_{0}(\bm{p})a_{0}(\bm{p})\]
\[N(\bm{p},1) \equiv a^{\dagger}_{1}(\bm{p}) a_{1}(\bm{p}) \quad N(\bm{p},2) \equiv a^{\dagger}_{2}(\bm{p}) a_{2}(\bm{p}) \quad N_T(\bm{p}) \equiv N(\bm{p},1) + N(\bm{p},2)\]
We have that
\[\langle \psi | N'(\bm{p}) | \psi\rangle = 0 \quad \langle \psi | N_T(\bm{p}) | \psi\rangle = \langle \psi_T | N_T(\bm{p}) | \psi_T\rangle\]
We can derive
\[ H = \int \widetilde{dp} \; \omega [N'(\bm{p}) + N_T(\bm{p})] + 2\mathcal{E}_0V\]
\[ \vec{P} = \int \widetilde{dp} \; \vec{p} [N'(\bm{p}) + N_T(\bm{p})]\]

So, the arbitrariness of $C_i(i=1,2,\cdots)$ does not affect the physical observables. Only the physical transverse polarizations
contribute to the result. Two states that differ only in their timelike and longitudinal photon content, $|\phi_n\rangle$ with $n \geq 1$ are said to be physically equivalent. We can think of the gauge symmetry of the classical theory as descending to the Hilbert space of the quantum theory.

It is important to note that although for the average values of the physical observables only the transverse polarizations contribute, the scalar and longitudinal polarizations are necessary for the consistency of the theory. In particular they show up when we consider complete sums over the intermediate states.

\subsubsection{Propagator}
\[G_F(x-y)_{\mu\nu} \equiv \langle 0 |T A_{\mu}(x) A_{\nu}(y) | 0 \rangle = \int \frac{d^4p}{(2\pi)^4} \frac{-i\eta_{\mu\nu}}{p^2-i\epsilon}  e^{ip(x-y)}\]
It is easy to verify that $G_F(x-y)_{\mu\nu}$ is the Green's function of the equation of motion, that for $\xi=1$ is the wave equation, that is
\[\partial^2 G_F(x-y)_{\mu\nu} = i\eta_{\mu\nu}\delta(x-y)\]
For the general case, $\xi \neq 0$, the equal times commutation relations are more complicated. And the propagator will be
\[G_F(x-y)_{\mu\nu}  = \int \frac{d^4p}{(2\pi)^4} \left[\frac{-i\eta_{\mu\nu}}{p^2-i\epsilon} + i(1-\xi)\frac{p_{\mu}p_{\nu}}{(p^2-i\epsilon)^2}\right] e^{ip(x-y)}\]

\section{Perturbation theory for canonical quantization}
\subsection{Lagrangian of QED}
\noindent
The Lagrangian of QED is
\[\mathcal{L} = -\frac{1}{4}F_{\mu\nu}F^{\mu\nu} + \overline{\Psi} (i\slashed{\partial}-m_0) \Psi + e_0j^{\mu} A_{\mu}, \]
where $j^{\mu} \equiv \overline{\Psi}\gamma^{\mu}\Psi$. Usually, we also define a covariant derivative,
\[D_{\mu}\Psi \equiv \partial_{\mu}\Psi -ie_0A_{\mu}\Psi \]
So, the Lagrangian can also be written as
\[\mathcal{L} = -\frac{1}{4}F_{\mu\nu}F^{\mu\nu} + \overline{\Psi} (i\slashed{D}-m_0) \Psi\]
The Lagrangian is invariant under the gauge transformation
\[A_{\mu}(x) \to A_{\mu}(x) + \frac{1}{e_0}\partial_{\mu}\alpha(x) \quad \Psi(x) \to e^{i\alpha(x)} \Psi(x)\]

\subsection{Coulomb gauge}
\noindent
The constraint equations are
\[\bm{\nabla} \cdot \bm{A} = 0 \quad \bm{\nabla}^2 A_0 = e_0j^0\]
The solution for $A_0$ is
\[A_0(\bm{x},t) = -e_0 \int d^3 x' \frac{j^0(\bm{x}',t)}{4\pi|\bm{x}-\bm{x}'|}\]
Finally, we can derive that
\[H = H_{D} + H_{M} + H_{\mathrm{int}}\]
where,
\[H_{D} = \int d^3 x \; -\Pi (\vec{\alpha} \cdot \vec{\nabla} + i \beta m)\Psi \quad H_{M} = \int d^3x \; \frac{1}{2}(\bm{\pi}^2 + \bm{B}^2)\]
\[H_{\mathrm{int}} = \int d^3x \; \left[-e_0\bm{j}\cdot\bm{A} + \frac{e_0^2}{2} \int d^3x' \frac{j^0(\bm{x}) j^0(\bm{x}')}{4\pi|\bm{x}-\bm{x}'|} \right]\]
The perturbation expansion of correlation functions is
\[\langle \Omega | T \{\Psi(x) \overline{\Psi}(y) A(z) \} | \Omega \rangle = \lim_{T \to \infty(1-i\epsilon)} \frac{\langle 0 | T \left\{ \Psi_I(x) \overline{\Psi}_I(y) A_I(z) \mathrm{exp} \left[ -i \int_{-T}^{T} dt H_I \right]\right\} | 0 \rangle}{\langle 0 | T \left\{ \mathrm{exp} \left[ -i \int_{-T}^{T} dt H_I \right]\right\} | 0 \rangle}\]
Here, $\int_{T}^{T} dt H_I$ can be written as
\[\left[-\int d^4x e_0\overline{\Psi}_I\bm{\gamma}\Psi_I\cdot\bm{A}_I \right] +\left[ \int d^4x \int d^4x' \frac{e_0^2\delta(t-t')}{4\pi|\bm{x}-\bm{x}'|}\; \frac{1}{2} \overline{\Psi}_I(\bm{x},t)\gamma^0 \Psi_I(\bm{x},t) \overline{\Psi}_I(\bm{x}',t')\gamma^0 \Psi_I(\bm{x}',t') \right]\]
Wick’s theorem for photons takes is similar to that in $\phi^4$ theory:
\[T \left\{ A_I(x_1) A_I(x_2)  A_I(x_3) \cdots \right\} = N \left\{A_I(x_1) A_I(x_2)  A_I(x_3) \cdots + \mbox{ all possible contractions }\right\} \]
\begin{example}
\begin{eqnarray}
&& \langle 0 | T \left\{ A_{Ii}(x_1) A_{Ij}(x_2) A_{Ik}(x_3) A_{Il}(x_4)\right\}| 0 \rangle \nonumber \\
&=&  G_F(x_1-x_2)_{ij} G_F(x_3-x_4)_{kl} + G_F(x_1-x_3)_{ik}G_F(x_2-x_4)_{jl} +  G_F(x_1-x_4)_{il}G_F(x_2-x_3)_{jk} \nonumber
\end{eqnarray}
\end{example}
\noindent
Now we can derive the Feynman rule for QED theory.
Firstly, we evaluate this term,
\[\langle 0 | T \left\{ \Psi_{Ia}(x) \overline{\Psi}_{Ib}(y) A_{Ii}(z) (i e_0) \int dw^4 \overline{\Psi}_I(w) \bm{\gamma} \Psi_I(w) \cdot A_I(w) \right\} | 0 \rangle\]
After contraction, it can be written as
\begin{eqnarray}
&-&  (ie_0) S_F(x-y)_{ab} \int d^4 w G_F(z-w)_{ik} \mathrm{Tr}[\gamma^k S_F(w-w)] \nonumber \\
&+&  (ie_0) \int d^4 w G_F(w-z)_{ik}  [S_F(x-w) \gamma^k S_F(w-y)]_{ab}  \nonumber
\end{eqnarray}
It can be represented by the following Feynman diagram.
\begin{figure}[!h]
\centering
\includegraphics[height=2cm ,width=5.4cm]{QFT/QEDFD1.png}
\caption{Feynman diagram representation of perturbation expansion}
\end{figure}\\
Secondly, we evaluate this term,
\[\langle 0 | T \left\{ \Psi_{Ia}(x) \overline{\Psi}_{Ib}(y) \int d^4w \int d^4w' \frac{-ie_0^2\delta(w^0-w'^0)}{4\pi|\bm{w}-\bm{w}'|}\; \frac{1}{2} \overline{\Psi}_I(w)\gamma^0 \Psi_I(w) \overline{\Psi}_I(w')\gamma^0 \Psi_I(w') \right\} | 0 \rangle\]
After contraction, it can be written as
\begin{eqnarray}
&-&  (ie_0)^2 \int d^4w d^4w' \frac{i\delta(w^0-w'^0)}{4\pi|\bm{w}-\bm{w}'|} [S_F(x-w)\gamma^0 S_F(w-y)]_{ab} \mathrm{Tr}[\gamma^0 S_F(w'-w')] \nonumber \\
&+&  (ie_0)^2 \int d^4w d^4w' \frac{i\delta(w^0-w'^0)}{4\pi|\bm{w}-\bm{w}'|} [S_F(x-w)\gamma^0 S_F(w-w') \gamma^0 S_F(w'-y)]_{ab} \nonumber \\
&+& \frac{1}{2} (ie_0)^2 S_F(x-y)_{ab} \int d^4w d^4w' \frac{i\delta(w^0-w'^0)}{4\pi|\bm{w}-\bm{w}'|} \mathrm{Tr}[\gamma^0 S_F(w-w)] \mathrm{Tr}[\gamma^0 S_F(w'-w')] \nonumber \\
&-& \frac{1}{2} (ie_0)^2 S_F(x-y)_{ab} \int d^4w d^4w' \frac{i\delta(w^0-w'^0)}{4\pi|\bm{w}-\bm{w}'|} \mathrm{Tr}[\gamma^0 S_F(w-w')\gamma^0 S_F(w'-w)] \nonumber
\end{eqnarray}
It can be represented by the following Feynman diagram.
\begin{figure}[!h]
\centering
\includegraphics[height=4.5cm ,width=6.85cm]{QFT/QEDFD2.png}
\caption{Feynman diagram representation of perturbation expansion}
\end{figure}\\
Before we write down Feymann rules, we notice that the offending non-local interaction comes from the $A_0$ component of the gauge field, we could try to redefine the propagator to include a $G_F(x-y)_{00}$ piece which will capture this term. We can verify that
\[\frac{i\delta(w^0-w'^0)}{4\pi|\bm{w}-\bm{w}'|} = \int \frac{d^4p}{(2\pi)^4} \frac{ie^{ip(w-w')}}{|\bm{p}|^2}\]
So we can combine the non-local interaction with the transverse photon propagator by defining a new photon propagator
\[G_F(p)_{\mu\nu} \equiv \begin{cases} \frac{i}{|\bm{p}|^2} \quad \mu,\nu=0\\  \frac{-i}{p^2-i\epsilon} \left(\delta_{ij} - \frac{p_ip_j}{|\bm{p}|^2}\right) \quad \mu = i \neq 0, \nu = j \neq 0 \\ 0 \quad \mbox{otherwise} \end{cases} \]
With this propagator, the wavy photon line now carries a $\mu \nu = 0,1,2,3$ index, with the extra $\mu=0$ component taking care of the instantaneous interaction.\\
The Feynman rules for QED are:
\begin{enumerate}
\item For each Fermion propagator from $y$ to $x$, $P = S_F(x-y)$
\item For each vector propagator, $P = G_F(x-y)$
\item For each vertex, $V = (ie_0\gamma^{\mu})\int d^4w$
\item For each external point, $E=1$
\item Divided by the symmetry factor
\end{enumerate}

\subsection{Lorentz Gauge}
\noindent
In Lorentz gauge,
\[\mathcal{H}_{int} = -e_0 \overline{\Psi} \gamma^{\mu} \Psi A_{\mu}\]
The Feynman rules for QED in Lorentz gauge will be the same as that in Coulomb gauge expect for that the vector propagator will be
\[G_F(p)_{\mu\nu}  = \frac{-i\eta_{\mu\nu}}{p^2-i\epsilon} + i(1-\xi)\frac{p_{\mu}p_{\nu}}{(p^2-i\epsilon)^2} \]
Especially, for Feynman gauge $\xi=1$, we have
\[G_F(p)_{\mu\nu}  = \frac{-i\eta_{\mu\nu}}{p^2-i\epsilon} \]

\section{Path integral quantization}
\subsection{Path integral formulation for free EM field}
\noindent
The correlation function is given by
\[\langle \Omega | T A_H(x_1) A_H(x_2)| \Omega \rangle = \lim_{T \to \infty(1-i\epsilon)} \frac{\int \mathcal{D}A \; \mathrm{exp} \left[ i\int_{-T}^T d^4x (-\frac{1}{4} F^{\mu\nu}F_{\mu\nu}) \right] A(x_1) A(x_2)}{\int \mathcal{D}A \; \mathrm{exp} \left[ i\int_{-T}^T d^4x (-\frac{1}{4} F^{\mu\nu}F_{\mu\nu}) \right]}\]
The generating function is 
\[Z[J] = \int \mathcal{D}A \; \mathrm{exp} \left[ i\int d^4x (-\frac{1}{4} F^{\mu\nu}F_{\mu\nu}) + J^{\mu} A_{\mu} \right]\]
We can verify that
\[S = \int d^4x (-\frac{1}{4} F^{\mu\nu}F_{\mu\nu}) = \frac{1}{2} \int d^4x A_{\mu}(x) (\partial^2\eta^{\mu\nu} - \partial^{\mu}\partial^{\nu})A_{\nu}(x)\]
Notice that $(\partial^2\eta^{\mu\nu} - \partial^{\mu}\partial^{\nu})$ is singular, since for any $\alpha(x)$, 
\[(\partial^2\eta^{\mu\nu} - \partial^{\mu}\partial^{\nu})\partial_{\mu}\alpha(x) = 0\]
This difficulty is due to gauge invariance: $\alpha(x)$ is gauge equivalent to $0$. The functional is badly defined because we are redundantly integrating over a continuous infinity of physically equivalent field configurations. To fix the problem, we would like to isolate the interesting part of the functional integral, which counts each physical configuration only once. \\
Let $G(A)$ be some function that we wish to set equal zero as a gauge-fixing condition. We could constrain the functional integral to cover only the configurations with $G(A) = 0$ by inserting a functional delta function, $\delta(G(A))$. To do so, we insert $1$ in the path integral:
\[ 1 = \int \mathcal{D}\alpha(x) \delta(G(A(\alpha))) \det \left( \frac{\delta G}{\delta \alpha} \right)\]
where,
\[A_{\mu}(\alpha(x)) = A_{\mu}(x) + \frac{1}{e_0}\partial_{\mu}\alpha(x)\]
We set the gauge fixing function as $G(A) = \partial^{\mu} A_{\mu} -\omega(x)$, so $G(A(\alpha)) = \partial^{\mu} A_{\mu} + \frac{1}{e_0}\partial^2 \alpha - \omega(x)$. It is obvious that $\det \left( \frac{\delta G}{\delta \alpha} \right)$ is equivalent to $\det(\partial^2)/e_0$, which is independent of $A$. So,
\[Z_0[A] = \det \left( \frac{\delta G}{\delta \alpha} \right) \int \mathcal{D}\alpha \int \mathcal{D}A e^{iS[A]} \delta(G(A(\alpha)))\]
Now change variables from $A$ to $A(\alpha)$. This is a simple shift, so $\mathcal{D}A = \mathcal{D}A(\alpha)$. Also, by gauge invariance, $S[A] = S[A(\alpha)]$. Since $A(\alpha)$ is now just a dummy integration variable, we can rename it bake to $A$, so
\[Z_0[A] = \det \left( \frac{\delta G}{\delta \alpha} \right) \int \mathcal{D}\alpha \int \mathcal{D}A e^{iS[A]} \delta(\partial^{\mu}A_{\mu} - \omega(x))\]
Since the above equation is hold for any $\omega(x)$,so we have
\begin{eqnarray}
Z_0[A] &=& N(\xi) \int \mathcal{D}\omega \exp\left[ -i \int d^4x \frac{\omega^2}{2\xi} \right] \det(\partial^2) \int \mathcal{D}\alpha \int \mathcal{D}A e^{iS[A]} \delta(\partial^{\mu}A_{\mu} - \omega(x)) \nonumber \\
&=& N(\xi) \frac{\det(\partial^2)}{e_0} \int \mathcal{D}\alpha \int \mathcal{D}A e^{iS[A]} \exp\left[ -i \int d^4x \frac{1}{2\xi} (\partial^{\mu}A_{\mu})^2\right] \nonumber \\
&=& W(\xi) \int \mathcal{D}A \exp\left[ \frac{i}{2} \int d^4x  A_{\mu} (\partial^2\eta^{\mu\nu} - \partial^{\mu}\partial^{\nu} + \frac{1}{\xi} \partial^{\mu} \partial^{\nu})A_{\nu}\right] \nonumber
\end{eqnarray}
And we rewrite the generating function as
\[Z[J] = W(\xi) \int \mathcal{D}A \exp\left[ \frac{i}{2} \int d^4x  A_{\mu} (\partial^2\eta^{\mu\nu} - \partial^{\mu}\partial^{\nu} + \frac{1}{\xi} \partial^{\mu} \partial^{\nu})A_{\nu} + J^{\mu} A_{\mu}\right]\]
Define
\[A'(x) = A(x) - i \int d^4y G_F(x-y)J(y)\]
Recall that
\[(\partial^2\eta^{\mu\nu} - \partial^{\mu}\partial^{\nu} + \frac{1}{\xi} \partial^{\mu} \partial^{\nu})G_F(x-y)_{\nu \rho} = i\delta^{\mu}_{\rho}\delta(x-y)\]
We can derive that
\[Z[J]=Z_0[A] \exp \left[ -\frac{1}{2} \int d^4x d^4y J^{\mu}(x) G_F(x-y)_{\mu\nu}J^{\nu}(y) \right]\]
The two point correlation functions are
\[\langle 0 | T A_H(x_1) A_H(x_2)| 0 \rangle = Z_0^{-1} \left(-i \frac{\delta}{\delta J(x_1)} \right) \left(-i \frac{\delta}{\delta J(x_2)} \right) Z[J]\bigg|_{J=0} = G_F(x_1-x_2) \]

\subsection{Perturbation theory for path integral quantization}
\noindent
We use QED theory as an example.
\[\mathcal{L} = -\frac{1}{4}F_{\mu\nu}F^{\mu\nu} + \overline{\Psi} (i\slashed{\partial}-m_0) \Psi + e_0\overline{\Psi}\gamma^{\mu}\Psi A_{\mu}\]
As we stated in section 4.1, the Lagrangian of QED is also invariant under a general gauge transformation. And we also notice that the measure $\mathcal{D}\Psi \mathcal{D}\overline{\Psi}$ is invariant under gauge transformation. By the similar method, we can show that
\[Z_0 \equiv \int \mathcal{D}A \mathcal{D}\Psi \mathcal{D}\overline{\Psi}e^{iS[A,\Psi,\overline{\Psi}]} = W(\xi)\int \mathcal{D}A \mathcal{D}\Psi \mathcal{D}\overline{\Psi}e^{iS[A,\Psi,\overline{\Psi}]} \exp\left[ -i \int d^4x \frac{1}{2\xi} (\partial^{\mu}A_{\mu})^2\right] \]
So,define
\[\mathcal{L}_0 \equiv -\frac{1}{4}F_{\mu\nu}F^{\mu\nu} - \frac{1}{2\xi} (\partial^{\mu}A_{\mu})^2 + \overline{\Psi} (i\slashed{\partial}-m_0) \Psi \quad \mathcal{L}_1 \equiv e_0\overline{\Psi}\gamma^{\mu}\Psi A_{\mu}\]
\begin{eqnarray}
Z[J] &=& W(\xi) \int \mathcal{D}A \mathcal{D}\overline{\Psi} \mathcal{D}\Psi e^{i\int d^4x [\mathcal{L}_0 + \mathcal{L}_1 + JA + \overline{\eta}\Psi + \overline{\Psi}\eta]} \nonumber \\
&=& W(\xi) e^{i\int d^4x \mathcal{L}_1(\frac{1}{i} \frac{\delta}{\delta J(x)},\frac{1}{i}\frac{\delta}{\delta \overline{\eta}(x)},-\frac{1}{i}\frac{\delta}{\delta \eta(x)})} \int \mathcal{D}\phi \mathcal{D}\overline{\Psi} \mathcal{D}A e^{i\int d^4y [\mathcal{L}_0 + JA + \overline{\eta}\Psi + \overline{\Psi}\eta]} \nonumber \\
&\propto & e^{i\int d^4x \mathcal{L}_1(\frac{1}{i} \frac{\delta}{\delta J(x)},\frac{1}{i}\frac{\delta}{\delta \overline{\eta}(x)},-\frac{1}{i}\frac{\delta}{\delta \eta(x)})} \mathrm{exp} [- \int d^4y d^4z  \frac{1}{2} J(y)G_F(y-z)J(z) + \overline{\eta}(y)S_F(y-z)\eta(z)] \nonumber \\
& =& \sum_{V=0}^{\infty} \frac{1}{V!} [ie_0 \int d^4x (\frac{1}{i} \frac{\delta}{\delta J^{\mu}(x)} \cdot -\frac{1}{i}\frac{\delta}{\delta \eta(x)} \cdot \gamma^{\mu} \cdot  \frac{1}{i}\frac{\delta}{\delta \overline{\eta}(x)})]^V \nonumber \\
&\times & \sum_{P_1=0}^{\infty} \frac{1}{P_1!} [-\frac{1}{2} \int d^4y_1 d^4z_1 J(y_1)G_F(y_1-z_1)J(z_1)]^{P_1} \nonumber \\
&\times &  \sum_{P_2=0}^{\infty} \frac{1}{P_2!} [-\int d^4y_2 d^4z_2 \overline{\eta}(y_2)S_F(y_2-z_2)\eta(z_2)]^{P_2} \nonumber
\end{eqnarray}
If we focus on a term with particular values of $V$, $P_1$ and $P_2$, then the number of surviving scalar sources is $E_1 = 2P_1-V$, the number of surviving fermion sources is $E_2 = 2P_2-2V$.
We can introduce Feynman diagrams as in the $\phi^4$ theory. In these diagrams, a wavy line segment stands for a vector propagator $G_F(x-y)$, a line with an arrow pointing from $y$ to $x$ for a fermion propagator $S_F(x-y)$, a filled circle at one end of a wavy line segment for a vector source $i\int d^4x J(x)$, a filled circle at the start of a line with an arrow for a fermion source $i\int d^4x \eta(x)$, a filled circle at the end of a line with an arrow for a anti-fermion source $i\int d^4x \overline{\eta}(x)$, a vertex joining three line segments for $ie_0\gamma^{\mu}\int d^4x$.

\subsection{Ward-Takahahsi identity (1)}
The Noether current of the symmetry $\Psi \to e^{i\alpha}\Psi$ is $j^{\mu} = \overline{\Psi}\gamma^{\mu}\Psi$. Recall the conservation law in functional formalism
\[\langle \partial_{\mu} j^{\mu}(x) \phi(x_1)\cdots\phi(x_n) \rangle  = \sum_{i=1}^{n} \langle \phi(x_1) \cdots (i\Delta \phi(x_i)\delta(x-x_i)) \cdots \phi(x_n) \rangle\]
So, we can write the charge conservation law as
\[ie_0\partial_{\mu} \langle \Omega | T j^{\mu} \Psi(x_1) \overline{\Psi}(x_2)| \Omega\rangle = -ie_0\delta(x-x_1)\langle \Omega | T \Psi(x_1) \overline{\Psi}(x_2)| \Omega\rangle + ie_0\delta(x-x_2)\langle \Omega | T \Psi(x_1) \overline{\Psi}(x_2)| \Omega\rangle\]
Notice that $i\langle \Omega | T j^{\mu} \psi(x_1) \overline{\Psi}(x_2)| \Omega\rangle$ can be represented by
\begin{figure}[!h]
\centering
\includegraphics[height=2cm ,width=4.46cm]{QFT/WDTA1.png}
\caption{Feynman diagram representation of correlation function}
\end{figure}\\
From the diagram, we have
\[\langle A_{\nu}(y) \rangle = \int d^4x G_F(x-y)_{\mu\nu} ie\langle j^{\mu}(x) \rangle  = \int \frac{d^4p}{(2\pi)^4} e^{-ipy} G_F(p)_{\mu\nu} \int d^4x e^{ipx} ie\langle j^{\mu}(x) \rangle\]
So,
\[\int d^4x \langle A_{\nu}(x) \rangle e^{ikx} = G_F(p)_{\mu\nu} \int d^4x e^{ikx} ie\langle j^{\mu}(x)\rangle\]
Compute the Fourier transformation of the equation of charge conservation by integrating
\[\int d^4x e^{ikx} \int d^4x_1 e^{-iqx_1} \int d^4x_2 e^{ipx_2}\]
We can get
\begin{figure}[!h]
\centering
\includegraphics[height=1.67cm ,width=11.36cm]{QFT/WDTA2.png}
\caption{Feynman diagram representation of Ward identity}
\end{figure}\\
Note that in the diagram above, the external leg of photon will be cut-off, but external leg of fermion will remain.
The above equation can be generated to the diagram with $n$ external fermions. 
%\begin{figure}[!h]
%\centering
%\includegraphics[height=1.87cm ,width=11.63cm]{QFT/WDTA3.png}
%\caption{Feynman diagram representation of Ward identity}
%\end{figure}\
Another proof of Ward-Takahashi identity by calculating the Feynman diagram directly can be found in chapter 7.4 of \emph{An introduction to quantum field theory (M.E.Peskin \& D.V.Schroeder)}

\section{Exact propagator of photon}
\subsection{Photon self-energy}
\noindent
The exact propagator of photon is
\[\mathcal{G}(x)_{\mu\nu} = \langle  \Omega | T \{ A_{\mu}(x)A_{\nu}(0)| \Omega \rangle_C\]
Its Fourier transformation can be represented by the following diagram. 
\begin{figure}[!h]
\centering
\includegraphics[height=1.08cm ,width=13.65cm]{QFT/QEDRG1.png}
\caption{Feynman diagram representation of exact propagator of photon}
\end{figure}\\
Let us define $i\Pi^{\mu\nu}$ to be the sum of all 1-particle-irreducible insertions into the photon propagator. So, we have
\[\mathcal{G}(k) = G_F(k) + G_F(k)(i\Pi(k))G_F(k) + \cdots = G_F(k) \frac{1}{1-i\Pi(k)G_F(k)}\]
Hence,
\[(i\mathcal{G})^{-1} = (iG_F)^{-1} - \Pi\]
Recall that
\[iG_F(p)_{\mu\nu}  = \frac{\eta_{\mu\nu}}{k^2-i\epsilon} - (1-\xi)\frac{k_{\mu}k_{\nu}}{(k^2-i\epsilon)^2} = \frac{1}{k^2-i\epsilon}(P^T_{\mu\nu} + \xi P^L_{\mu\nu})\]
Here, \[P^T_{\mu\nu} \equiv \eta_{\mu\nu} - \frac{k_{\mu}k_{\nu}}{k^2} \quad  P^L_{\mu\nu} \equiv \frac{k_{\mu}k_{\nu}}{k^2}\]
We can derive that
\[(iG_F)^{-1}(k)_{\mu\nu} = k^2 (P^T_{\mu\nu} + \frac{1}{\xi} P^L_{\mu\nu})\]
We may also expand $i\Pi^{\mu\nu}$ as
\[\Pi^{\mu\nu} = P_T^{\mu\nu}f_T(k^2) +  P_L^{\mu\nu}f_L(k^2) = \eta^{\mu\nu}f_T + \frac{k^{\mu}k^{\nu}}{k^2}(f_L-f_T)\]
Therefore,
\[(i\mathcal{G})^{-1}(k)_{\mu\nu} = (k^2-f_T(k^2))P^T_{\mu\nu} + (\frac{k^2}{\xi}-f_L(k^2)) P^L_{\mu\nu}\]
\[\mathcal{G}(k)_{\mu\nu} = \frac{-i}{k^2-f_T(k^2)}P^T_{\mu\nu} + \frac{-i}{\frac{k^2}{\xi}-f_L(k^2)} P^L_{\mu\nu}\]
We observe that if $f_{T,L}(k^2 = 0) \neq 0$, a mass will be generated for the photon. Because $\Pi(k)$ comes from 1PI diagrams, it should not be singular at $k^2 =0 $, and so $f_L - f_T = O(k^2)$, as $k \to 0$. We will show that gauge invariance ensures that no mass is generated from the loop corrections

\subsection{Ward identities(2)}
\noindent
We define the generating functional for connected diagrams
\[Z[J,\eta,\overline{\eta}] = e^{-iE[J,\eta,\overline{\eta}]}\]
For example,
\[\mathcal{G}(x-y)_{\mu\nu} = -i  \frac{\delta^2 E[J,\eta,\overline{\eta}]}{\delta J^{\mu}(x) \delta J^{\nu}(y)}\bigg|_{J,\eta,\overline{\eta}=0}\]
Recall, for infinitesimal gauge transformations, $\delta A_{\mu} = \partial_{\mu} \lambda $, $\delta \Psi = ie_0\lambda\Psi$ and $\delta \overline{\Psi}  = -ie_0 \lambda \overline{\Psi}$. For a change of
variables in the path integral, $Z[J,\eta,\overline{\eta}]$ will remain the same. 
Recall that
\[Z[J,\eta,\overline{\eta}] = W(\xi) \int \mathcal{D}A \mathcal{D}\overline{\Psi} \mathcal{D}\Psi e^{i\int d^4x [\mathcal{L}_0 + \mathcal{L}_1 + JA + \overline{\eta}\Psi + \overline{\Psi}\eta]} \]
The change of action is
\[\delta S = -\frac{1}{\xi} \int d^4x \partial_{\mu} A^{\mu} \partial^2 \lambda + \int d^4x J^{\mu}\partial_{\mu}\lambda + ie_0\overline{\eta}\Psi\lambda - ie_0\overline{\Psi}\eta\lambda\]
Hence, we must have
\[\int d^4x \lambda(x) W(\xi)\int \mathcal{D}A \mathcal{D}\overline{\Psi} \mathcal{D}\Psi e^{iS} \left[ -\frac{1}{\xi} \partial^2 \partial_{\mu} A^{\mu} - \partial_{\mu}J^{\mu}  + ie_0(\overline{\eta}\Psi - \overline{\Psi}\eta)\right] \]
Since
\[\langle A_{\mu}(x) \rangle_{J,\eta,\overline{\eta}} = - \frac{\delta E}{\delta J^{\mu}} \quad \langle \Psi(x) \rangle_{J,\eta,\overline{\eta}} = - \frac{\delta E}{\delta \overline{\eta}} \quad \langle \overline{\Psi}(x) \rangle_{J,\eta,\overline{\eta}} =  \frac{\delta E}{\delta \eta}\]
The above equation can be written as
\[\frac{1}{\xi} \partial^2 \partial^{\mu}\frac{\delta E}{\delta J^{\mu}} - \partial_{\mu}J^{\mu} - ie_0\left[ \overline{\eta}\frac{\delta E}{\delta \overline{\eta}} + \frac{\delta E}{\delta \eta} \eta \right]=0\]
We can derive that
\[\frac{1}{\xi} \partial^2 \partial^{\mu} \frac{\delta^2 E[J,\eta,\overline{\eta}]}{\delta J^{\mu}(x) \delta J^{\nu}(y)}\bigg|_{J,\eta,\overline{\eta}=0} - \partial_{\nu} \delta(x-y) = 0\]
that is,
\[\frac{i}{\xi}\partial^2 \partial^{\mu} \mathcal{G}(x-y)_{\mu\nu}+ \partial_{\nu} \delta(x-y) = 0 \]
or, written in momentum-space,
\[-\frac{i}{\xi}k^2 k^{\mu} \mathcal{G}(k)_{\mu\nu}+ k_{\nu} = 0\]
So
\[- \frac{k^2}{k^2-\xi f_L(k^2)} k_{\nu} + k_{\nu} = 0\]
Which means $f_L(k^2) =0$ and so, we have $f_T(k^2) \to O(k^2)$ as $k^2 \to 0$. The exact propagator of photon is
\[\mathcal{G}(k)_{\mu\nu} = \frac{-i}{k^2(1-\pi(k^2))}P^T_{\mu\nu} + \frac{-i\xi}{k^2} P^L_{\mu\nu}\]
where $\pi(k^2) \equiv \frac{f_T(k^2)}{k^2}$

\section{LSZ reduction formula}
\subsection{LSZ reduction formula and Feynman rules}
\noindent
Suppose that the probability for the quantum field to create or annihilate an exact one-particle eigenstate of H is $Z_3$, i.e.
\[\langle \Omega | A(0) | p,\lambda \rangle = \sqrt{Z_3} \epsilon_{\lambda}(p)\]
In Feynman gauge, because the norm of $|p,0\rangle$ is negative, the expansion of orthogonal complete set will be written as
\[\frac{d^3q}{(2\pi)^3} \frac{1}{2E_q} \eta^{\lambda\lambda'} | p,\lambda\rangle\langle p, \lambda' |\]
We have demonstrated that photon will remain massless when interacting with charged fermions. And recall that $\sum_{\lambda}\xi_{\lambda}(p)\xi^{*}_{\lambda}(p) = \eta_{\mu\nu}$. So, we can derive by the similar method in $\phi^4$ theory that
\[\int d^4x \; e^{-ipx} \langle \Omega | T A_{\mu}(x) A_{\nu}(0) | \Omega \rangle_C = \frac{-iZ_3\eta_{\mu\nu}}{p^2-i\epsilon} + \cdots \]
The LSZ reduction formula for fermions would take the form as \\ \\
\begin{newthem}[LSZ reduction formula]
\begin{eqnarray}
&\phantom{=}& \langle \bm{p}_1 \cdots \bm{p}_n \; | S | \; \bm{k}_1 \cdots \bm{k}_m \; \rangle  
\nonumber \\
&=& \quad \prod_1^n \int d^4 x_i e^{-i p_ix_i } \prod_1^m \int d^4 y_j e^{ik_jy_j} 
\nonumber \\
&\times & \left( \frac{i}{\sqrt{Z}_3} \right) ^{m+n}  [p_1^2 \epsilon^{*\mu_1}_{\lambda_1}(p_1)] \cdots [p_n^2 \epsilon^{*\mu_n}_{\lambda_n}(p_n)] [k_1^2 \epsilon^{ \nu_1}_{\lambda'_1}(k_1)] \cdots [k_m^2 \epsilon^{ \nu_m}_{\lambda'_m}(p_m)]
\nonumber \\
&\times & \langle \Omega | T \{A_{\mu_1}(x_1) \cdots A_{\mu_n}(x_n)
A_{\nu_1}(y_1) \cdots A_{\nu_m}(y_m) \} | \Omega \rangle
\nonumber
\end{eqnarray}
\end{newthem}
The LSZ reduction formula in other gauge would give similar procedure for calculating scattering amplitude: Fourier transform the Green function in position space to momentum space, cut-off the external legs and multiply the polarization vector of asymptotic states. (Note there is still an extra factor $\sqrt{Z_3}^{m+n}$ to multiply, similar to the $\phi^4$ theory).\\
Finally, we list the Feymann rules of QED in momentum space as follows:
\begin{enumerate}
\item For each incoming electron, draw a solid line with an arrow pointed towards the vertex, and label it with the electron's four-momentum, $p_i$.
\item For each outgoing electron, draw a solid line with an arrow pointed away from the vertex, and label it with the electron's four-momentum, $p'_i$.
\item For each incoming positron, draw a solid line with an arrow pointed away from the vertex, and label it with minus the positron's four-momentum, $-p_i$.
\item For each outgoing positron, draw a solid line with an arrow pointed towards the vertex, and label it with minus the positron's four-momentum, $-p'_i$.
\item For each incoming photon, draw a wavy line with an arrow pointed towards the vertex, and label it with the photon's four-momentum,$k_i$.
\item For each outgoing photon, draw a wavy line with an arrow pointed away from the vertex, and label it with the photon's four-momentum, $k'_i$.
\item The only allowed vertex joins two solid lines, one with an arrow pointing towards it and one with an arrow pointing away from it, and one wavy line. Using this vertex, join up all the external lines, including extra internal lines as needed. In this way, draw all possible diagrams that are topologically inequivalent.
\item Assign each internal line its own four-momentum. Think of the four-momenta as flowing along the arrows, and conserve four-momentum at each vertex. 
\item The value of a diagram consists of the following factors:
\begin{itemize}
\item for each incoming photon, $\epsilon^{\mu}_{\lambda}(k)$; 
for each outgoing photon, $\epsilon^{*\mu}_{\lambda}(k)$;
\item for each incoming electron, $u_{r}(\bm{k})$; for each outgoing electron, $\overline{u}_{s}(\bm{p})$;
\item for each incoming positron, $\overline{v}_{\overline{r}}(\overline{\bm{k}})$; for each outgoing positron, $v_{\overline{s}}(\overline{\bm{p}})$;
\item for each vertex, $ie_0 \gamma^{\mu}$;
for each internal photon, $G_F(p)$;
for each internal fermion, $S_F(p)$.
\end{itemize}
\item Spinor indices are contracted by starting at one end of a fermion line: specifically, the end that has the arrow pointing away from the vertex. The factor associated with the external line is either $\bar{u}$ or $\bar{v}$. Go along the complete fermion line, following the arrows backwards, and write down (in order from left to right) the factors associated with the vertices and propagators that you encounter. The last factor is either a $u$ or $v$. Repeat this procedure for the other fermion lines, if any. The vector index on each vertex is contracted with the vector index on either the photon propagator (if the attached photon line is internal) or the
photon polarization vector (if the attached photon line is external).
\item The overall sign of a tree diagram is determined by drawing all contributing diagrams in a standard form: all fermion lines horizontal, with their arrows pointing from left to right, and with the left endpoints labeled in the same fixed order (from top to bottom); if the ordering of the labels on the right endpoints of the fermion lines in a given diagram is an even (odd) permutation of an arbitrarily chosen fixed ordering, then the sign of that diagram is positive (negative)
\item Each closed fermion loop contributes an extra minus sign.
\item Value of $i\mathcal{M}$ is given by a sum over the values of the contributing diagrams.
\item $\langle f | S | i \rangle = (Z_2)^{\frac{n_f}{2}} (Z_3)^{\frac{n_p}{2}} i\mathcal{M}\delta(\sum p_f -\sum p_i)$
\end{enumerate}
\subsection{Ward Takahashi identity (3)}
Suppose the invariant matrix element for a process is $\mathcal{M}$, if we replace the polarization state vector $\epsilon_{\lambda}^{\mu}$ (or $\epsilon_{\lambda}^{*\mu}$) of one incoming (or outgoing) photon with its momentum vector $k^{\mu}$, we have
\[k^{\mu} \mathcal{M}_{\mu} = 0\]
\begin{newproof}
Without losing generality, we can consider a physical process with a single incoming and outgoing fermion lines respectively. So, the ward identities states that
\[-ik_{\mu} F^{\mu}(k;p,q) = ie_0\left[F_0(p+k,q)-F_0(p,q-k)\right]\]
Here, $F$ keeps the external fermion legs but cuts external photon lines. According to the LSZ reduction formula, from each diagram a contribution to an S matrix element by taking the coefficient of the product of poles
\[\left( \frac{-i}{\slashed{p}+m} \right) \left( \frac{-i}{\slashed{q}+m} \right)\] 
But the terms on the right hand side contain one of these poles, but neither contains both poles. So they contribute nothing to S-matrix. 
So, we can have
\[k^{\mu} \mathcal{M}_{\mu} = 0\] 
\end{newproof}
We also note that when calculating invariant matrix element, the main difference between different gauge are photon propagator. In Coulomb gauge, we have
\[G_F(p)_{\mu\nu} \equiv \begin{cases} \frac{i}{|\bm{p}|^2} \quad \mu,\nu=0\\  \frac{-i}{p^2-i\epsilon} \left(\delta_{ij} - \frac{p_ip_j}{|\bm{p}|^2}\right) \quad \mu = i \neq 0, \nu = j \neq 0 \\ 0 \quad \mbox{otherwise} \end{cases} \]
In Lorentz gauge, we have
\[G_F(p)_{\mu\nu}  = \frac{-i\eta_{\mu\nu}}{p^2-i\epsilon} + i(1-\xi)\frac{p_{\mu}p_{\nu}}{(p^2-i\epsilon)^2} \]
We now argue that the lead to the same $\mathcal{M}$ element.
For a general process, it can be represented as follows.
\begin{figure}[!h]
\centering
\includegraphics[height=2.86cm ,width=5.31cm]{QFT/WDTA4.png}
\caption{Feynman diagram representation of a QED process}
\end{figure}\\
The value of the diagram is
\[\mathcal{M}_{1}^{\mu} G_F(k)_{\mu\nu}\mathcal{M}_{2}^{\nu}\]
and $k_{\mu}\mathcal{M}_{1}^{\mu}=0$, $k_{\nu}\mathcal{M}_{2}^{\nu}=0$.
So the factor $\xi$ in Lorentz gauge is irrelevant to the value of $\mathcal{M}$. As for Coulomb gauge, denote $\mathcal{M}_{1}^{\mu}$ as $\alpha^{\mu}$, $\mathcal{M}_{2}^{\mu}$ as $\beta^{\mu}$, so 
\[\alpha^{\mu} G_F(k)_{\mu\nu} \beta^{\nu} = i\left( -\frac{\bm{\alpha} \cdot \bm{\beta}}{k^2} + \frac{(\bm{\alpha}\cdot\bm{k})(\bm{\beta}\cdot\bm{k})}{k^2 \bm{k}^2} + \frac{\alpha^0 \beta^0 }{\bm{k}^2} \right)\]
Using the fact that $\bm{\alpha}\cdot\bm{k} + \alpha^0 k_0 = 0$, we can verify that
\[\alpha^{\mu} G_F(k)_{\mu\nu} \beta^{\nu} = \alpha^{\mu} \left( -\frac{i\eta_{\mu\nu}}{k^2} \right)\beta^{\nu}\]
So, the invariant matrix element is gauge invariant.

\section{Renormalization}
\subsection{Renormalized quantum electrodynamics}
The Lagrangian of QED is
\[\mathcal{L} = -\frac{1}{4}F_{\mu\nu}F^{\mu\nu} + \overline{\Psi} (i\slashed{\partial}-m_0) \Psi + e_0j^{\mu} A_{\mu}, \]

Suppose the number of external photons is $N_{\gamma}$, the number of external fermions is $N_{e}$, the number of photon propagator (after amputation) is $P_{\gamma}$, the number of fermion propagator is $P_{e}$, the number of vertex is $V$. So we know the number of loops is
\[L = P_{\gamma}+P_{e} - V+1\]
We also know that
\[V = 2P_{\gamma} + N_{\gamma}  \quad 2V = 2P_{e} + N_{e}\]
So, the superficial divergence is 
\[D = 4L - 2P_{\gamma} - P_{e} = 4 -N_{\gamma} - \frac{3}{2}N_{e}\]

Recall that $C$ is a symmetry of QED, so $C|\Omega\rangle = | \Omega \rangle$. But $j^{\mu}$ changes sign under charge conjugation, $C j^{\mu}(x)C^{\dagger} = - j^{\mu}(x)$, so its vacuum expectation value must vanish:
\[\langle \Omega | T j^{\mu}(x) | \Omega \rangle = \langle \Omega | T C^{\dagger}C j^{\mu}(x)C^{\dagger}C | \Omega \rangle = -\langle \Omega | T j^{\mu}(x) | \Omega \rangle = 0 \]
So, the amplitude with $N_{\gamma} = 1,N_{e}=0$ will vanish. Similarly, we can verify that the diagram with $N_{\gamma} = 3,N_e=0$ will also vanish.

As for the amplitude with $N_{\gamma} = 4$, the superficial divergence $D = 0$. So the only divergence must be of the form $\log \Lambda$. But the ward identity implies that
\[K^{\mu}\mathcal{M}_{\mu\nu\sigma\rho} = 0\]
So, the divergent terms must vanish.

If we neglect the vacuum term with $N_{\gamma}=0,N_{e}=0$, there are only three divergent amplitude terms left. 
\begin{figure}[!h]
\centering
\includegraphics[height=5.44cm ,width=8.24cm]{QFT/QEDRG2.png}
\caption{Feynman diagram representation of divergent amplitude in QED}
\end{figure}\\
We need four counter terms to eliminate all the divergence. The Lagrangian can be written as
\[\mathcal{L} = \mathcal{L}_1 + \mathcal{L}_{ct}\]
Here,
\[\mathcal{L}_1 = -\frac{1}{4}F_r^{\mu\nu}F_{r\mu\nu} +  i\overline{\Psi}_r \gamma^{\mu} \partial_{\mu} \Psi_r - m \overline{\Psi}_r \Psi_r + e \overline{\Psi}_r\gamma^{\mu}\Psi_r A_{r\mu} \]
\[\mathcal{L}_{ct} = -\frac{1}{4}\delta_3 F_r^{\mu\nu}F_{r\mu\nu} + i\delta_{2}\overline{\Psi}_r \gamma^{\mu} \partial_{\mu} \Psi_r - \delta_m \overline{\Psi}_r \Psi_r + e \delta_1 \overline{\Psi}_r\gamma^{\mu}\Psi_r A_{r\mu} \]
\[A = \sqrt{Z_3}A_r \quad \Phi  = \sqrt{Z_2}\Phi_r \quad \delta_{3} = Z_3 - 1 \quad \delta_{2} = Z_2 - 1 \quad \]
\[\delta_m = Z_2 m_0 - m  \quad \delta_1 = Z_1 - 1 =(e_0/e) Z_2 Z_3^{\frac{1}{2}} - 1\]
The Feynman rules of counter terms are:
\[-\frac{1}{4}\delta_3 F_r^{\mu\nu}F_{r\mu\nu} \quad  -i(\eta^{\mu\nu}q^2-q^{\mu}q^{\nu})\delta_3\]
\[i\overline{\Psi}_r (\delta_{2}\slashed{\partial} - \delta_m)\Psi_r \quad -i(\delta_2\slashed{p}+\delta_m)\]
\[ e \delta_1 \overline{\Psi}_r\gamma^{\mu}\Psi_r A_{r\mu} \quad \quad ie \gamma^{\mu}\delta_1 \quad \quad  \quad \]
We also denote the renormalized 1PI component of exact propagator of photon as $i(\eta^{\mu\nu}q^2 -q^{\mu}q^{\nu})\Pi_r(q^2)$, the renormalized 1PI component of exact propagator of fermion as $-i\Sigma_r(\slashed{p})$, the renormalized exact amputated photon-fermion-antifermion vertex as $ie\Gamma^{\mu}_r(p,p')$. \\
So, renormalized exact propagator of photon is
\[\mathcal{G}_r(q)_{\mu\nu} = \frac{-i}{q^2(1-\Pi_r(q^2))}P^T_{\mu\nu}\]
renormalized exact propagator of fermion is
\[\mathcal{S}_r(p) = \frac{-i}{\slashed{p}+m+\Sigma_r(\slashed{p})}\]
The on-shell renormalization conditions are
\begin{eqnarray}
\Sigma_r(\slashed{p}=-m) &=& 0 \nonumber \\
\frac{d}{d\slashed{p}} \Sigma_r(\slashed{p}) \Big|_{\slashed{p} = -m} &=& 0 \nonumber \\
\Pi_r(q^2 = 0) &=& 0 \nonumber \\
ie\Gamma^{\mu}_r(p=p',p^2=-m^2) &=& ie\gamma^{\mu} \nonumber
\end{eqnarray}
Recall the ward identity, we have
\[ie\sqrt{Z_2}\partial_{\mu} \langle \Omega | T j_r^{\mu} \Psi_r(x_1) \overline{\Psi}_r(x_2)| \Omega\rangle = -ie\delta(x-x_1)\langle \Omega | T \Psi_r(x_1) \overline{\Psi}_r(x_2)| \Omega\rangle + ie\delta(x-x_2)\langle \Omega | T \Psi_r(x_1) \overline{\Psi}_r(x_2)| \Omega\rangle\]
In momentum space, we have
\[-k_{\mu} Z_2 Z_1^{-1} \mathcal{S}_r(p+k) [ie\Gamma^{\mu}_r(p+k,p)] \mathcal{S}_r(p) \frac{1}{1-\Pi_r(k^2)} = e(\mathcal{S}_r(p+k) - \mathcal{S}_r(p)) \]
So,
\[Z_2 Z_1^{-1} k_{\mu} [\Gamma^{\mu}_r(p+k,p)] \frac{1}{1-\Pi_r(k^2)} =\slashed{k}+\Sigma_r(\slashed{k}+\slashed{p})-\Sigma_r(\slashed
{p})\]
Since $\Gamma_r$, $\Pi_r$ and $\Sigma_r$ are all finite by renormalization, so $Z_1/Z_2$ must be finite.
In $\overline{MS}$ renormalization scheme, we immediately get
\[Z_1 = Z_2\]
In on-shell renormalization scheme, taking the limit of $k \to 0$, we can also get that
\[Z_1 = Z_2\]
So, we know
\[e = \sqrt{Z_3}e_0\]
This means that the relation between the bare and renormalized electric charge depends only on the photon field strength renormalization, not on quantities particular to the fermions, leading to a universal electric charge which has the same value for all species.\\
In the following subsection, we would omit the subscript $r$ unless it is necessary to emphasis the difference of bare field and renormalized fields.

\subsection{One loop structure of QED}
\subsubsection{Photon propagator}
\begin{figure}[!h]
\centering
\includegraphics[height=2.22cm ,width=9.21cm]{QFT/QEDRG3.png}
\caption{The one-loop and counterterm corrections to the photon propagator}
\end{figure}
\[\Pi(k^2) = -\frac{e^2}{\pi^2} \int_0^1 dx x(1-x) \left[ \frac{1}{\epsilon} - \frac{1}{2}\ln(\frac{D}{\mu^2})\right] - \delta_3 + O(e^4)\]
where $D = x(1-x)k^2+m^2-i\epsilon$ and $\mu^2 = 4\pi e^{-\gamma} \tilde{\mu}^2$.
Impose the OS renormalization condition $\Pi(0) = 0$, we have
\[\delta_3 = -\frac{e^2}{6\pi^2} \left[ \frac{1}{\epsilon} - \ln(\frac{m}{\mu})\right] + O(e^4)\]
\[\Pi(k^2) = \frac{e^2}{2\pi^2} \int_0^1 dx x(1-x) \ln(\frac{D}{m^2}) + O(e^4)\]
\subsubsection{Fermion propagator}
The exact renormalized fermion propagator in OS renormalization can be written in Lehmann–Kallen form as
\[i\mathcal{S}(\slashed{p}) = \frac{1}{\slashed{p}+m-i\epsilon} + \int_{m_{th}}^{\infty} ds \frac{\rho_{\Psi}(s)}{\slashed{p}+\sqrt{s}-i\epsilon}\]
We see that the first term has a pole at $\slashed{p}=-m$ with residue one. This residue corresponds to the field normalization that is needed for the validity of the LSZ formula.

There is a problem, however: in quantum electrodynamics, the threshold mass $m_{th}$ is $m$, corresponding to the contribution of a fermion and a zero energy photon. Thus the second term has a branch point at $\slashed{p}=-m$. The pole in the first term is therefore not isolated, and its residue is ill defined.

This is a reflection of an underlying infrared divergence, associated with the massless photon. To deal with it, we must impose an infrared cutoff that moves the branch point away from the pole. The most direct method is to change the denominator of the photon propagator from $k^2$ to $k^2+m_{\gamma}^2$, where $m_{\gamma}$ is a fictitious photon mass. Ultimately, we must deal with this issue by computing cross sections that take into account detector inefficiencies. In quantum electrodynamics, we must specify the
lowest photon energy $\omega_{min}$ that can be detected. Only after computing cross sections with extra undetectable photons, and then summing over them, is it safe to take the limit $m_{\gamma} \to 0$. 
\begin{figure}[!h]
\centering
\includegraphics[height=1.98cm ,width=7.77cm]{QFT/QEDRG4.png}
\caption{The one-loop and counterterm corrections to the fermion propagator}
\end{figure}
\[\Sigma(\slashed{p}) = \frac{e^2}{8\pi^2} \int_0^1 dx \left( (2-\epsilon)(1-x)\slashed{p} + (4-\epsilon)m \right) \left[ \frac{1}{\epsilon} - \frac{1}{2}\ln(\frac{D}{\mu^2})\right] + \delta_2 \slashed{p} + \delta_m + O(e^4) \]
where $D = x(1-x)p^2 + xm^2 + (1-x)m_{\gamma}^2$.
The fitness of $\Sigma(\slashed{p})$ requires that
\[\delta_2 =  - \frac{e^2}{8\pi^2} \left( \frac{1}{\epsilon} + \mbox{ finite } \right) + O(e^4)\]
\[\delta_m/m = - \frac{e^2}{2\pi^2} \left( \frac{1}{\epsilon} + \mbox{ finite } \right) + O(e^4) \]
Impose the OS renormalization condition $\Sigma(-m) = 0$ and $\Sigma'(-m) = 0$, we have
\[\Sigma(\slashed{p}) = -\frac{e^2}{8\pi^2} \int_0^1 dx \left( (1-x)\slashed{p} + 2m \right) \ln(\frac{D}{D_0})  + \kappa_2( \slashed{p} + m) + O(e^4) \]
where $D_0 = x^2m^2 +(1-x)m_{\gamma}^2$ and $\kappa_2 = -2 \ln(m/m_{\gamma})+1$. 
\subsubsection{Vertex}
\begin{figure}[!h]
\centering
\includegraphics[height=2.75cm ,width=4.11cm]{QFT/QEDRG5.png}
\caption{The one-loop correction to the photon-fermion-fermion vertex}
\end{figure}
\[\Gamma^{\mu}(p,p') = (1+\delta_1)\gamma^{\mu} + \frac{e^2}{8\pi^2} \left[\left(\frac{1}{\epsilon} - 1 - \frac{1}{2} \int dF_3 \ln (D/\mu^2)\right)\gamma^{\mu} + \frac{1}{4} \int dF_3 \frac{\tilde{N}^{\mu}}{D}\right] + O(e^4)\]
where
\[\int dF_3 = 2 \int_0^1 dx_1 dx_2 dx_3 \delta(x_1+x_2+x_3-1)\]
\[D = x_1(1-x_1)p^2 + x_2(1-x_2)p'^2 - 2x_1x_2p \cdot p' + (x_1+x_2)m^2 + x_3 m_{\gamma}^2\]
\[\tilde{N}^{\mu} = \gamma_{\nu}[x_1\slashed{p}-(1-x_2)\slashed{p}'+m]\gamma^{\mu}[-(1-x_1)\slashed{p}+x_2\slashed{p}'+m]\gamma^{\nu}\]
Fitness of $\Gamma^{\mu}$ requires that
\[\delta_1 = -  \frac{e^2}{8\pi^2}(\frac{1}{\epsilon}+\mbox{ finite }) + O(e^4)\]
Impose the OS renormalization condition $\Gamma^{\mu}_r(p=p',p^2=-m^2) = \gamma^{\mu}$, we have
\[\Gamma^{\mu}(p,p') = \gamma^{\mu} - \frac{e^2}{16\pi^2} \int dF_3 \left[\left(\ln (D/D_0)+2\kappa_1\right)\gamma^{\mu} - \frac{\tilde{N}^{\mu}}{2D}\right] + O(e^4)\]
where
\[D_0 = (1-x_3)^2m^2 + x_3 m_{\gamma}^2 \quad \kappa_1 = -2 \ln(m/m_{\gamma}) + \frac{5}{2}\]

\chapter{Vector Field}
\section{Vector field}
Consider a vector field $A^{\mu}(x)$. Here the index $\mu$ is a vector index that takes on four possible values. Under a Lorentz transformation, we have
\[U(\Lambda)^{-1} A^{\mu}(x) U(\Lambda) = \Lambda^{\mu}_{\phantom{\mu}\nu} A^{\nu}(\Lambda^{-1}x)\]
For an infinitesimal transformation, we can write
\[\delta^{\mu}_{\phantom{\mu}\nu}+\delta \omega ^{\mu}_{\phantom{\mu}\nu} = \delta^{\mu}_{\phantom{\mu}\nu} + \frac{i}{2} \delta \omega_{\rho \sigma} (S_V^{\rho \sigma})^{\mu}_{\phantom{\mu}\nu}\]
Here
\[(S_V^{\rho \sigma})^{\mu}_{\phantom{\mu}\nu} = -i(\eta^{\rho \mu}\delta ^{\sigma}_{\phantom{\sigma}\nu} - \eta^{\sigma \mu}\delta^{\rho}_{\phantom{\rho}\nu})\]
It is obvious that $A^{\dagger \mu}$ is also a vector field. 
We know that $\eta^{\mu \nu}$ is invariant under Lorentz transformation, i.e.
\[\Lambda^{\mu}_{\phantom{\mu}\rho} \Lambda^{\nu}_{\phantom{\mu}\sigma} \eta^{\rho \sigma} = \eta^{\mu \nu} \]
We can use $\eta^{\mu \nu}$ and and its inverse $\eta_{\mu\nu}$ to raise and lower vector indices of the vector field,
\[A_{\mu} \equiv \eta_{\mu \nu} A^{\nu}\]
And we can verify the following equations
\[\Lambda^{\mu}_{\phantom{\mu}\nu} \Lambda_{\mu}^{\phantom{\mu}\rho} = \delta^{\rho}_{\nu} \]
\[A^{\mu}(x) = \eta^{\mu \nu} A_{\nu}(x)\]
\[\Lambda_{\mu}^{\phantom{\mu}\rho} \Lambda_{\nu}^{\phantom{\nu}\sigma} \eta_{\rho \sigma} = \eta_{\mu \nu}\]
\[U(\Lambda)^{-1} A_{\mu}(x) U(\Lambda) = \Lambda_{\mu}^{\phantom{\mu}\nu} A_{\nu}(\Lambda^{-1}x)\]
Define $C_i \equiv \frac{1}{2}\epsilon_{ijk}S_V^{jk}$,$D_i \equiv S_V^{i0}$. For example, we have
\[(C_3)_{\mu}^{\phantom{\mu}\nu} = \left(\begin{array}{rrrr}
0 & 0 & 0 & 0 \\
0 & 0 & -i & 0 \\
0 & i & 0 & 0 \\
0 & 0 & 0 & 0
\end{array}\right)\]
The eigenvectors of $C_3$ are
\[\left[\left(-1, \left[\left(0,\,1,\,-i,\,0\right)\right], 1\right),
\left(1, \left[\left(0,\,1,\,i,\,0\right)\right], 1\right), \left(0,
\left[\left(1,\,0,\,0,\,0\right), \left(0,\,0,\,0,\,1\right)\right],
2\right)\right]\]
We further define $N_i \equiv \frac{1}{2}(C_i-iD_i)$ and $N^{\dagger}_i \equiv \frac{1}{2}(C_i + i D_i)$. For example, we have
\[(N_1)_{\mu}^{\phantom{\mu}\nu} = \left(\begin{array}{rrrr}
0 & -\frac{1}{2} & 0 & 0 \\
-\frac{1}{2} & 0 & 0 & 0 \\
0 & 0 & 0 & -\frac{1}{2} i \\
0 & 0 & \frac{1}{2} i & 0
\end{array}\right)\]
The eigenvectors of $N_1$ are
\[\left[\left(-\frac{1}{2}, \left[\left(1,\,1,\,0,\,0\right),
\left(0,\,0,\,1,\,-i\right)\right], 2\right), \left(\frac{1}{2},
\left[\left(1,\,-1,\,0,\,0\right), \left(0,\,0,\,1,\,i\right)\right],
2\right)\right]\]
And we can conclude that vector is in the $(2,2)$ representation of the Lie algebra of the Lorentz group.

\section{Electromagnetic field and gauge invariance}
\noindent
The Lagrangian of EM field is
\[\mathcal{L} = -\frac{1}{4}F_{\mu\nu}F^{\mu\nu}\]
Here,
\[F_{\mu\nu} = \partial_{\mu} A_{\nu} - \partial_{\nu} A_{\mu} \quad \mbox{and} \quad A^{\mu} = (\phi,\bm{A})\]
So,
\[F_{0i} = \dot{A}^i + \nabla_i \phi \equiv -E^i \quad \mbox{and} \quad F_{ij} = \nabla_i A^j - \nabla_j A^i \equiv \epsilon_{ijk}B^k\]
We can derive the equation of motion of the EM field by variation method,
\[\partial_{\mu}F^{\mu \nu} = 0\]
It can be rewritten in terms of $\bm{E}$ and $\bm{B}$, i.e. Maxwell equations:
\begin{eqnarray}
&\phantom{=}&\bm{\nabla} \cdot \bm{E} = 0 \quad \frac{\partial \bm{E}}{\partial t} = \bm{\nabla} \times \bm{B} \nonumber \\
&\phantom{=}& \bm{\nabla} \cdot \bm{B} = 0  \quad \frac{\partial \bm{B}}{\partial t} = - \bm{\nabla} \times \bm{E}\nonumber
\end{eqnarray}

The massless vector field $A_{\mu}$ has 4 components, which would naively seem to tell us that the gauge field has 4 degrees of freedom.But there are two related comments which will ensure that quantizing the gauge field $A_{\mu}$ gives rise to 2 degrees of freedom, rather than 4.
\begin{itemize}
\item The field $A_0$ has no kinetic term $\dot{A_0}$ in the Lagrangian: it is not dynamical. This means that if we are given some initial data $A_i$ and $\dot{A_i}$ at a time $t_0$, then the field $A_0$ is fully determined by the equation of motion $\bm{\nabla} \cdot \bm{E} = 0$,which, expanding out,
reads
\[\nabla^2 A_0 = \bm{\nabla} \cdot \frac{\partial \bm{A}}{\partial t}\]
So $A_0$ is not independent: we don't get to specify $A_0$ on the initial time slice.
\item If we transform the EM field as
\[A_{\mu} \to A_{\mu} + \partial_{\mu}\lambda(x) \]
we can derive that
\[F_{\mu\nu} \to F_{\mu \nu} \quad \mathcal{L} \to \mathcal{L}\]
The seemed infinite number of symmetries, one for each function $\lambda(x)$, is to be viewed as a redundancy in our description. That is, two states related by a gauge symmetry are to be identified: they are the same physical state. One way to see that this interpretation is necessary is to notice that Maxwell’s equations are not sufficient to specify the evolution of $A_{\mu}$.The equations read,
\[(\eta_{\mu\nu} \partial^2 - \partial_{\mu} \partial_{\nu}) A^{\nu} = 0\]
But the operator $(\eta_{\mu\nu} \partial^2 - \partial_{\mu} \partial_{\nu})$ is not invertible: it annihilates any function of
the form $\partial_{\mu} \lambda$. This means that given any initial data, we have no way to uniquely determine $A_{\mu}$ at a later time since we can't distinguish between $A_{\mu}$ and $A_{\mu} + \partial_{\mu} \lambda$. This would be problematic if we thought that $A_{\mu}$ is a physical object. However, if we're happy to identify $A_{\mu}$ and $A_{\mu} + \partial_{\mu} \lambda$ as corresponding to the same physical state, then our problems disappear. 
\end{itemize}

The picture that emerges for the theory of electromagnetism is of an enlarged phase space, foliated by gauge orbits. All states that lie along a given gauge orbit can be reached by a gauge transformation and are identified. To make progress, we pick a representative from each gauge orbit. It doesn't matter which representative we pick after all, they're all physically equivalent. But we should make sure that we pick a "good" gauge, in which we cut the orbits. Here we'll look at two different gauges:
\begin{itemize}
\item Coulomb Gauge: $\bm{\nabla} \cdot \bm{A} = 0$\\
We can make use of the residual gauge transformations in Lorentz gauge to pick $\bm{\nabla} \cdot \dot{\bm{A}} = 0$. We
have as a consequence $A_0 = 0$. Coulomb gauge is sometimes called radiation gauge.
\item Lorentz Gauge: $\partial^{\mu} A_{\mu} = 0$\\
In fact this condition doesn't pick a unique representative from the gauge orbit. We're always free to make further gauge transformations with $\partial^{\mu}\partial_{\mu} \lambda = 0$, which also has non-trivial solutions. As the name suggests, the Lorentz gauge has the advantage that it is Lorentz invariant.
\end{itemize}

\section{Canonical quantization of EM field}
\subsection{Canonical quantization in Coulomb gauge}

\subsubsection{Canonical momentum and Hamiltonian}
\[\pi^0 = \frac{\partial \mathcal{L}}{\partial \dot{A_0}} = 0 \quad  \pi^{i} = \frac{\partial \mathcal{L}}{\partial(\partial_0 A_i)} = \dot{A}^i + \nabla_i \phi = -E^i\]
\[\mathcal{H} = \frac{1}{2}(\bm{\pi}^2 + \bm{B}^2) + (\bm{\pi} \cdot \bm{\nabla}) A_0\]
Integration by parts can give
\[H = \int d^3x \frac{1}{2}(\bm{\pi}^2 + \bm{B}^2)\]

\subsubsection{Momentum and angular momentum}
\[P^0 = H \quad \vec{P} = \int - \bm{\pi} \vec{\nabla} \bm{A} d^3x\]
\[\vec{J} = - \int \bm{\pi} (\vec{x}\times \vec{\nabla} + i \vec{C})\bm{A} \; d^3x \quad \vec{S} = -i \int \bm{\pi} \vec{C} \bm{A} \; d^3x\]

\subsubsection{Canonical quantization}
\noindent
In Coulomb gauge, we have
\[A_0 = \pi^0 = 0 \quad \pi^i = \dot{A}^i \]
Three pairs of $A_i$ and $\pi^i$ are not independent from each other. They must satisfy the constraint equations
\[\nabla \cdot \bm{A} = 0 \quad \nabla \cdot \bm{\pi} = 0\]
A reasonable quantization condition can be written as
\[[A_i(\bm{x},t),A_j(\bm{x}',t)] = 0 \quad [\pi^i(\bm{x},t),\pi^j(\bm{x}',t)] = 0\]
\[[A_i(\bm{x},t),\pi^j(\bm{x}',t)] = i \left( \delta^{j}_i - \frac{\partial_i \partial^j}{\nabla^2} \right) \delta(\bm{x}-\bm{x}') \equiv i \int \frac{d^3k}{(2\pi)^3} \; (\delta^j_i - \frac{k_ik^j}{\bm{k}^2})e^{i\bm{k}\cdot(\bm{x}-\bm{x}')}\]
In this case, we can verify that
\[\dot{A}_i = -i[A_i(\bm{x},t),H] = \pi_i (\bm{x},t)\]
\[\dot{\pi}^i = -i[\pi^i(\bm{x},t),H] = \nabla^2 A^i(\bm{x},t)\]
It is constant with the field equation we derive from Euler-Lagrange equation.

\subsubsection{Fourier expansion}
\[\bm{A}(x) = \sum_{r = \pm} \int \widetilde{dp} [a_{r}(\bm{p}) \bm{\epsilon}_r(\bm{p})e^{ipx} + a^{\dagger}_{r}(\bm{p}) \bm{\epsilon}^*_r(\bm{p})e^{-ipx}]\]
And we can derive from constraint condition that
\[\bm{\epsilon} \cdot \bm{p} = 0\]
We will choose $\bm{\epsilon}$ to satisfy that
\[\bm{\epsilon}_r \cdot \bm{\epsilon}^*_s = \delta_{rs}\]
So, the completeness relation for the polarization vectors is
\[\sum_{r=\pm} \epsilon_r^i(\bm{p}) \epsilon_r^{*j}(\bm{p}) = \delta^{ij} - \frac{p^ip^j}{|\bm{p}|^2}\]
\begin{example}
If $\bm{p} = (0,0,p)$, we usually choose 
\[\bm{\epsilon}_{+} = \frac{1}{\sqrt{2}}(1,i,0) \quad \bm{\epsilon}_{-} = \frac{1}{\sqrt{2}}(1,-i,0) \]
$\bm{\epsilon}_{+}$ corresponds to left-handed rotation and it is the eigenvectors of the space-part of $C_3$ with eigenvalue $+1$. $\bm{\epsilon}_{-}$ corresponds to right-handed rotation and it is eigenvector of the space-part of $C_3$ with eigenvalue $- 1$.
\end{example}
\noindent
We can further derive from above discussion that
\[\bm{\pi}(x) = -i \sum_{r = \pm} \int \widetilde{dp} \omega [a_{r}(\bm{p}) \bm{\epsilon}_r(\bm{p})e^{ipx} - a^{\dagger}_{r}(\bm{p}) \bm{\epsilon}^*_r(\bm{p})e^{-ipx}]\]
\[a_r(\bm{p}) = \bm{\epsilon}^*_r \int d^3x e^{-ikx}(i\bm{\pi}+\omega\bm{A})\]
\[a^{\dagger}_r(\bm{p}) = \bm{\epsilon}_r \int d^3x e^{ikx}(-i\bm{\pi}+\omega\bm{A})\]
\[[a_r(\bm{p}),a_{r'}(\bm{p'})] = 0 \quad [a^{\dagger}_r(\bm{p}),a^{\dagger}_{r'}(\bm{p'})] = 0 \quad [a_r(\bm{p}),a^{\dagger}_{r'}(\bm{p'})] = (2\pi)^3 2\omega \delta_{rr'} \delta(\bm{p} - \bm{p}')\]

\subsubsection{Operator represented by $a$ and $a^{\dagger}$}
\noindent
Define that
\[N(\bm{p},r) \equiv a^{\dagger}_{r}(\bm{p}) a_r(\bm{p})\]
So, we can derive
\[H = \sum_{r = \pm} \int \widetilde{dp} \; \omega N(\bm{p},r) + 2\mathcal{E}_0V\]
\[\vec{P} = \sum_{r = \pm} \int \widetilde{dp} \; \vec{p} N(\bm{p},r) \]
\[\vec{S} = \sum_{r,s = \pm} \int \widetilde{dp} \; \frac{1}{2}(\bm{\epsilon}^*_{s} \vec{C}\bm{\epsilon}_{r} - \bm{\epsilon}_{r} \vec{C}\bm{\epsilon}^*_{s}) a^{\dagger}_{s}(\bm{p}) a_r(\bm{p})\]
From above equation, we can say that $a^{\dagger}_r(\bm{p})$ create an photon with energy $\omega$, momentum $\bm{p}$ and spin angular momentum along the direction of momentum $r$.

\subsubsection{Propagator}
\[G_{ij} \equiv \langle 0 |T A_i(x) A_j(y) | 0 \rangle = \int \frac{d^4p}{(2\pi)^4} \frac{-i}{p^2-i\epsilon} \left(\delta_{ij} - \frac{p_ip_j}{|\bm{p}|^2}\right) e^{ip(x-y)}\]

\subsection{Canonical quantization in Lorentz gauge}
\subsubsection{Undefined metric formalism}
\noindent
Modify the Maxwell Lagrangian introducing a new term
\[\mathcal{L} = -\frac{1}{4} F_{\mu\nu}F^{\mu\nu} - \frac{1}{2\xi} (\partial_{\mu} A^{\mu})^2\]
The equations of motion are now
\[\partial^2 A_{\mu} - (1-\frac{1}{\xi})\partial^{\mu}(\partial \cdot A) = 0\]
Canonical momentums are
\[\pi^0 = \frac{1}{\xi} \partial \cdot A = \frac{1}{\xi}(-\dot{A}_0 + \partial_i A^i) \quad \pi^i = \dot{A}^i + \nabla^i A^0 = -E^i\]
Hamiltonian is
\[\mathcal{H} = \frac{1}{2}(\bm{\pi}^2 + \bm{B}^2 - \xi \pi^0 \pi^0) + (\bm{\pi} \cdot \bm{\nabla}) A_0 + \pi^0 (\bm{\nabla} \cdot \bm{A})\]
\[H =  \int \left[ \frac{1}{2}(\bm{\pi}^2 + \bm{B}^2 - \xi \pi^0 \pi^0) -A_0(\bm{\nabla} \cdot \bm{\pi}) + \pi^0 (\bm{\nabla} \cdot \bm{A}) \right] d^3x \]
We remark that the above Lagrangian and the equations of motion, reduce to Maxwell theory in the gauge $\partial \cdot A = 0$. This why we say that our choice corresponds to a class of Lorenz gauges with parameter $\xi$. With this abuse of language (in fact we are not setting $\partial \cdot A = 0$, otherwise the problems would come back) the value of $\xi=1$ is known as the Feynman gauge and $\xi=0$ as the Landau gauge. From now on we will take the case of the so-called Feynman gauge, where $\xi=1$. Then the equation of motion coincide with the Maxwell theory in the Lorenz gauge. In Feymann gauge,the canonical quantization conditions can be written as
\[[A_{\mu}(\bm{x},t),A_{\nu}(\bm{x}',t)] = 0 \quad [\pi^{\mu}(\bm{x},t),\pi^{\nu}(\bm{x}',t)] = 0 \quad [A_{\mu}(\bm{x},t),\pi^{\nu}(\bm{x}',t)] = i\delta^{\nu}_{\mu} \delta(\bm{x}-\bm{x}')\]
we can also derive that
\[[\dot{A}_{\mu}(\bm{x},t),\dot{A}_{\nu}(\bm{x}',t)] = 0 \quad [A_{\mu}(\bm{x},t),\dot{A}_{\nu}(\bm{x}',t)] = i\eta_{\mu\nu} \delta(\bm{x}-\bm{x}')\]

\subsubsection{Fourier expansion}
\[A(x) = \sum_{\lambda=0}^{3} \int \widetilde{dp} [a_{\lambda}(\bm{p}) \epsilon_{\lambda}(\bm{p})e^{ipx} + a^{\dagger}_{\lambda}(\bm{p}) \epsilon^*_{\lambda}(\bm{p})e^{-ipx}]\]
where $\epsilon_{\lambda \mu}$ are a set of four independent 4-vectors.  We will now make a choice for these 4-vectors. We choose $\epsilon_{1\mu}$ and $\epsilon_{2\mu}$ orthogonal to $k^{\mu}$ and $n^{\mu}$, such that
\[\epsilon_{\lambda \mu} \epsilon^{*\mu}_{\lambda} = \delta_{\lambda \lambda'} \quad \lambda,\lambda' = 1,2\]
After, we choose $\epsilon_{3\mu}$ in the plane $(k^{\mu},n^{\mu})$ and perpendicular to $n^{\mu}$ such that
\[\epsilon_{3\mu} n^{\mu} = 0 \quad \epsilon_{3\mu} \epsilon^{*\mu}_{3} = 1\]
Finally we choose $\epsilon_{0\mu} = n_{\mu}$. The vectors $\epsilon_{1\mu}$ and $\epsilon_{2\mu}$ are called transverse polarizations, while $\epsilon_{3\mu}$ and $\epsilon_{0\mu}$ longitudinal and scalar polarizations, respectively.\\
In general we can show that
\[\epsilon_{\lambda} \cdot \epsilon^*_{\lambda'} = \eta_{\lambda \lambda'} \quad \eta^{\lambda \lambda'} \epsilon_{\lambda \mu} \epsilon^*_{\lambda' \nu} = \eta_{\mu \nu} \]
We can further derive from above discussion that
\[\dot{A}(x) = -i \sum_{\lambda=0}^{3} \int \widetilde{dp} \omega [a_{\lambda}(\bm{p}) \epsilon_{\lambda}(\bm{p})e^{ipx} - a^{\dagger}_{\lambda}(\bm{p}) \epsilon^*_{\lambda}(\bm{p})e^{-ipx}]\]
\[ a_{\lambda}(\bm{p}) =  \eta_{\lambda \lambda'} \epsilon^*_{\lambda'} \cdot \int d^3x e^{-ipx}(i\dot{A}+\omega A)\]
\[ a^{\dagger}_{\lambda}(\bm{p}) =  \eta_{\lambda \lambda'} \epsilon_{\lambda'} \cdot \int d^3x e^{ipx}(-i\dot{A}+\omega A)\]
\[[a_{\lambda}(\bm{p}),a_{\lambda'}(\bm{p'})] = 0 \quad [a^{\dagger}_{\lambda}(\bm{p}),a^{\dagger}_{\lambda'}(\bm{p'})] = 0 \quad [a_{\lambda}(\bm{p}),a^{\dagger}_{\lambda'}(\bm{p'})] = (2\pi)^3 2\omega \eta_{\lambda \lambda'} \delta(\bm{p} - \bm{p}')\]

\subsubsection{Indefinite metric problem}
We Introduce the vacuum state defined by
\[a_{\lambda}(\bm{p}) | 0 \rangle = 0\]
To see the problem with the sign we construct the one-particle state with scalar polarization, that is
\[|1\rangle = \int \widetilde{dp} a^{\dagger}_{0}(\bm{p})|0\rangle
\]
and calculate its norm
\[\langle 1 | 1 \rangle = -\langle 0 | 0 \rangle \int \widetilde{dp} |f(p)|^2\]
The state $| 1 \rangle$ has a negative norm.

To solve this problem we note that we are not working anymore with the classical Maxwell theory because we modified the Lagrangian. What we would like to do is to impose the condition $\partial \cdot A = 0$, but that is impossible as an equation for operators. We can, however, require that condition on a weaker form, as a condition only to be verified by the physical states.

More specifically, we require that the part of $\partial \cdot A$ that contains the annihilation operator (positive frequencies) annihilates the physical states,
\[\partial^{\mu} A^{+}_{\mu} | \psi \rangle = 0\]
The states $| \psi \rangle$ can be written in the form
\[| \psi \rangle = | \psi_T \rangle | \phi \rangle\]
where $| \psi_T \rangle$  is obtained from the vacuum with creation operators with transverse polarization and $| \phi \rangle$ with scalar and longitudinal polarization.

$\partial^{\mu} A^{+}_{\mu}$ contains only scalar and longitudinal polarizations
\[\partial^{\mu} A^{+}_{\mu} = i\sum_{\lambda=0,3} \int \widetilde{dp} a_{\lambda}(\bm{p}) (p \cdot \epsilon_{\lambda}(\bm{p}) ) e^{ipx} \]
Therefore the previous condition becomes
\[i\sum_{\lambda=0,3} (p \cdot \epsilon_{\lambda}(\bm{p})) a_{\lambda}(\bm{p})  | \phi \rangle = 0\]
The condition is equivalent to,
\[(a_{0}(\bm{p}) - a_{3}(\bm{p})) | \phi \rangle = 0\]
We can construct $| \phi \rangle$ as a linear combination of states $| \phi \rangle$ with $n$ scalar or longitudinal photons:
\[| \phi \rangle = C_0 | \phi_0 \rangle + C_1 | \phi \rangle + \cdots \quad \mbox{Here,} | \phi_0 \rangle \equiv | 0 \rangle\]
The states $|\phi_n\rangle$ are eigenstates of the operator number for scalar or longitudinal photons
\[N' | \phi_n \rangle = n | \phi_n \rangle\]
where,
\[N' = \int \widetilde{dp} [a^{\dagger}_{3}(\bm{p})a_{3}(\bm{p})-a^{\dagger}_{0}(\bm{p})a_{0}(\bm{p})] \]
Then
\[n \langle \phi_n | \phi_n \rangle = \langle \phi_n |N'| \phi_n \rangle = 0\]
This means that
\[\langle \phi_n | \phi_n \rangle = \delta_{n0}\]
that is, for $n \neq 0$, the state $| \phi_n \rangle$ has zero norm. We have then for the general state $| \phi \rangle$,
\[\langle \phi | \phi \rangle = |C_0|^2 \geq 0\]
and the coefficients $C_i(i=1,2,\cdots)$ are arbitrary.

\subsubsection{Operator represented by $a$ and $a^{\dagger}$}
\noindent
Define that
\[N'(\bm{p}) \equiv a^{\dagger}_{3}(\bm{p})a_{3}(\bm{p})-a^{\dagger}_{0}(\bm{p})a_{0}(\bm{p})\]
\[N(\bm{p},1) \equiv a^{\dagger}_{1}(\bm{p}) a_{1}(\bm{p}) \quad N(\bm{p},2) \equiv a^{\dagger}_{2}(\bm{p}) a_{2}(\bm{p}) \quad N_T(\bm{p}) \equiv N(\bm{p},1) + N(\bm{p},2)\]
We have that
\[\langle \psi | N'(\bm{p}) | \psi\rangle = 0 \quad \langle \psi | N_T(\bm{p}) | \psi\rangle = \langle \psi_T | N_T(\bm{p}) | \psi_T\rangle\]
We can derive
\[ H = \int \widetilde{dp} \; \omega [N'(\bm{p}) + N_T(\bm{p})] + 2\mathcal{E}_0V\]
\[ \vec{P} = \int \widetilde{dp} \; \vec{p} [N'(\bm{p}) + N_T(\bm{p})]\]

So, the arbitrariness of $C_i(i=1,2,\cdots)$ does not affect the physical observables. Only the physical transverse polarizations
contribute to the result. Two states that differ only in their timelike and longitudinal photon content, $|\phi_n\rangle$ with $n \geq 1$ are said to be physically equivalent. We can think of the gauge symmetry of the classical theory as descending to the Hilbert space of the quantum theory.

It is important to note that although for the average values of the physical observables only the transverse polarizations contribute, the scalar and longitudinal polarizations are necessary for the consistency of the theory. In particular they show up when we consider complete sums over the intermediate states.

\subsubsection{Propagator}
\[G_{\mu\nu} \equiv \langle 0 |T A_{\mu}(x) A_{\nu}(y) | 0 \rangle = \int \frac{d^4p}{(2\pi)^4} \frac{-i\eta_{\mu\nu}}{p^2-i\epsilon}  e^{ip(x-y)}\]
It is easy to verify that $G_{\mu\nu}(x-y)$ is the Green's function of the equation of motion, that for $\xi=1$ is the wave equation, that is
\[\partial^2 G_{\mu\nu} = i\eta_{\mu\nu}\delta(x-y)\]
For the general case, $\xi \neq 0$, the equal times commutation relations are more complicated. And the propagator will be
\[G_{\mu\nu}  = \int \frac{d^4p}{(2\pi)^4} \left[\frac{-i\eta_{\mu\nu}}{p^2-i\epsilon} + i(1-\xi)\frac{k_{\mu}k_{\nu}}{(k^2-i\epsilon)^2}\right] e^{ip(x-y)}\]
\include{QFT4}
\part{Statistical mechanics}
\documentclass[cyan]{elegantnote}
\author{Yuyang Songsheng}
\email{songshengyuyang@gmail.com}
\zhtitle{物理}
\entitle{Physics}
\version{1.00}
\myquote{Summary is the best way to say "Good Bye"}
\logo{logo.jpg}
\cover{cover.pdf}
%green color
   \definecolor{main1}{RGB}{210,168,75}
   \definecolor{seco1}{RGB}{9,80,3}
   \definecolor{thid1}{RGB}{0,175,152}
%cyan color
   \definecolor{main2}{RGB}{239,126,30}
   \definecolor{seco2}{RGB}{0,175,152}
   \definecolor{thid2}{RGB}{236,74,53}
%cyan color
   \definecolor{main3}{RGB}{127,191,51}
   \definecolor{seco3}{RGB}{0,145,215}
   \definecolor{thid3}{RGB}{180,27,131}


\usepackage{makecell}
\usepackage{lipsum}
\usepackage{amssymb}
\usepackage{float}
\usepackage{wrapfig}
\usepackage{latexsym}
\usepackage{hyperref}
\usepackage{feynmf}
\usepackage{exscale}
\usepackage{relsize}
\usepackage{slashed}
\usepackage{bm}%bold math, for vector


\begin{document}
\maketitle
\tableofcontents

\chapter{Thermodynamics}
\section{Introduction to thermodynamics}
A thermodynamic system is a macroscopic system whose behaviour is identified thanks to a small and finite number of quantities - the thermodynamic properties.
\\
There is a certain degree of circularity in the definition of thermodynamic parameters, which is resolved by experiment. One considers only a restricted set of manipulations on thermodynamic systems. In practice, one allows them to be put in contact with one another, or one acts upon them by changing a few macroscopic properties such as their volume or the electric or magnetic field in which they are immersed. One then identifies a number of properties such that, if they are known before the manipulation, their values after the manipulation can be predicted. The smallest set of properties that allows one to successfully perform such a prediction can be selected as the basis for a thermodynamic description of the system.
\\ \\
If the state of a thermodynamic system can be fully characterized by the values of the thermodynamic variables, and if these values are invariant over time, one says that it is in a state of thermodynamic equilibrium. Thermodynamic equilibrium occurs when all fast processes have already occurred, while the slow ones have yet to take place. Clearly the distinction between fast and slow processes is dependent on the observation time $\tau$ that is being considered.
\\ \\
A system can be shown to be in equilibrium if the observation time is fairly short, while it is no longer possible to consider it in equilibrium for longer observation times. A more curious situation is that the same system can be considered in equilibrium, but with different properties, for different observation times.
\\ \\
Let us consider two thermodynamic systems, 1 and 2, that can be made to interact with one another. Variables like the volume $V$, the number of particles $N$, and the internal energy $U$, whose value (relative to the total system) is equal to the sum of the values they assume in the single systems, are called additive or extensive.
Strictly speaking, internal energy is not extensive, unless the interaction between 1 and 2 can be neglected.
\\ \\
\textbf{The fundamental hypothesis of thermodynamics is that it should be possible to characterize the state of a thermodynamic system by specifying the values of a certain set $(X_0, X_1, \cdots, X_r)$ of extensive variables. For example, $X_0$ could be the internal energy $U$, $X_1$ the number of particles $N$, $X_2$ the volume $V$ of the system, and so on.}
\\
The central problem of thermodynamics is that \textbf{given the initial state of equilibrium of several thermodynamic systems that are allowed to interact, determine the final thermodynamic state of equilibrium}.
\\ \\
The interaction between thermodynamic systems is usually represented by idealized walls that allow the passage of one (or more) extensive quantities from one system to the other.
Among the various possibilities, the following are usually considered:
\begin{itemize}
\item \textbf{Thermally conductive walls} These allow the passage of energy, but not of volume or particles. 
\item \textbf{ Semipermeable walls} These allow the passage of particles belonging to a given chemical species (and consequently also of energy)
\end{itemize}
The space of possible states of equilibrium (compatible with constraints and initial conditions) is called the space of virtual states. The initial state is obviously a (specific) virtual state. The central problem of thermodynamics can obviously be restated as follows:
\\
\textbf{Characterize the actual state of equilibrium among all virtual states.}

\section{Entropy formulation of thermodynamics}
\subsection{Property of entropy function}
There exists a function S of the extensive variables $(X_0, X_1, \cdots, X_r)$, called the entropy, that assumes the maximum value for a state of equilibrium among all virtual states and that possesses the following properties:
\begin{enumerate}
\item \textbf{Extensivity}: If 1 and 2 are thermodynamic systems, then 
\[ S^{(1 \cup 2)} = S^{(1)} + S^{(2)} \]
\item \textbf{Convexity}: If $ X^1 = (X_0^1, X_1^1, \cdots, X_r^1)$ and $X^2 = (X_0^2, X_1^2, \cdots, X_r^2)$ are two thermodynamic states of the same system, then for any a between 0 and 1, one obtains
\[ S[(1-\alpha)X^1 + \alpha X^2] \geq (1 - \alpha)S(X^1) + \alpha S(X^2) \]
From this expression, if we take the derivative with respect to $\alpha$ at $\alpha = 0$, we obtain
\[\left. \sum_{i=0}^{r} \frac{\partial S}{\partial X_i} \right|_{X^1} (X_i^2 - X_i^1) \geq S(X^2) - S(X^1) \]
which expresses the fact that the surface $S(X_0, X_1, \cdots, X_r)$ is always below the plane that
is tangent to each of its points. (We adpot the convention that convex means upper convex).
\item \textbf{Monotonicity}: $S(U, X_1, \cdots, X_r)$ is a monotonically increasing function of the internal energy $U$:
\[\left. \frac{\partial S}{\partial U} \right|_{X_1,\cdots,X_r} = \frac{1}{T} > 0\]
\end{enumerate}
The entropy postulate allows one to solve the central problem of thermodynamics, by referring it back to the solution of a constrained extremum problem:
\textbf{The equilibrium state corresponds to the maximum entropy compatible with the constraints.}

\subsection{Simple problems}
\subsubsection{Thermal Contact}
Let us consider two systems, 1 and 2, that are in contact by means of a thermally conductive wall. The virtual state space is therefore defined by the relations:
\[U^{(1)} + U^{(2)} = U = \mathrm{const.}\]
\[X^{(1)}_{i} = \mathrm{const}. \quad X^{(2)}_{i} = \mathrm{const.}. \quad r = 1,\cdots,r \quad \]
Let us look for the maximum of $S$ as a function of $U^{(1)}$:
\[\frac{\partial S}{\partial U^{(1)}} = \left. \frac{\partial S^{(1)}}{\partial U^{(1)}} \right|_{U^{(1)}} -  \left. \frac{\partial S^{(2)}}{\partial U^{(2)}} \right|_{U-U^{(1)}}\]
If we denote the value of $U^{(1)}$ at equilibrium by $U^{(1)}_{\mathrm{eq}}$, then we have
\[\left. \frac{\partial S^{(1)}}{\partial U^{(1)}} \right|_{U^{(1)}_{\mathrm{eq}}} =  \left. \frac{\partial S^{(2)}}{\partial U^{(2)}} \right|_{U^{(2)}_{\mathrm{eq}}}\]
Due to entropy's convexity, we can futher derive that
\[\left[ \left. \frac{\partial S^{(1)}}{\partial U^{(1)}} \right|_{U^{(1)}_{\mathrm{in}}} -  \left. \frac{\partial S^{(2)}}{\partial U^{(2)}} \right|_{U^{(2)}_{\mathrm{in}}}\right] (U^{(1)}_{\mathrm{eq}} - U^{(1)}_{\mathrm{in}}) \geq 0\]
Let us introduce the quantity
\[T = \left( \frac{\partial S}{\partial U} \right)^{-1}\]
According to our hypotheses, this quantity is positive. We obtained the following results:
\begin{itemize}
\item At equilibrium, $T$ is the same in all subsystems that are in reciprocal contact by means of thermally conductive walls.
\item In order to reach equilibrium, energy shifts from systems with higher values of $T$ toward systems with lower values of $T$.
\end{itemize}
Later, we will show that $T$ is the temperature of the system.

\subsubsection{A Thermally Conductive and Mobile Wall}
In this case, the two systems can also exchange volume $V$, in addition to internal energy $U$. If we introduce the quantity $p$ by
\[\frac{p}{T} =  \frac{\partial S}{\partial V} \]
The two equilibrium conditions are
\[T^{(1)} = T^{(2)} \quad p^{(1)} = p^{(2)}\]
One can easily prove that between two systems, both initially at the same temperature, volume is initially released by the system in which $p$ is lower to the system in which $p$ is higher. Later, we will show that $p$ is the pressure of the system.

\subsubsection{A Semipermeable Wall}
Let us consider a system composed of several chemical species, and let us introduce the number of molecules $N_1, \cdots, N_{r}$ belonging to the chemical species that constitute it as part of the thermodynamic variables. Let us suppose that two systems of this type are separated by a wall that only allows the $k$-th chemical species to pass. Clearly, it is impossible for the exchange of molecules to occur without an exchange of energy. If we introduce the quantity $\mu_i$ by
\[\frac{\mu_i}{T} =  \frac{\partial S}{\partial N_i} \]
The equilibrium conditions will therefore be
\[T^{(1)} = T^{(2)} \quad \mu^{(1)} = \mu^{(2)}\]
We will define $\mu_i$ as the chemical potential of the specie $i$.

\subsection{Heat and Work}
From mechanics (and from electromagnetism), we can derive an expression for the infinitesimal mechanical work performed on the system by varying the extensive quantities. One usually adopts a sign convention according to which work is considered positive if the system performs work on the outside. Following this convention, the expression of infinitesimal work is given by
\[\delta W  =  -\sum_{i = 1}^{r} f_i dX_i\]
On the one hand, we have
\[dS = \frac{dU}{T} + \sum_{i = 1}^{r} \left. \frac{\partial S}{\partial X_i} \right|_{U,\cdots,X_r} dX_i\]
It can be written as
\[dU = TdS - \sum_{i = 1}^{r} \left. T \frac{\partial S}{\partial X_i} \right|_{U,\cdots,X_r} dX_i\]
On the other hand, we have
\[dU = \delta Q - \delta W\]
So we can get
\[\delta Q = TdS \quad \left. \frac{\partial S}{\partial X_i} \right|_{U,\cdots,X_r} = - \frac{f_i}{T}\]

\subsubsection{Temperature}
Let us consider a system made up of a thermal engine and two heat reservoirs with $T_1 > T_2$. A heat reservoir is a system for which $T$ is independent of $U$. The whole compound system is enclosed in a container that allows it to exchange energy with the environment only in a purely mechanical way. 
\\
Let the system evolve from an initial equilibrium condition, in which the first heat reservoir has internal energy $U_1$, the second has internal energy $U_2$ , and the thermal engine is in some equilibrium state, to a final equilibrium state in which the first heat reservoir has internal energy $U'_1$, the second has $U'_2$. So $W = (U_1 + U_2 ) - (U'_1 + U'_2)$, and the thermal engine is back to its initial state. By definition, the efficiency of the engine is given by $\eta = \frac{W}{U_1 - U'_1}$.
\\
In a transformation of this kind, the total entropy of the compound system cannot become smaller. Thus, we have
\[S^{(1)}(U_1) + S^{(2)}(U_2) \leq S^{(1)}(U'_1) + S^{(2)}(U'_2)\]
On the other hand, since we are dealing with heat reservoirs, we have
\[S^{(i)}(U'_i) = S^{(i)}(U_i) + \frac{U'_i - U_i}{T_i} \quad i = 1,2\]
Thus, we have
\[\frac{U_1 - U'_1}{T_1} \leq \frac{U'_2 - U_2}{T_2}\]
Therefore,
\[\eta \leq 1 - \frac{T_2}{T_1}\]
Compared with the maximum efficiency evaluated in elementary thermodynamics, we can conclude that $T$ is the absolute temperature, up to an overall factor, which can be fixed to $1$ by rescaling the $S$. 

\subsubsection{Pressure}
Let us consider an infinitesimal variation of $V$. In this case, mechanics tells us that the work performed by the system is given by $\delta W = PdV$. So we have
\[\left. \frac{\partial S}{\partial V} \right|_{U,\cdots,X_r} = \frac{P}{T}\]
This allows us to identify the pressure $P$ with the quantity $p$ we defined previousl.


\subsubsection{The Fundamental Equation}
The equation
\[S = S(X_0 = U, X_1, \cdots, X_r)\]
is called the fundamental equation, and it represents a complete description of the thermodynamics of the system being considered.

\section{Thermodynamic potential}
\subsection{Energy Scheme}
We can also use a different (but equivalent) formulation of the fundamental principle of thermodynamics, in which entropy assumes the role of an independent variable, while energy becomes a dependent variable that satisfies a variational principle. This formalism is known as the energy scheme. In this formalism, the maximum entropy principle is replaced by the principle of minimum internal energy:
\\ 
\textbf{Among all states with a specific entropy value, the state of equilibrium is that in which internal energy is minimal.}
\\ \\
Let $\Delta X$ be a virtual variation of the extensive variables (excluding internal energy $U$) with respect to the equilibrium value $X_{\mathrm{eq}}$. Then
\[\Delta S = S(U,X_{\mathrm{eq}} + \Delta X) - S(U,X_{\mathrm{eq}} ) \leq 0\]
Since $S$ is a monotonically increasing function of $U$, there exists a value $U' > U$ such that $S(U',X_{\mathrm{eq}} + \Delta X) = S(U, X_{\mathrm{eq}})$. Therefore, if $S$ is kept constant, as the system moves out of equilibrium, $U$ cannot but increase. This is what we wanted to prove.
\\
Therefore, the fundamental equation in the energy scheme is
\[U = U(S,X_1,\cdots,X_r)\]
$U$'s differential assumes the form
\[dU = TdS + \sum_{i=1}^r f_i dX_i\]
Further more, we can derive that
\[ U[(1-\alpha)Y^1 + \alpha Y^2] \leq (1 - \alpha)U(Y^1) + \alpha U(Y^2) \]
which implies that the internal energy function is concave.

\subsection{Intensive Variables and Thermodynamic Potentials}
The derivatives $f_i = \frac{\partial U}{\partial X_i}$ of the internal energy $U$ with respect to extensive quantities $S$, $\{X_i\}$ are called intensive quantities. For uniformity's sake, we define $f_0 \equiv \frac{\partial U}{\partial X_i} = T$. A given quantity $f_i$ is called the conjugate of the corresponding variable $X_i$, and vice versa. The temperature $T$ and entropy $S$ are therefore conjugates, as are the pressure (with the opposite sign) $-P$ and the volume $V$.
\\ \\
Since both $U$ and $X_i$ are extensive, in a homogeneous system, intensive variables are not dependent on system size. Moreover, if a system is composed of several subsystems that can exchange the extensive quantity $X_i$, the corresponding intensive variable $f_i$ assumes the same value in those subsystems that are in contact at equilibrium.
\\
We now want to identify the state of equilibrium among all states that exhibit a given value of an intensive variable $f_i$. Specifically, for $i = 0$, we are confronted with the case of system with a fixed temperature, i.e. heat reservoir.
\\ \\
Let us now define the Helmholtz free energy $F(T,X) $ by the relation
\[F(T,X) = U(S(T,X),X) - TS(T,X)\]
where $X = \{ X_1,\cdots,X_r\}$.
We can then show that the thermodynamical equilibrium in these conditions is characterized by the following variational principle:
\\
\textbf{The value of the Helmholtz free energy is minimal for the equilibrium state among all virtual states at the given temperature $T$.}
\\ \\
Let us now consider more generally the Legendre transform of the internal energy $U$ with respect to the intensive variable $f_i$:
\[\Phi(S,f_1,X_2,\cdots,X_r) = U(S,X_1(S,f_1,X_2,\cdots,X_r),X_2,\cdots,X_r) - f_1 X_1 (S,f_1,X_2,\cdots,X_r)\]
where $X_1(S,f_1,X_2,\cdots,X_r),X_2,\cdots,X_r)$ is determined by the condition
\[f_1 =\left. \frac{\partial U}{\partial X_1} \right |_{S,X_2,\cdots,X_r}\]
Then, the state of equilibrium is specified by the following criterion:
\\
\textbf{Among all the states that have the same value as $f_1$, the state of equilibrium is that which corresponds to the minimum value of $\Phi$.}
\\ \\
Let us observe that the partial derivative of $\Phi$, performed with respect to $f_1$, with the other extensive variables kept fixed, yields the value of the extensive variable $X_1$:
\[ \left. \frac{\partial \Phi}{\partial f_1} \right|_{S,X_2,\cdots,X_r} = -X_1(S,f_1,X_2,\cdots,X_r)\]
\\ \\
Nothing prevents us from considering two or more intensive variables as fixed—for example, $f_0 = T$ and $f_1$. Similar considerations will then lead us to introduce the thermodynamic potential $\Phi(T,f_1,X_2,\cdots,X_r)$, obtained as a Legendre transform of $U$ with respect to $S$ and $X_1$:
\[\Phi(T,f_1,X_2,\cdots,X_r) = U - TS - f_1X_1\]
\\
This thermodynamic potential assumes at equilibrium the minimum value among all the states with the same values of $T$ and $f_1$. We can therefore obtain a whole series of thermodynamic potentials, by using a Legendre transform with respect to the extensive variables $X_i$. We cannot however eliminate all extensive variables in this manner. We will see later that if we did this, the resulting thermodynamic potential would identically vanish. For the time being, it is sufficient to observe that the $\Phi$ potentials are extensive quantities, and one cannot see how they could be a function only of intensive quantities like the $f_i$.
\\
A general thermodynamic potential
\[\Phi(S,f_1,\cdots,f_k,X_{k+1},\cdots,X_r) = U - \sum_{i = 1}^{k} f_i X_i\]
is concave as a function of the remaining extensive variables, for fixed values of the intensive variables $f_1,\cdots,f_k$.
$\Phi$ on the other hand is convex as a function of the intensive variables $f_i$s, when the extensive variables are fixed.

\subsection{Free Energy and Maxwell Relations}
The natural variables of $F$ are the temperature $T$ and the extensive variables $X_1,\cdots,X_r$, entropy excluded. Entropy can be obtained by taking the derivative of $F$ with respect to $T$. The expression for the differential of $F$ is
\[dF = -SdT + \sum_{i = 1}^{r} f_i dX_i\]
More specifically, by setting $X_1 = V$, one has
\[-P =\left. \frac{\partial F}{\partial V} \right|_{T,X_2,\cdots,X_r}\]
If we now take this equation's derivative with respect to $T$ and we use the theorem of the equality of mixed derivatives, we obtain
\[\left. -\frac{\partial P}{\partial T} \right|_{V,X_2,\cdots,X_r} = -\frac{\partial^2 F}{\partial T \partial V} = \left. -\frac{\partial S}{\partial V} \right|_{T,X_2,\cdots,X_r}\]
These relations between thermodynamic derivatives that derive from the equality of mixed derivatives of thermodynamic potentials are called Maxwell relations.
\\
The free energy designation is derived from the following property. If a system is put in contact with a reservoir at temperature $T$, we can prove the maximum quantity of work $W_{\mathrm{max}}$ that it can perform on its environment is equal to the variation in free energy between the initial and final states. In other words, one has
\[W \leq F_{\mathrm{in}} - F_{\mathrm{fin}}\]

\subsection{Gibbs Free Energy and Enthalpy}
Transforming $F$ according to Legendre with respect to $V$, we obtain a new thermodynamic potential, called the Gibbs free energy:
\[G(T,P,X_2,\cdots,X_r) = F + PV = U - TS - PV\]
The variational principle satisfied by the Gibbs free
energy is the following:
\\
\textbf{Among all states that have the same temperature and pressure values, the state of equilibrium is that in which the Gibbs free energy assumes the minimum value.}
\\
$G$'s differential is expressed as follows:
\[dG = -SdT + VdP + \sum_{i=2}^r f_i dX_i \]
We can prove that if a system is brought toward equilibrium while temperature and pressure are kept constant, the maximum work that can be performed on its environment is given precisely by the difference between the initial and final values of $G$.
\\ \\
If on the other hand, we Legendre transform the internal energy $U$ with respect to $V$, we obtain a new thermodynamic potential, usually denoted by H and called enthalpy:
\[H(S,P,X_2,\cdots,X_r) = U + PV\]
Enthalpy governs the equilibrium of adiabatic processes that occur while pressure is constant:
\\
\textbf{Among all states that have the same entropy and pressure values, the state of equilibrium is the one that corresponds to the minimum value of enthalpy.}
\\ \\
If a system relaxes toward equilibrium while the pressure is kept constant, the maximum heat that can be produced by the system is equal to its variation in enthalpy. For this reason, enthalpy it is also called free heat.
The differential of $H$ is given by
$dH = TdS + VdP + \sum_{i=2}^r f_i dX_i$
\\ \\
The equality of the mixed derivatives of $G$ and $H$ yield two more Maxwell relations:
\[\left. -\frac{\partial S}{\partial P} \right|_{T,X_2,\cdots,X_r} = \left. \frac{\partial V}{\partial T} \right|_{P,X_2,\cdots,X_r}\]

\[\left. \frac{\partial T}{\partial P} \right|_{S,X_2,\cdots,X_r} = \left. \frac{\partial V}{\partial S} \right|_{P,X_2,\cdots,X_r}\]

\subsection{Other Thermodynamic Potentials}
The Legendre transform of $F$ with respect to $N$ produces a thermodynamic potential (often written as $\Omega$) that depends on $T$, on volume $V$, on chemical potential $\mu$, and on the other extensive variables:
\[\Omega(T,V,\mu) = F - \mu N\]
Its differential is expressed as follows:
\[d\Omega = -SdT - PdV - Nd\mu\]
If one transforms $U$ instead, one obtains a rarely used potential that depends on $S$, $V$, and $\mu$, which we will designate as $\Phi$:
\[\Phi(S,V,\mu) = U - \mu N\]
Its differential is given by
\[d\Phi = TdS - PdV - Nd\mu\]
We have the following Maxwell relations
\[\left. \frac{\partial S}{\partial \mu} \right|_{T,V} = \left. \frac{\partial N}{\partial T} \right|_{\mu,V}\]
\[\left. \frac{\partial S}{\partial N} \right|_{T,V} = - \left. \frac{\partial \mu}{\partial T} \right|_{N,V}\]

\section{The Euler and Gibbs-­Duhem Equations}
\[U(\lambda S, \lambda X_1, \cdots, \lambda X_r) = \lambda U(S,X_1,\cdots,X_r)\]
By taking the derivative of this equation with respect to $\lambda$ and setting $\lambda = 1$, we obtain the Euler equation
\[TS + \sum_{i=1}^{r} f_i X_i = U\]
More particularly, for simple fluids, one obtains
\[U = TS - PV + \mu N\]
which among other things implies that
\[\mu = (U - TS + PV)/N = G/N\]
\[\Omega = U - TS - \mu N = -PV\]
More particularly, from the Euler equation, it follows that the Legendre transform of $U$ with respect to all extensive variables vanishes identically. Let us note that the interpretation of the chemical potential as a per particle density of Gibbs free energy is valid only in the case of simple fluids—in the case of a mixture of several chemical species, it is no longer valid.
If we take the Euler equation's differential, we obtain
\[dU = TdS + SdT + \sum_{i=1}^r f_idX_i + X_i df_i\]
By subtracting both sides of this equation from the usual expression of $dU$, we obtain the Gibbs-­Duhem equation:
\[SdT + \sum_{i=1}^{r}X_idf_i = 0\]
In the case of simple fluids, for example, one arrives at
\[SdT -VdP + Nd\mu = 0\]
By dividing with respect to the number of particles $N$, one obtains the Gibbs-­Duhem equation in the form
\[d\mu = vdP -sdT\]
where $v$ represents volume per particle and $s$ entropy per particle.
\\ \\
Relations between the densities and the intensive variables obtained by deriving the fundamental equation are called equations of state.
If, for example, we consider the Gibbs free energy for a simple fluid, we arrive at
\[ V = \left. \frac{\partial G}{\partial P} \right|_{T,N}\]
from which we obtain
\[v = \frac{V}{N} = v(P,T)\]
where we have made use of the fact that $G$ is extensive, and therefore proportional to $N$.
In the case of the simple fluid, we have another independent equation of state:
\[s = \frac{S}{N} = -\frac{1}{N}  \left. \frac{\partial G}{\partial T} \right|_{P}\]
which expresses the entropy per particle $s$ as a function of $P$ and $T$. In reality, the two equations of state are not completely independent, because of the Maxwell relations:
\[\left. -\frac{\partial s}{\partial P} \right|_{T} = \left. \frac{\partial v}{\partial T} \right|_{P}\]
The ideal gas is a simple fluid that satisfies the equation of state
\[P = \frac{NT}{V}\]
Maxwell relations allow us to prove that in an ideal gas, the internal energy only depends on $N$ and $T$ (and not on the volume $V$). Moreover, entropy is the sum of a term that depends only on $T$ with one that depends only
on $V$.

\section{Thermodynamic systems with multi-components}
\subsection{Chemical Reactions}
Let us now consider a mixture of $r$ chemical species, $A_1,\cdots,A_r$ , which can be transformed into one other by a reaction of the following type:
\[\nu_1A_1+\cdots+\nu_kA_k \leftrightarrows \nu_{k+1}A_{k+1}+\cdots+\nu_rA_r\]
We can conventionally assign negative stoichiometric coefficients to the products, so as to write this formula as a formal equation:
\[\sum_{i=1}^r \nu_iA_i = 0\]
If temperature and pressure are kept fixed, we can calculate the variation of Gibbs free energy for a certain variation in the number of particles due to the reaction
\[\delta G = \sum_i \left. \frac{\partial G}{\partial N_i} \right|_{P,T} \delta N_i \propto \sum_i \left. \frac{\partial G}{\partial N_i} \right|_{P,T} \nu_i = \sum_{i} \mu_i\nu_i\]
Since at equilibrium one must have $\delta G = 0$ for any virtual variation of the $N_i$, one will have
\[\sum_{i} \mu_i\nu_i = 0\]
where the stoichiometric coefficients $\nu_i$ are taken with their corresponding signs.

\subsection{Phase Coexistence}
It frequently happens that two systems characterized by different thermodynamic density values can maintain thermodynamic equilibrium even in the absence of constraints on the mutual exchange of extensive quantities. This situation is called phase coexistence.
\\
In the case of a simple fluid, it is realized, for example, when a liquid coexists with its vapor inside a container. In this case, the intensive variables assume the same value in both systems, while densities assume different values. In these cases, we refer to each of the coexisting homogeneous systems as a phase.
\\
One can describe phase coexistence by saying that the equation of state
\[v = v(P,T)\]
does not admit of a unique solution, but instead allows for at least the two solutions $v = v_{\mathrm{liq}}$ and $v = v_{\mathrm{vap}}$ which correspond to the liquid and vapor, respectively.
\\
Since the liquid and vapor coexist and can exchange particles, the chemical potential of the liquid has to be equal to that of the vapor:
\[\mu_{\mathrm{liq}}(P,T) = \mu_{\mathrm{vap}}(P,T)\]
On the other hand, we know that for a simple fluid, the chemical potential is equal to the Gibbs free energy per particle. We can verify that the Gibbs free energy in the total system does not depend on the number of particles that make up the liquid and the vapor system:
\[G= G_{\mathrm{liq}} + G_{\mathrm{vap}} = N_{\mathrm{liq}}\mu_{\mathrm{liq}} + N_{\mathrm{vap}}\mu_{\mathrm{vap}} = (N_{\mathrm{liq}}+N_{\mathrm{vap}})\mu = N\mu\]
The volume per particle of the system is given by
\[v = \frac{V_{\mathrm{liq}}+V_{\mathrm{vap}}}{N} = \frac{N_{\mathrm{liq}}v_{\mathrm{liq}} + N_{\mathrm{vap}}v_{\mathrm{vap}}}{N} = x_{\mathrm{liq}}v_{\mathrm{liq}}+x_{\mathrm{vap}}v_{\mathrm{vap}}\]
where $x_{\mathrm{liq}}$ is the fraction of particles in the liquid and $x_{\mathrm{vap}} = 1 - x_{\mathrm{liq}}$ that of the particles in the vapor. As a consequence, the value of $v$ lies somewhere between $v_{\mathrm{liq}}$ and $v_{\mathrm{vap}}$.

\subsection{The Clausius-­Clapeyron Equation}
If we consider a thermodynamic system as a function of its intensive variables, we can identify some regions in which the thermodynamic properties vary regularly with variations of their values. These regions represent thermodynamically stable phases and are limited by curves that represent phase transitions. The phase transitions can be discontinuous, like the phase coexistence we just discussed, or continuous. In the first case, the densities present a discontinuity at the transition, while in the second, they vary with continuity, even though their derivatives can exhibit some singularities. One often also employs the following terminology: discontinuous transitions are called first order transitions, while continuous ones are called second order transitions.
\\
In the case of a simple fluid, it is possible to identify the transition curve within the plane of the intensive variables $(P,T)$ - in other words, the curve $P= P_t(T)$ - from the condition of equality of the chemical potential $\mu$ between the two coexisting phases:
\[\mu_{\mathrm{liq}}(P_t(T),T) = \mu_{\mathrm{vap}}(P_t(T),T)\]
The pressure $P_t(T)$ at which the transition occurs is also called vapor pressure. It is possible to relate this curve locally with the discontinuity of densities at transition. To obtain this relation, let us take the total derivative of this equation with respect to $T$, along the transition line $P_t(T)$. We obtain
\[\left. \frac{\partial \mu_{\mathrm{liq}}}{\partial P} \right|_{T} \frac{dP_t}{dT} + \left. \frac{\partial \mu_{\mathrm{liq}}}{\partial T} \right|_{P} = \left. \frac{\partial \mu_{\mathrm{vap}}}{\partial P} \right|_{T} \frac{dP_t}{dT} + \left. \frac{\partial \mu_{\mathrm{vap}}}{\partial T} \right|_{P}\]
Therefore,
\[\frac{dP_t}{dT} = \frac{s_{\mathrm{vap}}-s_{\mathrm{liq}}}{v_{\mathrm{vap}}-v_{\mathrm{liq}}}\]
This equation, which can be applied to each case of phase coexistence, is called the Clausius - Clapeyron equation.

\subsection{The Coexistence Curve}
We can represent the phase diagram in the plane $(v,T)$, in which the intensive variable $T$ is accompanied by the density $v$, the volume per particle. In this manner, phase coexistence is represented by the existence of a forbidden region $v_{\mathrm{liq}}(T) < v < v_{\mathrm{vap}}(T)$ in the plane. Outside this region, it is possible to obtain any given value of $v$ in a homogeneous system. Within this region, instead, the system separates into a liquid and a vapor phase. The $x_{\mathrm{liq}}$ fraction of particles in the liquid phase (and the analogous fraction in the vapor phase) are determined by the condition that the entire system's volume per particle be equal to $v$. One thus obtains
\[x_{\mathrm{liq}} = \frac{v_{\mathrm{vap}}-v}{v_{\mathrm{vap}}-v_{\mathrm{liq}}}\]
This result is known as the lever rule.

\subsection{Coexistence of Several Phases}
Let us now consider a mixture of particles belonging to $r$ different chemical species. Let us suppose that we are looking for the coexistence of $q$ phases. At equilibrium, all the intensive variables must assume the same value in the coexisting phases. We will therefore have a specific value for the pressure and the temperature, and in addition the chemical potential of each species will have to assume the same value in all the different phases. If we denote the chemical potential of species $i$ in phase $\alpha$ as $\mu_i^{\alpha}$ , we will then have
\[\mu_{i}^{\alpha} = \mu_i \quad i = 1,\cdots,r \quad \alpha = 1,\cdots,\]
In this equation, $\mu_i$ is the shared value taken by the chemical potential of species $i$. We
thus obtain $r(q-1)$ equations for $q(r-1)+2$ unknown values. These unknown values are $P$, $T$, and the $q(r-1)$ independent densities $x_i^{\alpha}$ of species $i$ in phase $\alpha$. Generically speaking, therefore, $f = 2 - q + r$ free parameters remain. For $f = 0$, coexistence will occur in isolated points of the phase diagram, for $f = 1$, along a line, and so on. The quantity $f$ is called variance.  It is called the Gibbs phase rule.

\subsection{The Critical Point}
Let us once again consider the simple fluid and observe that the coexistence of liquid and vapour cannot be obtained for temperatures higher than a certain temperature $T_c$, called the critical temperature. To be more precise, the transition curve ends at a point $(P_c,T_c)$, where $P_c$ is the critical pressure. For $T < T_c$, the difference $v_{\mathrm{vap}}-v_{\mathrm{liq}}$ tends continuously toward zero when $T$ gets close to $T_c$ - the discontinuity of thermodynamic densities tends to vanish, or in other words, the transition goes from being discontinuous to being continuous (and finally to disappear at higher temperatures).
\\
The critical point is a thermodynamic state with exceptional properties. For example, since for $T < T_c$ and $P = P_t(T)$ one gets $\partial P / \partial V |_{T} = 0$ (within the coexistence curve), this relation must ultimately be valid also at the critical point - in other words, the system's compressibility diverges:
\[\chi = - \frac{1}{V} \left. \frac{\partial V}{\partial P} \right|_{T} \to \infty \mbox{ for } T \to T_c \]
Various thermodynamic properties exhibit analogous singularities.

\chapter{Principles of Statistical Mechanics and Ensembles}
\section{Density matrix}
In quantum mechanics, the state of a system is a vector in Hilbert space, denoted as $|\psi\rangle$. A physical observable is an operator on this Hilbert space, denoted as $O$. The expectation value of the measurement of the observable is $\langle \psi | O | \psi \rangle$. It is easy to verify that
\[\langle \psi | O | \psi \rangle = \mathrm{Tr}(| \psi \rangle \langle \psi | O)\]
Now, consider a system whose space of states is the direct product of two subspace, i.e.
\[\mathcal{H} = \mathcal{H}_A \otimes \mathcal{H}_B\]
An arbitrary state can be decomposed as
\[|\psi\rangle = C_{iI}|i\rangle_A \otimes |I\rangle_B\]
So, we have
\[| \psi \rangle \langle \psi | = C_{iI}C^*_{jJ} |i,I\rangle \langle j,J |\]
We define the partial trace of $| \psi \rangle \langle \psi |$ on B as
\[\mathrm{Tr}_B (| \psi \rangle \langle \psi |) \equiv \sum_{I} C_{iI}C^*_{jI} |i\rangle \langle j |\]
It is an operator on $\mathcal{H}_A$. 
Now, suppose there is an observable which measures only on A, i.e.
\[\langle i,I | O | j,I \rangle = \delta_{IJ} \langle i | O_A | j \rangle\]
where $O_A$ is an operator on $\mathcal{H}_A$. We can verify that
\[\langle \psi | O | \psi \rangle = \mathrm{Tr}(| \psi \rangle \langle \psi | O) = \mathrm{Tr}_A \left[\mathrm{Tr}_B (| \psi \rangle \langle \psi |) O_A   \right]\]
Now, if we take A as the system and B the environment, a piratical observable measures only on system. For any system which is coupled to environment, its state can be described by an operator
\[\rho = \mathrm{Tr}_{\mathrm{env}} (| \psi \rangle \langle \psi |)\]
So the expectation value of the measurement on the system is
\[\mathrm{Tr}[\rho O_{\mathrm{sys}}]\]
We can verify that
\[\mathrm{Tr}\rho = 1 ,\quad \rho^{\dagger} = \rho\]
and any eigenvalue of $\rho$ must lie between $0$ and $1$. 
Suppose $rho$ can be diagonalized as
\[\rho = \sum_i p_i |i\rangle\langle i |\]
We have
\[\mathrm{Tr}[\rho O] = p_i \langle i | O | i \rangle\]
So it is reasonable to assume $p_i$ as the (classical) probability of the system in (pure) state $|i\rangle$.
One fundamental postulate of statistical mechanics is that the entropy operator of the system is
\[\hat{S} = -\ln \rho\]
So, the expectation value of entropy is
\[S = \mathrm{Tr}[-\rho\ln\rho]\]

\section{Statistical ensemble}
\subsection{Micro-canonical ensemble}
Micro-canonical ensemble describes a system which is weakly coupled to the environment. The volume $V$ and the number of the particles $N$ are fixed. The energy of the system lies in a narrow range between $E-\Delta E$ and $E + \Delta E$. The total number of distinct microstates accessible to a system is then denoted by the symbol $\Gamma(V,N,E;\Delta)$ and, by assumption, any one of these microstates is as likely to occur as any other. 
\\
Accordingly, the density matrix in the energy representation will be of the form
\[\rho_{mn} = \rho_m \delta_{mn}\]
with
\[\rho_n = \begin{cases} 1/\Gamma \mbox{ for each of the accessible states} \\ 0 \mbox{ for all other states} \end{cases}\]
The entropy of the system is
\[S = \ln \Gamma\]

\subsection{Canonical ensemble}
Canonical ensemble describes a system which can exchange energy with the environment. The density matrix of the system is
\[\rho = \frac{e^{-\beta H}}{\mathrm{Tr}[e^{-\beta H}]}\]
Now, we define
\[Z(\beta,V,N) \equiv \mathrm{Tr}[e^{-\beta H}] \quad F(\beta,V,N) \equiv -\ln Z/\beta\]
The energy of the system is
\[U = \mathrm{Tr}[\rho H] = -\left. \frac{\partial \ln Z}{\partial \beta} \right|_{V,N}\]
If we further define $T \equiv 1/\beta$, we have
\[U = F - T \left. \frac{\partial F}{\partial T} \right|_{V,N}\]
The entropy of the system is
\[S = \mathrm{Tr}[-\rho\ln\rho] = \frac{U-F}{T}\]
We can further derive that
\[\left. \frac{\partial U}{\partial S}\right|_{V,N} = T\]
Now, we can identify $T$ as the absolute temperature and $F$ as free energy in thermodynamics.

\subsection{Grand-canonical ensemble}
Grand canonical ensemble describes a system which can exchange energy and particles with the environment. The density matrix of the system is
\[\rho = \frac{e^{-\beta (H - \mu N)}}{\mathrm{Tr}[e^{-\beta (H - \mu N)}]}\]
Now, we define
\[Z_{\Omega}(\beta,V,\mu) \equiv \mathrm{Tr}[e^{-\beta (H-\mu N)}] \quad \Omega(\beta,V,N) \equiv -\ln Z_{\Omega}/\beta\]
The particle number of the system is
\[N = \mathrm{Tr}[\rho N] = \frac{1}{\beta}\left. \frac{\partial \ln Z_{\Omega}}{\partial \mu} \right|_{V,\beta} = -\left. \frac{\partial \Omega}{\partial \mu} \right|_{V,T}\]
We also have
\[U - \mu N = -\left. \frac{\partial \ln Z_{\Omega}}{\partial \beta} \right|_{V,\mu} = \Omega - T\left. \frac{\partial \Omega}{\partial T} \right|_{V,\mu} \]
The entropy of the system is
\[S = \mathrm{Tr}[-\rho\ln\rho] = \frac{U-\mu N - \Omega}{T}\]
We can further derive that
\[\left. \frac{\partial U}{\partial N}\right|_{V,S} = \mu\]
Now, we can identify $\mu$ as the chemical potential and $\Omega$ as grand canonical potential in thermodynamics.

\section{Fluctuations}
\subsection{Canonical Ensemble}
The density matrix for canonical ensemble is
\[\rho = \frac{e^{-\beta H}}{\mathrm{Tr}[e^{-\beta H}]}\]
We have
\[\left. \frac{\partial \rho}{\partial \beta} \right|_{N,V} = -\rho H + \rho \mathrm{Tr}[\rho H]\]
Since $U = \mathrm{Tr}[\rho H]$, we can derive that
\[\left. \frac{\partial U}{\partial \beta} \right|_{N,V} = -\mathrm{Tr}[\rho H^2] + (\mathrm{Tr}[\rho H])^2 = -\langle E^2 \rangle + \langle E \rangle^2 = -\langle (\Delta E)^2 \rangle\]
So, the relative fluctuation of energy in canonical ensemble is
\[\frac{\sqrt{\langle (\Delta E)^2 \rangle}}{\langle E \rangle} = \frac{T}{U}\sqrt{\left. \frac{\partial U}{\partial T} \right|_{N,V}} = T \frac{\sqrt{C_V}}{T} \sim O(N^{-1/2})\]
For large $N$ (which is true for every statistical system) the relative r.m.s. fluctuation in the values of $E$ is quite negligible.

\subsection{Grand canonical ensemble}
Density and energy fluctuations in the grand canonical ensemble is much more complicated. The detailed discussion can be found in section 4.5 of \emph{Statistical Mechanics (R.K.Pathria \& Paul D.Beale)}. 
\\
The density fluctuation is
\[\frac{\langle (\Delta n)^2 \rangle}{\langle n \rangle^2} = \frac{T}{V}\kappa_T\]
where $ n = \frac{N}{V}$ is the number density and $\kappa_T = -\frac{1}{v} \left. \frac{\partial v}{\partial P} \right|_{T}$ is the isothermal compressibility of the system.
\\ \\
Thus, the relative root-mean-square fluctuation in the particle density of the given system is ordinarily $O(N^{-1/2})$ and, hence, negligible. However, there are exceptions, like the ones met with in situations accompanying phase transitions. In those situations, the compressibility of a given system can become excessively large, as is evidenced by an almost "flattening" of the isotherms. In the region of phase transitions, especially at the critical points, we encounter unusually large fluctuations in the particle density of the system. Such fluctuations indeed exist and account for phenomena like critical opalescence. It is clear that under these circumstances the formalism of the grand canonical ensemble could, in principle, lead to results that are not necessarily identical to the ones following from the corresponding canonical ensemble. In such cases, it is the formalism of the grand canonical ensemble that will have to be preferred because only this one will provide a correct picture of the actual physical situation.
\\ \\
The energy fluctuation in grand canonical ensemble is 
\[\langle (\Delta E)^2 \rangle = T^2C_V + \left(\left. \frac{\partial U}{\partial N} \right|_{T,V} \right)^2 \langle (\Delta N)^2 \rangle\]
The mean-square fluctuation in the energy of a system in the grand canonical ensemble is equal to the value it would have in the canonical ensemble plus a contribution arising from the fact that now the particle number $N$ is also fluctuating. Again, under ordinary circumstances, the relative root-mean-square fluctuation in the energy density of the system would be practically negligible. However, in the region of phase transitions, unusually large fluctuations in the value of this variable can arise by virtue of the second term in the formula.

\chapter{Interaction-­Free Systems}

\chapter{Quantum field theory approach}

\chapter{Phase Transitions and the Renormalization Group}

\end{document}

\end{document}