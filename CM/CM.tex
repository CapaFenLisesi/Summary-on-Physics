\documentclass{article}
\usepackage[left=1.5cm, right=1.5cm, top=3cm, bottom = 3cm]{geometry}
\usepackage{amsmath}
\usepackage{mathrsfs}
\usepackage{amsfonts}
\usepackage{amssymb}
\usepackage{graphicx}
\usepackage{float}
\usepackage{wrapfig}
\usepackage{latexsym}
\usepackage{hyperref}
\usepackage{feynmf}
\usepackage{exscale}
\usepackage{relsize}
\usepackage{bm}%bold math, for vector
\linespread{1.1}


\author{Yuyang Songsheng}
\title{Summary on Classical Mechanics}

\begin{document}
\maketitle
\section{Lagrangian Formulation}
\[S=\int_{t_1}^{t_2}L(q_i,\dot{q_i},t)dt, \ \ \ \ \delta q_i(t_1) = \delta q_i(t_2) = 0\]
\[\delta S=0 \rightarrow \frac{d}{dt}(\frac{\partial L}{\partial \dot{q_i}}) - \frac{\partial L}{\partial q_i}=0\]
\subsection{Example}
The form of Lagrangian for a system of particles in inertial frame:
\[L=\sum_a \frac{1}{2}m_a v_a^2 -U(\vec{r_1},\vec{r_2},\cdots,)\]
To get the form of Lagrangian for a system of interacting particles, we must assume:\\
(1) Space and time are homogeneous and isotropic in inertial frame;\\
(2) Galileo's relativity principle and Galilean transformation;\\
(3) Spontaneous interaction between particles;\\

\section{Symmetry and Conservation Laws(1)}
\subsection{Nother's theorem}
For $q_i \to q_i+\delta q_i$ and $L \to L+\delta L$, if $\delta L= \frac{d f(q,\dot{q},t)}{dt}$,then we get
\[\frac{d}{dt}(p^i \delta q_i-f)=0 \ \ \ (p^i=\frac{\partial L}{\partial \dot{q_i}})\]

\subsection{Homogeneity of time}
If $\frac{\partial L}{\partial t}=0$,then we get
\[\frac{dH}{dt}=0 \ \ \ (H=\sum_i \dot{q_i}p^i-L)\]

\section{Hamilton formulation}
\[p^i = \frac{\partial L}{\partial \dot{q_i}}\]
\[H(q,p,t)=\sum_i p^i \dot{q_i}-L\]
\[\dot{p^i}=-\frac{\partial H}{\partial q_i} \ \ \ \ \ \ \dot{q_i}=\frac{\partial H}{\partial p^i}\]
 
\subsection{Poisson Brackets}
First, we assume the bracket operation has the following properties:
\[ \left[f,g\right]=-\left[g,f\right] \]
\[\left[\alpha_1 f_1+\alpha_2 f_2,\beta_1 g_1+\beta_2 g_2\right]=\alpha_1 \beta_1\left[f_1,g_1\right]
+\alpha_1 \beta_2\left[f_1,g_2\right]+\alpha_2 \beta_1\left[f_2,g_1\right]+\alpha_2 \beta_2\left[f_2,g_2\right]\]
\[\left[f_1 f_2,g_1 g_2\right]=f_1\left[f_2,g_1\right]g_2+f_1 g_1\left[f_2,g_2\right]+g_1\left[f_1,g_2\right]f_2 +\left[f_1,f_2\right]g_2 f_2 \]
\[\left[f,\left[g,h\right]\right]+\left[g,\left[h,f\right]\right]+\left[h,\left[f,g\right]\right]=0\]
Here, $f,g,h$ are functions of $p^i,q_i,t$.
Then, if we assume that
\[\left [q_i,p^k\right ]=\delta^{k}_{i}\]
we can deduce that 
\[ \left[f,g\right]=\sum_k(\frac{\partial f}{\partial q_k} \frac{\partial g}{\partial p^k} - \frac{\partial f}{\partial p^k} \frac{\partial g}{\partial q_k}  )\]
The Hamilton equation can be written as
\begin{equation}
\dot{p^i}=\left[ p^i,H \right] \ \ \ \ \dot{q_i}=\left[ q_i,H \right]
\end{equation}
And we can also derive that $\frac{df}{dt} = [f,H]$.

\section{Symmetry and Conservation Laws(2)}
Suppose $g$ is a function of $p$ and $q$. If the transformation of $q$ and $p$ can be described as
\[q \rightarrow q + \epsilon [q,g]\]
\[p \rightarrow p + \epsilon [p,g]\]
We can prove that 
\[H \rightarrow H + \epsilon[H,g]\]
So if $H$ is invariant under the transformation, then $[H,g] = 0$, that means $\frac{dg}{dt} = 0$, i.e. $g$ is a conserved quantity of the motion.

\section{Hamilton-Jacobi equation}
We define
\[S(q,t)=\left(\int_{q_0,t_0}^{q,t} L dt\right)|_{extremum}\]
We can prove that
\[p = \frac{\partial S}{\partial q}, \ \ \ \ H = -\frac{\partial S}{\partial t}\]
So, we have
\[-\frac{\partial S}{\partial t} = H (q,\frac{\partial S}{\partial q})\]
This is called Hamiltonian-Jacobi equation.

\section{Symmetry and Conservation Laws(3)}
If $S$ is invariant under transformation
$q_i \rightarrow q_i + \delta q_i$, then 
\[\delta S = (\sum_i p^i \delta q_i) |_{q_0,t_0}^{q,t} = 0\]
So, we have
\[\frac{d}{dt} (p^i \delta q_i) = 0\]
Further more, if
\[\delta S = (\sum_i p^i \delta q_i) |_{q_0,t_0}^{q,t} =  f(q_i,\dot{q}_i,t)|_{q_0,t_0}^{q,t}\]
we will have conserved quantity
\[\frac{d}{dt} (p^i \delta q_i -f) = 0\]
\end{document}