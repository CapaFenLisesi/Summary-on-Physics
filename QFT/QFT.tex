\documentclass{article}
\usepackage[left=1.5cm, right=1.5cm, top=3cm, bottom = 3cm]{geometry}
\usepackage{amsmath}
\usepackage{mathrsfs}
\usepackage{amsfonts}
\usepackage{amssymb}
\usepackage{graphicx}
\usepackage{float}
\usepackage{wrapfig}
\usepackage{latexsym}
\usepackage{hyperref}
\usepackage{feynmf}
\usepackage{exscale}
\usepackage{relsize}
\usepackage{bm}%bold math, for vector
\linespread{1.1}


\author{Yuyang Songsheng}
\title{Summary on QFT}

\begin{document}
\maketitle
\section{From classical field to quantum field}
\subsection{Heisenberg picture of fields}
The state of the field is described by an element $|\psi\rangle$ in Hilbert space. The measurement of the field is described by an operator field $\phi_a(\vec{x},t)$. In Heisenberg picture, the dynamic of the field satisfy the equation
\[\frac{d\phi_a(x)}{dt} = -i[\phi_a(x),H]\]
So, the mean value of the measurement of the field is described by Erenfest theorem
\[\frac{d\langle \psi| \phi_a | \psi \rangle}{dt} = -i \langle \psi | [\phi_a,H] | \psi \rangle\]
If $[\phi_a,H]_Q = i[\phi_a,H]_C$, we can reproduce the classical field equation. We also note that the bracket operation here $[A,B] = AB - BA$ has the same properties as the poission bracket in classical mechanics. So, what we need here is the canonical quantization
\[[\phi_a(\vec{x},t),\phi_b(\vec{y},t)] = 0 \quad [\pi^a(\vec{x},t),\pi^b(\vec{y},t)] = 0 \quad [\phi_a(\vec{x},t),\pi^b(\vec{y},t)] = i \delta_a^b \delta(\vec{x}-\vec{y}) \]
and the definition of $\mathcal{L}$,$\pi^a$ and $H$ is the same as those in corresponding classical theory. Then we can recover the classical field theory.

\subsection{Lorentz invariance in quantum field theory}
\[| \bar{\psi}\rangle = U(\Lambda)| \psi\rangle\]
Scalar fields:
\[\langle \bar{\psi} | \phi(x) | \bar{\psi}\rangle = \langle \psi | \phi(\Lambda^{-1}x) | \psi\rangle\]
\[U^{-1}(\Lambda) \phi(x) U(\Lambda) = \phi(\Lambda^{-1}x)\]
Vector fields:
\[\langle \bar{\psi} | A^{\mu}(x) | \bar{\psi}\rangle = \langle \psi | \Lambda^{\mu}_{\phantom{\mu}\nu} A^{\nu}(\Lambda^{-1}x) | \psi\rangle\]
\[U^{-1}(\Lambda) A^{\mu}(x) U(\Lambda) = \Lambda^{\mu}_{\phantom{\mu}\nu} A^{\nu}(\Lambda^{-1}x)\]
\paragraph{Lorentz invariance} Lagrangian is a scalar, or more loosely, action is invariant under Lorentz transformation.

\subsection{Momentum}
The definition of momentum is the same as that in classical theory.
\[T^{\mu \nu} \equiv -\frac{\partial \mathcal{L}}{\partial(\partial_{\mu}\phi_a)} \partial^{\nu} \phi_a + \eta^{\mu \nu} \mathcal{L} \quad \partial_{\mu} T^{\mu \nu} = 0\]
and
\[P^{\mu} = \int T^{0 \mu} d^3 x \quad \frac{d P^{\mu}}{dt} = 0\]
\[P^{0} = H, \quad P^{i} = \int -\pi^a \partial^i \phi_a d^3 x\]
And we can get the commutation relationship that
\begin{eqnarray}
\left[\phi_a,P^{\mu}\right] &=& -i\partial^{\mu} \phi_a \nonumber \\
\left[\pi^a,P^{\mu}\right] &=& -i\partial^{\mu} \pi^a \nonumber \\
\left[P^{\mu},P^{\nu}\right] &=& 0 \nonumber 
\end{eqnarray}
We denote the translation operator as $T(s)$, so
\[T^{-1}(s) \phi_a(x) T(s) = \phi_a(x-s)\]
we can deduce that
\[T(\epsilon) = 1 - i\epsilon_{\mu} P^{\mu} \quad T(s) = e^{-iP^{\mu}s_{\mu}}\]


\subsection{Angular Momentum}
The definition of Angular momentum is the same as that in classical theory.
\[M^{\mu \nu \rho} \equiv x^{\nu}T^{\mu \rho} - x^{\rho} T^{\mu \nu} - \frac{\partial \mathcal{L}}{\partial (\partial_{\mu}\phi_a)}(\Sigma^{\nu \rho})_{a}^{\phantom{a}b}\phi_b\]
and 
\[M^{\nu \rho} = \int M^{0 \nu \rho} d^3 x \quad \frac{dM^{\nu \rho}}{dt} = 0\]
\[M^{\mu \nu} = \int (x^{\mu}T^{0\nu}-x^{\nu}T^{0\mu}-\pi^a(\Sigma^{\mu \nu})_{a}^{\phantom{a}b}\phi_b) d^3 x\]
We denote that
\[M_{L}^{\mu \nu} = \int (x^{\mu}T^{0\nu}-x^{\nu}T^{0\mu}) d^3 x \quad M_S^{\mu \nu} = \int (-\pi^a(\Sigma^{\mu \nu})_{a}^{\phantom{a}b}\phi_b) d^3 x\]
\[(L^{\mu \nu})_a^{\phantom{a}b} = -i(x^{\mu}\partial^{\nu}-x^{\nu}\partial^{\mu})\delta_a^{\phantom{a}b} \quad (S^{\mu \nu})_a^{\phantom{a}b} = -i(\Sigma^{\mu \nu})_a^{\phantom{a}b}\]
And we have the commutation relationship that
\[M^{\mu \nu} = M_L^{\mu \nu} + M_S^{\mu \nu}\]
\[[\phi_a,M_L^{\mu \nu}] = (L^{\mu \nu})_a^{\phantom{a}b} \phi_b \quad [\phi_a,M_S^{\mu \nu}] = (S^{\mu \nu})_a^{\phantom{a}b} \phi_b\]
\[[\pi^a,M_L^{\mu \nu}] = (L^{\mu \nu})_b^{\phantom{b}a}\pi^{b}  \quad [\pi^a,M_S^{\mu \nu}] = - (S^{\mu \nu})_b^{\phantom{b}a} \pi^b \]
\[[M^{\mu \nu},M^{\rho \sigma}] = i(-g^{\nu \rho}M^{\mu \sigma} + g^{\sigma \mu}M^{\rho \nu} + g^{\mu \rho}M^{\nu \sigma} - g^{\sigma \nu}M^{\rho \mu})\]
We again define $J_i \equiv \frac{1}{2} \epsilon_{ijk} M^{jk}$ and $K_i \equiv M^{i0}$, the communication relationship can be written as
\begin{eqnarray}
\left[J_i,J_j\right] &=& i\epsilon_{ijk}J_k \nonumber \\
\left[J_i,K_j\right] &=& i\epsilon_{ijk}K_k \nonumber \\
\left[K_i,K_j\right] &=& -i\epsilon_{ijk}J_k \nonumber
\end{eqnarray}
Further more, 
\[[P^{\mu},M^{\rho \sigma}] = i(g^{\mu \sigma}P^{\mu} - g^{\mu \rho}P^{\sigma})\]
\begin{eqnarray}
\left[J_i,H\right] &=& 0 \nonumber \\
\left[J_i,P_j\right] &=& i\epsilon_{ijk}P_k \nonumber \\
\left[K_i,H\right] &=& iP_i \nonumber \\
\left[K_i,P_j\right] &=& i\delta_{ij}H \nonumber
\end{eqnarray}
At last, we define $L_i \equiv \frac{1}{2} \epsilon_{ijk} M_L^{jk}$ and $S_i \equiv \frac{1}{2} \epsilon_{ijk} M_S^{jk}$. So
\begin{eqnarray}
\left[L_i,S_j\right] &=& 0 \nonumber \\
\left[S_i,P_j\right] &=& 0 \nonumber \\
\left[L_i,P_j\right] &=& i\epsilon_{ijk}P_k \nonumber
\end{eqnarray}
We denote the rotation operator as $U(\Lambda)$, so
\[U^{-1}(\Lambda) \phi_a(x) U(\Lambda) = S_{a}^{\phantom{a}b}\phi_b(\Lambda^{-1}x)\]
and 
\[S_{a}^{\phantom{a}b} = \delta_{a}^{\phantom{a}b}+\frac{i}{2} \delta \omega_{\alpha \beta} (S^{\alpha \beta})_{a}^{\phantom{a}b} \]
we can deduce that
\[U(1+\delta \omega) = 1 + \frac{i}{2} \delta \omega_{\mu \nu} M^{\mu \nu} \quad U(\Lambda) = e^{\frac{i}{2} \theta_{\mu \nu} M^{\mu \nu}}\]
\[U^{-1}(\Lambda) P^{\mu} U(\Lambda) = \Lambda^{\mu}_{\phantom{\mu}\nu} P^{\nu}\]
\[U^{-1}(\Lambda) M^{\mu \nu} U(\Lambda) = \Lambda^{\mu}_{\phantom{\mu}\rho} \Lambda^{\nu}_{\phantom{\nu}\sigma}M^{\rho \sigma}\]

\section{Spin 0 Fields}
\subsection{Canonical quantization of Klein-Gordon fields}
\paragraph{Lagrangian}
\[\mathcal{L} = -\frac{1}{2} \partial^{\mu} \phi \partial_{\mu} \phi -\frac{1}{2}m^2 \phi^2 + \Omega_0\]
\paragraph{Field equation}
\[(\partial^{\mu} \partial_{\mu} - m^2) \phi = 0\]
\paragraph{Hamiltonian}
\[\pi = \dot{\phi}\]
\[\mathcal{H} = \frac{1}{2} \pi^2 + \frac{1}{2} (\nabla \phi)^2 + \frac{1}{2} m^2 \phi^2-\Omega_0\]
\[H = \int \mathcal{H} d^3 x\]
\paragraph{Momentum and angular momentum}
\[T^{\mu \nu} = \partial^{\mu} \phi \partial^{\nu} \phi - \eta^{\mu \nu}(\frac{1}{2}\partial^{\mu}\phi \partial_{\mu} \phi + \frac{1}{2}m^2 \phi^2 -\Omega_0)\]
\[P^0 = H \quad P^i = \int -\pi \nabla^i \phi d^3 x\]
\[J_k = \int - \pi \epsilon_{ijk} x^{j} \nabla^{k} \phi d^3 x\]
\paragraph{Canonical quantization}
\begin{eqnarray}
\left[\phi(\vec{x},t),\phi(\vec{y},t)\right] &=& 0 \nonumber \\
\left[\pi(\vec{x},t),\pi(\vec{y},t)\right] &=& 0 \nonumber \\
\left[\phi(\vec{x},t),\pi(\vec{y},t)\right] &=& i \delta(\vec{x}-\vec{y}) \nonumber
\end{eqnarray}
\paragraph{Fourier expansion}
\[\phi(\vec{x},t) = \int \widetilde{dk} \left[ a(\vec{k})e^{ikx} + a^{\dagger}(\vec{k})e^{-ikx} \right]\]
\[\pi(\vec{x},t) = -i \int  \widetilde{dk} \omega \left[ a(\vec{k})e^{ikx} - a^{\dagger}(\vec{k})e^{-ikx} \right]\]
Here, $k^2 = \mathbf{k}^2 - \omega^2 = -m^2$, $kx = \mathbf{k}\cdot \mathbf{x} - \omega t$, $\widetilde{dk} = \frac{d^3}{(2\pi)^2 2\omega}$
\[a(\vec{k}) = \int d^3 x e^{-ikx}(i\pi+\omega \phi)\]
\[a^{\dagger}(\vec{k}) = \int d^3 x e^{ikx}(-i\pi+\omega \phi)\]
\begin{eqnarray}
\left[a(\vec{p}),a(\vec{q})\right] &=& 0 \nonumber \\
\left[a^{\dagger}(\vec{p}),a^{\dagger}(\vec{q})\right] &=& 0 \nonumber \\
\left[a(\vec{p}),a^{\dagger}(\vec{q})\right] &=& (2\pi)^3 2\omega \delta(\vec{p}-\vec{q}) \nonumber
\end{eqnarray}
\paragraph{Operator represented by $a$ and $a^{\dagger}$}
\[H=\int \widetilde{dk}\, \omega\, a^{\dagger}(\vec{k})a(\vec{k}) + (\mathcal{E}_0 - \Omega_0)V \quad \mathcal{E}_0 = \frac{1}{2}(2\pi)^{-3}\int d^3 k \,\omega\]
\[P^{i}=\int \widetilde{dk}\, k^{i}\, a^{\dagger}(\vec{k})a(\vec{k}) \]
\paragraph{particles}
\[[H,a(\vec{k})] = -\omega a(\vec{k}) \quad [H,a^{\dagger}(\vec{k})] = \omega a^{\dagger}(\vec{k})\]
\[[P^i,a(\vec{k})] = -k^i a(\vec{k}) \quad [P^i,a^{\dagger}(\vec{k})] = k^i a^{\dagger}(\vec{k})\]
Let $|p\rangle = a^{\dagger}(\vec{p})|0\rangle $,so
\[H |p\rangle = \omega_p|p\rangle \quad P^i |p\rangle = p^i|p\rangle\]
So, we interpret the state $|\vec{p}\rangle$ as the momentum eigenstate of a single particle of mass $m$. We can also show that 
$J_i|\vec{p} = 0\rangle = 0$, so the particle carries no internal angular momentum.
\end{document}


