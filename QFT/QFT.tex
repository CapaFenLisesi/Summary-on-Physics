\documentclass{article}
\usepackage[left=1.5cm, right=1.5cm, top=3cm, bottom = 3cm]{geometry}
\usepackage{amsmath}
\usepackage{mathrsfs}
\usepackage{amsfonts}
\usepackage{amssymb}
\usepackage{graphicx}
\usepackage{float}
\usepackage{wrapfig}
\usepackage{latexsym}
\usepackage{caption,subcaption}
\usepackage{hyperref}
\usepackage{feynmf}
\usepackage{exscale}
\usepackage{relsize}
\usepackage{bm}%bold math, for vector
\linespread{1.1}


\author{Yuyang Songsheng}
\title{Summary on QFT}

\begin{document}
\maketitle
\section{From classical field to quantum field}
\subsection{Heisenberg picture of fields}
The state of the field is described by an element $|\psi\rangle$ in Hilbert space. The measurement of the field is described by an operator field $\phi_a(\vec{x},t)$. In Heisenberg picture, the dynamic of the field satisfy the equation
\[\frac{d\phi_a(x)}{dt} = -i[\phi_a(x),H]\]
So, the mean value of the measurement of the field is described by Erenfest theorem
\[\frac{d\langle \psi| \phi_a | \psi \rangle}{dt} = -i \langle \psi | [\phi_a,H] | \psi \rangle\]
If $[\phi_a,H]_Q = i[\phi_a,H]_C$, we can reproduce the classical field equation. We also note that the bracket operation here $[A,B] = AB - BA$ has the same properties as the poission bracket in classical mechanics. So, what we need here is the canonical quantization
\[[\phi_a(\vec{x},t),\phi_b(\vec{y},t)] = 0 \quad [\pi^a(\vec{x},t),\pi^b(\vec{y},t)] = 0 \quad [\phi_a(\vec{x},t),\pi^b(\vec{y},t)] = i \delta_a^b \delta(\vec{x}-\vec{y}) \]
and the definition of $\mathcal{L}$,$\pi^a$ and $H$ is the same as those in corresponding classical theory. Then we can recover the classical field theory.

\subsection{Lorentz invariance in quantum field theory}
\[| \bar{\psi}\rangle = U(\Lambda)| \psi\rangle\]
Scalar fields:
\[\langle \bar{\psi} | \phi(x) | \bar{\psi}\rangle = \langle \psi | \phi(\Lambda^{-1}x) | \psi\rangle\]
\[U^{-1}(\Lambda) \phi(x) U(\Lambda) = \phi(\Lambda^{-1}x)\]
Vector fields:
\[\langle \bar{\psi} | A^{\mu}(x) | \bar{\psi}\rangle = \langle \psi | \Lambda^{\mu}_{\phantom{\mu}\nu} A^{\nu}(\Lambda^{-1}x) | \psi\rangle\]
\[U^{-1}(\Lambda) A^{\mu}(x) U(\Lambda) = \Lambda^{\mu}_{\phantom{\mu}\nu} A^{\nu}(\Lambda^{-1}x)\]
\paragraph{Lorentz invariance} Lagrangian is a scalar, or more loosely, action is invariant under Lorentz transformation.

\subsection{Momentum}
The definition of momentum is the same as that in classical theory.
\[T^{\mu \nu} \equiv -\frac{\partial \mathcal{L}}{\partial(\partial_{\mu}\phi_a)} \partial^{\nu} \phi_a + \eta^{\mu \nu} \mathcal{L} \quad \partial_{\mu} T^{\mu \nu} = 0\]
and
\[P^{\mu} = \int T^{0 \mu} d^3 x \quad \frac{d P^{\mu}}{dt} = 0\]
\[P^{0} = H, \quad P^{i} = \int -\pi^a \partial^i \phi_a d^3 x\]
And we can get the commutation relationship that
\begin{eqnarray}
\left[\phi_a,P^{\mu}\right] &=& -i\partial^{\mu} \phi_a \nonumber \\
\left[\pi^a,P^{\mu}\right] &=& -i\partial^{\mu} \pi^a \nonumber \\
\left[P^{\mu},P^{\nu}\right] &=& 0 \nonumber 
\end{eqnarray}
We denote the translation operator as $T(s)$, so
\[T^{-1}(s) \phi_a(x) T(s) = \phi_a(x-s)\]
we can deduce that
\[T(\epsilon) = 1 - i\epsilon_{\mu} P^{\mu} \quad T(s) = e^{-iP^{\mu}s_{\mu}}\]


\subsection{Angular Momentum}
The definition of Angular momentum is the same as that in classical theory.
\[M^{\mu \nu \rho} \equiv x^{\nu}T^{\mu \rho} - x^{\rho} T^{\mu \nu} - \frac{\partial \mathcal{L}}{\partial (\partial_{\mu}\phi_a)}(\Sigma^{\nu \rho})_{a}^{\phantom{a}b}\phi_b\]
and 
\[M^{\nu \rho} = \int M^{0 \nu \rho} d^3 x \quad \frac{dM^{\nu \rho}}{dt} = 0\]
\[M^{\mu \nu} = \int (x^{\mu}T^{0\nu}-x^{\nu}T^{0\mu}-\pi^a(\Sigma^{\mu \nu})_{a}^{\phantom{a}b}\phi_b) d^3 x\]
We denote that
\[M_{L}^{\mu \nu} = \int (x^{\mu}T^{0\nu}-x^{\nu}T^{0\mu}) d^3 x \quad M_S^{\mu \nu} = \int (-\pi^a(\Sigma^{\mu \nu})_{a}^{\phantom{a}b}\phi_b) d^3 x\]
\[(L^{\mu \nu})_a^{\phantom{a}b} = -i(x^{\mu}\partial^{\nu}-x^{\nu}\partial^{\mu})\delta_a^{\phantom{a}b} \quad (S^{\mu \nu})_a^{\phantom{a}b} = -i(\Sigma^{\mu \nu})_a^{\phantom{a}b}\]
And we have the commutation relationship that
\[M^{\mu \nu} = M_L^{\mu \nu} + M_S^{\mu \nu}\]
\[[\phi_a,M_L^{\mu \nu}] = (L^{\mu \nu})_a^{\phantom{a}b} \phi_b \quad [\phi_a,M_S^{\mu \nu}] = (S^{\mu \nu})_a^{\phantom{a}b} \phi_b\]
\[[\pi^a,M_L^{\mu \nu}] = (L^{\mu \nu})_b^{\phantom{b}a}\pi^{b}  \quad [\pi^a,M_S^{\mu \nu}] = - (S^{\mu \nu})_b^{\phantom{b}a} \pi^b \]
\[[M^{\mu \nu},M^{\rho \sigma}] = i(-g^{\nu \rho}M^{\mu \sigma} + g^{\sigma \mu}M^{\rho \nu} + g^{\mu \rho}M^{\nu \sigma} - g^{\sigma \nu}M^{\rho \mu})\]
We again define $J_i \equiv \frac{1}{2} \epsilon_{ijk} M^{jk}$ and $K_i \equiv M^{i0}$, the communication relationship can be written as
\begin{eqnarray}
\left[J_i,J_j\right] &=& i\epsilon_{ijk}J_k \nonumber \\
\left[J_i,K_j\right] &=& i\epsilon_{ijk}K_k \nonumber \\
\left[K_i,K_j\right] &=& -i\epsilon_{ijk}J_k \nonumber
\end{eqnarray}
Further more, 
\[[P^{\mu},M^{\rho \sigma}] = i(g^{\mu \sigma}P^{\mu} - g^{\mu \rho}P^{\sigma})\]
\begin{eqnarray}
\left[J_i,H\right] &=& 0 \nonumber \\
\left[J_i,P_j\right] &=& i\epsilon_{ijk}P_k \nonumber \\
\left[K_i,H\right] &=& iP_i \nonumber \\
\left[K_i,P_j\right] &=& i\delta_{ij}H \nonumber
\end{eqnarray}
At last, we define $L_i \equiv \frac{1}{2} \epsilon_{ijk} M_L^{jk}$ and $S_i \equiv \frac{1}{2} \epsilon_{ijk} M_S^{jk}$. So
\begin{eqnarray}
\left[L_i,S_j\right] &=& 0 \nonumber \\
\left[S_i,P_j\right] &=& 0 \nonumber \\
\left[L_i,P_j\right] &=& i\epsilon_{ijk}P_k \nonumber
\end{eqnarray}
We denote the rotation operator as $U(\Lambda)$, so
\[U^{-1}(\Lambda) \phi_a(x) U(\Lambda) = S_{a}^{\phantom{a}b}\phi_b(\Lambda^{-1}x)\]
and 
\[S_{a}^{\phantom{a}b} = \delta_{a}^{\phantom{a}b}+\frac{i}{2} \delta \omega_{\alpha \beta} (S^{\alpha \beta})_{a}^{\phantom{a}b} \]
we can deduce that
\[U(1+\delta \omega) = 1 + \frac{i}{2} \delta \omega_{\mu \nu} M^{\mu \nu} \quad U(\Lambda) = e^{\frac{i}{2} \theta_{\mu \nu} M^{\mu \nu}}\]
\[U^{-1}(\Lambda) P^{\mu} U(\Lambda) = \Lambda^{\mu}_{\phantom{\mu}\nu} P^{\nu}\]
\[U^{-1}(\Lambda) M^{\mu \nu} U(\Lambda) = \Lambda^{\mu}_{\phantom{\mu}\rho} \Lambda^{\nu}_{\phantom{\nu}\sigma}M^{\rho \sigma}\]

\section{Spin 0 Fields}
\subsection{Canonical quantization of Klein-Gordon fields}
\paragraph{Lagrangian}
\[\mathcal{L} = -\frac{1}{2} \partial^{\mu} \phi \partial_{\mu} \phi -\frac{1}{2}m^2 \phi^2 + \Omega_0\]
\paragraph{Field equation}
\[(\partial^{\mu} \partial_{\mu} - m^2) \phi = 0\]
\paragraph{Hamiltonian}
\[\pi = \dot{\phi}\]
\[\mathcal{H} = \frac{1}{2} \pi^2 + \frac{1}{2} (\nabla \phi)^2 + \frac{1}{2} m^2 \phi^2-\Omega_0\]
\[H = \int \mathcal{H} d^3 x\]
\paragraph{Momentum and angular momentum}
\[T^{\mu \nu} = \partial^{\mu} \phi \partial^{\nu} \phi - \eta^{\mu \nu}(\frac{1}{2}\partial^{\mu}\phi \partial_{\mu} \phi + \frac{1}{2}m^2 \phi^2 -\Omega_0)\]
\[P^0 = H \quad P^i = \int -\pi \nabla^i \phi d^3 x\]
\[J_k = \int - \pi \epsilon_{ijk} x^{j} \nabla^{k} \phi d^3 x\]
\paragraph{Canonical quantization}
\begin{eqnarray}
\left[\phi(\vec{x},t),\phi(\vec{y},t)\right] &=& 0 \nonumber \\
\left[\pi(\vec{x},t),\pi(\vec{y},t)\right] &=& 0 \nonumber \\
\left[\phi(\vec{x},t),\pi(\vec{y},t)\right] &=& i \delta(\vec{x}-\vec{y}) \nonumber
\end{eqnarray}
\paragraph{Fourier expansion}
\[\phi(\vec{x},t) = \int \widetilde{dk} \left[ a(\vec{k})e^{ikx} + a^{\dagger}(\vec{k})e^{-ikx} \right]\]
\[\pi(\vec{x},t) = -i \int  \widetilde{dk} \omega \left[ a(\vec{k})e^{ikx} - a^{\dagger}(\vec{k})e^{-ikx} \right]\]
Here, $k^2 = \mathbf{k}^2 - \omega^2 = -m^2$, $kx = \mathbf{k}\cdot \mathbf{x} - \omega t$, $\widetilde{dk} = \frac{d^3}{(2\pi)^2 2\omega}$
\[a(\vec{k}) = \int d^3 x e^{-ikx}(i\pi+\omega \phi)\]
\[a^{\dagger}(\vec{k}) = \int d^3 x e^{ikx}(-i\pi+\omega \phi)\]
\begin{eqnarray}
\left[a(\vec{p}),a(\vec{q})\right] &=& 0 \nonumber \\
\left[a^{\dagger}(\vec{p}),a^{\dagger}(\vec{q})\right] &=& 0 \nonumber \\
\left[a(\vec{p}),a^{\dagger}(\vec{q})\right] &=& (2\pi)^3 2\omega \delta(\vec{p}-\vec{q}) \nonumber
\end{eqnarray}
\paragraph{Operator represented by $a$ and $a^{\dagger}$}
\[H=\int \widetilde{dk}\, \omega\, a^{\dagger}(\vec{k})a(\vec{k}) + (\mathcal{E}_0 - \Omega_0)V \quad \mathcal{E}_0 = \frac{1}{2}(2\pi)^{-3}\int d^3 k \,\omega\]
\[P^{i}=\int \widetilde{dk}\, k^{i}\, a^{\dagger}(\vec{k})a(\vec{k}) \]
\paragraph{Particles}
\[[H,a(\vec{k})] = -\omega a(\vec{k}) \quad [H,a^{\dagger}(\vec{k})] = \omega a^{\dagger}(\vec{k})\]
\[[P^i,a(\vec{k})] = -k^i a(\vec{k}) \quad [P^i,a^{\dagger}(\vec{k})] = k^i a^{\dagger}(\vec{k})\]
Let $|p\rangle = a^{\dagger}(\vec{p})|0\rangle $,so
\[H |p\rangle = \omega_p|p\rangle \quad P^i |p\rangle = p^i|p\rangle\]
So, we interpret the state $|\vec{p}\rangle$ as the momentum eigenstate of a single particle of mass $m$. We can also show that 
$J_i|\vec{p} = 0\rangle = 0$, so the particle carries no internal angular momentum.
\paragraph{Causality}
The amplitude for a particle to propagate from $y$ to $x$ is $\langle 0 | \phi(x) \phi(y) | 0 \rangle$, denoted by $D(x-y)$.
\[D(x-y) = \int \widetilde{dk} e^{ik(x-y)}\]
\[[\phi(x),\phi(y)] = D(x-y) -D(y-x)\]
If $x-y$ is space-like, a continuous Lorentz transformation can take $(x-y)$ to $-(x-y)$. So $[\phi(x),\phi(y)] =0$ for space-like $x-y$. A measurement performed at one point can not affect a measurement at another point whose separation is space-like.
\paragraph{The Klein-Gordon propagator}
\[D_R(x-y) \equiv \theta(x^0-y^0) \langle 0 | \phi(x) \phi(y) | 0 \rangle = \int \frac{d^4 p}{(2\pi)^4} \frac{-i}{p^2+m^2} e^{ip(x-y)}\]
\begin{figure}[!h]
\centering
\includegraphics[height=3cm ,width=14cm]{./pic/R_Green.png}
\caption{Retarded Green Function}
\end{figure}
\[(\partial^2-m^2) D_R(x-y) = i \delta(x-y)\]
\[D_F(x-y) \equiv \langle 0 | T\phi(x) \phi(y) | 0 \rangle = \int \frac{d^4 p}{(2\pi)^4} \frac{-i}{p^2+m^2-i\epsilon} e^{ip(x-y)}\]
\begin{figure}[!h]
\centering
\includegraphics[height=3cm ,width=14cm]{./pic/F_Green.png}
\caption{Feynman Green Function}
\end{figure}

\subsection{Perturbation theory}
\[\mathcal{L} = -\frac{1}{2}\partial_{\mu} \phi \partial^{\mu} \phi -\frac{1}{2}m_0^2 \phi^2 -\frac{\lambda_0}{4!}\phi^4\]
\[H = H_0 + H_{int} \quad H_{int} = \int d^3 x \frac{\lambda_0^4}{4!} \phi^4 (\vec{x})\]

\subsubsection{Perturbation expansion of correlation functions}
\[\phi(t_0,\vec{x}) = \int \frac{d^3p}{(2\pi)^3}( a(\vec{p})e^{i\vec{p}\cdot\vec{x}} + a^{\dagger}(\vec{p})e^{-i\vec{p}\cdot\vec{x}})\]
\[\phi(t,\vec{x}) = e^{iH(t-t_0)} \phi(t_0,\vec{x}) e^{-iH(t-t_0)}\]
\[\phi_I(t,\vec{x}) \equiv e^{iH_0(t-t_0)} \phi(t_0,\vec{x}) e^{-iH_0(t-t_0)}\]
\[H_I(x) = \int d^3x \frac{\lambda_0^4}{4!} \phi_I^4\]
The perturbation expansion of correlation functions is
\[\langle \Omega | T \{ \phi(x) \phi(y) \} | \Omega \rangle = \lim_{T \to \infty(1-i\epsilon)} \frac{\langle 0 | T \left\{ \phi_I(x) \phi_I(y) \mathrm{exp} \left[ -i \int_{-T}^{T} dt H_I \right]\right\} | 0 \rangle}{\langle 0 | T \left\{ \mathrm{exp} \left[ -i \int_{-T}^{T} dt H_I \right]\right\} | 0 \rangle}\]
The proof can be found in chapter 4.2 of \emph{An introduction to quantum field theory (M.E.Peskin \& D.V.Schroeder)}

\subsubsection{Wick's theorem}
\[T \left\{ \phi(x_1) \phi(x_2) \cdots \phi(x_n)\right\} = N \left\{ \phi(x_1) \phi(x_2) \cdots \phi(x_n) + \mbox{ all possible contractions }\right\} \]
Normal order : all the $a$'s are to the right of all the $a^{\dagger}$.
\paragraph{Example} 
\begin{eqnarray}
\langle 0 | T \left\{ \phi(x_1) \phi(x_2) \phi(x_3) \phi(x_4)\right\}| 0 \rangle &=& D_F(x_1-x_2)D_F(x_3-x_4) \nonumber \\
&=& D_F(x_1-x_3)D_F(x_2-x_4) \nonumber \\
&=& D_F(x_1-x_4)D_F(x_2-x_3) \nonumber
\end{eqnarray}

\subsubsection{Feynman diagram}
Expand $\langle 0 | T \left\{ \phi(x) \phi(y) \mathrm{exp} \left[ -i \int_{-T}^{T} dt H_I \right]\right\} | 0 \rangle$ to the first order of $\lambda_0$
\begin{eqnarray}
& &\langle 0 | T \left\{ \phi(x) \phi(y) \frac{-i\lambda_0}{4!} \int dz^4 \phi(z) \phi(z) \phi(z) \phi(z) \right\} | 0 \rangle \nonumber \\
&=& 3 \cdot (\frac{-i\lambda_0}{4!}) D_F(x-y) \int d^4 z D_F(z-z) D_F(z-z) \nonumber \\
&+& 12 \cdot (\frac{-i\lambda_0}{4!}) \int d^4 z  D_F(x-z) D_F(y-z) D_F(z-z) \nonumber
\end{eqnarray}
It can be represented by the so called Feynman diagram.
\begin{figure}[!h]
\centering
\includegraphics[height=2cm ,width=14cm]{./pic/FD1.png}
\caption*{}
\end{figure}
\\
The symmetry factor of the first diagram is $S = \frac{4!}{3} = 8$.
The symmetry factor of the second diagram is $S = \frac{4!}{12} = 2$.
The Feynman rules for $\phi^4$ theory are:\\
(1) For each propagator, $P = D_F(x-y)$;\\
(2) For each vertex, $V = (-i\lambda_0)\int d^4z$;\\
(3) For each external point, $E=1$;\\
(4) Divided by the symmetry factor.\\
At last, we can prove that
\[\langle \Omega | T \{ \phi(x_1) \phi(x_2) \cdots \phi(x_n) \} | \Omega \rangle = \mbox{ sum of all E-connected diagrams with n external points}\]
Here, the "E-disconnected" means disconnected from all external points", being called "vacuum bubbles". They vacuum bubbles in $\langle 0 | T \left\{ \phi(x_1) \phi(x_2) \cdots \phi(x_n) \mathrm{exp} \left[ -i \int_{-T}^{T} dt H_I \right]\right\} | 0 \rangle$ are all cancelled by the $\langle 0 | T \left\{ \mathrm{exp} \left[ -i \int_{-T}^{T} dt H_I \right]\right\} | 0 \rangle$.

\subsection{LSZ reduction formula}
\subsubsection{Field strength renormalization}
The completeness relation:
\[\mathbf{1} = |\Omega\rangle\langle\Omega| +  \sum_{\lambda} \int \frac{d^3p}{(2\pi)^3} \frac{1}{2E_{\mathbf{p}}} |\lambda_{\mathbf{p}}\rangle\langle\lambda_{\mathbf{p}}|\]
Here, $E_{\mathbf{p}} = \sqrt{m_{\lambda}^2 + \mathbf{p}^2}$\\
\begin{figure}[!h]
\centering
\includegraphics[height=7cm ,width=12cm]{./pic/FSR1.png}
\caption*{}
\end{figure}
\\
Assume for now $x^0 > y^0$ and define $\langle \Omega | \phi(x) \phi(y) | \Omega \rangle_{C} = \langle \Omega | \phi(x) \phi(y) | \Omega \rangle - \langle \Omega | \phi(x)| \Omega \rangle \langle | \Omega \phi(y) | \Omega \rangle$ as connected two point function. (The term $\langle \Omega | \phi(x)| \Omega \rangle \langle | \Omega \phi(y) | \Omega \rangle$ is usually zero by symmetry; for higher spin fields, it is zero by Lorentz invariance.) The connected two point function is
\[\langle \Omega | \phi(x) \phi(y) | \Omega \rangle_{C} = \sum_{\lambda} \int \frac{d^3p}{(2\pi)^3} \frac{1}{2E_{\mathbf{p}}} \langle \Omega | \phi(x) |\lambda_{\mathbf{p}}\rangle\langle\lambda_{\mathbf{p}}| \phi(y) | \Omega \rangle\]
It can be verified that
\[\langle \Omega | \phi(x) |\lambda_{\mathbf{p}}\rangle = \langle \langle \Omega | \phi(0) | \lambda_0 \rangle e^{ipx} |_{p^0 = E_{\mathbf{p}}}\]
So,
\[\langle \Omega | \phi(x) \phi(y) | \Omega \rangle_C = \sum_{\lambda} \int \frac{d^4p}{(2\pi)^4} \frac{-i}{p^2 + m_{\lambda}^2 -i\epsilon} e^{ip(x-y)} |\langle \Omega | \phi(0) | \lambda_0 \rangle|^2\]
Analogous expressions hold for the case $y^0 > x^0$, and both cases can be summarized as
\[\langle \Omega | T \phi(x) \phi(y) | \Omega \rangle_C = \int_0^{\infty} \frac{dM^2}{2\pi} \rho(M^2) D_F(x-y;M^2)\]
and 
\[\rho(M^2) = \sum_{\lambda} (2\pi) \delta(M^2-m^2)|\langle \Omega | \phi(0) | \lambda_0 \rangle|^2 \]
\begin{figure}[!h]
\centering
\includegraphics[height=5cm ,width=12cm]{./pic/FSR2.png}
\caption*{The structure of the spectral density function $\rho(M^2)$}
\end{figure}\\
The one-particle state contribute an isolated delta function to the spectral density function, so
\[\rho(M^2) = 2\pi \delta (M^2 -m^2) \cdot Z + \mbox{ (nothing else until $M^2 > \sim (2m)^2$) }\]
$Z = |\langle \Omega | \phi(0) | \lambda_0 \rangle|^2$ is called field-strength renormalization. $m$ is the physical mass of a single particle of the $\phi$ boson. The Fourier transformation of the two point function is
\[\int d^4x e^{-ipx} \langle \Omega | T \phi(x) \phi(0) | \Omega \rangle_C  = \int_{0}^{\infty} \frac{dM^2}{2\pi} \rho(M^2) \frac{-i}{p^2+M^2-i\epsilon} = \frac{-iZ}{p^2+m^2-i\epsilon} +  \int_{\sim 4m^2}^{\infty} \frac{dM^2}{2\pi} \rho(M^2) \frac{-i}{p^2+M^2-i\epsilon}\]
\begin{figure}[!h]
\centering
\includegraphics[height=4cm ,width=12cm]{./pic/FSR3.png}
\caption*{The structure of the two point function in Fourier space}
\end{figure}\\
\subsubsection{LSZ reduction formula}
\begin{eqnarray}
&& \quad \prod_1^n \int d^4 x_i e^{-ip_ix_i} \prod_1^m d^4 y_i e^{ik_jy_j} \langle \omega | T \{\phi(x_1) \cdots \phi(x_n) \phi(y_1) \cdots \phi(y_m)\} | \Omega \rangle \nonumber \\
&& \underset{ p_i^0 \to E_{\mathbf{p}_i}\, k_i^0 \to E_{\mathbf{k}_i}}{\sim}  \left( \prod_1^n \frac{-\sqrt{Z} i}{p_i^2 + m^2 -i\epsilon} \right) \left( \prod_1^m \frac{-\sqrt{Z} i}{k_i^2 + m^2 -i\epsilon} \right) \langle \mathbf{p}_1 \cdots \mathbf{p}_n | S | \mathbf{k}_1 \cdots \mathbf{k}_m \rangle \nonumber
\end{eqnarray}
The $\sim$ means the two sides of the expression share the same singular structure around $p_i^0 \to E_{\mathbf{p}_i}$, $k_i^0 \to E_{\mathbf{k}_i}$.
The proof can be found in chapter 7.2 of \emph{An introduction to quantum field theory (M.E.Peskin \& D.V.Schroeder)}.
To express the formula above in the language of Feynman diagrams, we consider the S-matrix element for 2-particle $\to$ 2-particle for example. Note the disconnected diagram should be disregarded because they do not have the singularity structure with a product of four poles indicated by on the right side of the LSZ reduction formula. So, the exact four point function
\[\prod_1^2 \int d^4 x_i e^{-ip_ix_i} \prod_1^2 d^4 y_i e^{ik_jy_j} \langle \omega | T \{\phi(x_1)\phi(x_2)\phi(y_1) \phi(y_2)\} | \Omega \rangle \]
has the general form showed as below.
\begin{figure}[!h]
\centering
\includegraphics[height=4cm ,width=5cm]{./pic/LSZ1.png}
\caption*{}
\end{figure}\\
We can sum up the corrections to each external leg. Let $-iM^2(p^2)$ denote the sum of all one-particle-irreducible (1PI) insertions into the scalar propagator:
\begin{figure}[!h]
\centering
\includegraphics[height=2cm ,width=15cm]{./pic/LSZ2.png}
\caption*{}
\end{figure}\\
Then the exact propagator can be written as a geometric series as
\begin{figure}[!h]
\centering
\includegraphics[height=1.5cm ,width=15cm]{./pic/LSZ3.png}
\caption*{}
\end{figure}\\
The result is $\frac{-i}{p^2 + m_0^2 + M^2}$. If we expand each resummed propagator about the physical particle pole, we see that each external leg of the four-point amplitude contributes
\[\frac{-i}{p^2 + m_0^2 + M^2} \underset{p^0 \to E_{\mathbf{p}}}{\sim} \frac{-iZ}{p^2+m^2} + \mbox{ (regular) }\]
Thus, the sum of diagrams contains a product of four point poles:
\[\frac{-iZ}{p_1^2 + m^2} \frac{-iZ}{p_2^2 + m^2} \frac{-iZ}{k_1^2 + m^2} \frac{-iZ}{k_2^2 + m^2}\]
So, the S matrix element can be represented by 
\begin{figure}[!h]
\centering
\includegraphics[height=2cm ,width=8cm]{./pic/LSZ4.png}
\caption*{}
\end{figure}\\
It is easy to be generalized to the more complicated scattering cases. After Fourier transforming the n-point function to momentum space and cutting off the external legs, the Feynman rules for S-matrix element can be stated as follows:\\
(1) For each propagator, $P = \frac{-i}{p^2 + m_0^2 -i\epsilon}$;\\
(2) For each vertex, $V = -i\lambda_0$;\\
(3) For each external point, $E=1$;\\
(4) Impose momentum conservation at each vertex;\\
(5) Integrate over each undetermined loop momentum: $\int \frac{d^4p}{(2\pi)^4}$;\\
(6) Divided by the symmetry factor;\\
(7) Multiply the total momentum conservation factor $(2\pi)^4 \delta(\sum p_f - \sum p_i)$ 
We can write $\langle f | S | i \rangle = i \mathcal{M} (2\pi)^4 \delta(\sum p_f - \sum p_i)$ for convenience.

\subsection{Renormalization}
\subsubsection{Counting of ultraviolet divergence}
Consider a pure scalar theory in $d$ dimensions with a $\phi^n$ interaction term
\[\mathcal{L} = -\frac{1}{2} \partial^{\mu} \phi \partial_{\mu} \phi -\frac{1}{2}m^2 \phi^2 - \frac{\lambda}{n!}\phi^n\]
Let $N$ be the number of external lines in the diagram, $P$ the number of propagators, $V$ the number of vertices. The number of the loops in the diagram is $L=P-V+1$.  There are $n$ lines meeting at each vertex, so $nV = 2P+N$. Loosely speaking, each loop has an integral $d^d p$, each propagator has a factor $p^{-2}$, so the superficial degrees of divergence is
\[D = dL - 2P = d + [n(\frac{d-2}{2})-d)]V - (\frac{d-2}{2})N\]
According the superficial degrees of divergence of the diagram. These three possible types of ultraviolet behaviour of quantum field theories. We will refer to them as follows:
(1) Super-Renormalizable theory: Only a finite number of Feynman diagrams superficially diverge.\\
(2) Renormalizable theory: Only a finite number of amplitudes superficially diverge; however, divergences
occur at all orders in perturbation theory. \\
(3) Non-Renormalizable theory: All amplitudes are divergent at a sufficiently high order in perturbation
theory.\\
So, for $\phi^4$ theory in four dimension, $D = 4 - N$. It is a renormalizable theory. For $\phi^3$ theory in four dimension, $D = 4 - V -N$. It is a super-renormalizable theory. For $\phi^6$ theory in four dimension, $D = 4 + 2V -N$. It is a Non-renormalizable theory. \\
The superficial degrees of freedom can also be derived from dimensional analysis. The dimension of $\lambda$ is $d - \frac{n(d-2)}{2}$. Now consider an arbitrary diagram with $N$ external lines. One way that such a diagram could arise is from an interaction term $\eta \phi^N$ in the Lagrangian. The dimension of $\eta$ would then be $d - \frac{N(d-2)}{2}$, and therefore we conclude that any (amputated) diagram with $N$ external lines has dimension $d - \frac{N(d-2)}{2}$. In our theory with only the $\lambda \phi^n$ vertex, if the diagram has $V$ vertices, its divergent part is proportional to $\lambda^V \Lambda^D$, where $\Lambda$ is a high momentum cut-off and $D$ is the superficial degree of divergence.  Applying dimensional analysis, we find
\[d - \frac{N(d-2)}{2} = V[d - \frac{n(d-2)}{2}] + D\]
Note that the quantity that multiplies $V$ in this expression is just the dimension of the coupling constant $\lambda$. Thus we can characterize the three degrees of renormalizability in a second way:\\
(1) Super-Renormalizable: Coupling constant has positive mass dimension.\\
(2) Renormalizable: Coupling constant is dimensionless.\\
(3) Non-Renormalizable: Coupling constant has negative mass dimension.

\subsubsection{Renormalized Perturbation Theory}
It is generally true that the divergences in a renormalizable quantum field theory never show up in observable quantities. 
The Lagrangian of $\phi^4$ theory is 
\[\mathcal{L} = -\frac{1}{2} \partial^{\mu} \phi \partial_{\mu} \phi -\frac{1}{2}m_0^2 \phi^2 - \frac{\lambda_0}{4!}\phi^4\]
We write $m_0$ and $\lambda_0$, to emphasize that these are the bare values of the mass and coupling constant, not the values measured in experiments.
Since the theory is invariant under $\phi \to -\phi$, all amplitudes with an odd number of external legs vanish. The only divergent amplitudes are therefore
\begin{figure}[!h]
\centering
\includegraphics[height=5cm ,width=11cm]{./pic/RG1.png}
\caption*{}
\end{figure}\\
Ignoring the vacuum diagram, these amplitudes contain three infinite constants. Our goal is to absorb these constants into the three unobservable parameters of the theory: the bare mass, the bare coupling constant, and the field strength. To accomplish this goal, it is convenient to reformulate the perturbation expansion so that these unobservable quantities do not appear
explicitly in the Feynman rules. Recall that the exact two-point function has the form
\[\int d^4x \langle \Omega | \phi(x) \phi(0) | \Omega \rangle e^{-ipx} = \frac{-iZ}{p^2+m^2} + \mbox{ terms regular at } p^2 = m^2\]
We can eliminate the $Z$ from this equation by rescaling the field:
$\phi = Z^{\frac{1}{2}} \phi_r$
We also define
\[\delta_Z = Z -1 \quad \delta_m = Zm_0^2 - m^2 \quad \delta_{\lambda} = \lambda_0 Z^2 - \lambda\]
Then the Lagrangian becomes
\[\mathcal{L} = -\frac{1}{2} \partial^{\mu} \phi_r \partial_{\mu} \phi_r -\frac{1}{2}m^2 \phi_r^2 - \frac{\lambda}{4!}\phi_r^4 -\frac{1}{2} \delta_Z \partial^{\mu} \phi_r \partial_{\mu} \phi_r -\frac{1}{2}\delta_m m^2 \phi_r^2 - \frac{\delta \lambda}{4!}\phi_r^4\]
The last three terms, known as counterterms, have absorbed the infinite but unobservable shifts between the bare parameters and the physical parameters.
We give precise definitions of the physical mass and coupling constant as follows.
\begin{figure}[!h]
\centering
\includegraphics[height=3cm ,width=11cm]{./pic/RG2.png}
\caption*{}
\end{figure}\\
These equations are called renormalization conditions.
Our new Lagrangian gives a new set of Feynman rules,
\begin{figure}[!h]
\centering
\includegraphics[height=6cm ,width=9cm]{./pic/RG3.png}
\caption*{}
\end{figure}\\
We can use these new Feynman rules to compute any amplitude in $\phi^4$ theory. The procedure is as follows. Compute the desired amplitude as the sum of all possible diagrams created from the propagator and vertices shown above. The loop integrals in the diagrams will often diverge, so one must introduce a regulator. The result of this computation will be a function of the three unknown parameters $\delta_Z$, $\delta_m$, and $\delta_{\lambda}$. Adjust ( or "renormalize") these three parameters as necessary to maintain the renormalization conditions. After this adjustment, the expression for the amplitude should be finite and independent of the regulator. \\
This procedure, using Feynman rules with counterterms, is known as renormalized perturbation theory. 

\subsubsection{One loop structure of $\phi^4$ theory}

\end{document}

