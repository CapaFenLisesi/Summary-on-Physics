\documentclass{article}
\usepackage[left=1.5cm, right=1.5cm, top=3cm, bottom = 3cm]{geometry}
\usepackage{amsmath}
\usepackage{mathrsfs}
\usepackage{amsfonts}
\usepackage{amssymb}
\usepackage{graphicx}
\usepackage{float}
\usepackage{wrapfig}
\usepackage{latexsym}
\usepackage{hyperref}
\usepackage{feynmf}
\usepackage{exscale}
\usepackage{relsize}
\usepackage{bm}%bold math, for vector
\linespread{1.1}


\author{Yuyang Songsheng}
\title{Summary on QFT}

\begin{document}
\maketitle
\section{Canonical quantization for particles}
\subsection{Classical Field Theory}
Field:
\[\phi_{a}(\vec{x},t)\]
Lagrangian density:
\[\mathcal{L}(\phi_a,\dot{\phi_a},\nabla\phi_a)\]
Action:
\[ S=\int d^4x\mathcal{L} \]
Hamilton principle:
\begin{equation}
\delta S=0
\end{equation}
Euler-Lagrange equation:
\begin{equation}
\partial_{\mu}( \frac{\partial \mathcal{L}}{\partial(\partial_{\mu}\phi_{a})})-\frac{\partial \mathcal{L}}{\partial \phi_a}=0
\end{equation}

\subsubsection{Locality}
There are many terms in the Lagrangian coupling $\phi(\vec{x},t)$ directly to $\phi(\vec{y},t)$ with$\vec{x} \neq \vec{y}$.The closet we got for $\vec{x}$ label is coupling between $\phi(\vec{x})$ and $\phi(\vec{x}+\delta \vec{x})$ through the gradient term $(\nabla \phi)^2$.

\subsubsection{Lorentz Invariance}
Lorentz transformation of fields:
\begin{equation}
\phi_{a} '(x)=S_{ab}(\Lambda)\phi(\Lambda^{-1}x)
\end{equation}
Examples:
\begin{equation}
\partial_{\mu}\phi'(x)=(\Lambda^{-1})^{\nu}_{\ \mu}(\partial_{\nu}\phi)(\Lambda^{-1}x)
\end{equation}
Lagrangian density is a scalar, or more loosely, action is invariant under Lorentz transformation.

\subsubsection{Symmetries}
\paragraph{Nother's Theorem} Every continuous symmetry of the Lagrangian gives rise to a conserved current $j^{\mu}(x)$ such that the equation of motion imply $\partial_{\mu}j^{\mu}=0$.For $\phi_{a} \to \phi_{a}+\delta \phi_{a},\mathcal{L} \to \mathcal{L}+\delta \mathcal{L}$,if $\delta \mathcal{L}=\partial_{\mu}K^{\mu}$,then
\begin{equation}
j^{\mu}=\frac{\partial\mathcal{L}}{\partial(\partial_{\mu}\phi_{a})}\delta\phi_{a}-K^{\mu}
\end{equation}
\paragraph{infinitesimal translation}
$A^{\mu} \to A^{\mu}+\epsilon^{\mu}$
\begin{equation}
j^{\mu}=\epsilon_{\nu}T^{\mu \nu}
\end{equation}
\begin{equation}
T^{\mu \nu}=\frac{\partial\mathcal{L}}{\partial(\partial_{\mu}\phi_{a})}\partial^{\nu}\phi_{a}-g^{\mu\nu}\mathcal{L}
\end{equation}
\begin{equation}
\partial_{\mu} T^{\mu\nu}=0
\end{equation}
\begin{equation}
P^{\mu}=\int d^3x T^{0\mu}
\end{equation}
\end{document}