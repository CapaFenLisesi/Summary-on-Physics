\documentclass[cyan]{elegantnote}
\author{Yuyang Songsheng}
\email{songshengyuyang@gmail.com}
\zhtitle{物理}
\entitle{Physics}
\version{1.00}
\myquote{Summary is the best way to say "Good Bye"}
\logo{logo.jpg}
\cover{cover.pdf}
%green color
   \definecolor{main1}{RGB}{210,168,75}
   \definecolor{seco1}{RGB}{9,80,3}
   \definecolor{thid1}{RGB}{0,175,152}
%cyan color
   \definecolor{main2}{RGB}{239,126,30}
   \definecolor{seco2}{RGB}{0,175,152}
   \definecolor{thid2}{RGB}{236,74,53}
%cyan color
   \definecolor{main3}{RGB}{127,191,51}
   \definecolor{seco3}{RGB}{0,145,215}
   \definecolor{thid3}{RGB}{180,27,131}


\usepackage{makecell}
\usepackage{lipsum}
\usepackage{amssymb}
\usepackage{float}
\usepackage{wrapfig}
\usepackage{latexsym}
\usepackage{hyperref}
\usepackage{feynmf}
\usepackage{exscale}
\usepackage{relsize}
\usepackage{slashed}
\usepackage{bm}%bold math, for vector


\begin{document}
\maketitle
\tableofcontents
\chapter{Gauge Field}
\section{Nonabelian gauge theory}
\subsection{Nonabelian symmetries}
Consider the theory of $N$ real scalar fields $\phi_i$
\[\mathcal{L} = -\frac{1}{2}\partial_{\mu}\phi_i \partial^{\mu}\phi_i - \frac{1}{2}m^2\phi_i\phi_i - \frac{1}{16}\lambda(\phi_i\phi_i)^2\]
This lagrangian is clearly invariant under the $SO(N)$ transformation
\[\phi_i(x) \to R_{ij}\phi_j(x)\]
where $R$ is an orthogonal matrix with a positive determinant: $R^T = R^{-1}$ and $\det R = +1$.
\\
Consider an infinitesimal $SO(N)$ transformation
\[R_{ij} = \delta_{ij} + \theta_{ij} + O(\theta^2)\]
Orthogonality of $R_{ij}$ implies that $\theta_{ij}$ is real and antisymmetric. It is convenient to express $\theta_{ij}$ in terms of a basis set of hermitian matrices $T^a_{ij}$. The index a runs from $1$ to $\frac{1}{2}N(N-1)$, the number of linearly independent, hermitian, antisymmetric, $N \times N$ matrices. Commonly, we demand these matrices obey the normalization condition
\[\mathrm{Tr}(T^a T^b) = 2\delta^{ab}\]
In terms of them, we can write
\[\theta_{ij} = -i\theta^a T^a_{ij}\]
The $T^a$s are the generator matrices of $SO(N)$. The product of any two $SO(N)$ transformations is another $SO(N)$ transformation; this implies that the commutator of any two generator matrices must be a linear combination of generator matrices,
\[[T^a,T^b] = if^{abc}T^c\]
The numerical factors $f^{abc}$ are the structure coefficients of the group. If $f^{abc} = 0$, the group is abelian. Otherwise, it is nonabelian. Under our normalization condition, we have
\[f^{abc} = -\frac{i}{2} \mathrm{Tr} \left([T^a,T^b]T^c \right)\]
Using the cyclic property of the trace, we find that $f^{abc}$ must be completely antisymmetric. Taking the complex conjugate of equation above, we find that $f^{abc}$ must be real.

\begin{example}
The simplest nonabelian group is $SO(3)$. In this case, we can choose $T^a_{ij} = \epsilon^{aij}$. The commutation relations become
\[[T^a,T^b] = i\epsilon^{abc}T^c\]
\end{example}

\noindent
Consider now the theory of $N$ complex scalar fields $\phi_i$
\[\mathcal{L} = -\partial_{\mu}\phi^{\dagger}_i \partial^{\mu}\phi_i - m^2\phi_i^{\dagger}\phi_i - \frac{1}{4}\lambda(\phi_i^{\dagger}\phi_i)^2\]
This lagrangian is clearly invariant under the $U(N)$ transformation
\[\phi_i(x) \to U_{ij}\phi_j(x)\]
where $U$ is a unitary matrix, $R^{\dagger} = R^{-1}$. We can write $U_{ij} = e^{-i\theta} \widetilde{U}_{ij}$, where $\theta$ is a real parameter and $\det \widetilde{U} = 1$.
$\widetilde{U}_{ij}$ is called a special unitary matrix.
Clearly the product of two special unitary matrices is another special unitary matrix; the $N \times N$ special unitary matrices form the group $SU(N)$. 
The group $U(N)$ is the direct product of the group $U(1)$ and the group $SU(N)$.\\
Consider an infinitesimal $SU(N)$ transformation
\[\widetilde{U}_{ij} = \delta_{ij} - i\theta^aT^a_{ij} + O(\theta^2)\]
where $\theta^a$ is a set of real, infinitesimal parameters. Unitarity of $\widetilde{U}$ implies that the generator matrices $T$ are hermitian, and $\det \widetilde{U} = 1$ implies that each $T$ is traceless.
The index a runs from $1$ to $N^2 - 1$, the number of linearly independent, hermitian, traceless, $N \times N$ matrices. We can choose these matrices to obey the normalization condition
\[\mathrm{Tr}(T^a T^b) =2\delta^{ab}\]

\begin{example}
For $SU(2)$, we can choose $T^a_{ij} = \frac{1}{2}\sigma^a_{ij}$. The commutation relations become
\[[T^a,T^b] = i\epsilon^{abc}T^c\]
\end{example}

\subsection{Nonabelian gauge theory}
Consider a lagrangian with $N$ scalar or spinor fields $\phi^i(x)$ that is invariant under a continuous $SU(N)$ symmetry,
\[\phi_i(x) = U_{ij}\phi_j(x)\]
It is called a global symmetry transformation, because the matrix $U$ does not depend on the space-time label $x$.
\\
If we want to generalize the symmetry of lagrangian to local transformation
\[\phi_i(x) = U_{ij}(x)\phi_j(x)\]
terms with derivatives, such as $\partial^{\mu}\psi^{\dagger} \partial_{\mu}\phi_i$, will not remain invariant under local transformation. 
So we must include a traceless hermitian $N \times N$ gauge field $A_{\mu}(x)$, and promote ordinary derivatives $\partial_{\mu}$ to covariant derivatives $D_{\mu} = \partial_{\mu} - igA_{\mu}$ to ensure that
\[D_{\mu}\phi \to UD_{\mu}\phi\]
As a result, the gauge field must transform as
\[A_{\mu}(x) \to U(x)A_{\mu}(x)U^{\dagger}(x) + \frac{i}{g}U(x)\partial_{\mu}U^{\dagger}(x)\]
Replacing all ordinary derivatives in $\mathcal{L}$ with covariant derivatives renders $\mathcal{L}$ gauge invariant (assuming, of course, that $\mathcal{L}$ originally had a global $SU(N)$ symmetry).\\
We can write $U(x)$ in terms of the generator matrices as
$\exp[-ig\Gamma(x)T^a]$. If the structure constant $f^{abc} \neq 0$, we have a nonabelian gauge theory.
\\ \\
We still need a kinetic term for $A_{\mu}(x)$. Let us define the field strength
\[F_{\mu\nu}(x) \equiv \frac{i}{g}[D_{\mu},D_{\nu}] = \partial_{\mu}A_{\nu} - \partial_{\nu}A_{\mu} - ig[A_{\mu},A_{\nu}]\]
We can verify that the field strength transform as
\[F_{\mu\nu}(x) \to U(x)F_{\mu\nu}(x)U^{\dagger}(x)\]
Therefore, a reasonable kinetic term is
\[\mathcal{L}_{\mathrm{kin}} = - \frac{1}{2} \mathrm{Tr}(F^{\mu\nu}F_{\mu\nu})\]
Since we have taken $A_{\mu}$ to be hermitian and traceless, we can expand it in terms of the generator matrices:
\[A_{\mu}(x) = A^a_{\mu}(x)T^a\]
Similarly, we have
\[F_{\mu\nu}(x) = F^a_{\mu\nu}(x)T^a \]
We can get
\[F^{c}_{\mu\nu} = \partial_{\mu}A^c_{\nu} - \partial_{\nu}A^c_{\mu} + gf^{abc}A^a_{\mu}A^b_{\nu}\]
\[\mathcal{L}_{\mathrm{kin}} = -\frac{1}{4}F^{c\mu\nu}F_{c\mu\nu}\]
\\
Everything we have just said about $SU(N)$ also goes through for $SO(N)$, with unitary replaced by orthogonal, and traceless replaced by antisymmetric. There is also another class of compact nonabelian groups called $Sp(2N)$,
and five exceptional compact groups: $G(2)$, $F(4)$, $E(6)$, $E(7)$ and $E(8)$. Compact means that $\mathrm{Tr}(T^aT^b)$ is a positive definite matrix. Nonabelian gauge theory must be based on a compact group, because otherwise some of the
terms in $\mathcal{L}_{\mathrm{kin}}$ would have the wrong sign, leading to a Hamiltonian that is unbounded below.
\\ \\
As a specific example, let us consider quantum chromodynamics, or QCD, which is based on the gauge group $SU(3)$. There are several Dirac fields corresponding to quarks. Each quark comes in three colors; these are the
values of the $SU(3)$ index. 
There are also six flavours: up, down, strange, charm, bottom, and top. Thus we consider the Dirac field $\Psi_{iI}(x)$, where $i$ is the color index and $I$ is the flavour index. The Lagrangian is
\[\mathcal{L} = i\overline{\Psi}_{iI}\slashed{D}_{ij}\Psi_{jI} - m_I\overline{\Psi}_{I}\Psi_{I} - \frac{1}{2}\mathrm{Tr}(F^{\mu\nu}F_{\mu\nu})\]
The different quark flavours have different masses, ranging from a few MeV for the up and down quarks to $174$ GeV for the top quark. The covariant derivative is
\[D_{\mu ij} = \delta_{ij}\partial_{\mu} - igA^a_{\mu}T^a_{ij}\]
The index $a$ on $A^a_{\mu}$ runs from $1$ to $8$, and the corresponding massless spin-one particles are the eight gluons.
\\ \\
In a nonabelian gauge theory in general, we can consider scalar or spinor fields in different representations of the group. A representation of a compact nonabelian group is a set of finite-dimensional hermitian matrices $T^a_{R}$ that obey the same commutation relations as the original generator matrices $T^a$. 
Given such a set of $D(R)\times D(R)$ matrices, and
a field $\phi(x)$ with $D(R)$ components, we can write its covariant derivative as $D_{\mu} = \partial_{\mu} -igA^a_{\mu}T^a_{R}$. 
Under a gauge transformation, $\phi(x) \to U_R(x)\phi(x)$. The theory will be gauge invariant provided that
\[A^c_{\mu} \to A^c_{\mu} + g\theta^aA^b_{\mu}f^{abc} - \partial_{\mu}\theta^c\]
under infinitesimal transformation, which is independent of representation.

\end{document}
