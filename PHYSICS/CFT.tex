\chapter{Mechanics within special relativity}
\section{Basic Assumption}
First, we assume there is an upper limit of velocity of propagation of interaction $c$. Second, we assume that inertial reference frame are all the same in describing the law of physics. Then, we can find the invariant intervals when transforming from one inertial reference frame to another, $ds^2 = -c^2 dt^2 + dx^2 + dy^2 + dz^2$. 
(In the following, we assume that $c=1$.)This transformation is called Lorentz transformation, which can be written as
\[\bar{x}^{\mu} = \Lambda^{\mu}_{\phantom{\mu}\nu} x^{\nu}\]
and it is easy to verify that
\[\eta_{\mu \nu} \Lambda^{\mu}_{\phantom{\mu}\rho} \Lambda^{\nu}_{\phantom{\nu}\sigma} = \eta_{\rho \sigma},\]
where,
\[\eta_{\mu \nu} = \left[ 
\begin{matrix} 
-1& & & \\ 
& +1 & & \\
& & +1 & \\
& & & +1
\end{matrix} 
\right]\]
and
\[(\Lambda^{-1})^{\rho}_{\phantom{\rho}\nu} = \Lambda_{\nu}^{\phantom{\mu}\rho}\]

In a special case when the new reference frame move along $\hat{1}$ direction with velocity $\beta$, we have
\[\bar{x}^{0} = \gamma x^0 - \gamma \beta x^1\]
\[\bar{x}^{1} = -\gamma \beta x^0 + \gamma x^1\]

Some physical quantity will behave like a tensor (vector,scalar) when transforming form one inertial frame to another. For example,
\paragraph{scalar} proper time: $d \tau$, mass: $m$, electrical charge $e$
\paragraph{vector} four velocity: $v^{\mu} = \frac{dx^{\mu}}{d \tau}$, four momentum: $p^{\mu} = m v^{\mu}$, four acceleration: $a^{\mu} = \frac{du^{\mu}}{d \tau}$, four force: $f^{\mu} = m a^{\mu}$

\section{"Three vector"}
\noindent
three velocity: $\hat{u}^{i} = \frac{dx^i}{dt}$
\[u^0 = \gamma_v, u^i = \gamma \hat{u}^i\]
transformation of three velocity when we boost along $\hat{1}$ direction:
\[\bar{\hat{v}}^1 = \frac{\hat{v}^1 - \beta}{1-\hat{v}^1 \beta}\]
\[\bar{\hat{v}}^2 = \frac{\hat{v}^2}{\gamma(1-\hat{v}^2 \beta})\]
\[\bar{\hat{v}}^3 = \frac{\hat{v}^3}{\gamma(1-\hat{v}^3 \beta})\]
three momentum: $\hat{p}^{i} = p^i$
\[\hat{p}^{i} \gamma_v \hat{v}^i\]
three acceleration: $\hat{a}^{i} = \frac{dv^i}{dt}$\\
three force: $\hat{f}^i = \frac{d\hat{p}^i}{dt}$
\[f^i = \gamma_v \hat{f}^i\]
Energy: $E = p^0 = m u^0 = \gamma_v m$

\section{Mechanics}
\noindent
Revised newton's second law:
\[f^{\mu} = \frac{dp^{\mu}}{d\tau}\]
It can be written in three vector language as
\[\hat{f}^i = \gamma_v m \hat{a}^i + \gamma_v^3 (\hat{a}^j \hat{v}_j) m \hat{v}^i\]

\section{Lagrangian formulation}
\[S=-m\int_{a}^{b} d\tau, \ \ \ \ \delta x^{\mu}(a) = \delta x^{\mu}(b) = 0\]
\[\delta S = 0 \Rightarrow m\frac{du^{\mu}}{d\tau} = 0\]

\section{Hamiltonian formulation}
\[S = -m \int_{t_1}^{t_2} \sqrt{1-\dot{x}_i\dot{x}^i} dt\]
\[L = - m \sqrt{1-\dot{x}_i\dot{x}^i}\]
\[\pi^i = \frac{\partial L}{\partial \dot{x}_i} = \gamma m \eta^{ij}\dot{x}_j\]
\[H = \pi^i \dot{x}_i - L = \gamma m = \sqrt{m^2 + \pi^i \pi_i}\]

\subsubsection{Hamilton equation}
\[\dot{\pi}^i = 0, \ \ \ \ \dot{x_i} = \eta_{ij}\frac{\pi^j}{\sqrt{m^2 + \pi^k \pi_k}}\]

\subsubsection{Hamiltonian-Jacobi equation}
\[H = -\frac{\partial S}{\partial t}, \ \ \ \ \pi^i = \frac{\partial S}{\partial x_i}\]
If we define $p^0 = H$, $p^i = \pi^i$, then we can verify that $p^{\mu} = \frac{\partial S}{\partial x_{\mu}}$. So, $p^{\mu}$ is a vector under Lorentz transformation. The Hamiltonian-Jacobi equation can be written as
\[(\frac{\partial S}{\partial t})^2 = m^2 + (\frac{\partial S}{\partial x})^2 + (\frac{\partial S}{\partial y})^2 + (\frac{\partial S}{\partial z})^2\]

\section{Symmetry and conservation law}
\subsubsection{Translational symmetry and conservation of momentum}
\[\bar{x}^{\mu} = x^{\mu} + \delta x^{\mu}\]
\[\delta S = \sum mu_{\mu} \delta x^{\mu}|_a^b = 0 \]
$\sum p^{\mu}$ is conserved.
\subsubsection{Rotational symmetry and conservation of angular momentum}
\[\bar{x}^{\mu} = x^{\mu} + x_{\nu}\delta \Omega^{\mu \nu}\]
\[\delta S = \sum mu^{\mu} x^{\nu} \delta \Omega_{\mu \nu}|_a^b = 0 \]
$\sum M^{\mu \nu} $ is conserved, where $M^{\mu \nu} = x^{\mu}p^{\nu} - x^{\nu}p^{\mu}$.

\chapter{Classical field theory}
\section{Lagrangian formulation}
\[S = \int \mathcal{L}(\phi_a,\dot{\phi}_a,\nabla \phi_a) d^4 x, \ \ \ \ \delta \phi_a |_{\Sigma} = 0\]
\[\delta S = 0 \Rightarrow \partial_{\mu} \left (\frac{\partial \mathcal{L}}{\partial (\partial_{\mu} \phi_a)} \right ) - \frac{\partial \mathcal{L}}{\partial \phi_a} = 0\]
\subsubsection{Locality of the theory}
There are no terms in the Lagrangian coupling $\phi(\vec{x},t)$ directly to  $\phi(\vec{y},t)$ with $\vec{x} \neq \vec{y}$. The closet we get for the $\vec{x}$ label is coupling between $\phi(\vec{x},t)$ and $\phi(\vec{x}+\delta\vec{x},t)$ through the gradient term $\nabla \phi$.
\subsubsection{Lorentz invariance}
\noindent
Scalar fields:
\[\bar{\phi}(x) = \phi(\Lambda^{-1} x)\]
Vector fields:
\[\bar{A}^{\mu}(x) = \Lambda^{\mu}_{\phantom{\mu}\nu} A^{\nu}(\Lambda^{-1}x)\]
\[\bar{A}_{\mu}(x) = (\Lambda^{-1})^{\nu}_{\phantom{\mu}\mu} A_{\nu}(\Lambda^{-1}x) = \Lambda_{\mu}^{\phantom{\mu}\nu}A_{\nu}(\Lambda^{-1}x)\]
\[\overline{\partial_{\mu}\phi}(x) = (\Lambda^{-1})^{\nu}_{\phantom{\mu}\mu} \partial_{\nu} \phi (\Lambda^{-1}x) = \Lambda_{\mu}^{\phantom{\mu}\nu} \partial_{\nu} \phi (\Lambda^{-1}x)\]
Lagrangian is a scalar, or more loosely, action is invariant under Lorentz transformation.

\section{Symmetry and conservation law}
\begin{newthem}[Noether's theorem]
Every continuous symmetry of the Lagrangian gives rise to a conserved current $j^{\mu}(x)$ such that the equation of motion imply $\partial_{\mu} j^{\mu} = 0$.
Suppose that the infinitesimal transformation is
\[\phi_a \rightarrow \phi_a + \delta \phi_a\]
\[\mathcal{L} \rightarrow + \mathcal{L} + \delta \mathcal{L} \]
and if $\delta \mathcal{L} = \partial_{\mu} K^{\mu} = 0$, we can get
\[j^{\mu} = \frac{\partial \mathcal{L}}{\partial (\partial_{\mu} \phi_a)} \delta \phi_a - K^{\mu}\]
\end{newthem}

\subsubsection{space-time translation}
\noindent
$\bar{x} = x - a$ 
\[j^{\mu} = -a_{\nu} T^{\mu \nu}\]
\[T^{\mu \nu} \equiv -\frac{\partial \mathcal{L}}{\partial(\partial_{\mu}\phi_a)} \partial^{\nu} \phi_a + \eta^{\mu \nu} \mathcal{L}\]
If we define $P^{\mu} = \int T^{0 \mu} d^3 x$,then we have
\[\frac{d P^{\mu}}{dt} = 0\]

\subsubsection{Lorentz Transformation} 
\noindent
$\bar{x}^{\mu} = x^{\mu} + \delta \omega^{\mu}_{\phantom{\mu}\nu} x^{\nu}$\\
The infinitesimal Lorentz transformation can be written as $I+\delta \omega^{\mu}_{\phantom{\mu}\nu}$
\[\delta \omega^{\mu}_{\phantom{\mu}\nu} = \left[ 
\begin{matrix} 
0       & \beta_1   & \beta_2   & \beta_3   \\ 
\beta_1 & 0         & -\theta_3 & \theta_2  \\
\beta_2 & \theta_3  & 0         & -\theta_1 \\
\beta_3 & -\theta_2 & \theta_1  & 0
\end{matrix} 
\right]\]
This time, we assume that
\[\bar{\phi}_a(x) = S_{a}^{\phantom{a}b}\phi_b(\Lambda^{-1}x)\]
In the limit of infinitesimal Lorentz transformation, we have
\[S_{a}^{\phantom{a}b} = \delta_{a}^{\phantom{a}b}+\frac{1}{2} \delta \omega_{\alpha \beta} (\Sigma^{\alpha \beta})_{a}^{\phantom{a}b} \]
\[j^{\mu} = -\frac{1}{2} M^{\mu \nu \rho}  \delta \omega_{\nu \rho}\]
\[M^{\mu \nu \rho} \equiv x^{\nu}T^{\mu \rho} - x^{\rho} T^{\mu \nu} - \frac{\partial \mathcal{L}}{\partial (\partial_{\mu}\phi_a)}(\Sigma^{\nu \rho})_{a}^{\phantom{a}b}\phi_b\]
If we define $M^{\nu \rho} = \int M^{0 \nu \rho} d^3 x$, then we have
\[\frac{dM^{\nu \rho}}{dt} = 0\]

\section{Functional derivatives}
\begin{newdef}[Functional derivatives]
Given a manifold $M$ representing (continuous/smooth) functions $\rho$ (with certain boundary conditions etc.), and a functional $F$ defined as
\[F\colon M\rightarrow \mathbb {R} \quad {\mbox{or}}\quad F\colon M\rightarrow \mathbb {C} \,,\]
the functional derivative of $F[\rho]$, denoted $\frac{\delta F}{\delta \rho}$,is defined by
\[{\begin{aligned}\int {\frac {\delta F}{\delta \rho }}(x)\phi (x)\;dx&=\lim _{\varepsilon \to 0}{\frac {F[\rho +\varepsilon \phi ]-F[\rho ]}{\varepsilon }}\\&=\left[{\frac {d}{d\epsilon }}F[\rho +\epsilon \phi ]\right]_{\epsilon =0},\end{aligned}}\]
where $\phi$ is an arbitrary function. The quantity $\epsilon \phi$ is called the variation of $\rho$. 
\end{newdef}

Like the derivative of a function, the functional derivative satisfies the following properties, where $F[\rho]$ and $G[\rho]$ are functionals:\\
Linearity:
\[{\frac {\delta (\lambda F+\mu G)[\rho ]}{\delta \rho (x)}}=\lambda {\frac {\delta F[\rho ]}{\delta \rho (x)}}+\mu {\frac {\delta G[\rho ]}{\delta \rho (x)}},\]
where $\lambda$, $\mu$ are constants.\\
Product rule:
\[{\frac {\delta (FG)[\rho ]}{\delta \rho (x)}}={\frac {\delta F[\rho ]}{\delta \rho (x)}}G[\rho ]+F[\rho ]{\frac {\delta G[\rho ]}{\delta \rho (x)}}\,,\]
Chain rules:\\
If $F$ is a functional and $G$ an operator, then
\[{\displaystyle \displaystyle {\frac {\delta F[G[\rho ]]}{\delta \rho (y)}}=\int dx{\frac {\delta F[G]}{\delta G(x)}}_{G=G[\rho ]}\cdot {\frac {\delta G[\rho ](x)}{\delta \rho (y)}}\ .}\]
If $G$ is an ordinary differentiable function $g$, then this reduces to
\[{\displaystyle \displaystyle {\frac {\delta F[g(\rho )]}{\delta \rho (y)}}={\frac {\delta F[g(\rho )]}{\delta g[\rho (y)]}}\ {\frac {dg(\rho )}{d\rho (y)}}\ .} \]
\begin{newprop}[Properties of functional derivatives]
\[\frac{\delta F}{\delta \rho} (y) = \lim_{\epsilon \to \infty} \frac{1}{\epsilon} \{ F[\rho(x) + \epsilon \delta(x-y)] - F[\rho(x)] \}\]
\[\frac{\delta f(x)}{\delta f(y)} = \delta(x-y)\]
\[\frac{\delta}{\delta f(y)} \int g\left( f(x)\right) dx =  g'(f(y))\]
\[\frac{\delta f'(x)}{\delta f(y)} = \frac{d}{dx}\delta(x-y)\]
\[\frac{\delta}{\delta f(y)} \int g\left( f'(x)\right) dx = -\frac{d}{dy} g'(f'(y))\]
\end{newprop}

\section{Hamiltonian formulation}
\[\pi^a(x) = \frac{\partial \mathcal{L}}{\partial \dot{\phi_a}}\]
\[\mathcal{H}(\phi_a,\nabla \phi_a,\pi^a) = \pi^a \dot{\phi_a} - \mathcal{L}\]
\[H = \int \mathcal{H} d^3 x\]
Now, we can get the Hamilton equation form $\delta S =0$,
\[\dot{\phi_a}(\vec{x},t) = \frac{\delta}{\delta \pi^a(\vec{x},t)} H = \frac{\partial \mathcal{H}}{\partial \pi^a}\]
\[\dot{\pi^a}(\vec{x},t) = -\frac{\delta}{\delta \phi_a(\vec{x},t)} H = - \frac{\partial \mathcal{H}}{\partial \phi_a} + \left(\frac{\partial \mathcal{H}}{\partial \phi_{a,b}}\right)_{,b}\]

\subsection{Poission bracket}
\noindent
First, we demand that
\[[\phi_a(\vec{x}),\phi_b(\vec{y})] = [\pi^a(\vec{x}),\phi_b(\vec{y})]= 0\]
\[[\phi_a(\vec{x}),\pi^b(\vec{y})] = \delta^{b}_{a} \delta(\vec{x}-\vec{y})\]
then, we assume the bracket operation has the same properties as the Poission bracket in classical mechanics. And we also assume that
\[[\partial_x A(\vec{x}),B(\vec{y})] = \partial_x [A(\vec{x}),B(\vec{y})]\]
and
\[\left[\int d^3 x A(\vec{x}),B(\vec{y})\right] = \int d^3 x [A(\vec{x}),B(\vec{y})]\]
We can verify that
\[[W[\phi(\vec{x}),\pi(\vec{x})],Z[\phi(\vec{x}),\pi(\vec{x})]] = \int d^3x \left\{ \frac{\delta W}{\delta \phi(\vec{x})} \frac{\delta Z}{\delta \pi(\vec{x})} - \frac{\delta W}{\delta \pi(\vec{x})} \frac{\delta Z}{\delta \phi(\vec{x})} \right\}\]
Specially,
\[[\phi_a(\vec{x}),H] = \frac{\delta }{\delta \pi^a(\vec{x})} H, \ \ \ \ [\pi^a(\vec{x}),H] = -\frac{\delta }{\delta \phi_a(\vec{x})} H\]
So, the Hamilton equation can be written as
\[\dot{\phi_a} = [\phi_a,H], \ \ \ \ \dot{\pi^a} = [\pi^a,H]\]
Further more, we can prove
\[\frac{dO(\phi,\pi,t)}{dt} = [O,H] + \frac{\partial O}{\partial t}\]
and
\[\frac{d[A,B]}{dt} = [A,\frac{dB}{dt}] + [\frac{dA}{dt},B]  \]

\subsection{Momentum}
\noindent
It is easy to verify that
\[P^{0} = H, \ \ \ \ P^{i} = \int -\pi^a \partial^i \phi_a d^3 x\]
And we can get the commutation relationship that
\begin{eqnarray}
\left[\phi_a,P^{\mu}\right] &=& -\partial^{\mu} \phi_a \nonumber \\
\left[\pi^a,P^{\mu}\right] &=& -\partial^{\mu} \pi^a \nonumber \\
\left[P^{\mu},P^{\nu}\right] &=& 0 \nonumber 
\end{eqnarray}


\subsection{Angular momentum}
\noindent
It is easy to verify that
\[M^{\mu \nu} = \int (x^{\mu}T^{0\nu}-x^{\nu}T^{0\mu}-\pi^a(\Sigma^{\mu \nu})_{a}^{\phantom{a}b}\phi_b) d^3 x\]
We denote that
\[M_{L}^{\mu \nu} = \int (x^{\mu}T^{0\nu}-x^{\nu}T^{0\mu}) d^3 x \quad M_S^{\mu \nu} = \int (-\pi^a(\Sigma^{\mu \nu})_{a}^{\phantom{a}b}\phi_b) d^3 x\]
\[(L^{\mu \nu})_a^{\phantom{a}b} = -(x^{\mu}\partial^{\nu}-x^{\nu}\partial^{\mu})\delta_a^{\phantom{a}b} \quad (S^{\mu \nu})_a^{\phantom{a}b} = -(\Sigma^{\mu \nu})_a^{\phantom{a}b}\]
So, we have the commutation relationship that
\[M^{\mu \nu} = M_L^{\mu \nu} + M_S^{\mu \nu}\]
\[[\phi_a,M_L^{\mu \nu}] = (L^{\mu \nu})_a^{\phantom{a}b} \phi_b \quad [\phi_a,M_S^{\mu \nu}] = (S^{\mu \nu})_a^{\phantom{a}b} \phi_b\]
\[[\pi^a,M_L^{\mu \nu}] = (L^{\mu \nu})_b^{\phantom{b}a}\pi^{b}  \quad [\pi^a,M_S^{\mu \nu}] = - (S^{\mu \nu})_b^{\phantom{b}a} \pi^b \]
Because $\frac{d M^{\mu \nu}}{dt} = 0$, we can prove that
\[[[\phi(x),M^{\mu \nu}],M^{\rho \sigma}] = (L^{\mu \nu}+S^{\mu \nu})(L^{\rho \sigma}+S^{\rho \sigma})\phi(x)\]
and then we can get the communication relationship from the Jacobi identity,
\[[\phi(x),[M^{\mu \nu},M^{\rho \sigma}]] = (L^{\mu \nu}L^{\rho \sigma}-L^{\rho \sigma}L^{\mu \nu} + S^{\mu \nu}S^{\rho \sigma}-S^{\rho \sigma}S^{\mu \nu})\phi(x)\]
We can prove that
\[L^{\mu \nu}L^{\rho \sigma}-L^{\rho \sigma}L^{\mu \nu} = -g^{\nu \rho}L^{\mu \sigma} + g^{\sigma \mu}L^{\rho \nu} + g^{\mu \rho}L^{\nu \sigma} - g^{\sigma \nu}L^{\rho \mu}\]
If we demand that
\[S^{\mu \nu}S^{\rho \sigma}-S^{\rho \sigma}S^{\mu \nu} = -g^{\nu \rho}S^{\mu \sigma} + g^{\sigma \mu}S^{\rho \nu} + g^{\mu \rho}S^{\nu \sigma} - g^{\sigma \nu}S^{\rho \mu}\]
We can get get the communication relationship of the $M^{\mu \nu}$,
\[[M^{\mu \nu},M^{\rho \sigma}] = -g^{\nu \rho}M^{\mu \sigma} + g^{\sigma \mu}M^{\rho \nu} + g^{\mu \rho}M^{\nu \sigma} - g^{\sigma \nu}M^{\rho \mu}\]
up to the possibility of a term on the right-hand side that commutes with $\phi(x)$ and its derivatives.\\
We now define $J_i \equiv \frac{1}{2} \epsilon_{ijk} M^{jk}$ and $K_i \equiv M^{i0}$,so
\[M^{\mu \nu} = \left[ 
\begin{matrix} 
0   & -K_1 & -K_2 & -K_3 \\ 
K_1 & 0    & J_3  & -J_2 \\
K_2 & -J_3 & 0    &  J_1 \\
K_3 & J_2  & -J_1 &  0
\end{matrix} 
\right]\] 
the communication relationship can be written as
\begin{eqnarray}
\left[J_i,J_j\right] &=& \epsilon_{ijk}J_k \nonumber \\
\left[J_i,K_j\right] &=& \epsilon_{ijk}K_k \nonumber \\
\left[K_i,K_j\right] &=& -\epsilon_{ijk}J_k \nonumber
\end{eqnarray}
We can use the similar method to derive that
\[[P^{\mu},M^{\rho \sigma}] = g^{\mu \sigma}P^{\mu} - g^{\mu \rho}P^{\sigma}\]
It can also be written as
\begin{eqnarray}
\left[J_i,H\right] &=& 0 \nonumber \\
\left[J_i,P_j\right] &=& \epsilon_{ijk}P_k \nonumber \\
\left[K_i,H\right] &=& P_i \nonumber \\
\left[K_i,P_j\right] &=& \delta_{ij}H \nonumber
\end{eqnarray}
At last, we define $L_i \equiv \frac{1}{2} \epsilon_{ijk} M_L^{jk}$ and $S_i \equiv \frac{1}{2} \epsilon_{ijk} M_S^{jk}$
we can demonstrate that
\begin{eqnarray}
\left[L_i,S_j\right] &=& 0 \nonumber \\
\left[S_i,P_j\right] &=& 0 \nonumber \\
\left[L_i,P_j\right] &=& \epsilon_{ijk}P_k \nonumber
\end{eqnarray}
